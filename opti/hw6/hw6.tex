\documentclass{article}
\usepackage[sexy, hdr, fancy]{evan}
\setlength{\droptitle}{-4em}

\lhead{Homework 6}
\rhead{Introduction to Optimization}
\lfoot{}
\cfoot{\thepage}

\begin{document}
\title{Homework 6}
\maketitle
\thispagestyle{fancy}

\begin{enumerate}
	\item Recall that by strong duality there are exactly four possibilities for any linear program LP and its dual DP. Namely,
		\begin{enumerate}[i)]
				\ii LP and DP are both feasible and have equal optimal objective function values
				\ii LP is unbounded and DP is infeasible
				\ii LP is infeasible and DP is unbounded
				\ii LP and DP are both infeasible
		\end{enumerate}

		Consider the linear program (LP) min $c^T x$ such that $Ax=b$ and $x\ge 0$ where $A$ happens to be a matrix of all zeros. Determine and show which of the four above scenarios occurs in (each possible) different choices of $b$ and $c.$
		\begin{soln}
			The dual program (DP) is given by max $y^T b$ such that $A^T y \le c, y$ unrestricted.
			\begin{enumerate}[i)]
				\item In this case, since $A$ is a matrix of all 0, we must have $b=\vec{0}$ in order for (LP) to be feasible. Then $A^T y = 0\le c$ so we must have $c\ge 0$ in order for (DP) to be feasible. Then the optimal objective function value is exactly 0.

				\item In this case, if (LP) is feasible, then we must have $b=\vec{0}$ again. Then if any entry of $c^T$ is negative, the objective function value is unbounded since any $x\ge 0$ satisfies $Ax=0.$ This also corresponds to $A^T y = 0 \le c$ not being feasible if $c$ is not non-negative.

				\item In this case, if (DP) is feasible, then we must have $A^T y = 0\le c.$ Then if any entry of $b$ is non-zero, then (LP) is infeasible, which also corresponds to the (DP) being unbounded since $y$ can be anything. 

				\item In this case, if $b\neq 0$ and $c\not\ge 0,$ both (LP) and (DP) are infeasible.
					
			\end{enumerate}
		\end{soln}

	\item Consider the following LP:
		\begin{align*}
			\max \quad 4x_1-6x_2 & \\
			\text{s.t.}\quad x_1+3x_2 &\ge 3 \\
			5x_1-x_2 &\le 7 \\
			-2x_1+3x_2 &= 9 \\
			x_1\ge 0 &\quad x_2\text{ unrestricted}
		\end{align*}

		\begin{enumerate}[a)]
			\item Write LP in standard form, then write its standard form dual.
				\begin{soln}
					Let $x_2=a-b$ where $a, b\ge 0.$ Maximizing the objective function is equivalent to minimizing its negative. Substitute this into the constraints and the objective function:
					\begin{align*}
						\min\quad -4x_1+6a-6b & \\
						\text{s.t.}\quad x_1+3a-3b &\ge 3 \\
						5x_1-a+b &\le 7 \\
						-2x_1+3a-3b &= 9 \\
						x_1, a, b&\ge 0
					\end{align*}
					Add slack variables $c, d\ge 0$ the first and second constraints become
					\begin{align*}
						x_1+3a-3b-c &= 3 \\
						5x_1-a+b+d &= 7
					\end{align*} so finally the standard form is given by min $c^T x$ where $Ax=b, x\ge 0$ where \[A=\begin{bmatrix}
							1 & 3 & -3 & -1 & 0 \\
							5 & -1 & 1 & 0 & 1 \\
							-2 & 3 & -3 & 0 & 0
						\end{bmatrix}\quad b=\begin{bmatrix}
							3 \\ 7 \\ 9
						\end{bmatrix}\quad c=\begin{bmatrix}
							-4 & 6 & -6 & 0 & 0
					\end{bmatrix}^T\] The corresponding dual is given by
					\begin{align*}
						\max\quad b^T y & \\
						\text{s.t.}\quad A^T y &\le c \\
						y\text{ unrestricted}
					\end{align*} or 
					\begin{align*}
						\max\quad 3y_1+7y_2+9y_3 & \\
						\text{s.t.}\quad y_1+5y_2-2y_3 &\le -4 \\
						3y_1-y_2+3y_3 &\le 6 \\
						-3y_1+y_2-3y_3 &\le -6 \\
						-y_1 &\le 0 \\
						y_2 &\le 0 \\
						y_1, y_2, y_3&\text{ unrestricted}
					\end{align*}
				\end{soln}

			\item Write LP in canonical form, then write its canonical form dual.
				\begin{soln}
					As before, we substitute $x_2=a-b$ and convert maximization to minimization:
					\begin{align*}
						\min\quad -4x_1+6a-6b & \\
						\text{s.t.}\quad x_1+3a-3b &\ge 3 \\
						5x_1-a+b &\le 7  \\
						-2x_1+3a-3b &= 9 \\
						x_1, a, b\ge 0
					\end{align*}
					Now, flip the sign in the second constraint, and convert the third constraint into two inequalities:
					\begin{align*}
						-5x_1+a-b &\ge -7 \\
						-2x_1+3a-3b &\ge 9 \\
						2x_1-3a+3b &\ge -9
					\end{align*} so finally the canonical form is given by $Ax\ge b, x\ge 0$ where \[ A=\begin{bmatrix}
							1 & 3 & -3 \\
							-5 & 1 & -1 \\
							-2 & 3 & -3 \\
							2 & -3 & 3
						\end{bmatrix}\quad b=\begin{bmatrix}
							3 \\ -7 \\ 9 \\ -9
						\end{bmatrix}\quad c=\begin{bmatrix}
							-4 & 6 & -6
					\end{bmatrix}^T\] The corresponding dual is given by
					\begin{align*}
						\max\quad b^T z & \\
						\text{s.t.}\quad A^T z&\le c \\
						z\ge 0
					\end{align*} or
					\begin{align*}
						\max\quad 3z_1-7z_2+9z_3-9z_4 & \\
						\text{s.t.}\quad z_1-5z_2-2z_3+2z_4 &\le -4 \\
						3z_1+z_2+3z_3-3z_4 &\le 6 \\
						-3z_1-z_2-3z_3+3z_4 &\le -6 \\
						z_1, z_2, z_3, z_4 &\ge 0
					\end{align*}
				\end{soln}

			\item Show how the dual programs from part a, b, are equivalent.
				\begin{proof}
					In part a, the condition $-y_1\le 0$ means that $y_1\ge 0,$ so let $y_1=z_1\ge 0.$ Then we have $y_2\le 0$ so let $-y_2=z_2\ge 0.$ Then since $y_3$ is unrestricted, let $y_3=z_3-z_4,$ where $z_3, z_4\ge 0.$ Making these substitutions into the result from part a, we have
					\begin{align*}
						\max\quad 3z_1-7z_2+9z_3-9z_4 & \\
						\text{s.t.}\quad z_1-5z_2-2z_3+2z_4 &\le -4 \\
						3z_3+z_2+3z_3-3z_4 &\le 6 \\
						-3z_1-z_2-3z_3+3z_4 &\le -6 \\
						z_1, z_2, z_3, z_4 &\ge 0
					\end{align*} which is exactly the problem we obtained in part b.
					
				\end{proof}

		\end{enumerate}

	\item Consider the linear program (LP) min $c^T x$ such that $Ax=b, x\ge 0$ where \[A=\begin{bmatrix}
				-6 & -5 & 25 & 3 & -85 & 4 & 30 \\
				24 & -2 & 28 & 6 & -55 & 1 & -9 \\
				9 & -5 & 11 & 2 & -55 & -1 & 10
			\end{bmatrix}\quad b=\begin{bmatrix}
				62 \\ 62 \\ 3
			\end{bmatrix}\quad c=\begin{bmatrix}
				23 & 1 & -17 & -1 & 52 & -6 & -12
		\end{bmatrix}^T\]
		Write the dual program (DP) and then solve the dual problem.
		\begin{soln}
			The dual program (DP) is given by max $y^T b$ such that $A^T y\le c.$ That is, 
		\end{soln}<++>

	\item Concerning the specific LP discussed in the previous problem:

		\begin{enumerate}[a)]
			\item Suppose you may change the value of $b_1$ (currently 62) to anything you want. To what value should you set $b_1$ in order to have the adjusted LP have optimal objective function value -70? Compute the optimal solution for the adjusted LP.

			\item Suppose you may change the value of $b_2$ (currently 62) to anything you want. To what value should you set $b_2$ in order to have the adjusted LP have optimal objective function value -68.5? Compute the optimal solution for the adjusted LP.

		\end{enumerate}

	\item Consider the linear program (LP) min $c^T x$ such that $Ax=b, x\ge 0$ where \[ A=\begin{bmatrix}
				7 & 7 & 45 & -1 & 3 & -53 & -68 \\
				9 & -5 & 27 & -115 & 7 & -129 & 42 \\
				5 & -3 & 63 & -96 & 10 & -109 & 86
			\end{bmatrix}\quad b=\begin{bmatrix}
				26 \\ 18 \\ 34
			\end{bmatrix}\quad c=\begin{bmatrix}
				1 & 7 & -37 & 94 & -9 & 76 & -146
		\end{bmatrix}^T\] Write the dual program (DP) and then solve the dual problem. Compute the optimal dual variables using one of the optimal bases for (LP), and then repeat this for the other optimal basis.

\end{enumerate}

\end{document}

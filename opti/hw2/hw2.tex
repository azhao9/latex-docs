\documentclass{article}
\usepackage[sexy, hdr, fancy]{evan}
\setlength{\droptitle}{-4em}

\lhead{Homework 2}
\rhead{Introduction to Optimization}
\lfoot{}
\cfoot{\thepage}

\begin{document}
\title{Homework 2}
\maketitle
\thispagestyle{fancy}

\begin{enumerate}
	\item Geometrically solve the following LP:
		\begin{align*}
			\min \quad x_1-2x_2 & \\
			\text{s.t.}\quad x_1-x_2&\ge 4 \\
			-3x_1-2x_2&\ge -18 \\
			-3x_1+x_3&\ge-9 \\
			x_1, x_2 &\ge 0
		\end{align*}
		Again geometrically solve for each of the LPs with the same feasible region as above but with the respective objective functions $-x_1+2x_2,$ and $-3x_1-x_2,$ and $3x_1+x_2,$ and $3x_1-3x_2.$ 

	\item Consider the matrix \[A=\begin{bmatrix}
				4 & 3 & 6 & 9 \\
				3 & 7 & 1 & 8 \\
				7 & 2 & 2 & 1
		\end{bmatrix}\] by considering the appropriate row operations and their associated elementary matrices, compute an invertible matrix $B\in\RR^{3\times 3}$ such that \[B\cdot A = \begin{bmatrix}
				1 & 0 & * & * \\
				0 & 1 & * & * \\
				0 & 0 & * & *
		\end{bmatrix} \] where $*$ denotes a nonzero number.
		\begin{soln}
			The row operations we perform and the corresponding elementary matrices are:
			\begin{align*}
				\begin{bmatrix}
					4 & 3 & 6 & 9 \\
					3 & 7 & 1 & 8 \\
					7 & 2 & 2 & 1
				\end{bmatrix} \to \begin{bmatrix}
					1 & 3/4 & * & * \\
					3 & 7 & 1 & 8 \\
					7 & 2 & 2 & 1
				\end{bmatrix} &\implies \begin{bmatrix}
					1/4 & 0 & 0 \\
					0 & 1 & 0 \\
					0 & 0 & 1
				\end{bmatrix} \\
				\begin{bmatrix}
					1 & 3/4 & * & * \\
					3 & 7 & 1 & 8 \\
					7 & 2 & 2 & 1
				\end{bmatrix} \to \begin{bmatrix}
					1 & 3/4 & * & * \\
					0 & 19/4 & * & * \\
					7 & 2 & 2 & 1
				\end{bmatrix} &\implies \begin{bmatrix}
					1 & 0 & 0 \\
					-3 & 1 & 0 \\
					0 & 0 & 1
				\end{bmatrix} \\
				\begin{bmatrix}
					1 & 3/4 & * & * \\
					0 & 19/4 & * & * \\
					7 & 2 & 2 & 1
				\end{bmatrix} \to \begin{bmatrix}
					1 & 3/4 & * & * \\
					0 & 19/4 & * & * \\
					0 & -13/4 & * & * 
				\end{bmatrix} &\implies \begin{bmatrix}
					1 & 0 & 0 \\
					0 & 1 & 0 \\
					-7 & 0 & 1
				\end{bmatrix} \\
				\begin{bmatrix}
					1 & 3/4 & * & * \\
					0 & 19/4 & * & * \\
					0 & -13/4 & * & *
				\end{bmatrix} \to \begin{bmatrix}
					1 & 3/4 & * & * \\
					0 & 1 & 0 * 0 \\
					0 & -13/4 & * & *
				\end{bmatrix} &\implies \begin{bmatrix}
					1 & 0 & 0 \\
					0 & 4/19 & 0 \\
					0 & 0 & 1
				\end{bmatrix} \\
				\begin{bmatrix}
					1 & 3/4 & * & * \\
					0 & 1 & * & * \\
					0 & -13/4 & * & *
				\end{bmatrix} \to \begin{bmatrix}
					1 & 3/4 & * & * \\
					0 & 1 & * & * \\
					0 & 0 & * & *
				\end{bmatrix} &\implies \begin{bmatrix}
					1 & 0 & 0 \\
					0 & 1 & 0 \\
					0 & -13/4 & 1
				\end{bmatrix} \\
				\begin{bmatrix}
					1 & 3/4 & * & * \\
					0 & 1 & * & * \\
					0 & 0 &* & *
				\end{bmatrix} \to \begin{bmatrix}
					1 & 0 & * & * \\
					0 & 1 & * & * \\
					0 & 0 & * & *
				\end{bmatrix} &\implies \begin{bmatrix}
					1 & -3/4 & 0 \\
					0 & 1 & 0 \\
					0 & 0 & 1
				\end{bmatrix}
			\end{align*}
			Multiplying these elementary matrices, going left to right from bottom to top, we have 
		\end{soln}<++>

	\item Recall the nonlinear optimization problem which we have previously considered: 
		\begin{align*}
			\max \quad 1+x_1^2(x_2&-1)^3e^{-x_1-x_2} \\
			\text{s.t.}\quad x_2&\ge\log x_1 \\
			x_1+x_2 &\le 6 \\
			x_1, x_2 &\ge 0
		\end{align*}
		We are interested in finding out where $x_1+x_2=6$ intersects $x_2=\log x_1.$ Note that $x_1$ would solve $x_1=6-\log x_1;$ this is called \textit{fixed point form} since for the function $f(x)=6-\log x$ it would hold that $x_1=f(x_1).$ Do the following in MATLAB: Start with any value $z$ which you guess is close to $x_1,$ then evaluate $f(z), f(f(z)), f(f(f(z))),$ until the sequence seems to converge\ldots and this sequence converges to $x_1$ if all goes well.

	\item Recall the nonlinear optimization problem which we have previously considered: 
		\begin{align*}
			\max \quad 1+x_1^2(x_2&-1)^3e^{-x_1-x_2} \\
			\text{s.t.}\quad x_2&\ge\log x_1 \\
			x_1+x_2 &\le 6 \\
			x_1, x_2 &\ge 0
		\end{align*}
		Suppose that we further restricted the feasible region by additionally requiring that $x_1+x_2=6.$ Use calculus to solve the problem exactly.
		\begin{soln}
			If $x_1+x_2=6,$ let $x_2=6-x_1.$ Then the objective function becomes \[f(x_1)=1+x_1^2[(6-x_1)-1)^3e^{-6}=1+e^{-6}x_1^2(5-x_1)^3\] and its derivative is 
			\begin{align*}
				\frac{d}{dx_1}f(x_1) &= \frac{d}{dx_1}\left[ 1+e^{-6}x_1^2(5-x_1)^3 \right] \\
				&= e^{-6}\left[ 2x_1(5-x_1)^3 + x_1^2(-3)(5-x_1)^2 \right] \\
				&= e^{-6} x_1(5-x_1)^2 \left[ 2(5-x_1)-3x_1 \right] \\
				&= e^{-6}x_1(5-x_1)^2(10-5x_1)
			\end{align*} and the extreme points to consider are where this derivative equals 0 and the boundary points, which are $x_1=0, 2, 5, 6.$ However, $x_1=0$ is not in the feasible region because its logarithm is not defined, and $x_1=5$ and $x_1=6$ are not feasible either because their corresponding $x_2$ values don't satisfy $x_2\ge\log x_1.$ 
			
			Thus, the only value to consider is $x_1=2,$ which gives an objective function value of $f(2)=1.2677.$ If we compute the second derivative at the point $x_1=2,$ we find 
			\begin{align*}
				\frac{d}{dx_1}f'(x_1) &= \frac{d}{dx_1} \left[ 5e^{-6}x_1(5-x_1)^2(2-x_1) \right] \\
				&= 5e^{-5}\left( -4x^3+36x^2-90x+50 \right) \\
				\implies f''(2) &= -0.223 < 0
			\end{align*} thus the point $(2, 4)$ is indeed a maximum for the objective function.
			
		\end{soln}

	\item Lizzie’s Dairy produces cream cheese and cottage cheese. Milk and cream are blended to produce these two products. Both high-fat and low-fat milk can be used to produce cream cheese and cottage cheese. High-fat milk is 60\% fat; low-fat milk is 30\% fat. The milk used to produce cream cheese must average at least 50\% fat, and that for cottage cheese at least 35\% fat. At least 40\% (by weight) of the inputs to cream cheese and at least 20\% (by weight) of the inputs to cottage cheese must be cream. Both cream cheese and cottage cheese are produced by putting milk and cream through the cheese machine. It costs \$0.40 to process 1 lb of inputs into a pound of cream cheese. It costs \$0.40 to produce 1 lb of cottage cheese, but every pound of input for cottage cheese yields 0.9 lb of cottage cheese and 0.1 lb of waste. Cream is produced by evaporating high-fat and low-fat milk. It costs \$0.40 to evaporate 1 lb of high-fat milk, and each pound of high-fat milk that is evaporated yields 0.6 lb of cream. It costs \$0.40 to evaporate 1 lb of low-fat milk, and each pound of low-fat milk that is evaporated yields 0.3 lb of cream. Each day, up to 3000 lb of input may be sent through the cheese machine. Each day, at least 1000 lb of cream cheese and 1000 lb of cottage cheese must be produced. Up to 1500 lb of cream cheese and 2000 lb of cottage cheese can be sold each day. Cream cheese is sold for \$1.50 per lb	and cottage cheese for \$1.20 per lb. High-fat milk is purchased for \$0.80 per lb, and low-fat milk for \$0.40 per lb. The evaporator can process at most 2000 lb of milk daily. Formulate a linear program in canonical form that can be used to maximize Lizzie’s daily profit.
		In working on this problem, provide the matrix $A$ and vectors $b$ and $c$ for the LP. Declare
		\begin{align*}
			x_1 &= \text{lb of high-fat milk that will go to cream cheese in cheese machine} \\
			x_2 &= \text{lb of low-fat milk that will go to cream cheese in cheese machine} \\
			x_3 &= \text{lb of cream that will go to cream cheese in cheese machine} \\
			x_4 &= \text{lb of high-fat milk that will go to cottage cheese in cheese machine} \\
			x_5 &= \text{lb of low-fat milk that will go to cottage cheese in cheese machine} \\
			x_6 &= \text{lb of cream that will go to cottage cheese in cheese machine} \\
			x_7 &= \text{lb of high-fat milk that will go to cream} \\
			x_8 &= \text{lb of low-fat milk that will go to cream.} \\
		\end{align*}
		\begin{soln}
			We have the following constraints:
			\begin{align*}
				\frac{0.6x_1+0.3x_2}{x_1+x_2} \ge 0.5 \implies x_1-2x_2 &\ge 0 \\
				\frac{0.6x_4+0.3x_5}{x_4+x_5} \ge 0.35 \implies 5x_4-x_5 &\ge 0 \\
				\frac{x_3}{x_1+x_2+x_3} \ge 0.4 \implies -5x_1-5x_2+3x_2 &\ge 0 \\
				\frac{x_6}{x_4+x_5+x_6} \ge 0.2 \implies -5x_4-5x_5+4x_6 &\ge 0 \\
				x_1+x_2+x_3+x_4+x_5+x_6 &\le 3000 \\
				x_7+x_8 &\le 2000 \\
				x_1+x_2+x_3 &\ge 1000 \\
				0.9(x_4+x_5+x_6) &\ge 1000
			\end{align*}
		\end{soln}<++>

\end{enumerate}<++>

\end{document}

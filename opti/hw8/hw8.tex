\documentclass{article}
\usepackage[sexy, hdr, fancy, diagrams]{evan}
\usepackage{tikz}
\usetikzlibrary{arrows}
\setlength{\droptitle}{-4em}

\lhead{Homework 8}
\rhead{Introduction to Optimization}
\lfoot{}
\cfoot{\thepage}

\begin{document}
\title{Homework 8}
\maketitle
\thispagestyle{fancy}

\begin{enumerate}
	\item Give an example of a shortest path instance with no negative length cycles such that Dijkstra's Algorithm fails to give the correct shortest paths.
		\begin{soln}
			Consider the graph
			\begin{diagram}
				A & \rTo^2 & B \\
				& \rdTo_1 & \dTo_{-3} \\
				& & C
			\end{diagram}

			If we want to find the shortest $A-C$ path, we execute Dijkstra's Algorithm as follows:
			\[
				\begin{tabular}{c|c}
					nodes & distance \\
					\hline
					*A & 0 \\
					B & $\infty$ \\
					C & $\infty$
				\end{tabular} \implies
				\begin{tabular}{c|c}
					nodes & distance \\
					\hline
					A & 0 \\
					\hline
					B & 2 \\
					*C & 1 \\
				\end{tabular} \implies
				\begin{tabular}{c|c}
					nodes & distance \\
					\hline
					A & 0 \\
					C & 1 \\
					\hline
					*B & 2
				\end{tabular} \implies
				\begin{tabular}{c|c}
					nodes & distance \\
					\hline 
					A & 0 \\
					C & 1 \\
					B & 2 \\
					\hline
				\end{tabular}
			\]
			At the conclusion, the shortest discovered $A-C$ path is $A\to C$ with length 1, but the real shortest length path is $A\to B\to C$ with length -1, so Dijkstra's Algorithm fails.			
		\end{soln}

	\item Solve the Max Flow problem from HW1 using the Ford-Fulkerson Algorithm. Be sure to provide the flow and residual network at each algorithm iteration. Then, at the conclusion, provide the Max Flow and the Min Cut.
		\begin{soln}
			The network and its capacities are given as
			\begin{center}
				\begin{tikzpicture}[->, >=stealth', shorten >=1pt, auto, node distance=3cm, thick, main node/.style={circle, draw, font=\sffamily\Large\bfseries}]

					\node[main node] (1) {A};
					\node[main node] (2) [right of=1] {B};
					\node[main node] (3) [right of=2] {D};
					\node[main node] (4) [right of=3] {F};
					\node[main node] (5) [below of=2] {C};
					\node[main node] (6) [right of=5] {E};

					\path[every node/.style={font=\sffamily\small}]
					(1) edge node [above] {5.1} (2)
					edge node [below left] {7.2} (5)
					(2) edge node [above] {5.9} (3)
					(3) edge node [above] {4.0} (4)
					edge node [right] {2.9} (6)
					(5) edge node[right] {2.1} (2)
					edge node[below] {3.1} (6)
					(6) edge node[below right] {10.5} (4);
				\end{tikzpicture}
			\end{center}
			\newpage
			We start with the feasible flow $x$ given by
			\begin{center}
				\begin{tikzpicture}[->, >=stealth', shorten >=1pt, auto, node distance=3cm, thick, main node/.style={circle, draw, font=\sffamily\Large\bfseries}]

					\node[main node] (1) {A};
					\node[main node] (2) [right of=1] {B};
					\node[main node] (3) [right of=2] {D};
					\node[main node] (4) [right of=3] {F};
					\node[main node] (5) [below of=2] {C};
					\node[main node] (6) [right of=5] {E};

					\path[every node/.style={font=\sffamily\small}]
					(1) edge node [above] {0/5.1} (2)
					edge node [below left] {0/7.2} (5)
					(2) edge node [above] {0/5.9} (3)
					(3) edge node [above] {0/4.0} (4)
					edge node [right] {0/2.9} (6)
					(5) edge node[right] {0/2.1} (2)
					edge node[below] {0/3.1} (6)
					(6) edge node[below right] {0/10.5} (4);
				\end{tikzpicture}
			\end{center}
			and the associated residual network $G^x$ is just the original network. Use the path $P=A\to B\to D\to F,$ where $\mu^x(P)=4.0,$ and augment along $P$ by 4.0 to get the new flow $x$
			\begin{center}
				\begin{tikzpicture}[->, >=stealth', shorten >=1pt, auto, node distance=3cm, thick, main node/.style={circle, draw, font=\sffamily\Large\bfseries}]

					\node[main node] (1) {A};
					\node[main node] (2) [right of=1] {B};
					\node[main node] (3) [right of=2] {D};
					\node[main node] (4) [right of=3] {F};
					\node[main node] (5) [below of=2] {C};
					\node[main node] (6) [right of=5] {E};

					\path[every node/.style={font=\sffamily\small}]
					(1) edge node [above] {4.0/5.1} (2)
					edge node [below left] {0/7.2} (5)
					(2) edge node [above] {4.0/5.9} (3)
					(3) edge node [above] {4.0/4.0} (4)
					edge node [right] {0/2.9} (6)
					(5) edge node[right] {0/2.1} (2)
					edge node[below] {0/3.1} (6)
					(6) edge node[below right] {0/10.5} (4);
				\end{tikzpicture}
			\end{center}
			and the residual network 
			\begin{center}
				\begin{tikzpicture}[->, >=stealth', shorten >=1pt, auto, node distance=3cm, thick, main node/.style={circle, draw, font=\sffamily\Large\bfseries}]

					\node[main node] (1) {A};
					\node[main node] (2) [right of=1] {B};
					\node[main node] (3) [right of=2] {D};
					\node[main node] (4) [right of=3] {F};
					\node[main node] (5) [below of=2] {C};
					\node[main node] (6) [right of=5] {E};

					\path[every node/.style={font=\sffamily\small}]
					(1) edge node [above] {1.1} (2)
					edge node [below left] {7.2} (5)
					(2) edge node [above] {1.9} (3)
					edge [bend right] node[above] {4.0} (1)
					(3) edge node [right] {2.9} (6)
					edge [bend right] node[above] {4.0} (2)
					(4) edge [bend right] node[above] {4.0} (3)
					(5) edge node[right] {2.1} (2)
					edge node[below] {3.1} (6)
					(6) edge node[below right] {10.5} (4);
				\end{tikzpicture}
			\end{center}
			Next, choose the path $A\to C\to C\to F,$ where $\mu^x(P)=3.1,$ so augmenting by 3.1 gives the new flow $x$
			\begin{center}
				\begin{tikzpicture}[->, >=stealth', shorten >=1pt, auto, node distance=3cm, thick, main node/.style={circle, draw, font=\sffamily\Large\bfseries}]

					\node[main node] (1) {A};
					\node[main node] (2) [right of=1] {B};
					\node[main node] (3) [right of=2] {D};
					\node[main node] (4) [right of=3] {F};
					\node[main node] (5) [below of=2] {C};
					\node[main node] (6) [right of=5] {E};

					\path[every node/.style={font=\sffamily\small}]
					(1) edge node [above] {4.0/5.1} (2)
					edge node [below left] {3.1/7.2} (5)
					(2) edge node [above] {4.0/5.9} (3)
					(3) edge node [above] {4.0/4.0} (4)
					edge node [right] {0/2.9} (6)
					(5) edge node[right] {0/2.1} (2)
					edge node[below] {3.1/3.1} (6)
					(6) edge node[below right] {3.1/10.5} (4);
				\end{tikzpicture}
			\end{center}
			and the residual network 
			\begin{center}
				\begin{tikzpicture}[->, >=stealth', shorten >=1pt, auto, node distance=3cm, thick, main node/.style={circle, draw, font=\sffamily\Large\bfseries}]

					\node[main node] (1) {A};
					\node[main node] (2) [right of=1] {B};
					\node[main node] (3) [right of=2] {D};
					\node[main node] (4) [right of=3] {F};
					\node[main node] (5) [below of=2] {C};
					\node[main node] (6) [right of=5] {E};

					\path[every node/.style={font=\sffamily\small}]
					(1) edge node [above] {1.1} (2)
					edge node [below left] {4.1} (5)
					(2) edge node [above] {1.9} (3)
					edge [bend right] node[above] {4.0} (1)
					(3) edge node [right] {2.9} (6)
					edge [bend right] node[above] {4.0} (2)
					(4) edge [bend right] node[above] {4.0} (3)
					edge [bend left] node[below right] {3.1} (6)
					(5) edge node[right] {2.1} (2)
					edge [bend left] node[below left] {3.1} (1)
					(6) edge node[below right] {7.4} (4)
					edge [bend left] node[below] {3.1} (5);
				\end{tikzpicture}
			\end{center}		
			Next, choose the path $A\to B\to D\to E\to F$ where $\mu^x(P)=1.1,$ so augmenting by 1.1 gives the new flow $x$
			\begin{center}
				\begin{tikzpicture}[->, >=stealth', shorten >=1pt, auto, node distance=3cm, thick, main node/.style={circle, draw, font=\sffamily\Large\bfseries}]

					\node[main node] (1) {A};
					\node[main node] (2) [right of=1] {B};
					\node[main node] (3) [right of=2] {D};
					\node[main node] (4) [right of=3] {F};
					\node[main node] (5) [below of=2] {C};
					\node[main node] (6) [right of=5] {E};

					\path[every node/.style={font=\sffamily\small}]
					(1) edge node [above] {5.1/5.1} (2)
					edge node [below left] {3.1/7.2} (5)
					(2) edge node [above] {5.1/5.9} (3)
					(3) edge node [above] {4.0/4.0} (4)
					edge node [right] {1.1/2.9} (6)
					(5) edge node[right] {0/2.1} (2)
					edge node[below] {3.1/3.1} (6)
					(6) edge node[below right] {4.2/10.5} (4);
				\end{tikzpicture}
			\end{center}
			and the residual network
			\begin{center}
				\begin{tikzpicture}[->, >=stealth', shorten >=1pt, auto, node distance=3cm, thick, main node/.style={circle, draw, font=\sffamily\Large\bfseries}]

					\node[main node] (1) {A};
					\node[main node] (2) [right of=1] {B};
					\node[main node] (3) [right of=2] {D};
					\node[main node] (4) [right of=3] {F};
					\node[main node] (5) [below of=2] {C};
					\node[main node] (6) [right of=5] {E};

					\path[every node/.style={font=\sffamily\small}]
					(1) edge node [below left] {4.1} (5)
					(2) edge node [above] {0.8} (3)
					edge [bend right] node[above] {5.1} (1)
					(3) edge node [right] {1.8} (6)
					edge [bend right] node[above] {5.1} (2)
					(4) edge [bend right] node[above] {4.0} (3)
					edge [bend left] node[below right] {4.2} (6)
					(5) edge node[right] {2.1} (2)
					edge [bend left] node[below left] {3.1} (1)
					(6) edge node[below right] {6.3} (4)
					edge [bend left] node[left] {1.1} (3)
					edge [bend left] node[below] {3.1} (5);
				\end{tikzpicture}
			\end{center}		
			Next, choose the path $A\to C\to B\to D\to E\to F,$ with $\mu^x(P)=0.8,$ so augmenting by 0.8 gives the new flow $x$
			\begin{center}
				\begin{tikzpicture}[->, >=stealth', shorten >=1pt, auto, node distance=3cm, thick, main node/.style={circle, draw, font=\sffamily\Large\bfseries}]

					\node[main node] (1) {A};
					\node[main node] (2) [right of=1] {B};
					\node[main node] (3) [right of=2] {D};
					\node[main node] (4) [right of=3] {F};
					\node[main node] (5) [below of=2] {C};
					\node[main node] (6) [right of=5] {E};

					\path[every node/.style={font=\sffamily\small}]
					(1) edge node [above] {5.1/5.1} (2)
					edge node [below left] {3.9/7.2} (5)
					(2) edge node [above] {5.9/5.9} (3)
					(3) edge node [above] {4.0/4.0} (4)
					edge node [right] {1.9/2.9} (6)
					(5) edge node[right] {0.8/2.1} (2)
					edge node[below] {3.1/3.1} (6)
					(6) edge node[below right] {5.0/10.5} (4);
				\end{tikzpicture}
			\end{center}
			with residual network
			\begin{center}
				\begin{tikzpicture}[->, >=stealth', shorten >=1pt, auto, node distance=3cm, thick, main node/.style={circle, draw, font=\sffamily\Large\bfseries}]

					\node[main node] (1) {A};
					\node[main node] (2) [right of=1] {B};
					\node[main node] (3) [right of=2] {D};
					\node[main node] (4) [right of=3] {F};
					\node[main node] (5) [below of=2] {C};
					\node[main node] (6) [right of=5] {E};

					\path[every node/.style={font=\sffamily\small}]
					(1) edge node [below left] {3.3} (5)
					(2) edge [bend right] node[above] {5.1} (1)
					edge [bend right] node[left] {0.8} (5)
					(3) edge node [right] {1.8} (6)
					edge [bend right] node[above] {5.9} (2)
					(4) edge [bend right] node[above] {4.0} (3)
					edge [bend left] node[below right] {5.0} (6)
					(5) edge node[right] {1.3} (2)
					edge [bend left] node[below left] {3.9} (1)
					(6) edge node[below right] {5.5} (4)
					edge [bend left] node[left] {1.1} (3)
					edge [bend left] node[below] {3.1} (5);
				\end{tikzpicture}
			\end{center}		

			At this point, there no longer exists an $A-F$ path, so the algorithm concludes. The effective flow is 9.0. The Min Cut is also 9.0, where we cut the arcs from $B\to D$ and $C\to E.$ 
		\end{soln}

	\item Suppose families $F_1, F_2, \cdots, F_n$ each respectively consist of $f_1, f_2, \cdots, f_n$ people, and cars $C_1, C_2, \cdots, C_m$ can respectively seat $c_1, c_2, \cdots, c_m$ people. Everyone needs to travel from Baltimore to Pittsburgh in these cars to see the Ravens-Steelers game, but no two members of any family can ride in the same car, as they will fight the whole time. Formulate a max flow instance such that the max flow will provide everyone with seating assignments for this very important trip.
		\begin{soln}
			I will attempt to describe this because drawing seems disgusting.

			There is a starting node from Baltimore and the ending node at Pittsburgh. There is a node for each family $F_1, \cdots, F_n,$ and there exists an arc from Baltimore to each $F_i$ with capacity $f_i.$ Then there is a node for each car $C_1, \cdots, C_m,$ and for every pair $F_i, C_j,$ there exists an arc from $F_i\to C_j$ with capacity 1. Then from each node $C_i$ to Pittsburgh, there exists an arc with capacity $c_i.$ This max flow problem will have a solution that transports everyone.
		\end{soln}

		\newpage
	\item Suppose that $G=(V, E), s, t\in V,$ and $\ell:E\to \RR$ is a shortest path problem instance such that there is a negative cycle. Prove that there do not exist distance labels $d:V\to\RR$ such that $d$ satisfies the triangle inequality.
		\begin{proof}
			Suppose a distance label $d$ exists that satisfies the triangle inequality. Suppose a cycle $C=(a_1, a_2, \cdots, a_n, a_1)$ has a negative length. If $d$ satisfies the triangle inequality, then we must have
			\begin{align*}
				d(a_2)-d(a_1)&\le \ell(a_1, a_2) \\
				d(a_3)-d(a_2) &\le \ell(a_2, a_3) \\
				&\vdots \\
				d(a_n)-d(a_{n-1})&\le \ell(a_{n-1}, a_n) \\
				d(a_1)-d(a_n) &\le \ell(a_n, a_1)
			\end{align*}
			If we sum all of these inequalities, the LHS sums to 0 because everything cancels, whereas the RHS becomes $\ell(C),$ which we know is negative. This is a contradiction: 0 can't be less than or equal to a negative number. Thus, no such $d$ exists that satisfies the triangle inequality,a
		\end{proof}

\end{enumerate}

\end{document}

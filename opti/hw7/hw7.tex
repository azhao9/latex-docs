\documentclass{article}
\usepackage[sexy, hdr, fancy]{evan}
\setlength{\droptitle}{-4em}

\lhead{Homework 7}
\rhead{Introduction to Optimization}
\lfoot{}
\cfoot{\thepage}

\begin{document}
\title{Homework 7}
\maketitle
\thispagestyle{fancy}

\begin{enumerate}
	\item For a symmetric (LP) min $c^T x$ such that $Ax\ge b, x\ge 0,$ recall that the dual problem is (DP) max $b^T y$ such that $A^T y\le c, y\ge 0.$ Suppose that $x$ is feasible in (LP) and $y$ is feasible in (DP). Prove that $x$ is optimal in (LP) and $y$ is optimal in (DP) if an only if $x$ is complementary to the dual slackness $c-A^t y,$ and $y$ is complementary to the primal slackness $Ax-b.$

	\item Consider the canonical form linear program min $c^T x$ such that $Ax\ge b$ and $x\ge 0$ where \[A=\begin{bmatrix}
				6 & 2 & -5 & 7 & 9 & -3 & 1 & 9 \\
				5 & -6 & 5 & 4 & 1 & 9 & 8 & -1 \\
				1 & 7 & 3 & -1 & -1 & 4 & 0 & 8
			\end{bmatrix} \quad b=\begin{bmatrix}
				4 \\ 2 \\ 7
			\end{bmatrix}\quad c^T = \begin{bmatrix}
				5 & 1 & 7 & 3 & 0 & 1 & 2 & 3
		\end{bmatrix}\] Run the dual simplex method starting from the basis introduced by the slack variables. Show the successive tableaux, give the optimal solution to the primal linear program and to the dual of this linear program. 

	\item Consider the canonical form linear program min $c^T x$ such that $Ax\ge b$ and $x\ge 0$ where \[A=\begin{bmatrix}
				-1 & 3 & 4 & 6 & -2 & 4 & 1 \\
				8 & -2 & 5 & -3 & 0 & 1 & 5
			\end{bmatrix} \quad b=\begin{bmatrix}
				6 \\ 3
			\end{bmatrix} \quad c^T = \begin{bmatrix}
				5 & 2 & 1 & 0 & 1 & 3 & 2
		\end{bmatrix}\] Run the dual simplex method starting from the basis introduced by the slack variables. Show the successive tableaux, report and note any degeneracy encountered, and give the optimal solution to the primal linear program and to the dual of this linear program.

	\item Consider the canonical form linear program min $c^T x$ such that $Ax\ge b$ and $x\ge 0$ where \[A=\begin{bmatrix}
				1 & -2 & -2 & -2 \\
				-2 & 1 & -2 & -2
			\end{bmatrix}\quad b=\begin{bmatrix}
				3 \\ 4
			\end{bmatrix}\quad c^T=\begin{bmatrix}
				1 & 2 & 3 & 4
		\end{bmatrix}\] Run the dual simplex method starting from the basis introduced by the slack variables. Show the successive tableaux, and report the conclusion given by the algorithm.
		
\end{enumerate}

\end{document}

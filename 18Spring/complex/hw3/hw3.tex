\documentclass{article}
\usepackage[sexy, hdr, fancy]{evan}
\usepackage{graphicx}
\graphicspath{.}
\setlength{\droptitle}{-4em}

\DeclareMathOperator{\re}{Re}
\DeclareMathOperator{\im}{Im}

\lhead{Homework 3}
\rhead{Complex Analysis}
\lfoot{}
\cfoot{\thepage}

\begin{document}
\title{Homework 3}
\maketitle
\thispagestyle{fancy}

\section*{Section 2.3}

\begin{itemize}
	\item[4.] Using Definition 4, show that each of the following functions is nowhere differentiable.
		\begin{itemize}
			\item[(a)] $\re z$
				\begin{proof}
					Suppose $\re z$ was differentiable at $z_0=a_0+b_0i.$ Then
					\begin{align*}
						\frac{d(\re z)}{dz} (z_0) &= \lim_{h\to 0}\frac{\re(z_0+h)-\re z}{h} = \lim_{a+bi\to 0}\frac{\re\left[ (a_0+b_0i)+(a+bi) \right] - \re(a_0+b_0i)}{a+bi} \\
						&= \lim_{a+bi\to 0}\frac{(a_0+a)-a_0}{a+bi} = \lim_{a+bi\to 0} \frac{a}{a+bi}
					\end{align*}
					This limit does not exist because if we go along the real axis, the limit is 1, but if we go along the imaginary axis, the limit is 0. Thus, $\re z$ is not differentiable at any point.
				\end{proof}

			\item[(c)] $\abs{z}$
				\begin{proof}
					Suppose $\abs{z}$ was differentiable at $z_0=a_0+b_0i.$ Then
					\begin{align*}
						\frac{d(\abs z)}{dz} (z_0) &= \lim_{h\to 0}\frac{\abs{z_0+h}-\abs{z_0}}{h} = \lim_{a+bi\to 0}\frac{\abs{(a_0+b_0i)+(a+bi)} - \abs{a_0+b_0i}}{a+bi} \\
						&= \lim_{a+bi\to 0} \frac{\sqrt{(a_0+a)^2+(b_0+b)^2}-\sqrt{a_0^2+b_0^2}}{a+bi}
					\end{align*}
					If we approach along the real axis, $b=0,$ so the limit is
					\begin{align*}
						\lim_{a\to 0} \frac{\sqrt{(a_0+a)^2+b_0^2}-\sqrt{a_0^2}}{a} \to\infty
					\end{align*}
					so the limit does not exist.
				\end{proof}

		\end{itemize}

	\item[8.] Suppose that $f$ is analytic at $z_0$ and $f'(z_0)\neq 0.$ Show that
		\begin{align*}
			\lim_{z\to z_0} \frac{\abs{f(z)-f(z_0)}}{\abs{z-z_0}} = \abs{f'(z_0)}
		\end{align*}
		and
		\begin{align*}
			\lim_{z\to z_0} \left\{ \arg\left[ f(z)-f(z_0) \right]-\arg(z-z_0) \right\} = \arg f'(z_0)
		\end{align*}
		\begin{proof}
			If $f$ is analytic at $z_0,$ then using the substitution $z=z_0+h\implies h=z-z_0$ we have
			\begin{align*}
				\abs{f'(z_0)} &= \abs{\lim_{h\to 0}\frac{f(z_0+h)-f(z_0)}{h}} = \lim_{z\to z_0} \frac{\abs{f(z)-f(z_0)}}{\abs{z-z_0}}
			\end{align*}
			as desired. Since $\arg z_1-\arg z_2 = \arg \frac{z_1}{z_2},$ we have
			\begin{align*}
				\lim_{z\to z_0} \left\{ \arg\left[ f(z)-f(z_0) \right] - \arg(z-z_0) \right\} &= \lim_{z\to z_0} \arg\left( \frac{f(z)-f(z_0)}{z-z_0} \right) \\
				&= \arg \left( \lim_{z\to z_0} \frac{f(z)-f(z_0)}{z-z_0} \right) = \arg f'(z_0)
			\end{align*}
			by the same substitution.
		\end{proof}

	\item[11.] Discuss the analyticity of each of the following functions.
		\begin{itemize}
			\item[(b)] $\frac{z}{\overline z+2}$
				\begin{soln}
					We have the derivative at point $z$ given by
					\begin{align*}
						\lim_{h\to 0}\frac{f(z+h)-f(z)}{h} &= \lim_{h\to 0} \frac{\frac{z+h}{\overline{z+h}+2} - \frac{z}{\overline z + 2}}{h} = \lim_{h\to 0} \frac{(z+h)(\overline z + 2) - z(\overline z+\overline h + 2)}{h(\overline z+\overline h + 2)(\overline z + 2)} \\
						&= \lim_{h\to 0} \frac{z\overline z + 2z +h\overline z + 2h - z\overline z - z\overline h - 2z}{h(\overline z + \overline h + 2)(\overline z + 2)} = \lim_{h\to 0} \frac{h\overline z + 2h - z\overline h}{h(\overline z + \overline h + 2)(\overline z + 2)}
					\end{align*}
					At $z=0,$ the limit is
					\begin{align*}
						\lim_{h\to 0} \frac{2h}{h(\overline h+2)(2)} = \lim_{h\to 0} \frac{1}{\overline h + 2} = \frac{1}{2}
					\end{align*}
					Otherwise if $z\neq 0,$ then if we approach along the real axis, $\overline h=h,$ so this limit is
					\begin{align*}
						\lim_{h\to 0} \frac{h(\overline z + 2 - z)}{h(\overline z + \overline h + 2)(\overline z + 2)} = \lim_{h\to 0} \frac{\overline z + 2 - z}{(\overline z + h + 2)(\overline z + 2)} = \frac{\overline z + 2 - z}{(\overline z+2)^2}
					\end{align*}
					but if we approach along the imaginary axis, $\overline h = -h,$ so this limit is
					\begin{align*}
						\lim_{h\to 0}\frac{h(\overline z + 2+ z)}{h(\overline z - h + 2)(\overline z + 2)} = \frac{\overline z + 2 + z}{(\overline z+2)^2}
					\end{align*}
					These two limits are not equal as long as $z\neq 0,$ so this function is not differentiable except at 0. Since 0 is not an open set in $\CC,$ this function is nowhere analytic.
				\end{soln}

			\item[(f)] $\left( x+\frac{x}{x^2+y^2} \right)+i\left( y-\frac{y}{x^2+y^2} \right)$
				\begin{soln}
					If $z=x+yi,$ then $ z+\frac{1}{z} = (x+yi) + \frac{x-yi}{x^2+y^2} = \left( x+\frac{x}{x^2+y^2} \right) + i\left( y-\frac{y}{x^2+y^2} \right).$ Since $z$ is analytic everywhere and $1/z$ is analytic everywhere except 0, this is analytic everywhere but 0.
				\end{soln}

			\item[(g)] $\abs{z}^2+2z$
				\begin{soln}
					Since $\abs{z}$ is nowhere analytic, this is also nowhere analytic.
				\end{soln}

		\end{itemize}

\end{itemize}

\newpage
\section*{Section 2.4}

\begin{itemize}
	\item[3.] Use Theorem 5 to show that $g(z)=3x^2+2x-3y^2-1+i(6xy+2y)$ is entire. Write this function in terms of $z.$
		\begin{proof}
			Here, $u=3x^2+2x-3y^2-1$ and $v=6xy+2y.$ We have
			\begin{align*}
				\frac{\partial u}{\partial x} &= 6x + 2 \\
				\frac{\partial v}{\partial y} &= 6x + 2 \\
				\frac{\partial u}{\partial y} &= -6y \\
				\frac{\partial v}{\partial x} &= 6y
			\end{align*}
			so the Cauchy-Riemann equations are satisfied, and they are satisfied at all points in $\CC.$ The first partials are also all continuous, so $g$ is entire.

			Using the identities $x=\frac{z+\bar z}{2}$ and $y=\frac{z-\bar z}{2i},$ we have
			\begin{align*}
				g(z) &= \left[ 3\left( \frac{z+\bar z}{2} \right)^2 + 2\left( \frac{z+\bar z}{2} \right) - 3\left( \frac{z-\bar z}{2i} \right)^2 - 1 \right] + i\left[ 6\left( \frac{z+\bar z}{2} \right)\left( \frac{z-\bar z}{2i} \right) + 2\left( \frac{z-\bar z}{2i} \right) \right] \\
				&= \left[ 3\left( \frac{z^2+2z\bar z + \bar z^2}{4} \right) + (z+\bar z) - 3\left( \frac{z^2-2z\bar z+\bar z^2}{-4} \right) - 1 \right] + \frac{3}{2}(z^2-\bar z^2) + (z-\bar z) \\
				&= \frac{3}{2}z^2 + \frac{3}{2}\bar z^2 + z + \bar z - 1 + \frac{3}{2}z^2 - \frac{3}{2}\bar z^2 + z - \bar z \\
				&= 3z^2+2z-1
			\end{align*}
		\end{proof}

	\item[5.] Show that the function $f(z)=e^{x^2-y^2}\left[ \cos(2xy)+i\sin(2xy) \right]$ is entire, and find its derivative.
		\begin{proof}
			Here, $u=e^{x^2-y^2}\cos(2xy)$ and $v=e^{x^2-y^2}\sin(2xy).$ We have
			\begin{align*}
				\frac{\partial u}{\partial x} &= 2xe^{x^2-y^2}\cos(2xy) -2ye^{x^2-y^2}\sin(2xy) \\
				\frac{\partial v}{\partial y} &= -2ye^{x^2-y^2}\sin(2xy) + 2xe^{x^2-y^2}\cos(2xy) \\
				\frac{\partial u}{\partial y} &= -2ye^{x^2-y^2}\cos(2xy) - 2xe^{x^2-y^2}\sin(2xy) \\
				\frac{\partial v}{\partial x} &= 2xe^{x^2-y^2}\sin(2xy) + 2ye^{x^2-y^2}\cos(2xy) 
			\end{align*}
			so the Cauchy-Riemann equations are satisfied, the first partials are continuous, and thus $f$ is analytic at every point in $\CC,$ so $f$ is also entire. By De Moivre's theorem, the derivative is
			\begin{align*}
				f(z) &= e^{x^2-y^2}\left[ \cos(2xy)+i\sin(2xy) \right] = e^{x^2-y^2}e^{2xyi} \\
				&= e^{x^2+2xyi - y^2} = e^{(x+yi)^2} = e^{z^2} \\
				\implies f'(z) &= 2ze^{z^2}
			\end{align*}
		\end{proof}

	\item[8.] Show that if $f$ is analytic in a domain $D$ and either $\re f(x)$ or $\im f(x)$ is constant in $D,$ then $f(z)$ must be constant in $D.$
		\begin{proof}
			Here, $u=\re f(x)$ and $v=\im f(x).$ If $f$ is analytic, and $u\equiv c_1$ is constant, then $u$ and $v$ must satisfy the Cauchy-Riemann equations:
			\begin{align*}
				\frac{\partial u}{\partial x} = 0 = \frac{\partial v}{\partial y} \implies v \equiv c_2
			\end{align*}
			Similarly, if $v\equiv c_2,$ we get $u\equiv c_1$ in order to satisfy the Cauchy-Riemann equations, and thus $f(x)=c_1+c_2i,$ which is a constant function.
		\end{proof}

	\item[15.] The Jacobian of a mapping 
		\begin{align*}
			u=u(x, y), \quad v=v(x, y)
		\end{align*}
		from the $xy$-plane to the $uv$-plane is defined to be the determinant
		\begin{align*}
			J(x_0, y_0) := \det{\begin{bmatrix}
				\frac{\partial u}{\partial x} & \frac{\partial u}{\partial y} \\
				\frac{\partial v}{\partial x} & \frac{\partial v}{\partial y}
			\end{bmatrix}}
		\end{align*}
		where the partial derivatives are all evaluated at $(x_0, y_0).$ Show that if $f=u+iv$ is analytic at $z_0=x_0+iy_0,$ then $J(x_0, y_0)=\abs{f'(z_0)}^2.$
		\begin{proof}
			If $f=u +iv$ is analytic at $(x_0, y_0)$ then by the Cauchy-Riemann equations, we must have $\frac{\partial u}{\partial x} = \frac{\partial v}{\partial y}$ and $\frac{\partial u}{\partial y} = -\frac{\partial v}{\partial x}$ where the partials are evaluated at $(x_0, y_0).$ The Jacobian is
			\begin{align*}
				\det \begin{bmatrix}
					\frac{\partial u}{\partial x} & \frac{\partial u}{\partial y} \\
					\frac{\partial v}{\partial x} & \frac{\partial v}{\partial y}
				\end{bmatrix} &= \frac{\partial u}{\partial x} \cdot \frac{\partial v}{\partial y} - \frac{\partial u}{\partial y}\cdot \frac{\partial v}{\partial x} = \left( \frac{\partial u}{\partial x} \right)^2 + \left( \frac{\partial v}{\partial x} \right)^2 \\ 
				&= \left( \sqrt{\left( \frac{\partial u}{\partial x} \right)^2 + \left( \frac{\partial v}{\partial x} \right)^2} \right)^2 = \abs{\frac{\partial u}{\partial x} + i\cdot\frac{\partial v}{\partial x}}^2 \\
				&= \abs{f'(z_0)}^2
			\end{align*}
			as desired.
		\end{proof}

\end{itemize}

\section*{Section 2.5}

\begin{itemize}
	\item[8.] Suppose that the functions $u$ and $v$ are harmonic in a domain $D.$
		\begin{enumerate}[(a)]
			\item Is the sum $u+v$ necessarily harmonic in $D?$
				\begin{soln}
					We have
					\begin{align*}
						\frac{\partial^2}{\partial x^2}(u+v) + \frac{\partial^2}{\partial y^2}(u+v) &= \frac{\partial^2u}{\partial x^2} + \frac{\partial^2v}{\partial x^2} + \frac{\partial^2u}{\partial y^2} + \frac{\partial^2 v}{\partial y^2} \\
						&= \left( \frac{\partial^2u}{\partial x^2} + \frac{\partial^2u}{\partial y^2} \right) + \left( \frac{\partial^2 v}{\partial x^2} + \frac{\partial^2v}{\partial y^2} \right) = 0
					\end{align*}
					by the linearity of the differential operator, so $u+v$ is also harmonic in $D.$
				\end{soln}

			\item Is the product $uv$ necessarily harmonic in $D?$
				\begin{soln}
					This is not necessarily true. Take $u=y=xy.$ Then $\frac{\partial^2u}{\partial x^2}+\frac{\partial^2u}{\partial y^2} = 0$ so $u$ and $v$ are both harmonic, but
					\begin{align*}
						\frac{\partial^2(uv)}{\partial x^2} + \frac{\partial^2 (uv)}{\partial y^2} = \frac{\partial^2 (x^2y^2)}{\partial x^2} + \frac{\partial^2 (x^2y^2)}{\partial y^2} = 2y^2 + 2x^2\neq 0
					\end{align*}
				\end{soln}

			\item Is $\partial u/\partial x$ harmonic in $D?$
				\begin{soln}
					We have
					\begin{align*}
						\frac{\partial^2}{\partial x^2} \left( \frac{\partial u}{\partial x} \right) + \frac{\partial^2}{\partial y^2} \left( \frac{\partial u}{\partial x} \right) &= \frac{\partial}{\partial x} \left( \frac{\partial^2u}{\partial x^2} \right) + \frac{\partial}{\partial x} \left( \frac{\partial^2u}{\partial y^2} \right) = \frac{\partial}{\partial x} \left( \frac{\partial^2u}{\partial x^2} + \frac{\partial^2 u}{\partial y^2} \right) = \frac{\partial}{\partial x} [0] = 0
					\end{align*}
					because we can take partial derivatives in any order, so $\partial u/\partial x$ is also harmonic in $D.$
				\end{soln}

		\end{enumerate}

	\item[12.] Prove that if $r$ and $\theta$ are polar coordinates, then the functions $r^n \cos n\theta$ and $r^n\sin n\theta,$ where $n$ is an integer, are harmonic as functions of $x$ and $y.$
		\begin{proof}
			If $x=r\cos \theta$ and $y=r\sin \theta,$ then we have 
			\begin{align*}
				r^n\cos n\theta + i \cdot r^n\sin n\theta = \left( re^{i\theta} \right)^n = z^n
			\end{align*}
			Thus, since $f(z)=z^n$ is analytic, it follows that $\re f(z)=r^n\cos n\theta$ and $\im f(z) = r^n\sin n\theta$ are both harmonic.
		\end{proof}

	\item[13.] Find a function harmonic inside the wedge bounded by the non-negative $x$-axis and the half-line $y=x\quad(x\ge 0)$ that goes to 0 on these sides but is not identically zero. 
		\begin{soln}
			If $z=re^{i\theta},$ then $f(z)=\im z^4=\im (r^4e^{4\theta i})= r^4\sin 4\theta=0$ exactly on the half lines $\theta=0$ and $\theta=\pi/4,$ but it is not identically zero. 
		\end{soln}

\end{itemize}

\end{document}

\documentclass{article}
\usepackage[sexy, hdr, fancy]{evan}
\setlength{\droptitle}{-4em}

\DeclareMathOperator{\re}{Re}
\DeclareMathOperator{\im}{Im}

\lhead{Homework 1}
\rhead{Complex Analysis}
\lfoot{}
\cfoot{\thepage}

\begin{document}
\title{Homework 1}
\maketitle
\thispagestyle{fancy}

\section*{Section 1.1}

\begin{itemize}
	\item[8.] Write the number in the form $a+bi.$
		\begin{align*}
			\frac{(8+2i)-(1-i)}{(2+i)^2}
		\end{align*}

	\item[10.] Write the number in the form $a+bi.$
		\begin{align*}
			\left[ \frac{2+i}{6i-(1-2i)} \right]^2
		\end{align*}

\end{itemize}

\section*{Section 1.2}

\begin{itemize}
	\item[7.] (e) Describe the set of points $z$ in the complex plane that satisfies $z=\re z+2.$

	\item[16.] Prove that if $\abs{z}=1$ ($z\neq 1$), then $\re[1/(1-z)]=\frac{1}{2}.$
		
\end{itemize}

\section*{Section 1.3}

\begin{itemize}
	\item[5.] (d) Find the value of
		\begin{align*}
			\abs{\frac{(\pi+i)^{100}}{(\pi-i)^{100}}}
		\end{align*}

	\item[7.] (h) Find the argument of this complex number and write it in polar form. 
		\begin{align*}
			\frac{-\sqrt{7}(1+i)}{\sqrt{3}+i}
		\end{align*}

	\item[28.] Let the crankshaft pivot $O$ lie at the right of the origin of the coordinate system, and let $z$ be the complex number giving the location of the base of the piston rod, as depicted in Fig 1.14,
		\begin{align*}
			z=\ell+id
		\end{align*}
		where $\ell$ gives the piston's linear excursion and $d$ is a fixed offset. The crank arm is described by $A=a(\cos \theta_1+i\sin\theta_1)$ the connecting arm by $B=b(\cos\theta_2+i\sin\theta_2)$ ($\theta_2$ is negative in Fig 1.14). Exploit the obvious identity $A+B=z=\ell+id$ to derive the expression relating the piston position to the crankshaft angle:
		\begin{align*}
			\ell=\cos\theta_1+b\cos\left[ \sin\inv\left( \frac{d-a\sin\theta_1}{b} \right) \right]
		\end{align*}
		\begin{soln}
			Because of the identity
			\begin{align*}
				A+B &= a(\cos\theta_1+i\sin\theta_1)+b(\cos\theta_2+i\sin\theta_2) \\
				&= (a\cos \theta_1+b\cos\theta_2) + i(a\sin\theta_1+b\sin\theta_2) \\
				&= \ell+id
			\end{align*}
			we must have 
			\begin{align*}
				a\cos\theta_1+b\cos\theta_2 &= \ell \\
				a\sin\theta_1+b\sin\theta_2 &= d \implies \theta_2 = \sin\inv\left( \frac{d-a\sin\theta_1}{b} \right) \\
				\implies \ell &= a\cos \theta_1 + b\cos\left[ \sin\inv\left( \frac{d-a\sin\theta_1}{b} \right) \right]
			\end{align*}
			as desired.
		\end{soln}
		
\end{itemize}

\section*{Section 1.4}

\begin{itemize}
	\item[11.] Determine which of the following properties of the real exponential function remain true for the complex exponential function
		\begin{enumerate}[(a)]
			\item $e^x$ is never zero.

			\item $e^x$ is a one-to-one function.

			\item $e^x$ is defined for all $x.$

			\item $e^{-x}=1/e^x.$
				
		\end{enumerate}

	\item[18.] Sketch the curves that are given for $0\le t\le 2\pi$ by
		\begin{enumerate}[(a)]
			\item $z(t)=e^{(1+i)t}$ 
				
			\item $z(t)=e^{(1-i)t}$ 

			\item $z(t)=e^{(-1+i)t}$

			\item $z(t)=e^{-1-i)t}$
				
		\end{enumerate}

	\item[22.] Show that if $n$ is an integer then
		\begin{align*}
			\int_0^{2\pi} e^{in\theta}\, d\theta = \int_0^{2\pi}\cos(n\theta)\, d\theta + i\int_0^{2\pi}\sin(n\theta)\, d\theta = \begin{cases}
				2\pi & \text{if }n=0 \\
				0 &\text{if }n\neq 0
			\end{cases}
		\end{align*}
		
\end{itemize}

\section*{Section 1.5}

\begin{itemize}
	\item[4.] Use the identity (1) to show that
		\begin{enumerate}[(a)]
			\item $(\sqrt{3}-i)^7=-64\sqrt{3}+64i$

			\item $(1+i)^{95}=2^{47}(1-i)$
				
		\end{enumerate}

	\item[5.] (f) Find the value of $\left( \frac{2i}{1+i} \right)^{1/6}$

\end{itemize}

\end{document}

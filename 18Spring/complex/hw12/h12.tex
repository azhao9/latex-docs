\documentclass{article}
\usepackage[sexy, hdr, fancy]{evan}
\usepackage{graphicx}
\graphicspath{.}
\setlength{\droptitle}{-4em}

\DeclareMathOperator{\re}{Re}
\DeclareMathOperator{\im}{Im}

\lhead{Homework 12}
\rhead{Complex Analysis}
\lfoot{}
\cfoot{\thepage}

\begin{document}
\title{Homework 12}
\maketitle
\thispagestyle{fancy}

\section*{Section 6.3}

\begin{itemize}
	\item[1.] $\int_{-\infty}^\infty \frac{dx}{x^2+2x+2}\, dx=\pi$
		\begin{soln}
			This satisfies the required form where $2+\deg P\le\deg Q.$ The poles are at the roots $z=-1\pm i,$ so taking the loop around the pole in the upper half plane, we have the residue at $-1+i$ to be
			\begin{align*}
				\frac{1}{x-(-1-i)}\bigg\vert_{-1+i} = \frac{1}{2i}
			\end{align*}
			so by the residue theorem, the integral is $\frac{1}{2i}\cdot 2\pi i = \pi.$
		\end{soln}

	\item[2.] $\int_{-\infty}^\infty \frac{x^2}{(x^2+9)^2}\, dx=\frac{\pi}{6}$
		\begin{soln}
			This satisfies the required form where $2+\deg P\le \deg Q.$ The poles are at the roots $z=\pm 3i,$ so taking the loop around the pole in the upper half plane, we have the residue at $3i$ to be
			\begin{align*}
				\frac{d}{dz} \left[ \frac{z^2}{(z+3i)^2} \right]\bigg\vert_{3i} = \frac{6iz}{(z+3i)^3}\bigg\vert_{3i} = \frac{6i(3i)}{(6i)^3} = \frac{1}{12i}
			\end{align*}
			so by the residue theorem, the integral is $\frac{1}{12i}\cdot 2\pi i = \frac{\pi}{6}.$
		\end{soln}

	\item[6.] $\int_0^\infty \frac{x^2}{(x^2+1)(x^2+4)}\, dx=\frac{\pi}{6}$
		\begin{soln}
			This satisfies the required form. We have $\int_0^{\infty} \frac{x^2}{(x^2+1)(x^2+4)}\, dx = \frac{1}{2}\int_{-\infty}^\infty \frac{x^2}{(x^2+1)(x^2+4)}\, dx,$ so consider the second improper integral. The poles are at the roots $z=\pm i$ and $z=\pm 2i,$ so taking the loop around the poles in the upper half plane, we have the residues at $i$ and $2i$ to be
			\begin{align*}
				\frac{z^2}{(z+i)(z^2+4)}\bigg\vert_{i} &= -\frac{1}{6i} \\
				\frac{z^2}{(z^2+1)(z+2i)}\bigg\vert_{2i} &= \frac{1}{3i}
			\end{align*}
			so by the residue theorem, the integral is $\frac{1}{2}\left( \frac{1}{3i}-\frac{1}{6i} \right)\cdot 2\pi i = \frac{\pi}{6}.$
		\end{soln}

	\item[11.] Show that
		\begin{align*}
			\int_0^\infty \frac{dx}{x^3+1}\, dz=\frac{2\pi\sqrt{3}}{9}
		\end{align*}
		by integrating $1/(z^3+1)$ around the boundary of the circular sector $S_p=\left\{ z=re^{i\theta}: 0\le \theta\le 2\pi/3, 0\le r\le p \right\}$ and letting $p\to\infty.$
		\begin{soln}
			Let $\gamma_1$ be the segment from $0$ to $p$ on the real axis, $\gamma_2$ be the arc, and $\gamma_3$ be the segment back to the origin. We have the parametrizations
			\begin{align*}
				\gamma_1(t) &= t, \quad 0\le t\le p \\
				\gamma_2(t) &= pe^{it}, \quad 0\le t\le \frac{2\pi}{3} \\
				\gamma_3(t) &= e^{2\pi i/3}(p-t), \quad 0\le t\le p
			\end{align*}
			and thus the integral over $S_p$ is given by
			\begin{align*}
				\int_{\gamma_1} \frac{dz}{z^3+1}+ \int_{\gamma_2}\frac{dz}{z^3+1} + \int_{\gamma_3} \frac{dz}{z^3+1} &= \int_0^p \frac{1}{t^3+1}\, dt + \int_0^{2\pi/3} \frac{pie^{it}}{p^3e^{3it}+1}\, dt + \int_0^p \frac{-e^{2\pi i/3}}{(p-t)^3+1}\, dt \\
				&= \left( 1-e^{2\pi i/3} \right)\int_0^p \frac{1}{t^3+1}\, dt +\int_0^{2\pi/3}\frac{pie^{it}}{p^3e^{3it}+1}\, dt
			\end{align*}
			Now, taking the limit as $p\to\infty,$ we have
			\begin{align*}
				\abs{\int_0^{2\pi/3} \frac{pie^{it}}{p^3e^{3it}+1}\, dt} &\le \int_0^{2\pi/3} \frac{\abs{pie^{it}}}{\abs{p^3e^{3it}+1}}\, dt \le \int_0^{2\pi/3} \frac{p}{p^3-1} = \frac{2\pi}{3}\cdot \frac{p}{p^3-1}\to 0
			\end{align*}
			Now, the integrand has a pole at $e^{\pi i/3}$ in $S_p,$ where the residue is
			\begin{align*}
				\frac{1}{\frac{d}{dz}\left[ z^3+1 \right]\bigg\vert_{e^{\pi i/3}}} = \frac{1}{3e^{2\pi i/3}}
			\end{align*}
			so by the residue theorem the integral around $S_p$ is $2\pi i\cdot \frac{1}{3e^{2\pi i/3}} = \frac{2\pi i}{3}e^{-2\pi i/3}.$ Equating the two, we have
			\begin{align*}
				\frac{2\pi i}{3}e^{-2\pi i/3} &= \lim_{p\to\infty} \left( 1-e^{2\pi i/3} \right)\int_0^p \frac{1}{t^3+1}\, dt \implies \int_0^{\infty} \frac{1}{t^3+1}\, dt = \frac{\frac{2\pi i}{3}e^{-2\pi i/3}}{1-e^{2\pi i/3}} \\
				&= \frac{\frac{2\pi i}{3}\left( -\frac{1}{2} -i\frac{\sqrt{3}}{2} \right)}{1-\left( -\frac{1}{2} + i\frac{\sqrt{3}}{2} \right)} =\frac{2\pi \sqrt{3}}{9}
			\end{align*}
		\end{soln}

	\item[13.] Show that
		\begin{align*}
			\int_{-\infty}^\infty \frac{1}{(1+x^2)^{n+1}}\, dz=\frac{\pi(2n)!}{2^{2n}(n!)^2}, \quad n=0, 1, 2, \cdots
		\end{align*}
		\begin{soln}
			We have a pole at $i$ in the upper half plane, and the residue is given by
			\begin{align*}
				\frac{1}{n!} \frac{d^n}{dz^n} \left[ \frac{1}{(z+i)^{n+1}} \right]\bigg\vert_i = \frac{1}{n!} (-1)^n (n+1)(n+2)\cdots(2n)\cdot \frac{1}{(2i)^{2n+1}} = \frac{(2n)!}{(n!)^2}(-1)^n \cdot \frac{1}{2^{2n+1} (-1)^n \cdot i}
			\end{align*}
			so by the residue theorem, the integral is
			\begin{align*}
				2\pi i\cdot \frac{(2n)!}{(n!)^2}\cdot \frac{1}{2^{2n+1} i} = \frac{\pi(2n)!}{2^{2n}(n!)^2}
			\end{align*}
		\end{soln}
		
\end{itemize}

\section*{Section 6.4}

\begin{itemize}
	\item[1.] $\int_{-\infty}^\infty \frac{\cos 2x}{x^2+1}\, dx=\frac{\pi}{e^2}$
		\begin{soln}
			We have
			\begin{align*}
				\int \frac{\cos 2z}{z^2+1}\, dz = \frac{1}{2}\int \frac{e^{2iz}}{z^2+1}\, dz + \frac{1}{2}\int \frac{e^{-2iz}}{z^2+1}
			\end{align*}
			These have poles at $\pm i.$ Integrating the first along the upper half plane and the second along the lower half plane, we find that the residues at $i$ and $-i$ are
			\begin{align*}
				\frac{e^{2iz}}{z+i}\bigg\vert_{i} &= \frac{e^{-2}}{2i} \\
				\frac{e^{-2iz}}{z-i}\bigg\vert_{-i} &= \frac{e^{-2}}{-2i}
			\end{align*}
			so since the bottom half plane integral is clockwise, by the residue theorem the integral is
			\begin{align*}
				2\pi i\cdot \frac{1}{2}\left( \frac{e^{-2}}{2i} - \frac{e^{-2}}{-2i} \right) = \pi e^{2}
			\end{align*}
		\end{soln}

	\item[2.] $\int_{-\infty}^\infty \frac{x\sin x}{x^2-2x+10}\, dx = \frac{\pi}{3e^3}(3\cos 1+\sin 1)$
		\begin{soln}
			We have
			\begin{align*}
				\int \frac{z\sin z}{z^2-2z+10}\, dz = \frac{1}{2i}\int \frac{ze^{iz}}{z^2-2z+10}\, dz - \frac{1}{2i}\int \frac{ze^{-iz}}{z^2-2z+10}\, dz
			\end{align*}
			These have poles at $1\pm 3i.$ Integrating the first along the upper half plane and the second along the lower half plane, we find that the residues at $1+3i$ and $1-3i$ are
			\begin{align*}
				\frac{ze^{iz}}{z-(1-3i)}\bigg\vert_{1+3i} &= \frac{(1+3i)e^{-3+i}}{6i} \\
				\frac{ze^{-iz}}{z-(1+3i)}\bigg\vert_{1-3i} &= \frac{(1-3i)e^{-3-i}}{-6i}
			\end{align*}
			so since the bottom half plane integral is clockwise, by the residue theorem the integral is
			\begin{align*}
				2\pi i\cdot \frac{1}{2i}\left( \frac{(1+3i)e^{-3+i}}{6i} + \frac{(1-3i)e^{-3-i}}{-6i} \right) &= \pi e^{-3} \left( \frac{e^i - e^{-i}}{6i} + \frac{e^i+e^{-i}}{2} \right) = \pi e^{-3} \left( \frac{1}{3}\sin 1 + \cos 1 \right)
			\end{align*}
		\end{soln}


	\item[6.] $\int_{-\infty}^{\infty} \frac{e^{-2ix}}{x^2+4}\, dx$
		\begin{soln}
			The poles are at $\pm 2i.$ Integrating along the lower half plane, we find that the residue is
			\begin{align*}
				\frac{e^{-2iz}}{z-2i}\bigg\vert_{-2i} &= \frac{e^{-4}}{-4i}
			\end{align*}
			so since the bottom half plane integral is clockwise, by the residue theorem the integral is 
			\begin{align*}
				-2\pi i\cdot \frac{e^{-4}}{-4i} = \frac{\pi e^{-4}}{2}
			\end{align*}
		\end{soln}

	\item[7.] $\int_{-\infty}^\infty \frac{\cos x}{(x^2+1)(x^2+4)}\, dx$
		\begin{soln}
			We have
			\begin{align*}
				\int \frac{\cos z}{(z^2+1)(z^2+4)}\, dz = \frac{1}{2}\int \frac{e^{iz}}{(z^2+1)(z^2+4)}\, dz + \frac{1}{2} \int \frac{e^{-iz}}{(z^2+1)(z^2+4)}\, dz
			\end{align*}
			These have poles at $\pm i$ and $\pm 2i.$ Integrating the first along the upper half plane and the second along the lower half plane, we find that the residues at $i, -i, 2i, -2i$ are
			\begin{align*}
				\frac{e^{iz}}{(z+i)(z^2+4)}\bigg\vert_i &= \frac{e\inv}{6i} \\
				\frac{e^{-iz}}{(z-i)(z^2+4)}\bigg\vert_{-i} &= \frac{e\inv}{-6i} \\
				\frac{e^{iz}}{(z^2+1)(z+2i)}\bigg\vert_{2i} &= \frac{e^{-2}}{-12i} \\
				\frac{e^{-iz}}{(z^2+1)(z-2i)}\bigg\vert_{-2i} &= \frac{e^{-2}}{12i}
			\end{align*}
			so since the bottom half plane integral is clockwise, by the residue theorem the integral is
			\begin{align*}
				2\pi i \cdot \frac{1}{2} \left[ \left( \frac{e\inv}{6i} + \frac{e^{-2}}{-12i} \right) - \left( \frac{e\inv}{-6i} + \frac{e^{-2}}{12i} \right) \right] = \pi\left( \frac{e\inv}{3} - \frac{e^{-2}}{6} \right)
			\end{align*}
		\end{soln}
		
\end{itemize}

\end{document}

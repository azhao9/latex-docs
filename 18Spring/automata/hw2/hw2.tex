\documentclass{article}
\usepackage[sexy, hdr, fancy]{evan}
\setlength{\droptitle}{-4em}
\usepackage{tikz}
\usetikzlibrary{automata, arrows}

\lhead{Homework 2}
\rhead{Automata and Computation Theory}
\lfoot{}
\cfoot{\thepage}

\begin{document}
\title{Homework 2}
\maketitle
\thispagestyle{fancy}

\begin{enumerate}
	\item Give the state diagram of a DFA recognizing the following language. The alphabet is $\left\{ 0, 1 \right\}.$
		\begin{align*}
			\left\{ w:w\text{ has length exactly 3 and its last symbol is different from its first symbol} \right\}
		\end{align*}
		\begin{soln}
			Let $q_0$ be the start state.
			\begin{center}
				\begin{tikzpicture}[->, >=stealth', shorten >=1pt, auto, node distance=2cm, thick, main node/.style={circle, draw, font=\sffamily\Large\bfseries}]

					\node[initial, state] (0) {$q_0$};
					\node[state] (1) [above right of = 0] {$q_1$};
					\node[state] (2) [right of = 1] {$q_2$};
					\node[state, accepting] (3) [right of = 2] {$q_3$};
					\node[state] (4) [below right of = 0] {$q_4$};
					\node[state] (5) [right of = 4] {$q_5$};
					\node[state, accepting] (6) [right of = 5] {$q_6$};
					\node[state] (7) [above of = 3] {$q_7$};
					\node[state] (8) [below of = 6] {$q_8$};

					\path[every node/.style={font=\sffamily\small}]
					(0) edge node {1} (1)
					edge node {0} (4)
					(1) edge node {0, 1} (2)
					(2) edge node {0} (3)
					edge node {1} (7)
					(3) edge [right] node {0, 1} (7)
					(4) edge node {0, 1} (5)
					(5) edge node {1} (6)
					edge node {0} (8)
					(6) edge node {0, 1} (8)
					(7) edge [loop right] node {0, 1} (7)
					(8) edge [loop right] node {0, 1} (8);
				\end{tikzpicture}
			\end{center}
		\end{soln}

		\newpage
	\item Give a DFA (both a state diagram and a formal description) recognizing the following language. The alphabet is $\left\{ 0, 1 \right\}.$
		\begin{align*}
			\left\{ w:w\text{ has odd length or contains an even number of 0s} \right\}
		\end{align*}
		\begin{soln}
			Let $q_0$ be the start state. Then let
			\begin{align*}
				q_1 &:= \text{ odd length, even number of 0s} \\
				q_2 &:= \text{ odd length, odd number of 0s} \\
				q_3 &:= \text{ even length, even number of 0s} \\
				q_4 &:= \text{ even length, odd number of 0s}
			\end{align*}
			Thus, states $q_1, q_2, q_3$ are accepting states. Then $M=(Q, \Sigma, \delta, q_0, F)$ where
			\begin{align*}
				Q &= \left\{ q_0, q_1, q_2, q_3, q_4 \right\} \\
				F &= \left\{ q_1, q_2, q_3 \right\}
			\end{align*}
			and the transition function is described as
			\begin{center}
				\begin{tabular}{c|cc}
					$\delta$ & 0 & 1 \\
					\hline
					$q_0$ & $q_2$ & $q_1$ \\
					$q_1$ & $q_4$ & $q_3$ \\
					$q_2$ & $q_3$ & $q_4$ \\
					$q_3$ & $q_2$ & $q_1$ \\
					$q_4$ & $q_1$ & $q_2$
				\end{tabular}
			\end{center}
			Thus, the state diagram is given by

			\begin{center}
				\begin{tikzpicture}[->, >=stealth', shorten >=1pt, auto, node distance=3cm, thick, main node/.style={circle, draw, font=\sffamily\Large\bfseries}]

					\node[initial, state] (0) {$q_0$};
					\node[state, accepting] (1) [above right of = 0] {$q_1$};
					\node[state, accepting] (2) [below right of = 0] {$q_2$};
					\node[state, accepting] (3) [right of = 1] {$q_3$};
					\node[state] (4) [right of = 2] {$q_4$};

					\path[every node/.style={font=\sffamily\small}]
					(0) edge node {1} (1)
					edge node {0} (2)
					(1) edge [bend left] node  {1} (3)
					edge [bend left] node {0} (4)
					(2) edge [bend left] node {1} (4)
					edge [bend left] node {0} (3)
					(3) edge node {1} (1)
					edge node [left] {0} (2)
					(4) edge node {1} (2)
					edge node [right] {0} (1);
				\end{tikzpicture}
			\end{center}
		\end{soln}

		\newpage
	\item Show that the following language is regular, where the alphabet is $\left\{ 0, 1 \right\}.$
		\begin{align*}
			\left\{ w:w\text{ contains an equal number of occurrences of the substrings 01 and 10} \right\}
		\end{align*}
		\begin{soln}
			Let $q_0$ be the start state. The DFA represented by the following state diagram accepts this language, so it is regular, as desired.
			\begin{center}
				\begin{tikzpicture}[->, >=stealth', shorten >=1pt, auto, node distance=3cm, thick, main node/.style={circle, draw, font=\sffamily\Large\bfseries}]

					\node[initial, state, accepting] (0) {$q_0$};
					\node[state, accepting] (1) [above right of = 0] {$q_1$};
					\node[state, accepting] (2) [below right of = 0] {$q_2$};
					\node[state] (3) [right of = 1] {$q_3$};
					\node[state] (4) [right of = 2] {$q_4$};

					\path[every node/.style={font=\sffamily\small}]
					(0) edge node {1} (1) 
					edge node {0} (2)
					(1) edge [loop above] node {0} (1)
					edge [bend left] node {1} (3)
					(2) edge [loop below] node {1} (2)
					edge [bend right] node  {0} (4)
					(3) edge [loop above] node {1} (3)
					edge [bend left] node {0} (1)
					(4) edge [loop below] node {0} (4)
					edge [bend right] node {1} (2);
				\end{tikzpicture}
			\end{center}
		\end{soln}

		\newpage
	\item For any string $w=w_1w_2\cdots w_n,$ the reverse of $w,$ written as $w^{\mathcal R},$ is the string $w$ in reverse order $w_n\cdots w_2 w_1.$ For any language $A,$ let $A^{\mathcal R}=\left\{ w^{\mathcal R}:w\in A \right\}.$ Show that if $A$ is regular, so is $A^{\mathcal R}.$
		\begin{proof}
			Since $A$ is regular, it is accepted by some DFA $M=(Q, \Sigma, \delta, q_0, F).$ Construct the following NFA $N=(Q', \Sigma', \delta', q_0', F')$ where
			\begin{align*}
				Q' &= Q \cup \left\{ q \right\}\\
				\Sigma' &= \Sigma \\
				\delta'(q_i, w_j) &= \left\{ q:\delta(q, w_j)=q_i \right\} , \forall q_i\in Q \\
				\delta'(q, \varepsilon) &= F \\
				q_0' &= q \\
				F' &= q_0
			\end{align*}
			In this NFA $N,$ we have reversed the direction of every transition in $M.$ We created a new dummy start state that transitions to each of the original accept states under $\varepsilon,$ and the old start state became the new accept state. 
			
			By construction, $N$ accepts $A^{\mathcal R}$ because if $M$ accepts $w_1w_2\cdots w_n\in A,$ then the series of transitions from $q_0$ ends up in $F.$ Then in $N,$ starting at $q,$ we can go to any of the original accept states under $\varepsilon,$ then all the transitions are done in reverse order, so we will end up at $q_0,$ which is the accept state in $N.$ Since every NFA is equivalent to some DFA, it follows that a DFA accepts $A^{\mathcal R},$ so it is regular, as desired.
		\end{proof}

\end{enumerate}

\end{document}

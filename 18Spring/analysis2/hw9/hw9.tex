\documentclass{article}
\usepackage[sexy, hdr, fancy]{evan}
\setlength{\droptitle}{-4em}

\lhead{Homework 9}
\rhead{Honors Analysis II}
\lfoot{}
\cfoot{\thepage}

\begin{document}
\title{Homework 9}
\maketitle
\thispagestyle{fancy}

\section*{Chapter 18: The Lebesgue Integral}

\begin{itemize}
	\item[4.] Find a sequence $(f_n)$ of non-negative measurable functions such that $\lim_{n\to\infty} f_n=0,$ but $\lim_{n\to\infty} \int f_n=1.$ In fact, show that $(f_n)$ can be chosen to converge uniformly to 0.
		\begin{soln}
			Let $f_n=\chi_{[n, n+1]}.$ Then $\lim_{n\to\infty} f_n=0,$ but $\lim_{n\to\infty} \int f_n=\lim_{n\to\infty} 1 = 1.$
		\end{soln}

	\item[6.] Suppose that $f$ and $(f_n)$ are non-negative measurable functions, that $(f_n)$ decreases pointwise to $f,$ and that $\int f_k<\infty$ for some $k.$ Prove that $\int f=\lim_{n\to\infty} \int f_n.$ (Hint: Consider $(f_k-f_n)$ for $n>k.$) Give an example showing that this fails without the assumption that $\int f_k<\infty$ for some $k.$
		\begin{proof}
			Let $g_n=f_k-f_n$ for $n>k.$ Then since $(f_n)$ is decreasing, we have $0\le g_n\le g_{n+1}\le f_k-f,$ so by the monotone convergence theorem, we have
			\begin{align*}
				\lim_{n\to\infty} \int g_n &= \int\lim_{n\to\infty} g_n \\
				\implies \lim_{n\to\infty}\int (f_k-f_n)= \int f_k - \lim_{n\to\infty} \int f_n &=\int \lim_{n\to\infty} (f_k-f_n) = \int (f_k-f) = \int f_k - \int f \\
				\implies \lim_{n\to\infty} \int f_n &= \int f
			\end{align*}

			If we take $f_n=\frac{1}{x+n}\cdot \chi_{[1, \infty)},$ then $(f_n)$ decreases pointwise to 0, but $\int f_n=\infty$ for all $n.$
		\end{proof}

	\item[10.] If $f$ is non-negative and measurable, show that $\int_{-\infty}^\infty f = \lim_{n\to\infty} \int_{-n}^n f = \lim_{n\to\infty} \int_{ \left\{f\ge (1/n) \right\}}f.$
		\begin{proof}
			If $f$ is non-negative and measurable, then if $g_n=f\cdot \chi_{[-n, n]},$ we have $0\le g_n\le g_{n+1}\le f,$ so by the monotone convergence theorem, we have
			\begin{align*}
				\lim_{n\to\infty} \int g_n &= \int \lim_{n\to\infty} g_n \\
				\implies \lim_{n\to\infty} \int f\cdot \chi_{[-n, n]} = \lim_{n\to\infty} \int_{-n}^n f &= \int \lim_{n\to\infty} f\cdot \chi_{[-n, n]} = \int f\cdot \chi_{\RR} = \int_{-\infty}^\infty f
			\end{align*}
			which establishes the first equality. Similarly, we can let $h_n=f\cdot \chi_{ \left\{ f\ge(1/n) \right\}},$ so $0\le h_n\le h_{n+1}\le f,$ so by the monotone convergence theorem, we have
			\begin{align*}
				\lim_{n\to\infty}\int h_n &= \int\lim_{n\to\infty} h_n \\
				\implies \lim_{n\to\infty} \int f\cdot \chi_{ \left\{ f\ge (1/n) \right\}} = \lim_{n\to\infty}\int_{ \left\{ f\ge(1/n) \right\}} f &= \int\lim_{n\to\infty} f\cdot \chi_{ \left\{ f\ge (1/n) \right\}} = \int f\cdot \chi_{ \left\{ f\ge 0 \right\}}
			\end{align*}
			and since $f$ is non-negative, we have $\chi_{ \left\{ f\ge 0 \right\}}=\chi_{\RR},$ and the second equality follows.
		\end{proof}

	\item[15.] Let $f$ be non-negative and measurable. Prove that $\int f< \infty$ if and only if $\sum_{k=-\infty}^{\infty} 2^k m\left\{ f>2^k \right\}<\infty.$
		\begin{proof}
			Let $E_n:=\left\{ f > 2^n \right\}$ and let $F_n:=\left\{ 1\ge f> 2^{-n+1} \right\}$ for $n\ge 0.$ Then we have
			\begin{align*}
				\bigcup_{n=0}^\infty E_n\setminus E_{n+1} &= \left\{ f > 1 \right\} \\
				\bigcup_{n=0}^\infty F_n\setminus F_{n+1} &= \left\{ 1\ge f \ge 0 \right\}
			\end{align*}
			where the sets $E_n\setminus E_{n+1}$ are pairwise disjoint, and likewise for $F_n\setminus F_{n+1}.$ Then let
			\begin{align*}
				g = \sum_{n=0}^{\infty} 2^{n+1} \chi_{E_n\setminus E_{n+1}} + \sum_{n=0}^{\infty} 2^{-n+1}\chi_{F_n\setminus F_{n+1}}
			\end{align*}
		\end{proof}

	\item[23.] If $(f_n)$ is a sequence of Lebesgue integrable functions on $[a, b],$ and $f_n\implies f$ on $[a, b],$ prove that $f$ is integrable and that $\int_a^b \abs{f_n-f}\to 0.$
		\begin{proof}
			Let $\varepsilon>0.$ Then there exists $N$ such that for all $x\in[a, b],$ we have $\abs{f_n(x)-f(x)}<\varepsilon/(b-a)$ whenever $n\ge N.$ Thus, taking $n$ sufficiently large, we have
			\begin{align*}
				\int_a^b \abs{f} \le \int_a^b \abs{f-f_n} + \int_a^b \abs{f_n} < \int_a^b \frac{\varepsilon}{b-a} + \int_a^b\abs{f_n} = \varepsilon + \int_a^b \abs{f_n}
			\end{align*}
			and since $f_n$ is Lebesgue integrable, it follows that $\abs{f_n}$ is LI, and thus $\abs{f}$ is as well, so $f$ is LI. We have
			\begin{align*}
				\int_a^b \abs{f_n-f} <\int_a^b \frac{\varepsilon}{b-a} = \varepsilon
			\end{align*}
			so $\int_a^b \abs{f_n-f}\to 0.$
		\end{proof}

	\item[24.] Prove that $\int_0^\infty e^{-x}\, dx = \lim_{n\to\infty} \int_0^n (1-(x/n))^n\, dx=1.$ (Hint, for $x$ fixed, $(1-(x/n))^n$ increases to $e^{-x}$ as $n\to\infty.$)
		\begin{proof}
			Let $f_n(x)=(1-(x/n))^n\cdot \chi_{[0, n]}(x)$ Then $0\le f_n(x)\le f_{n+1}(x)\le e^{-x}\cdot \chi_{[0, \infty)}(x),$ so by the monotone convergence theorem, we have
			\begin{align*}
				\lim_{n\to\infty} \int f_n &= \int \lim_{n\to\infty} f_n \\
				\implies \lim_{n\to\infty} \int \left( 1-\frac{x}{n} \right)^n\cdot\chi_{[0, n]} = \lim_{n\to\infty} \int_0^n \left( 1-\frac{x}{n} \right)^n &= \int \lim_{n\to\infty} \left( 1- \frac{x}{n} \right)^n \cdot \chi_{[0, n]}= \int e^{-x}\cdot \chi_{[0, \infty)} = \int_0^\infty e^{-x}
			\end{align*}
			We can evaluate this integral as
			\begin{align*}
				\lim_{n\to\infty} \int_0^n \left( 1-\frac{x}{n} \right)^n = \lim_{n\to\infty}\left[ -\frac{n}{n+1}\left( 1-\frac{x}{n} \right)^{n+1} \right]\bigg|_0^n = 1
			\end{align*}
		\end{proof}

	\item[28.] Suppose that $f, g,$ and $h$ are measurable and that $f\le g\le h$ a.e. If $f$ and $h$ are Lebesgue integrable, does it follow that $g$ is Lebesgue integrable? Explain.
		\begin{soln}
			Yes, since
			\begin{align*}
				f\le g\le h \implies \abs{g}\le \abs{f}+\abs{h}
			\end{align*}
			and since both $\abs{f}$ and $\abs{h}$ are integrable because $f$ and $h$ are, it follows that $\abs{g}$ is as well.
		\end{soln}

	\item[37.] Check that the operations $a[f]=[af]$ for $a\in\RR, [f]+[g]=[f+g],$ and $[f]\le [g]$ whenever $f\le g$ a.e. are well defined, and that the collection of equivalence classes is a vector lattice when supplied with this arithmetic. What is $\abs{[f]}$ in this lattice? Is it $[\abs{f}]?$
		\begin{proof}
			We have
			\begin{align*}
				a[f] &= a\left\{ g:f\sim g \right\}  = \left\{ ag:f\sim g \right\}
			\end{align*}
			so if $ag\in a[f],$ it follows that $ag\sim af$ since $g\sim f,$ so $ag\in [af],$ and likewise if $h\in [af],$ we have $h\sim af\implies h\in a[f],$ so equality is well defined.

			Then we have
			\begin{align*}
				[f] + [g] &= \left\{ h + j:h\sim f, j\sim g \right\}
			\end{align*}
			so if $h+j\in [f] + [g],$ it follows that $h+j\sim f+g$ so $h+j\in[f+g],$ and likewise if $k\in [f+g],$ then $k\sim f+g$ where $f\sim f$ and $g\sim g,$ so $k\in [f]+[g].$

			Then if $f\le g$ a.e. it follows that if $h\sim f$ and $j\sim g,$ we have $h\le j$ a.e., so $[f]\le [g].$
		\end{proof}
		
\end{itemize}

\end{document}

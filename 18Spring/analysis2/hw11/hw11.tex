\documentclass{article}
\usepackage[sexy, hdr, fancy]{evan}
\setlength{\droptitle}{-4em}

\lhead{Homework 11}
\rhead{Honors Analysis II}
\lfoot{}
\cfoot{\thepage}

\begin{document}
\title{Homework 11}
\maketitle
\thispagestyle{fancy}

\section*{Chapter 15: Fourier Series}

\begin{itemize}
	\item[7.] Define $f(x)=(\pi-x)^2$ for $0\le x\le 2\pi,$ and extend $f$ to a $2\pi$-periodic continuous function on $\RR$ in the obvious way. Show that the Fourier series for $f$ is $\pi^2/3+4\sum_{n=1}^{\infty} \cos nx/n^2.$ Since the series is uniformly convergent, it actually converges to $f.$ In particular, note that setting $x=0$ yields the familiar formula $\sum_{n=1}^{\infty} 1/n^2=\pi^2/6.$
		\begin{proof}
			We have
			\begin{align*}
				\frac{a_0}{2}&= \frac{1}{2\pi} \int_{0}^{2\pi} (\pi-x)^2\, dt = \frac{1}{2\pi} \left[ -\frac{1}{3}(\pi-x)^3 \right]\bigg\vert_{0}^{2\pi} = \frac{\pi^2}{3} \\
				a_n &= \frac{1}{\pi} \int_{0}^{2\pi} (\pi-x)^2 \cos nx\, dx = \frac{1}{\pi} \left( \int_0^{2\pi} \pi^2\cos nx\, dx - 2\pi \int_0^{2\pi} x\cos nx\, dx + \int_0^{2\pi} x^2\cos nx\, dx \right)
			\end{align*}
			Integrating by parts, we have
			\begin{align*}
				\int_0^{2\pi} x\cos nx\, dx &= \left[ \frac{x}{n} \sin nx \right]\bigg\vert_0^{2\pi} - \frac{1}{n} \int_0^{2\pi} \sin nx\, dx = 0 \\
				\int_0^{2\pi} x^2\cos nx\, dx &= \left[ \frac{x^2}{n} \sin nx \right]\bigg\vert_0^{2\pi} - \frac{2}{n}\int_0^{2\pi} x\sin nx\, dx = -\frac{2}{n} \left( \left[-\frac{x}{n}\cos nx\right]\bigg\vert_0^{2\pi} - \int_0^{2\pi} -\frac{1}{n} \cos nx \right) \\
				&= \frac{4\pi}{n^2}
			\end{align*}
			Thus, since $\int_0^{2\pi} \pi^2\cos nx\, dx=0,$ we have
			\begin{align*}
				a_n &= \frac{1}{\pi} \cdot \frac{4\pi}{n^2} = \frac{4}{n^2}
			\end{align*}

			Similarly, we have
			\begin{align*}
				b_n &= \frac{1}{\pi} \left( \int_0^{2\pi} \pi^2\sin nx\, dx - 2\pi \int_0^{2\pi} x\sin nx\, dx + \int_0^{2\pi} x^2\sin nx\, dx \right) 
			\end{align*}
			Integrating by parts, we have
			\begin{align*}
				\int_0^{2\pi} x\sin nx\, dx &= \left[ -\frac{x}{n} \cos nx \right]\bigg\vert_0^{2\pi} = -\frac{2\pi}{n} \\
				\int_0^{2\pi} x^2\sin nx\, dx &= \left[-\frac{x^2}{n} \cos nx\right]\bigg\vert_0^{2\pi} - \frac{2}{n}\int_0^{2\pi} -x\cos nx\, dx = -\frac{4\pi^2}{n}
			\end{align*}
			Thus, since $\int_0^{2\pi} \pi^2\sin nx\, dx=0,$ we have
			\begin{align*}
				b_n &= \frac{1}{\pi} \left( -2\pi \cdot \left( -\frac{2\pi}{n} \right) + \left( -\frac{4\pi^2}{n} \right) \right) = 0
			\end{align*}
			so the Fourier series of $(\pi-x)^2$ is given by
			\begin{align*}
				(\pi-x)^2 &= \frac{a_0}{2} + \sum_{n=1}^{\infty} (a_n\cos nx + b_n\sin nx) = \frac{\pi^2}{3} + 4\sum_{n=1}^{\infty} \frac{\cos nx}{n^2}
			\end{align*}
			as desired. Substituting $x=0,$ we have
			\begin{align*}
				\pi^2 &= \frac{\pi^3}{3} + 4\sum_{n=1}^{\infty} \frac{1}{n^2} \implies \sum_{n=1}^{\infty} \frac{1}{n^2}= \frac{\pi^2}{6}
			\end{align*}
		\end{proof}

	\item[8.] Fix $n\ge 1$ and $\varepsilon>0.$
		\begin{enumerate}[(a)]
			\item Show that there is a continuous function $f\in C^{2\pi}$ satisfying $\left\lVert f \right\rVert_\infty=1$ and $(1/\pi) \int_{-\pi}^\pi \abs{f(t)-\sign D_n(t)}\, dt<\varepsilon/(n+1).$

			\item Show that $s_n(f)(0)\ge\lambda_n-\varepsilon$ and hence, that $\left\lVert s_n(f) \right\rVert_\infty\ge \lambda-\varepsilon.$

		\end{enumerate}

	\item[9.] Prove that $\left\lVert \sigma_n(f) \right\rVert_2\le \left\lVert f \right\rVert_2$ and $\left\lVert \sigma_n(f) \right\rVert_\infty \le \left\lVert f \right\rVert_\infty.$
		\begin{proof}
			We have
			\begin{align*}
				\left\lVert \sigma_n(f) \right\rVert_2 &= \left\lVert \frac{1}{n} \sum_{k=0}^{n-1} s_k(f) \right\rVert_2 = \frac{1}{n} \left\lVert \sum_{k=0}^{n-1} s_k(f) \right\rVert_2 \le \frac{1}{n} \sum_{k=0}^{n-1} \left\lVert s_k(f) \right\rVert_2
			\end{align*}
			and by Bessel's inequality, we have
			\begin{align*}
				\frac{1}{n} \sum_{k=0}^{n-1} \left\lVert s_k(f) \right\rVert_2\le \frac{1}{n} \sum_{k=0}^{n-1} \left\lVert f \right\rVert_2 = \left\lVert f \right\rVert_2
			\end{align*}
		\end{proof}
		
\end{itemize}

\section*{Chapter 19: Additional Topics}

\begin{itemize}
	\item[1.] Find a sequence of integrable functions $(f_n)$ such that $\int \abs{f_n}\to 0$ for $f_n\not\to 0$ pointwise a.e.
		\begin{soln}
			Let $f_1 = \chi_{[0, 1]}, f_2=\chi_{[0, 1/2]}, f_3=\chi_{[1/2, 1]}, f_4=\chi_{[0, 1/3]}, f_5=\chi_{[1/3, 2/3]},$ etc. Then the integrals approach 0 but $f_n$ does not converge pointwise to anything. 
		\end{soln}

	\item[2.] Find a sequence of integrable functions $(f_n)$ such that $f_n\to0$ uniformly but $\int\abs{f_n}=1$ for all $n.$
		\begin{soln}
			Let $f_n=\frac{1}{n}\cdot \chi_{[0, n]}.$ Then $\int\abs{f_n}= \frac{1}{n}\cdot n = 1$ for all $n,$ but $f_n\to 0$ uniformly.
		\end{soln}

	\item[26.] If $m(E)<\infty$ and $f\in L_p(E),$ show that $\left\lVert f \right\rVert_p\le \left( m(E) \right)^{1/p - 1/q}\left\lVert f \right\rVert_q$ for $1\le p<q<\infty.$ Thus, as sets, $L_q(E)\subset L_p(E)$ whenever $m(E)<\infty.$ (Hint: Holder's inequality). In particular, if $E=[0, 1],$ notice that the $L_p$-norms increase with $p;$ that is, $\left\lVert f \right\rVert_p\le \left\lVert f \right\rVert_q$ for $1\le p<q<\infty.$
		\begin{proof}
			Let $1/p'=\frac{p}{q}$ and $1/q'=1-\frac{p}{q},$ where $p', q'>0$ since $p<q$ and $1/p'+1/q'=1.$ Then if we take $g\equiv 1,$ by Holder's inequality, we have
			\begin{align*}
				\int_E\abs{f^{p}} &\le \left\lVert f^{p} \right\rVert_{p'} \left\lVert g \right\rVert_{q'} = \left( \int_E \abs{f^{p}}^{q/p} \right)^{p/q} \left( \int_E 1^{1/\left(1-\frac{p}{q}\right)} \right)^{ 1-\frac{p}{q}} \\ 
				\implies \left( \int_E \abs{f}^p \right)^{1/p} &= \left( \int_E \abs{f}^q \right)^{1/q} \left( m(E) \right)^{\frac{1}{p} \left( 1-\frac{p}{q} \right)} \\
				\implies \left\lVert f \right\rVert_p &\le \left\lVert f \right\rVert_q \left( m(E) \right)^{1/p-1/q}
			\end{align*}
		\end{proof}

	\item[33.] If $f$ and $g$ are disjointly supported elements of $L_p,$ that is, if $fg=0$ a.e., show that $\left\lVert f+g \right\rVert_p^p=\left\lVert f \right\rVert_p^p + \left\lVert g \right\rVert_p^p.$
		\begin{proof}
			We can partition $\RR$ as
			\begin{align*}
				\RR = \left\{ f=0, g\neq 0 \right\} \cup \left\{ f\neq 0, g=0 \right\}\cup \left\{ f=0, g=0 \right\}\cup \left\{ f\neq0, g\neq 0 \right\}
			\end{align*}
			So we can write
			\begin{align*}
				\left\lVert f+g \right\rVert_p^p &= \int \abs{f+g}^p \\
				&= \int_{ \left\{ f=0, g\neq 0 \right\}}\abs{f+g}^p + \int_{ \left\{ f\neq 0, g=0 \right\}} \abs{f+g}^p + \int_{ \left\{ f=0, g=0 \right\}}\abs{f+g}^p + \int_{ \left\{ f\neq 0, g\neq 0 \right\}}\abs{f+g}^p
			\end{align*}
			Since $fg=0$ a.e., we have $m\left(\left\{ f\neq 0, g\neq 0 \right\}\right)=0,$ so the last integral equals 0, and the third integral trivially equals 0, so this is equal to
			\begin{align*}
				\int_{ \left\{ f=0, g\neq 0 \right\}}\abs{f+g}^p + \int_{ \left\{ f\neq 0, g=0 \right\}} \abs{f+g}^p &= \int_{ \left\{ g\neq0 \right\}} \abs{g}^p + \int_{ \left\{ f\neq 0 \right\}} \abs{f}^p \\
				&=\left[ \left\lVert g \right\rVert_p^p -\int_{ \left\{ g=0 \right\}}\abs{g}^p\right] + \left[ \left\lVert f \right\rVert_p^p -\int_{ \left\{ f=0 \right\}}\abs{f}^p\right] \\
				&= \left\lVert g \right\rVert_p^p + \left\lVert f \right\rVert_p^p
			\end{align*}
			since integrating $\abs{g}^p$ and $\abs{f}^p$ over $\left\{ f=0 \right\}$ and $\left\{ g=0 \right\}$ trivially result in 0.
		\end{proof}
		
\end{itemize}

\end{document}

\documentclass{article}
\usepackage[sexy, hdr, fancy]{evan}
\setlength{\droptitle}{-4em}

\lhead{Homework 1}
\rhead{Investment Science}
\lfoot{}
\cfoot{\thepage}

\newcommand{\var}{\mathrm{Var}}
\newcommand{\cov}{\mathrm{Cov}}

\begin{document}
\title{Homework 1}
\maketitle
\thispagestyle{fancy}

\section*{Chapter 2: The Basic Theory of Interest}

\begin{itemize}
	\item[2.] The number of years $n$ required for an investment at interest rate $r$ to double in value must satisfy $(1+r)^n=2.$ Using $\ln 2 = 0.69$ and the approximation $\ln(1+r)\approx r$ valid for small $r,$ show that $n\approx 69/i,$ where $i$ is the interest rate percentage. Using the better approximation $\ln(1+r)\approx r-\frac{1}{2}r^2,$ show that for $r\approx 0.08$ there holds $n\approx 72/i.$
		\begin{proof}
			Solving $(1+r)^n=2,$ we have
			\begin{align*}
				(1+r)^n = 2 &\implies \ln(1+r)^n = n\ln (1+r)= \ln 2 \\
				\implies nr\approx 0.69 &\implies n \approx \frac{0.69}{r} = \frac{69}{i}
			\end{align*}

			Using a better approximation $\ln(1+r)\approx r-\frac{1}{2}r^2,$ we have
			\begin{align*}
				n\ln(1+r) = \ln 2 &\implies n\left( r-\frac{1}{2}r^2 \right)\approx 0.69 \\
				\implies n\left( 0.08-\frac{1}{2}\cdot 0.08^2 \right) = 0.0768n = 0.69 &\implies n \approx 8.984 \approx \frac{72}{8} = \frac{72}{i}
			\end{align*}
		\end{proof}

	\item[3.] Find the corresponding effective rates for:
		\begin{enumerate}[(a)]
			\item 3\% compounded monthly
				\begin{answer*}
					\begin{align*}
						r &= \left( 1+\frac{0.03}{12} \right)^{12} - 1 = 3.042\%
					\end{align*}
				\end{answer*}

			\item 18\% compounded monthly
				\begin{answer*}
					\begin{align*}
						r &= \left( 1+\frac{0.18}{12} \right)^{12} - 1 = 19.562\%
					\end{align*}
				\end{answer*}

			\item 18\% compounded quarterly 
				\begin{answer*}
					\begin{align*}
						r &= \left( 1+\frac{0.18}{4} \right)^4 - 1 = 19.252\%
					\end{align*}
				\end{answer*}

		\end{enumerate}

	\item[6.] A young couple has made a nonrefundable deposit of the first month's rent (equal to \$1000) on a 6-month apartment lease. The next day they find a different apartment that they like just as well, but its monthly rent is only \$900. They plan to be in the apartment only 6 months. Should they switch to the new apartment? What if they plan to stay 1 year? Assume an interest rate of 12\%.
		\begin{soln}
			The monthly interest rate is $12\%/12 = 1\%.$ By switching, the couple will have to pay \$1900 the first month, then \$900 for 5 months, whereas by keeping, the couple will be paying \$1000 for 6 months. The present values are
			\begin{align*}
				NPV_k &= \sum_{i=0}^{5} \frac{-1000}{1.01^i} = -5853.43 \\
				NPV_s &= -1900 - \sum_{i=1}^{5} \frac{-900}{1.01^i} = -6268.09
			\end{align*}
			Thus, the couple should keep the current apartment since the NPV is greater.

			If they plan to stay for 1 year, then the net present values are
			\begin{align*}
				NPV_k &= \sum_{i=0}^{11} \frac{-1000}{1.01^i} = -11367.6 \\
				NPV_s &= -1900 - \sum_{i=1}^{11} \frac{-900}{1.01^i} = -11230.9
			\end{align*}
			In this case, it makes sense to switch because the NPV of the alternative is greater.
		\end{soln}

	\item[9.] You are considering the purchase of a nice home. It is in every way perfect for you and in excellent condition, except for the roof. The roof has only 5 years of life remaining. A new roof would last 20 years, but would cost \$20,000. The house is expected to last forever. Assuming that costs will remain constant and that the future interest rate is 5\%, what value would you assign to the existing roof?
		\begin{soln}
			Since a roof lasts for 20 years and costs \$20,000 up front, this is equivalent to costing $C$ each year, where $C$ is the value such that
			\begin{align*}
				20, 000 &= \sum_{i=0}^{19} \frac{C}{(1+0.05)^i} \implies C = 1528.4
			\end{align*}
			Now, if the existing roof will last 5 more years, its present value should be
			\begin{align*}
				\sum_{i=0}^{4} \frac{1528.4}{(1+0.05)^i} = 6948
			\end{align*}
		\end{soln}

		\newpage
	\item[11.] Consider the two projects whose cash flows are shown in Table 2.8. Find the IRRs of the two projects and the NPVs at 5\%. Show that the IRR and NPV figures yield different recommendations. Can you explain this?
		\begin{center}
			\begin{tabular}{lcccccc}
				& & & Years & & & \\
				\hline
				& 0 & 1 & 2 & 3 & 4 & 5 \\
				\hline
				Project 1 & -100 & 30 & 30 & 30 & 30 & 30 \\
				Project 2 & -150 & 42 & 42 & 42 & 42 & 42 \\
				\hline
			\end{tabular}
		\end{center}
		\begin{soln}
			Project 1: For IRR, we must find the rate $r_1$
			\begin{align*}
				0 &= -100 + \frac{30}{1+r_1} + \frac{30}{(1+r_1)^2} + \frac{30}{(1+r_1)^3} + \frac{30}{(1+r_1)^4} + \frac{30}{(1+r_1)^5} \\
				\implies r_1 &= 15.2\%
			\end{align*}
			The NPV is calculated using $d=\frac{1}{1+0.05}:$
			\begin{align*}
				NPV_1 &= -100 + \frac{30}{1+0.05} + \frac{30}{(1+0.05)^2} + \frac{30}{(1+0.05)^3} + \frac{30}{(1+0.05)^4} + \frac{30}{(1+0.05)^5} = 29.88
			\end{align*}

			Project 2: For IRR, we must find the rate $r_2$ 
			\begin{align*}
				0 &= -150 + \frac{42}{1+r_2} + \frac{42}{(1+r_2)^2} + \frac{42}{(1+r_2)^3} + \frac{42}{(1+r_2)^4} + \frac{42}{(1+r_2)^5} \\
				\implies r_2 &= 12.4\%
			\end{align*}
			The NPV is calculated using $d=\frac{1}{1+0.05}:$
			\begin{align*}
				NPV_2 &= -150 + \frac{42}{1+0.05} + \frac{42}{(1+0.05)^2} + \frac{42}{(1+0.05)^3} + \frac{42}{(1+0.05)^4} + \frac{42}{(1+0.05)^5} = 31.84
			\end{align*}

			IRR recommends taking project 1 because $r_1>r_2,$ but NPV recommends taking project 2 because $NPV_2>NPV_1.$ These criteria disagree because the sizes of the projects are different. 
		\end{soln}

\end{itemize}

\end{document}

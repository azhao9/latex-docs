\documentclass{article}
\usepackage[sexy, hdr, fancy]{evan}
\setlength{\droptitle}{-4em}

\lhead{Homework 2}
\rhead{Honors Analysis I}
\lfoot{}
\cfoot{\thepage}

\DeclareMathOperator{\Ima}{Im}

\begin{document}
\title{Homework 2}
\maketitle
\thispagestyle{fancy}

\section*{Chapter 2: Countable and Uncountable Sets}

\begin{itemize}
	\item[3.] Given finitely many countable sets $A_1, \cdots, A_n,$ show that $A_1\cup\cdots\cup A_n$ and $A_1\times\cdots\times A_n$ are countable sets.
		\begin{proof}
			Since $A_i$ are countable, we have
			\begin{align*}
				A_1 &= \left\{ a_{11}, a_{12}, a_{13}, \cdots \right\} \\
				A_2 &= \left\{ a_{21}, a_{22}, a_{23}, \cdots \right\} \\
				\vdots \\
				A_n &= \left\{ a_{n1}, a_{n2}, a_{n3}\cdots \right\}
			\end{align*}
			The set of all the $a_{ij}$ (including possible duplicates) is countable by Cantor's diagonal argument, so the union $A_1\cup\cdots\cup A_n$ is countable.

			Since each of the $A_i$ is countable, there exists bijections $f_i:\NN\to A_i.$ We can construct a map 
			\begin{align*}
				g:\NN^n &\to A_1\times\cdots\times A_n \\
				(b_1, b_2, \cdots, b_n) &\mapsto (f_1(b_1), f_2(b_2), \cdots, f_n(b_n))
			\end{align*}
			which is clearly bijective since each of the $f_i$ is bijective. Since $\NN^n$ is countable by Cantor's diagonal argument, the product $A_1\times\cdots\times A_n$ is countable.
		\end{proof}

	\item[7.] Let $A$ be countable. If $f:A\to B$ is onto, show that $B$ is countable; if $g:C\to A$ is 1-1, show that $C$ is countable.
		\begin{proof}
			Since $A$ is countable, suppose its elements are $\left\{ a_1, a_2, a_3, \cdots \right\}.$ Then since $f$ is onto, we must have $B\subset \left\{ f(a_1), f(a_2), f(a_3), \cdots \right\},$ which is a countable set, and therefore $B$ itself is countable.

			Since $g$ is injective, it is a bijection from $C\to g(C)=\left\{ g(c):c\in C \right\}.$ Since $g(C)\subset A,$ it is countable, so there exists a bijection $h:g(C)\to A.$ Thus, the composition $h\circ g:C\to g(C)\to A$ is a bijection, so $C\sim A$ and thus $C$ is countable.
		\end{proof}

	\item[8.] Show that $(0, 1)$ is equivalent to $[0, 1]$ and to $\RR.$
		\begin{proof}
			Consider the function $f:(0, 1)\to\RR$ given by $f(x)=\tan\left( \frac{\pi}{2}x - \frac{\pi}{2} \right).$ This is bijective since it has a well defined inverse $f\inv = \frac{2}{\pi}\left( \arctan x + 1 \right).$ Thus, $(0, 1)\sim\RR.$

			To show that $[0, 1]\sim (0, 1),$ consider a countably infinite subset $(0, 1)\supset S=\left\{ x_1, x_2, x_3, \cdots \right\}.$ Let $T=\left\{ 0, 1\right\}.$  Then since $S\cup T$ is a countable union of countable sets, it is countable, and therefore $(S\cup T)\sim S,$ so there exists a bijection $h:(S\cup T)\to S.$ Now, define $g:[0, 1]\to(0, 1)$ as
			\begin{align*}
				g(x) &= \begin{cases}
					x & \text{if }x\in[1, 0]\setminus (S\cup T) \\
					h(x) & \text{if } x\in (S\cup T)
				\end{cases}
			\end{align*}
			Here, $g$ is defined on $[0, 1],$ its image is $[1, 0]\setminus T = (0, 1),$ and is bijective because of how we have defined it. Thus, $[0, 1]\sim(0, 1).$
	\end{proof}

	\item[16.] The algebraic numbers are those real or complex numbers that are the roots of polynomials having integer coefficients. Prove that the set of algebraic numbers is countable.
		\begin{proof}
			Consider the set of polynomials with integer coefficients, $\ZZ[x].$ Within $\ZZ[x],$ consider the set $S_n$ of polynomials of degree $n.$ Now, we construct a bijective map $g:\ZZ^{n+1}\to S_n:$
			\begin{align*}
				(a_0, a_1, \cdots, a_n) \mapsto \begin{cases}
					a_0 + a_1x + \cdots + (a_n+1) x^n, \quad &a_n\ge 0 \\
					a_0+a_1x+\cdots+a_n x^n, \quad\quad &a_n<0	
				\end{cases}
			\end{align*}
			This is obviously injective, and it is surjective because if
			\begin{align*}
				g=b_0+b_1x+\cdots+b_n x^n
			\end{align*}
			then if $b_n>0,$ we can recover $(b_0, b_1, \cdots, b_n-1)$ and if $b_n<0,$ we can recover $(b_0, b_1, \cdots, b_n).$ The case where $b_n=0$ is impossible since the polynomial has degree $n.$ Thus, $S_n$ is countable since $\ZZ^{n+1}$ is a countable union of countable sets $\ZZ.$ Since $S_n$ is countable, we can enumerate the polynomials $f_i\in S_n$ with $\NN$ and their corresponding set of $n$ (possibly repeated) roots $R_i$:
			\begin{align*}
				R_1 &= \left\{ r_{11}, r_{12}, r_{13}, \cdots, r_{1n} \right\} \\
				R_2 &= \left\{ r_{21}, r_{22}, r_{23}, \cdots, r_{2n} \right\} \\
				R_3 &= \left\{ r_{31}, r_{32}, r_{33}, \cdots, r_{3n} \right\} \\
				\vdots
			\end{align*}
			The set of all roots is countable by Cantor's diagonal argument, and denote this set by $T_n$ for $S_n.$ Now, the set of all algebraic numbers is
			\begin{align*}
				\bigcup_{i=1}^\infty T_i
			\end{align*}
			which is a countable union of countable sets $T_i,$ and therefore countable, as desired.
		\end{proof}

	\item[17.] If $A$ is uncountable and $B$ is countable, show that $A$ and $A\setminus B$ are equivalent. In particular, conclude that $A\setminus B$ is uncountable.
		\begin{proof}
			Let $S\subset A\setminus B$ be a countably infinite set. Then $B\cup S$ is a countable union of countable sets, and therefore countable, so there exists a bijection $g:(B\cup S)\to S.$ Now, define the mapping
			\begin{align*}
				f:A &\to A\setminus B \\
				a &\mapsto \begin{cases}
					a & \text{if } a\in A\setminus(B\cup S) \\
					g(a) & \text{if } a\in (B\cup S)
				\end{cases}
			\end{align*}
			This function is bijective as defined, so $A\sim A\setminus B,$ as desired. Equivalent sets have the same cardinality, so it follows that $A\setminus B$ is also uncountable.
		\end{proof}

		\newpage
	\item[18.] Show that the set of all real numbers in the interval $(0, 1)$ whose base 10 decimal expansion contains no 3s or 7s is uncountable.
		\begin{proof}
			Suppose the set is countable. Then elements can be indexed with $\NN,$ so suppose the list
			\begin{align*}
				x_1 &= 0.a_{11}a_{12}a_{13}\cdots \\
				x_2 &= 0.a_{21}a_{22}a_{23}\cdots \\
				x_3 &= 0.a_{31}a_{32}a_{33}\cdots \\
				\vdots
			\end{align*}
			is exhaustive, where $a_{ij}\in\left\{ 0, 1, 2, 4, 5, 6, 8, 9 \right\}.$ Now, construct a new number
			\begin{align*}
				y &= 0.b_1b_2b_3\cdots, \quad b_i = \begin{cases}
					4, \quad a_{ii} = 5 \\
					5, \quad a_{ii}\neq 5
				\end{cases}
			\end{align*}
			This element is not equal to any of the $x_i,$ but it still fits the criteria of the set. Contradiction, so such a set must be uncountable.
		\end{proof}
		
\end{itemize}

\end{document}

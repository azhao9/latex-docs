\documentclass{article}
\usepackage[sexy, hdr, fancy]{evan}
\setlength{\droptitle}{-4em}

\lhead{Homework 3}
\rhead{Honors Analysis I}
\lfoot{}
\cfoot{\thepage}

\begin{document}
\title{Homework 3}
\maketitle
\thispagestyle{fancy}

\section*{Chapter 2: Countable and Uncountable Sets}

\begin{itemize}
	\item[22.] Show that $\Delta$ contains no nonempty open intervals. In particular, show that if $x, y\in \Delta$ with $x<y,$ then there is some $z\in[0, 1]\setminus \Delta$ with $x<z<y.$
		\begin{proof}
			Suppose $(a, b)\subset [0, 1],$ and let $\Delta=\bigcap_{j=0}^\infty I_j.$ We will show that some $I_n$ does not contain $(a, b).$
			\begin{align*}
				0 < b-a <1 \implies -\infty<\log_3(b-a)<0
			\end{align*}
			Thus, there exists some $n>0$ such that $-n<\log_3(b-a).$ Then we have
			\begin{align*}
				-n<\log_3(b-a)\implies 3^{-n}<3^{\log_3(b-a)} = b-a
			\end{align*}
			Since the measure of $I_n$ is $3^{-n},$ it cannot contain an interval of strictly greater measure, so $(a, b)\not\subset \Delta.$
		\end{proof}

	\item[23.] The endpoints of $\Delta$ are those points in $\Delta$ having a finite-length base 3 decimal expansion (not necessarily in the proper form), that is, all of the points in $\Delta$ of the form $a/3^n$ for some integers $n$ and $0\le a\le 3^n.$ Show that the endpoints of $\Delta$ other than 0 and 1 can be written as $0.a_1a_2\cdots a_{n+1}$ (base 3), where each $a_k$ is 0 or 2, except $a_{n+1},$ which is either 1 or 2. That is, the discarded "middle third" intervals are of the form $(0.a_1a_2\cdots a_n1, 0.a_1a_2\cdots a_n2),$ where both entries are points of $\Delta$ written in base 3.
		\begin{proof}
			By Theorem 2.15, all elements of $\Delta$ can be written as the infinite sum $\sum_{n=1}^{\infty} \frac{a_n}{3^n}$ for $a_i\in\left\{ 0, 2 \right\}.$ Then if $x$ is an endpoint, it can be expressed as a finite sum. It obviously can't end in 0, since otherwise it would end at the farthest non-zero digit. Now,
			
			Case 1: $a_i=0$ for all $i\ge k$ for some $k,$ in which case $x$ ends with 2. 

			Case 2: $a_i=2$ for all $i\ge k$ for some $k,$ in which case $x$ ends with 1. 

			Thus, $x$ must take on the desired form.
		\end{proof}

	\item[26.] Let $f:\Delta\to [0, 1]$ be the Cantor function and let $x, y\in \Delta$ with $x<y.$ Show that $f(x)\le f(y).$ If $f(x)=f(y),$ show that $x$ has two distinct binary expansions. Finally show that $f(x)=f(y)$ if and only if $x$ and $y$ are "consecutive" endpoints of the form $x=0.a_1a_2\cdots a_n1$ and $y=0.a_1a_2\cdots a_n2$ (base 3).
		\begin{proof}
			Since $x, y\in \Delta,$ we may write them as
			\begin{align*}
				x &= \sum_{n=1}^{\infty} \frac{2a_n}{3^n}, & y=\sum_{n=1}^{\infty} \frac{2b_n}{3^n} \\
				\implies f(x) &= \sum_{n=1}^{\infty} \frac{a_n}{2^n}, & f(y) = \sum_{n=1}^{\infty}\frac{b_n}{2^n}
			\end{align*}
			Now suppose that the first $k$ terms of both sums are equal, but that $b_{k+1}-a_{k+1}=1.$ Then
			\begin{align*}
				y-x &= \frac{2}{3^{k+1}} + \sum_{n=k+2}^{\infty} \frac{2(b_n-a_n)}{3^n} \ge \frac{2}{3^{k+1}} + \sum_{n=k+2}^{\infty} \frac{2(-1)}{3^n} = \frac{1}{3^{k+1}} > 0
			\end{align*}
			so this is sufficient for $x<y.$ Now, 
			\begin{align*}
				f(y)-f(x) &= \frac{1}{2^{k+1}} + \sum_{n=k+2}^{\infty}\frac{b_n-a_n}{2^n} \ge \frac{1}{2^{k+1}} + \sum_{n=k+2}^{\infty} \frac{-1}{2^n} = 0 \\
				\implies f(y)&\ge f(x)
			\end{align*}
			as desired. Equality occurs if and only if $b_i-a_i=-1,$ that is, if $b_i=0$ and $a_i=1$ for all $i\ge k+2.$
			\begin{align*}
				y &= 0.c_1c_2\cdots c_k2 \\
				x &= 0.c_1c_2\cdots c_k 0222\cdots \\
				&= 0.c_1c_2\cdots c_k 1 \\
				c_i &\in \left\{ 0, 2 \right\}
			\end{align*}
			Thus, $x$ has two ternary (question says binary, but assuming this is a typo) expansions, and $x$ and $y$ are consecutive endpoints of the desired form.
		\end{proof}

	\item[29.] Prove that the extended Cantor function $f:[0, 1]\to [0, 1]$ is increasing.
		\begin{proof}
			Suppose $p, q\in [0, 1]$ and WLOG $q\ge p.$ We wish to show $f(q)\ge f(p).$ Consider 4 cases:
			
			Case 1: $p, q\in \Delta.$ Then $f(q)\ge f(p)$ by the first part of 26.

			Case 2: $p, q\in [1, 0]\setminus \Delta.$ Since $p\le q,$ it follows that
			\begin{align*}
				\left\{ f(y):y\in \Delta, y\le p \right\} &\subseteq \left\{ f(y):y\in \Delta, y\le q \right\} \\
				\implies f(q) = \sup\left\{ f(y):y\in \Delta, y\le p \right\} &\le \sup\left\{ f(y):y\in \Delta, y\le q \right\} = f(q)
			\end{align*}

			Case 3: $p\in \Delta, q\in [1, 0]\setminus \Delta.$ Then since $p\le q,$ we have
			\begin{align*}
				f(p)&\in \left\{ f(y):y\in \Delta, y\le q \right\} \\
				\implies f(p)&\le \sup\left\{ f(y):y\in \Delta, y\le q \right\} = f(q)
			\end{align*}

			Case 4: $p\in [1, 0]\setminus \Delta, q\in \Delta.$ From case 1, we have shown that $q\ge y\implies f(q)\ge f(y)$ for $q, y\in \Delta.$ Thus, $f(q)$ is an upper bound for $\left\{ f(y):y\in\Delta, y\le p \right\}$ since $y\le p\le q.$ Thus,
			\begin{align*}
				f(p)=\sup\left\{ f(y):y\in \Delta, y\le p \right\} \le f(q)
			\end{align*}

			Thus, $f$ is increasing, as desired.
		\end{proof}

	\item[30.] Check that the construction of the generalized Cantor set with parameter $\alpha,$ as described above, leads to a set of measure $1-\alpha;$ that is, check that the discarded intervals now have total length $\alpha.$
		\begin{proof}
			Going from $I_n$ to $I_{n+1},$ we will be removing a total of $2^{n}$ middle segments, each of length $\alpha 3^{-n-1}.$ The total measure of the removed intervals is thus
			\begin{align*}
				\sum_{n=0}^{\infty} 2^{n}\cdot \alpha 3^{-n-1} = \frac{\alpha}{3}\sum_{n=0}^{\infty}\cdot\left( \frac{2}{3} \right)^n = \frac{\alpha}{3}\cdot \frac{1}{1-\frac{2}{3}} = \alpha
			\end{align*}
		\end{proof}

	\item[32.] Deduce from Theorem 2.17 that a monotone function $f:\RR\to\RR$ has points of continuity in every open interval.
		\begin{soln}
			Consider an open interval $(a, b)\subset \RR.$ Then $f$ restricted to $(a, b)$ must also be monotone, so it has at most countably many points of discontinuity in $(a, b).$ Since $(a, b)$ is uncountable, there must exist a point of continuity in $(a, b),$ as desired.
		\end{soln}

	\item[33.] Let $f:[a, b]\to \RR$ be monotone. Given $n$ distinct points $a<x_1<x_2<\cdots<x_n<b,$ show that $\sum_{i=1}^{n} \abs{f(x_i+)-f(x_i-)}\le \abs{f(b)-f(a)}.$ Use this to give another proof that $f$ has at most countably many (jump) discontinuities. 
		\begin{proof}
			WLOG $f$ is is monotone increasing. Then it holds that $f(x_i+)\ge f(x_i-).$ Thus,
			\begin{align*}
				\sum_{i=1}^{n} \abs{f(x_i+)-f(x_i-)} &= \sum_{i=1}^{n} \left[ f(x_i+)-f(x_i-) \right] \\
				&= -f(x_1-) + \sum_{i=1}^{n-1} \left[ f(x_i+)-f(x_{i+1}-) \right] + f(x_n+)
			\end{align*}
			Now, since $f$ is monotone increasing and $x_{i+1}>x_i$ for all $i,$ each term of the summation is non-positive. Additionally, $a<x_1\implies -f(x_1-)\le -f(a)$ and $b>x_n\implies f(x_n+)\le f(b).$ Thus, we have
			\begin{align*}
				\sum_{i=1}^{n}\abs{f(x_i+)-f(x_i-)} \le -f(a) + f(b) = \abs{f(b)-f(a)}
			\end{align*}
			as desired.
		\end{proof}
		
\end{itemize}

\end{document}

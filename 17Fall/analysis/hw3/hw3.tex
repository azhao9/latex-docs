\documentclass{article}
\usepackage[sexy, hdr, fancy]{evan}
\setlength{\droptitle}{-4em}

\lhead{Homework 3}
\rhead{Honors Analysis I}
\lfoot{}
\cfoot{\thepage}

\begin{document}
\title{Homework 3}
\maketitle
\thispagestyle{fancy}

\section*{Chapter 2: Countable and Uncountable Sets}

\begin{itemize}
	\item[22.] Show that $\Delta$ contains no nonempty open intervals. In particular, show that if $x, y\in \Delta$ with $x<y,$ then there is some $x\in[0, 1]\setminus \Delta$ with $x<z<y.$

	\item[23.] The endpoints of $\Delta$ are those points in $\Delta$ having a finite-length base 3 decimal expansion (not necessarily in the proper form), that is, all of the points in $\Delta$ of the form $a/3^n$ for some integers $n$ and $0\le a\le 3^n.$ Show that the endpoints of $\Delta$ other than 0 and 1 can be written as $0.a_1a_2\cdots a_{n+1}$ (base 3), where each $a_k$ is 0 or 2, except $a_{n+1},$ which is either 1 or 2. That is, the discarded "middle third" intervals are of the form $(0.a_1a_2\cdots a_n1, 0.a_1a_2\cdots a_n2),$ where both entries are points of $\Delta$ written in base 3.

	\item[26.] Let $f:\Delta\to [0, 1]$ be the Cantor function and let $x, y\in \Delta$ with $x<y.$ Show that $f(x)\le f(y).$ If $f(x)=f(y),$ show that $x$ has two distinct binary expansions. Finally show that $f(x)=f(y)$ if and only if $x$ and $y$ are "consecutive" endpoints of the form $x=0.a_1a_2\cdots a_n1$ and $y=0.a_1a_2\cdots a_n2$ (base 3).

	\item[29.] Prove that the extended Cantor function $f:[0, 1]\to [0, 1]$ is increasing.

	\item[30.] Check that the construction of the generalized Cantor set with parameter $\alpha,$ as described above, leads to a set of measure $1-\alpha;$ that is, check that the discarded intervals now have total length $\alpha.$

	\item[32.] Deduce from Theorem 2.17 that a monotone function $f:\RR\to\RR$ has points of continuity in every open interval.

	\item[33.] Let $f:[a, b]\to \RR$ be monotone. Given $n$ distinct points $a<x_1<x_2<\cdots<x_n<b,$ show that $\sum_{i=1}^{n} \abs{f(x_i+)-f(x_i-)}\le \abs{f(b)-f(b)}.$ Use this to give anothe rproof that $f$ has at most countably many (jump) discontinuities. 
		
\end{itemize}

\end{document}

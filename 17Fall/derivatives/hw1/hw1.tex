\documentclass{article}
\usepackage[sexy, hdr, fancy]{evan}
\setlength{\droptitle}{-4em}

\lhead{Homework 1}
\rhead{Introduction to Financial Derivatives}
\lfoot{}
\cfoot{\thepage}

\newcommand{\var}{\mathrm{Var}}
\newcommand{\cov}{\mathrm{Cov}} 

\begin{document}
\title{Homework 1}
\maketitle
\thispagestyle{fancy}

\section*{Chapter 1: Introduction}

\begin{itemize}
	\item[17.] A company knows that it is due to receive a certain amount of a foreign currency in 4 months. What type of option contract is appropriate for hedging?
		\begin{answer*}
			The company can take a long position in a put option on the foreign currency. If the exchange rate improves, the company can exchange the foreign currency. If the exchange rate worsens, the company can exercise its option and sell the foreign currency at the strike price.
		\end{answer*}

	\item[18.] A US company expect to have to pay 1 million Canadian dollars in 6 months. Explain how the exchange rate risk can be hedged using (a) a forward contract and (b) an option.
		\begin{answer*}
			\begin{enumerate}[(a)]
				\item The company can take a long position in a forward contract to buy 1 million CAD in 6 months for a pre-determined USD-CAD exchange rate. 

				\item The company can take a long position in a call option on the USD-CAD exchange rate. If the exchange rate improves, the company can exchange USD for CAD for payout. If the exchange rate worsens, the company can exercise its option and buy CAD at the strike price for payout.
					
			\end{enumerate}
		\end{answer*}

	\item[22.] Describe the profit from the following portfolio: a long forward contract on a asset and a long European put option on the asset with the same maturity as the forward contract and a strike price that is equal to the forward price of the asset at the time the portfolio is set up.
		\begin{answer*}
			The payoff from the long forward contract at maturity is $S_T-K.$ If the price of the option is $P,$ then the payoff from the put option at maturity is $\max\left\{ K-S_T, 0 \right\}-P.$ If $K<S_T,$ then the profit is $S_T-K-P,$ and if $K\ge S_T,$ then the profit is $-P.$
		\end{answer*}

	\item[23.] In the 1980s, Bankers Trust developed index currency option notes. These are bonds in which the amount received by the holder at maturity varies with a foreign exchange rate. One example was its trading with the Long Term Credit Bank of Japan. The ICON specified that if the yen-US dollar exchange rate, $S_T,$ is greater than 169 yen per dollar at maturity (in 1995), the holder of the bond receives \$1000. If it is less than 169 yen per dollar, the amount received by the holder of the bond is
		\[1000-\max\left\{ 0, 1000\left( \frac{169}{S_T}-1 \right) \right\}\]
		When the exchange rate is below 84.5, nothing is received by the holder at maturity. Show that this ICON is a combination of a regular bond and two options.
		\begin{soln}
			This ICON can be created synthetically using a portfolio with a bond paying \$1000 at maturity, a long position on a call option at $S_T=84.5$ to buy yen, and a short position on a call option at $S_T=169$ to buy yen. If $S_T>169,$ then neither option will be exercised and the payout will be \$1000. If $84.5\le S_T\le 169,$ then the counterparty will exercise its call option. The holder buys 169, 000 yen at $S_T$ for $\frac{169, 000}{S_T},$ then sell to the counterparty at 169, for a payoff (including the bond) of
			\[1000-\frac{169, 000}{S_T} + \frac{169, 000}{169} = 1000\left( 2 - \frac{169}{S_T} \right) = 1000 + \max\left\{ 0, 1000\left( \frac{169}{S_T} - 1 \right) \right\}\]
			If $S_T<84.5,$ then both options will be exercised. The holder will buy 169, 000 yen at 84.5, then sell to the counterparty at 169, for a payoff (including the bond) of
			\[1000-\frac{169, 000}{84.5} + \frac{169, 000}{169} = 0\]
			Thus, this portfolio behaves exactly like the ICON.
		\end{soln}

	\item[39.] Suppose that in the situation of Table 1.1 a corporate treasurer said: "I will have \pounds1 million to sell in 6 months. If the exchange rate is less than 1.52, I want you to give me 1.52. If it is greater than 1.58, I will accept 1.58. If the exchange rate is between 1.52 and 1.58, I will sell the sterling for the exchange rate." How could you use options to satisfy the treasurer?
		\begin{answer*}
			Have the treasurer take a long position on a put option at 1.52, and a short position on a call option at 1.58. If the exchange rate goes below 1.52, he will exercise the option and get an exchange rate of 1.52. If it goes above 1.58, the counterparty will exercise their option and get an exchange rate of 1.58. If it's in between, the transaction will occur at the spot rate.
		\end{answer*}

	\item[40.] Describe how foreign currency options can be used for hedging in the situation considered in Section 1.7 so that (a) ImportCo is guaranteed that its exchange rate will be less than 1.5700 and (b) ExportCo is guaranteed that is exchange rate will be at least 1.5300. Use DerivaGem to calculate the cost of setting up the hedge in each case assuming that the exchange rate volatility is 12\%, interest rates in the US are 5\%, and the interests rates in Britain are 5.7\%. Assume that the current exchange rate is the average of the bid and offer in Table 1.1.
		\begin{soln}
			\begin{enumerate}[(a)]
				\item ImportCo can take a long position on a 3 month call option at 1.5700. If the exchange rate goes above 1.5700, it can exercise the option and buy the \pounds10 million at 1.5700 for payout. If it goes below 1.5700, it will buy the sterling at the USD-EUR spot rate. According to DerivaGem, the cost of setting up this hedge is \$0.0289 per pound, which is
					\begin{align*}
						\frac{\$0.0289}{\pounds1}\cdot(\pounds 1, 000, 000) = \$28, 900
					\end{align*}

				\item ExportCo can take a long position on a 3 month put option at 1.5300. If the exchange rate goes below 1.5300, it can exercise the option and sell the \pounds30 million at 1.5300. If it goes above 1.5300, it will sell the sterling at the USD-EUR spot rate. According to DerivaGem, the cost of setting up this hedge is \$0.0278 per pound, which is
					\begin{align*}
						\frac{\$0.0278}{\pounds1}\cdot(\pounds30, 000, 000) = \$834, 000
					\end{align*}

			\end{enumerate}
		\end{soln}
		
\end{itemize}

\section*{Chapter 2: Mechanics of Futures Markets}

\begin{itemize}
	\item[15.] At the end of one day a clearing house member is long 100 contracts, and the settlement price is \$50, 000 per contract. The original margin is \$2000 per contract. On the following day the member becomes responsible for clearing an additional 20 long contracts, entered into at a price of \$51, 000 per contract. The settlement price at the end of this day is \$50, 200. How much does the member have to add to its margin account with the exchange clearing house?
		\begin{soln}
			The member initially has $\$2000\cdot 100=\$200, 000$ in his margin account. At the end of the second day, the member has gained $\$200\cdot 100 = \$20, 000$ from the original 100 contracts, but has lost $\$800\cdot 20 = \$16, 000$ for a balance of $\$204, 000.$ Since the member owns 120 contracts now, he must have $\$2000\cdot 120 = \$240, 000$ in his margin account, so he must deposit $\$36, 000$ into his margin account.
		\end{soln}

	\item[21.] What do you think would happen if an exchange started trading a contract in which the quality of the underlying asset was incompletely specified?
		\begin{answer*}
			The quality of the underlying asset being delivered will deteriorate over time as short positions try to deliver lower and lower quality assets. Fewer parties would be willing to go long, so the contract loses value over time.
		\end{answer*}

	\item[22.] "When a futures contract is traded on the floor of the exchange, it may be the case that the open interest increases by one, stays the same, or decreases by one." Explain this statement.
		\begin{answer*}
			Open interest can increase by 1 if the futures contract is opening a new position for both parties. It can decrease by 1 if the contract is closing positions for both parties. For open interest to stay the same, suppose party $A$ is short 2 contracts of asset $a,$ and parties $B$ and $C$ are each long 1 contract of asset $a,$ for a total of 2 outstanding contracts. If $B$ goes long and $C$ goes short on a contract of $a,$ then the net result is $A$ being short 2 contracts of $a$ and $B$ being long 2 contracts of $a,$ so open interest stays the same.
		\end{answer*}

	\item[28.] Explain what is meant by open interest. Why does the open interest usually decline during the month preceding the delivery month? On a particular day, there were 2000 trades in a particular futures contract. This means that there were 2000 buyers and 2000 sellers. Of the 2000 buyers, 1400 were closing out positions and 600 were entering into new positions. Of the 2000 sellers, 1200 were closing out positions and 800 were entering into new positions. What is the impact of the day's trading on open interest?
		\begin{answer*}
			Open interest is the number of outstanding futures contracts. Open interest usually declines during the month preceding the delivery month because traders typically like to close positions to avoid delivery of assets. 

			Considering the number of long positions: the buyers opened 600 new long positions, and the sellers closed out 1200 long positions, for a total decrease of 600 in open interest.
		\end{answer*}

	\item[31.] Suppose that there are no storage costs for crude oil and the interest rate for borrowing or lending is 5\% per annum. How could you make money if the June and December futures contracts for a particular year trade at \$80 and \$86, respectively?
		\begin{answer*}
			Since the spot price of oil should be the same as the futures price in June, borrow \$80 to buy oil, and short the December futures contract. In December, deliver the oil for \$86, and pay off the loan of $\$80(1+2.5\%)=\$82$ for a profit of $\$4.$
		\end{answer*}
		
\end{itemize}

\end{document}

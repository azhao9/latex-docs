\documentclass{article}
\usepackage[sexy, hdr, fancy]{evan}
\setlength{\droptitle}{-4em}

\lhead{Homework 2}
\rhead{Introduction to Financial Derivatives}
\lfoot{}
\cfoot{\thepage}

\newcommand{\var}{\mathrm{Var}}
\newcommand{\cov}{\mathrm{Cov}}

\begin{document}
\title{Homework 2}
\maketitle
\thispagestyle{fancy}

\section*{Chapter 3: Hedging Strategies Using Futures}

\begin{itemize}
	\item[4.] Under what circumstances does a minimum variance hedge portfolio lead to no hedging at all?
		\begin{answer*}
			If the correlation between the spot price and the futures price is 0, then the optimal number of contracts is 0.
		\end{answer*}

	\item[7.] A company has a \$20 million portfolio with a beta of 1.2 It would like to use futures contracts on a stock index to hedge its risk. The index futures price is currently standing at 1080, and each contract is for deliver of \$250 times the index. What is the hedge that minimizes risk? What should the company do if it wants to reduce the beta of the portfolio to 0.6?
		\begin{soln}
			We have
			\begin{align*}
				N^* &= \beta\frac{V_A}{V_F} = 1.2\cdot \frac{20, 000, 000}{1080\cdot 250} \approx 89
			\end{align*}
			so the company should short approximately 89 futures contract on the stock index. 

			To reduce the beta of the portfolio from 1.2 to 0.6, the company should take a short position in
			\begin{align*}
				(1.2-0.6)\cdot \frac{20, 000, 000}{1080\cdot 250} \approx 45
			\end{align*}
			futures contracts.
		\end{soln}

	\item[10.] Explain why a short hedger's position improves when the basis strengthens unexpectedly and worsens when the basis weakens unexpectedly.
		\begin{answer*}
			The effective price that is obtained for the asset with hedging is $F_1+b_2,$ so if the basis strengthens, the position improves, and vice versa.
		\end{answer*}

	\item[17.] A corn farmer argues "I do not use futures contracts for hedging. My real risk is not the price of corn. It is that my whole crop gets wiped out by the weather." Discuss this viewpoint. Should the farmer estimate his or her expected production of corn and hedge to try to lock in a price for expected production?

	\item[18.] On July 1, an investor holds 50, 000 shares of a certain stock. The market price is \$30 per share. The investor is interested in hedging against movements in the market over the next month and decides to use the September Mini S\&P 500 futures contract. The index futures price is currently 1500 and one contract is for deliver of \$50 times the index. The beta of the stock is 1.3. What strategy should the investor follow? Under what circumstances will it be profitable?
		\begin{soln}
			The investor should short futures contract on stock index. The optimal number is
			\begin{align*}
				N^* &= \beta\frac{V_A}{V_F} = 1.3\cdot \frac{50000\cdot 30}{1500\cdot50} = 26
			\end{align*}
			contracts. This strategy will be profitable if the stock over-performs relative to the stock index. 
		\end{soln}

	\item[20.] A futures contract is used for hedging. Explain why the daily settlement of the contract can give rise to cash-flow problems.

	\item[22.] Suppose that the 1-year gold lease rate is 1.5\% and the 1-year risk-free rate is 5.0\%. Both rates are compounded annually. Use the discussion in Business Snapshot 3.1 to calculate the maximum 1-year gold forward price Goldman Sachs should quote to the gold-mining company when the spot price is \$1200.

	\item[32.] It is now October 2014. A company anticipates that it will purchase 1 million pounds of copper in each of 02/15, 08/15, 02/16, and 08/16. The company has decided to use the futures contracts traded in the COMEX division of the CME Group to hedge its risk. One contract is for the delivery of 25, 000 pounds of copper. The initial margin is \$2000 per contract and the maintenance margin is \$1500 per contract. The company's policy is to hedge 80\% of its exposure. Contracts with maturities up to 13 months in the future are considered to have sufficient liquidity to meet the company's needs. Devise a hedging strategy for the company. 

		Assume the market prices (in cents per pound) today and at future dates are as in the following table. What is the impact of the strategy you propose on the price the company pays for copper? What is the initial margin requirement in October 2014? Is the company subject to any margin calls?
		\begin{center}	
			\begin{tabular}{lccccc}
				\hline
				Date & Oct 2014 & Feb 2015 & Aug 2015 & Feb 2016 & Aug 2016 \\
				\hline
				Spot price & 372.00 & 369.00 & 365.00 & 377.00 & 388.00 \\
				Mar 2014 futures price & 372.00 & 369.10 & & & \\
				Sept 2014 futures price & 372.80 & 370.20 & 364.80 & & \\
				Mar 2015 futures price & & 370.70 & 364.30 & 376.70 & \\
				Sept 2015 futures price & & & 364.20 & 376.50 & 388.20 \\
				\hline
			\end{tabular}
		\end{center}
		
\end{itemize}

\end{document}

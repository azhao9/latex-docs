\documentclass{article}
\usepackage[sexy, hdr, fancy]{evan}
\setlength{\droptitle}{-4em}

\lhead{Homework 3}
\rhead{Introduction to Financial Derivatives}
\lfoot{}
\cfoot{\thepage}

\newcommand{\var}{\mathrm{Var}}
\newcommand{\cov}{\mathrm{Cov}}

\begin{document}
\title{Homework 3}
\maketitle
\thispagestyle{fancy}

\section*{Chapter 4: Interest Rates}

\begin{itemize}
	\item[5.] Suppose that zero interest rates with continuous compounding are as follows:
		\begin{center}
			\begin{tabular}{cc}
				Maturity (months) & Rate (\% per annum) \\
				\hline
				3 & 8.0 \\
				6 & 8.2 \\
				9 & 8.4 \\
				12 & 8.5 \\
				15 & 8.6 \\
				18 & 8.7
			\end{tabular}
		\end{center}
		Calculate forward interest rates for the 2nd, 3rd, 4th, 5th, and 6th quarters.
		\begin{soln}
			Using the formula $r_F=\frac{R_2T_2-R_1T_1}{T_2-T_1},$ we the forward rates are
			\begin{align*}
				R_{2F} &= \frac{8.2\cdot 0.5-8.0\cdot0.25}{0.5-0.25} = 8.4\% \\
				R_{3F} &= \frac{8.4\cdot 0.75-8.2\cdot 0.5}{0.75-0.5} = 8.8\% \\
				R_{4F} &= \frac{8.5\cdot 1 - 8.4\cdot 0.75}{1-0.075} = 8.8\% \\
				R_{5F} &= \frac{8.6\cdot 1.25 - 8.5\cdot 1}{1.25-1} = 9.0\% \\
				R_{6F} &= \frac{8.7\cdot 1.5-8.6\cdot 1.25}{1.5-1.25} = 9.2\%
			\end{align*}
		\end{soln}

	\item[8.] What does duration tell you about the sensitivity of a bond portfolio to interest rates? What are the limitations of the duration measure?
		\begin{answer*}
			The longer the duration, the more sensitive the bond portfolio is to interest rate changes. Limitations of the duration measure are that it assumes the yields of all bonds in the portfolio change by approximately the same amount, and the duration equation is only valid for small changes in interest rates.
		\end{answer*}

	\item[9.] What rate of interest with continuous compounding is equivalent to 15\% per annum with monthly compounding?
		\begin{soln}
			15\% per annum is equivalent to $15\%/12=1.25\%$ per month, so we must solve for the rate $r$
			\begin{align*}
				e^r &= (1+1.25\%)^{12} \implies r = 14.91\%
			\end{align*}
		\end{soln}

	\item[11.] Suppose that 6, 12, 18, 24, and 30-month zero rates are, respectively, 4\%, 4.2\%, 4.4\%, 4.6\%, and 4.8\% per annum, with continuous compounding. Estimate the cash price of a bond with a face value of 100 that will mature in 30 months and pays a coupon of 4\% per annum semiannually.
		\begin{soln}
			We discount each coupon of \$2 at the corresponding zero rate to get
			\begin{align*}
				P &= 2e^{-0.04\cdot 0.5} + 2e^{-0.042} + 2e^{-0.044\cdot 1.5} + 2e^{-0.046\cdot 2} + 102e^{-0.048\cdot 2.5} \\
				&= 98.04
			\end{align*}
		\end{soln}

	\item[12.] A 3-year bond provides a coupon of 8\% semiannually and has a cash price of 104. What is the bond's yield?
		\begin{soln}
			We solve for the rate $r$ such that the sum of future discounted cash flows equals the price.
			\begin{align*}
				104 &= 4e^{-0.5r} + 4e^{-r} + 4e^{-1.5r} + 4e^{-2r} + 4e^{-2.5r} + 104e^{-3r} \\
				\implies r &= 6.41\%
			\end{align*}
		\end{soln}

	\item[14.] Suppose that zero interest rates with continuous compounding are as follows:
		\begin{center}
			\begin{tabular}{cc}
				Maturity (years) & Rate (\% per annum) \\
				\hline
				1 & 2.0 \\
				2 & 3.0 \\
				3 & 3.7 \\
				4 & 4.2 \\
				5 & 4.5
			\end{tabular}
		\end{center}
		Calculate forward interest rates for the 2nd, 3rd, 4th, and 5th years.
		\begin{soln}
			Using the formula $r_F=\frac{R_2T_2-R_1T_1}{T_2-T_1},$ we the forward rates are
			\begin{align*}
				R_{2F} &= \frac{3.0\cdot 2-2.0\cdot 1}{2-1} = 4.0\% \\
				R_{3F} &= \frac{3.7\cdot3-3.0\cdot2}{3-2} = 5.1\% \\
				R_{4F} &= \frac{4.2\cdot 4-3.7\cdot3}{4-3} = 5.7\% \\
				R_{5F} &= \frac{4.5\cdot5-4.2\cdot 4}{5-4} = 5.7\%
			\end{align*}
		\end{soln}

	\item[16.] A 10-year 8\% coupon bond currently sells for \$90. A 10-year 4\% coupon bond currently sells for \$80. What is the 10-year zero rate? (Hint: Consider taking a long position in two of the 4\% coupon bonds and a short position in one of the 8\% coupon bonds.)
		\begin{soln}
			If we take a long position in 2 of the 4\% bonds and a short position in 1 of the 8\% bonds, the initial cash flow is $-2\cdot 80 + 90 = -70,$ and the cash flow at maturity is $2\left( 100+\frac{4}{m} \right)-\left( 10+\frac{8}{m} \right) = 100,$ where $m$ is the number of coupon payments per year. The intermediate coupon payments cancel out, so the zero rate is the rate $r$ such that
			\begin{align*}
				70e^{10r} = 100 \implies r = 3.57\%
			\end{align*}
		\end{soln}

	\item[22.] A 5-year bond with a yield of 11\% (continuously compounded) pays an 8\% coupon at the end of each year.
		\begin{enumerate}[(a)]
			\item What is the bond's price?
				\begin{soln}
					Each coupon is \$8, so discounting future cash flows at the rate of the yield, we have
					\begin{align*}
						P &= 8e^{-0.11}+8e^{-0.11\cdot2}+8e^{-0.11\cdot3}+8e^{-0.11\cdot4}+108e^{-0.11\cdot5} \\
						&= 86.80
					\end{align*}
				\end{soln}

			\item What is the bond's duration?
				\begin{soln}
					The bond's duration is
					\begin{align*}
						D &= \sum_{i=1}^{n} t_i\cdot \frac{c_i e^{-yt_i}}{B} = \sum_{i=1}^{5} i\cdot \frac{8e^{-0.11i}}{86.80} + 5\cdot \frac{100e^{-0.11\cdot 5}}{86.80} \\
						&= 4.256
					\end{align*}
				\end{soln}

			\item Use the duration to calculate the effect on the bond's price of a 0.2\% decrease in its yield.
				\begin{soln}
					For a small change $\Delta y$ in the yield, we have
					\begin{align*}
						\Delta B &\approx -BD\Delta y = -86.80\cdot 4.256\cdot (-0.002) \\
						&= 0.74
					\end{align*}
					so the price of the bond will increase by \$0.74 to \$87.54.
				\end{soln}

			\item Recalculate the bond's price on the basis of a 10.8\% per annum yield and verify that the result is in agreement with your answer to (c).
				\begin{soln}
					Discounting the \$8 coupons at the new yield, we have
					\begin{align*}
						P' &= 8e^{-0.108}+8e^{-0.108\cdot 2} + 8e^{-0.108\cdot 3}+8e^{-0.108\cdot 4}+108e^{-0.108\cdot5} \\
						&= 87.54
					\end{align*}
					as expected.
				\end{soln}
				
		\end{enumerate}

	\item[34.] The following table gives the prices of bonds:
		\begin{center}
			\begin{tabular}{cccc}
				Bond principal (\$) & Time to maturity (years) & Annual coupon (\$, SA) & Bond price (\$) \\
				\hline
				100 & 0.50 & 0.0 & 98 \\
				100 & 1.00 & 0.0 & 95 \\
				100 & 1.50 & 6.2 & 101 \\
				100 & 2.00 & 8.0 & 104
			\end{tabular}
		\end{center}
		\begin{enumerate}[(a)]
			\item Calculate zero rates for maturities of 6, 12, 18, and 24 months.
				\begin{soln}
					For 6 and 12 months, there are no coupons, so the zero rates are
					\begin{align*}
						100 &= 98e^{0.5\cdot R_1} \implies R_1 = 4.04\% \\
						100 &= 95e^{R_2} \implies R_2 = 5.13\%
					\end{align*}
					For 18 and 24 months, the coupon payments are \$3.10 and \$4 every 6 months, respectively, and discounting them at the respective zero rates, we have
					\begin{align*}
						101 &= 3.1e^{-0.0404\cdot 0.5} + 3.1e^{-0.0513\cdot 1} + 103.1e^{-1.5R_3} \\
						\implies R_3 &= 5.44\% \\
						104 &= 4e^{-0.0404\cdot 0.5}+4e^{-0.0513\cdot 1} + 4e^{-0.0544\cdot 1.5} + 104e^{-2R_4} \\
						\implies R_4 &= 5.81\%
					\end{align*}
				\end{soln}
			
			\item What are the forward rates for the following periods: 6 to 12 months, 12 to 18 months, and 18 to 24 months?
				\begin{soln}
					Using the zero rates above, and the formula $R_F = \frac{R_2T_2-R_1T_1}{T_2-T_1},$ we have the forward rates
					\begin{align*}
						R_{1F} &= \frac{5.13\cdot 1-4.04\cdot 0.5}{1-0.5} = 6.22\% \\
						R_{2F} &= \frac{5.44\cdot 1.5-5.13\cdot 1}{1.5-1} = 6.06\% \\
						R_{3F} &= \frac{5.81\cdot 2-5.44\cdot 1.5}{2-1.5} = 6.92\%
					\end{align*}
				\end{soln}

			\item What are the 6, 12, 18, and 24-month par yields for bonds that provide semiannual coupon payments?
				\begin{soln}
					The par yields are the coupon rates that cause the price of the bond to be \$100:
					\begin{align*}
						\left( 100+\frac{c_1}{2} \right)e^{-0.0404\cdot 0.5} = 100 \implies c_1 = 4.08 \\
						\frac{c_2}{2}e^{-0.0404\cdot0.5} + \left( 100+\frac{c_2}{2} \right)e^{-0.0513\cdot 1} = 100 \implies c_2 = 5.18 \\
						\frac{c_3}{2}e^{-0.0404\cdot0.5} + \frac{c_3}{2}e^{-0.0513\cdot 1} + \left( 100+\frac{c_3}{2} \right)e^{-0.0544\cdot 1.5} = 100 \implies c_3 = 5.50 \\
						\frac{c_4}{2}e^{-0.0404\cdot0.5}+\frac{c_4}{2}e^{-0.0513\cdot1}+\frac{c_4}{2}e^{-0.0544\cdot 1.5} + \left( 100+\frac{c_4}{2} \right)e^{-0.0581\cdot 2} = 100 \implies c_4 = 5.86
					\end{align*}
				\end{soln}

			\item Estimate the price and yield of a 2-year bond providing a semiannual coupon of 7\% per annum.
				\begin{soln}
					Using the above zero rates on the semiannual coupon payments of \$3.50, we have
					\begin{align*}
						P &= 3.5e^{-0.0404\cdot 0.5} + 3.5e^{-0.0513\cdot1}+3.5e^{-0.0544\cdot 1.5} + 103.5e^{-0.0581\cdot2} \\
						&= 102.13
					\end{align*}
					Then the yield is the rate $r$ such that the PV is equal to the price:
					\begin{align*}
						102.13 &= 3.5e^{-0.5r}+3.5e^{-r}+3.5e^{-1.5r}+103.5e^{-2r} \\
						\implies r &= 5.77\%
					\end{align*}
				\end{soln}
				
		\end{enumerate}
		
\end{itemize}

\end{document}

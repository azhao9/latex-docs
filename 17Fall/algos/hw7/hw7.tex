\documentclass{article}
\usepackage[sexy, hdr, fancy]{evan}
\setlength{\droptitle}{-4em}
\usepackage{tikz}
\usetikzlibrary{arrows}

\lhead{Homework 7}
\rhead{Intro Algorithms}
\lfoot{}
\cfoot{\thepage}

\begin{document}
\title{Homework 7}
\maketitle
\thispagestyle{fancy}


\section{Consistent Lifetimes (34 points)}

Suppose that you are an historian, and you are trying to figure out possible dates for when various historical figures may have lived.  In particular, there are $n$ people $P_1, P_2, \dots, P_n$ who you are studying.  Based on your research, you have discovered a collection of $m$ facts about when these people lived relative to each other.  Each of these facts has one of the following two forms:
\begin{itemize}
	\item For some $i$ and $j$, person $P_i$ died before person $P_j$ was born; or
	\item For some $i$ and $j$, the lifetimes of $P_i$ and $P_j$ overlapped.
\end{itemize}

Unfortunately, since the historical record is never fully trustworthy, it's possible that some of these facts are incorrect.  Design an $O(n+m)$ time algorithm to at least check whether they are \emph{internally consistent}, i.e., whether there is a birth and a death date for each person so that all of the facts are true.  As always, prove running time and correctness (i.e., prove that if there is a possible set of dates then your algorithm will return yes, and if no such dates are possible then your algorithm will return no).
\begin{soln}
	For each person $P_i,$ create vertices $P_{i, b}$ and $P_{i, d},$ representing their birth and death dates, respectively. For each fact, if it states that person $P_i$ died before person $P_j,$ then add the following edges:
	\begin{center}
		\begin{tikzpicture}[->, >=stealth', shorten >=1pt, auto, node distance=2cm, thick, main node/.style={circle, draw, font=\sffamily\Large\bfseries}]

			\node[main node] (1) {$P_{i, b}$};
			\node[main node] (2) [right of=1] {$P_{j, b}$};
			\node[main node] (3) [below of=1] {$P_{i, d}$};
			\node[main node] (4) [right of=3] {$P_{j, d}$};

			\path[every node/.style={font=\sffamily\small}]
			(1) edge node {} (2)
			edge node {} (3)
			edge node {} (4)
			(2) edge node {} (4)
			(3) edge node {} (2)
			edge node {} (4);
		\end{tikzpicture}
	\end{center}
	where an edge from $v$ to $u$ indicates that $v$ happened chronologically before $u.$ Similarly, if the fact states that the lifetimes of $P_i$ and $P_j$ overlapped, then add the following edges:
	\begin{center}
		\begin{tikzpicture}[->, >=stealth', shorten >=1pt, auto, node distance=2cm, thick, main node/.style={circle, draw, font=\sffamily\Large\bfseries}]

			\node[main node] (1) {$P_{i, b}$};
			\node[main node] (2) [right of=1] {$P_{j, b}$};
			\node[main node] (3) [below of=1] {$P_{i, d}$};
			\node[main node] (4) [right of=3] {$P_{j, d}$};

			\path[every node/.style={font=\sffamily\small}]
			(1) edge node {} (3)
			edge node {} (4)
			(2) edge node {} (3)
			edge node {} (4);
		\end{tikzpicture}
	\end{center}

	If there exists a directed cycle in the graph, say $(v_1, v_2, \cdots, v_n, v_1),$ then $v_1$ occurred earlier than $v_n$ which occurred earlier than $v_1,$ which is an inconsistency. On the other hand, this is the only way for an inconsistency to happen, so directed cycles are equivalent to inconsistencies.

	Constructing the graph, we add $2n$ vertices and at most $6m$ edges, which is $O(n+m).$ Then we perform a DFS on this graph, being sure to perform it on every connected component, which takes $O(n+m)$ if we keep track of which vertices have been visited already. If there is a back edge, then there is a directed cycle and therefore an inconsistency, and if not, then the data is consistent.
\end{soln}

\newpage
\section{Minimum Spanning Trees (33 points)}

\begin{enumerate}[(a)]
	\item Suppose that we are given a connected graph $G = (V, E)$ where all edge weights are distinct.  Prove that there is a \emph{unique} minimum spanning tree. 
		\begin{proof}
			Suppose there are two distinct minimum spanning trees $T_1$ and $T_2.$ Then there are edges that are either in $T_1$ and not $T_2$ or $T_2$ but not in $T_1.$ Take the edge $e$ of minimum weight in this set, and WLOG $e\in T_2.$ Suppose $e=\left\{ x, y \right\}.$ Then consider the path from $x$ to $y$ in $T_1.$ There must exist an edge $e'$ on this path not in $T_2,$ since if they were all in $T_2,$ there would be a cycle. Since $e\notin T_1,$ it follows that $T'=T_1\cup \left\{ e \right\}\setminus\left\{ e'\right\}$ is also a spanning tree. Now, $\left\{ a, b \right\}$ is in only one of $T_1$ and $T_2,$ so
			\begin{align*}
				w\left( \left\{ a, b \right\} \right) > w(e) \implies w(T') = w(T_1) + w(e) - w\left( \left\{ a, b \right\} \right) < w(T_1)
			\end{align*}
			which is a contradiction, since $T_1$ was assumed to be an MST. Thus, the MST is unique.
		\end{proof}
\end{enumerate}

As in the previous part, suppose that we are given a connected graph $G = (V, E)$ where all edge weights are distinct.  But instead of trying to construct an MST, we are trying to construct a something else.  Given a spanning tree $T = (V, E')$ of $G$, the \emph{bottleneck edge} of $T$ is the edge in $E'$ of maximum weight.  A spanning tree $T$ is a \emph{minimum-bottleneck spanning tree} if there is no spanning tree $T'$ of $G$ with a lower weight bottleneck edge.

\begin{enumerate}[(a), resume]
	\item Is every minimum-bottleneck spanning tree of $G$ also a minimum spanning tree of $G$?  Prove or give a counterexample (and explanation).
		\begin{soln}
			This is not true. Consider the following graph:
			\begin{center}
				\begin{tikzpicture}[-, auto, node distance=2cm, thick, main node/.style={circle, draw, font=\sffamily\Large\bfseries}]

					\node[main node] (1) {A};
					\node[main node] (2) [right of=1] {B};
					\node[main node] (3) [right of=2] {C};
					\node[main node] (4) [below of=3] {D};

					\path[every node/.style={font=\sffamily\small}]
					(1) edge node {10} (2)
					(2) edge node {1} (4)
					edge node {2} (3)
					(3) edge node {3} (4);
				\end{tikzpicture}
			\end{center}
			An MBST is given by 
			\begin{center}
				\begin{tikzpicture}[-, auto, node distance=2cm, thick, main node/.style={circle, draw, font=\sffamily\Large\bfseries}]

					\node[main node] (1) {A};
					\node[main node] (2) [right of=1] {B};
					\node[main node] (3) [right of=2] {C};
					\node[main node] (4) [below of=3] {D};

					\path[every node/.style={font=\sffamily\small}]
					(1) edge node {10} (2)
					(2) edge node {1} (4)
					(3) edge node {3} (4);
				\end{tikzpicture}
			\end{center}
			but the unique MST is different, as shown below:
			\begin{center}
				\begin{tikzpicture}[-, auto, node distance=2cm, thick, main node/.style={circle, draw, font=\sffamily\Large\bfseries}]

					\node[main node] (1) {A};
					\node[main node] (2) [right of=1] {B};
					\node[main node] (3) [right of=2] {C};
					\node[main node] (4) [below of=3] {D};

					\path[every node/.style={font=\sffamily\small}]
					(1) edge node {10} (2)
					(2) edge node {1} (4)
					edge node {2} (3);
				\end{tikzpicture}
			\end{center}
		\end{soln}

	\item Is every minimum spanning tree of $G$ a minimum-bottleneck spanning tree of $G$?  Prove or give a counterexample (and explanation).
		\begin{proof}
			Let $T_1$ be a minimum-bottleneck spanning tree of $G,$ and let $T_2$ be the unique minimum spanning tree of $G.$ Let $e_1\in T_1$ and $e_2\in T_2$ be the edges of maximum weight in their respective trees. Suppose $T_2$ is not an MBST. Then $e_2\notin T_1$ and $w(e_2)>w(e_1)\ge w(e)$ for any $e\in T_1.$ Suppose $e_2=\left\{ x, y \right\},$ and consider any $e\in T_1$ on the path from $x$ to $y.$ Then $T'=T_2\cup\left\{ e \right\}\setminus\left\{ e_2 \right\}$ is a spanning tree where
			\begin{align*}
				w(T') &= w(T_2) + w(e) - w(e_2) < w(T_2) 
			\end{align*}
			which is a contradiction since $T_2$ was assumed to be an MST. Thus, $T_2$ is also an MBST.
		\end{proof}
\end{enumerate}

\newpage
\section{Shortest Paths (33 points)}
\begin{enumerate}[(a)]
	\item We saw in class that Dijkstra's algorithm for computing shortest paths in a directed graph does not work if there are negative edge lengths.  Consider the following idea to fix this.
		\begin{quote}
			First add a large enough positive constant to every edge weight so that the ``revised" edge lengths are all positive (for example, you could add the value $1 + \max_{v \in V} |len(v)|$).  To find the shortest path between two nodes $s$ and $t$ in the original graph, run Dijkstra's algorithm using the revised edge lengths to find the shortest path between $s$ and $t$ using the revised lengths, and return the path which it finds.
		\end{quote}
		Does this work?  That is, will this always return the shortest path under the original edge lengths?  If so, give a proof.  If not, give a counterexample (and explain it). 
		\begin{soln}
			This does not work. Consider the following graph:
			\begin{center}
				\begin{tikzpicture}[->, >=stealth', shorten >=1pt, auto, node distance=2cm, thick, main node/.style={circle, draw, font=\sffamily\Large\bfseries}]

					\node[main node] (1) {A};
					\node[main node] (2) [right of=1] {B};
					\node[main node] (3) [below of=1] {C};

					\path[every node/.style={font=\sffamily\small}]
					(1) edge node [left] {1} (3)
					edge node [above] {2} (2)
					(2) edge node [below] {-5} (3);
				\end{tikzpicture}
			\end{center}
			The shortest path from $A$ to $C$ is $A\to B\to C.$ If we add a large constant, say 10 to each edge, then the revised graph becomes
			\begin{center}
				\begin{tikzpicture}[->, >=stealth', shorten >=1pt, auto, node distance=2cm, thick, main node/.style={circle, draw, font=\sffamily\Large\bfseries}]

					\node[main node] (1) {A};
					\node[main node] (2) [right of=1] {B};
					\node[main node] (3) [below of=1] {C};

					\path[every node/.style={font=\sffamily\small}]
					(1) edge node [left] {11} (3)
					edge node [above] {12} (2)
					(2) edge node [below] {5} (3);
				\end{tikzpicture}
			\end{center}
			Now, using Dijkstra's algorithm, the shortest path from $A$ to $C$ is just $A\to C.$
		\end{soln}

	\item Now suppose that instead of adding the same value to every length, we instead \emph{multiply} every length by the same value $\alpha > 0$ (note that this preserves the sign of each edge length).  Let $P$ be a shortest path from $s$ to $t$ under the original edge lengths.  Is $P$ still a shortest path from $s$ to $t$ under the new edge lengths?  If yes, give a proof.  If no, give a counterexample (and explain it).  
		\begin{proof}
			This does work. Suppose $P=(s, v_1, v_2, \cdots, v_n, t)$ is a shortest path in the original graph $G,$ and $P'=(s, u_1, u_2, \cdots, u_m, t)$ is a shortest path in the modified graph $G'.$ Then since $P$ is a path in $G'$ and $P'$ is a path in $G,$ we have the following:
			\begin{align*}
				len_G(P) &\le len_G(P') \\
				len_{G'} (P') &= \alpha\left[ len(s, u_1) + len(u_1, u_2) + \cdots + len(u_m, t) \right] \\
				&\le len_{G'}(P) = \alpha\left[ len(s, v_1) + len(v_1, v_2) + \cdots + len(v_n, t) \right] \\
				\implies len(s, u_1)+len(u_1, u_2) + \cdots + len(u_m, t) &\le len(s, v_1) + len(v_1, v_2) + \cdots + len(v_n, t) \\
				\implies len_G(P') &\le len_G(P) \\
				\implies len_G(P) &= len_G(P') \\
				\implies len_{G'}(P) = \alpha \cdot len_G(P) &= \alpha\cdot len_G(P') = len_{G'} (P')
			\end{align*}
			so $P$ is also a shortest path in $G'.$
		\end{proof}

\end{enumerate}

\end{document}

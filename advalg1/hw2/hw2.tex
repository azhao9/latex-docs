\documentclass{article}
\usepackage[sexy, hdr, fancy]{evan}
\setlength{\droptitle}{-4em}

\lhead{Homework 2}
\rhead{Advanced Algebra I}
\lfoot{}
\cfoot{\thepage}

\begin{document}
\title{Homework 2}
\maketitle
\thispagestyle{fancy}

\section*{Section 1.1: Induction}
\begin{itemize}
	\item[14.] (a) Show that \[\binom{n}{0} + \binom{n}{1}+\binom{n}{2} + \cdots + \binom{n}{n} = 2^n\] for all $n\ge 0.$
		\begin{proof}
			We proceed by induction. The base case is $n=0.$ In this case, $\binom{0}{0}=1=2^0$ thus the statement is true for $n=0.$

			Next, assume that the hypothesis is true for arbitrary $k\in\NN,$ so that \[2^k=\sum_{i=0}^k \binom{k}{i}. \] Consider the sums
			\begin{align*}
				S_{k+1} =\sum_{i=0}^{k+1}\binom{k+1}{i}&= \binom{k+1}{0}+\binom{k+1}{1}+\cdots+\binom{k+1}{k}+\binom{k+1}{k+1} \\
				S_{k} =\sum_{i=0}^k \binom{k+0}{i} &=\binom{k+0}{0} + \binom{k+0}{1} +\cdots+\binom{k+0}{k}
			\end{align*} Then \[S_{k+1} - S_k = \binom{k+1}{k+1} + \binom{k+1}{0}-\binom{k}{0} + \sum_{i=1}^{k} \left[ \binom{k+1}{i}-\binom{k}{i} \right] \] where the summand can be simplified as
			\begin{align*}
				\binom{k+1}{i}-\binom{k}{i} &= \frac{(k+1)!}{i!(k+1-i)!}-\frac{k!}{i!(k-i)!} \\
				&= \frac{k!}{i!(k-i)!} \left( \frac{k+1}{k+1-i}-1 \right) \\
				&= \frac{k!}{i!(k-i)!}\cdot\frac{i}{k+1-i} \\
				&= \frac{k!}{(i-1)!(k+1-i)!} \\
				&= \binom{k}{i-1}
			\end{align*} Thus, 
			\begin{align*}
				S_{k+1}-S_k &= \binom{k+1}{k+1} + \sum_{i=1}^k\binom{k}{i-1} \\
				&= \binom{k}{k} +\sum_{j=0}^{k-1} \binom{k}{j} \\
				&= \sum_{j=0}^k \binom{k}{j} \\
				&= 2^k.
			\end{align*} Thus, $S_{k+1}= S_k+2^k=2^k+2^k=2^{k+1},$ so the hypothesis is true for $k+1,$ completing the proof. 

		\end{proof}

		(b) Show that \[\binom{n}{0} - \binom{n}{1} + \binom{n}{2} - \cdots \pm \binom{n}{n} = 0\] if $n>0.$ 
		\begin{proof}
			We proceed by induction. The base case is $n=1.$ In this case, $\binom{1}{0}-\binom{1}{1}=0$ thus the statement is true for $n=1.$

			Next, assume the hypothesis is true for arbitrary $k\in\NN,$ so that \[0=\sum_{i=1}^k (-1)^i\binom{k}{i}.\] Consider the sums
			\begin{align*}
				S_{k+1} = \sum_{i=0}^{k+1}(-1)^i \binom{k+1}{i} &= \binom{k+1}{0} - \binom{k+1}{1} + \cdots \pm\binom{k+1}{k}\mp\binom{k+1}{k+1} \\
				S_k = \sum_{i=0}^k (-1)^i \binom{k+0}{i} &= \binom{k+0}{0} - \binom{k+0}{1} + \cdots \pm\binom{k+0}{k}
			\end{align*}
			Then 
			\begin{align*}
				S_{k+1} - S_k &= \mp\binom{k+1}{k+1} + \sum_{i=1}^k (-1)^i \binom{k}{i-1} \\
				&= \mp\binom{k}{k}-\sum_{j=0}^{k-1}(-1)^j\binom{k}{j} \\
				&= -\left( \sum_{j=0}^k (-1)^j \binom{k}{j} \right) \\
				&= 0
			\end{align*} so $S_{k+1}=S_k=0,$ completing the proof. 
			
		\end{proof}

	\item[18.] (b) Conjecture a formula for $a_n$ and prove it by induction: \[a_0 = 1, a_1 = -2, a_{n+2} = 2a_n - a_{n+1}, n\ge 0.\]
		\begin{conjecture}
			We claim that $a_n$ is given by the closed form \[a_n=4-3\cdot2^n\] for all $n\ge0.$
		\end{conjecture}

		\begin{proof}
			We proceed by strong induction. The base cases are $n=0, 1.$ We have 
			\begin{align*}
				a_0 &= 1 = 4 - 3\cdot2^0 \\
				a_1 &= -2 = 4 - 3\cdot 2^1
			\end{align*} so the hypothesis is true for the base cases.

			Next we assume that $a_k = 4 - 3\cdot 2^k$ for all $1\le k\le m + 1.$ Thus 
			\begin{align*}
				a_m &= 4 - 3\cdot 2^m \\
				a_{m+1} &= 4 - 3\cdot 2^{m+1}
			\end{align*} so that 
			\begin{align*}
				a_{m+2} &= 2a_m - a_{m+1} = 2(4-3\cdot 2^m) - (4-3\cdot 2^{m+1}) \\
				&= (8 - 3\cdot 2^{m+1}) - (4 - 3\cdot2^{m+1}) \\
				&= 4 - 6\cdot 2^{m+1} \\
				&= 4 - 3\cdot 2^{m+2}
			\end{align*} which is exactly the closed form in the conjecture, thus proven.
			
		\end{proof}

\end{itemize}

\section*{Section 1.2: Divisors and Prime Factorization}

\begin{itemize}
	\item[18.] If $\gcd{(m, n)} = 1,$ let $d = \gcd{(m + n, m - n)}.$ Show that $d = 1$ or $d = 2.$
		\begin{proof}
			Since $d|m+n$ and $d|m-n,$ it follows that 
			\begin{align*}
				d|[(m+n)+(m-n)] &\implies d|2m \\
				d|[(m+n)-(m-n)] &\implies d|2n
			\end{align*} 
			Let $h=\gcd(2m, 2n).$ Since $d$ divides both $2m$ and $2n,$ it follows that $d$ must divide their gcd, $h.$ But since $\gcd(m, n)=1\implies \gcd(2m, 2n)=2,$ the fact that $d$ must divide $h$ means that $d$ must divide $2,$ so $d=1$ or $d=2,$ as desired.
			
		\end{proof}

	\newpage

	\item[22.] If $d_1, \cdots, d_r$ are divisors of $n$ and if $\gcd{(d_i, d_j)} = 1$ whenever $i\neq j,$ show that $d_1d_2\cdots d_r$ divides $n.$ 
		\begin{proof}
			By Theorem 5, if $d_1$ and $d_2$ are relatively prime and $d_1|n$ and $d_2|n,$ then their product $d_1d_2|n.$ Next, since $d_3$ is relatively prime to $d_1$ and $d_2,$ then $d_3$ is relatively prime to the product $d_1d_2.$ Thus since $d_3|n$ and $d_1d_2|n,$ it follows that $d_1d_2d_3|n.$ Continuing in this fashion, we conclude that $d_1d_2\cdot d_r|n,$ as desired.
			
		\end{proof}

	\item[38.] If $q$ is a rational number such that $q^2$ is an integer, show that $q$ is an integer.
		\begin{proof}
			Since $q$ is rational, we may write $q=\frac{m}{n}$ for $m, n\in\ZZ.$ Thus $q^2=\frac{m^2}{n^2}\in\ZZ,$ so it must be that $n^2|m^2.$ Let the prime factorization of $m$ be \[m=p_1^{e_1}p_2^{e_2}\cdots p_k^{e_k}\] where $p_i$ are distinct primes and $e_i\ge1.$ Then the prime factorization of $m^2$ is \[m^2 = p_1^{2e_1}p_2^{2e_2}\cdots p_k^{2e_k}.\] By Theorem 8, all divisors $d$ of $m^2$ are of the form \[d = p_1^{d_1}p_2^{d_2}\cdots p_k^{d_k}\] where $1\le d_i\le 2e_i$ for all $1\le i\le k.$ Since $n^2$ divides $m^2,$ it too has this form, and since it is the square of an integer, its exponents must all be even. Thus write \[n^2 = p_1^{2f_1}p_2^{2f_2}\cdots p_k^{2f_k}\] where $1\le 2f_i\le 2e_i$ for all $i.$ Then taking the square root, we have \[n=p_1^{f_1}p_2^{f_2}\cdots p_k^{f_k}.\] Since $2f_i\le 2e_i,$ it follows that $f_i\le e_i,$ so then $n$ must divide $m$ by Theorem 8, so $q=\frac{m}{n}$ is an integer, as desired. 
			
		\end{proof}
	
	\item[42.] Show that $\gcd{(a, b, c)} = \gcd{[a, \gcd{(b, c)}]}.$
		\begin{lemma}
			For any integers $x, y, z\in\ZZ,$ we have \[\max{(x, y, z)}=\max{(x, \max{(y, z)})}.\]
		\end{lemma}
		\begin{proof}
			We have 3 cases, one where each of $x, y, z$ is the maximum of the three. 

		Case 1: $x$ is maximum. Then $\max{(y, z)}\le x,$ so \[\max{(x, \max{(y, z)})} = x = \max{(x, y, z)}.\]

		Case 2: $y$ is maximum. Then $\max{(y, z)}=y\ge x,$ so \[\max{(x, \max{(y, z)})} = \max{(x, y)} = y = \max{(x, y, z)}.\]

		Case 3: $z$ is maximum. This is identical to Case 2. 

		Thus \[\max{(x. y. z)} = \max{(x, \max{(y, z)})}, \] as desired.
		\end{proof}
		

		\begin{proof}
			WLOG all of $a, b, c$ are positive, since negative numbers only differ by a factor of $-1.$ Let $p_k$ be the greatest prime that divides any of $a, b, c.$ Then we may write factorizations of $a, b, c$ as 
			\begin{align*}
				a &= p_1^{e_1}p_2^{e_2}\cdots p_k^{e_k} \\
				b &= p_1^{f_1}p_2^{f_2}\cdots p_k^{f_k} \\
				c &= p_1^{g_1}p_2^{g_2}\cdots p_k^{g_k}
			\end{align*} where $p_i$ are all primes ranging from 2 to $p_k,$ and $e_i, f_i, g_i$ are all non-negative integers, where they are 0 if $a, b, c$ are not divisible by $p_i,$ respectively. Then, by Theorem 9, we have \[\gcd{(a, b, c)} = \prod_{i=1}^k p_i^{\max{(e_i, f_i, g_i)}}.\] We also have \[\gcd{(b, c)}=\prod_{i=1}^k p_i^{\max{(f_i, g_i)}}\] and then 
			\begin{align*}
				\gcd{[a, \gcd{(b, c)}]} &= \gcd\left( a, \prod_{i=1}^k p_i^{\max{(f_i, g_i)}} \right) \\
				&= \gcd\left( \prod_{i=1}^k p_i^{e_i}, \prod_{i=1}^k p_i^{\max{(f_i, g_i)}} \right) \\
				&= \prod_{i=1}^k p_i^{\max{(e_i, \max{(f_i, g_i)})}} \\
				&= \prod_{i=1}^k p_i^{\max{(e_i, f_i, g_i)}}
			\end{align*} where the last step is due to Lemma 0.2. Thus \[\gcd{(a, b,  c)} = \gcd{(a, \gcd{(b, c)})}\] as desired.

		\end{proof}
		
\end{itemize}


\section*{Section 1.3: Integers Modulo $n$}

\begin{itemize}
	\item[8.] (b) Find the remainder when $8^{391}$ is divided by $5.$
		\begin{soln}
			By Fermat's Little Theorem, we have $8^4\equiv1\pmod{5}.$ Then $8^{388}\equiv(8^4)^{97}\equiv1^{97}\equiv1\pmod{5}.$ Finally $8^{391}\equiv8^3\cdot(8^4)^{97}\equiv8^3\cdot1\equiv512\equiv\boxed{2\pmod{5}.}$
			
		\end{soln}

	
	\newpage
	
	\item[21.] (b) Let $n=d_kd_{k-1}\cdots d_2d_1d_0$ be the decimal representation of $n.$ Show that $11|n$ if and only if 11 divides $(d_0-d_1+d_2-d_3+\cdots\pm d_k).$
		\begin{proof}
			Rewrite $n$ as \[n = 10^kd_k + 10^{k-1}d_{k-1}+\cdots+10^2d_2+10^1d_1+10^0d_0.\] We have $10\equiv-1\pmod{11},$ which means $10^k\equiv-1\pmod{11}$ for odd $k$ and $10^k\equiv1\pmod{11}$ for even $k.$ Then taking $n\pmod{11},$ we have 
			\begin{align*}
				n\pmod{11} &\equiv 10^0d_0+10^1d_1+\cdots+10^kd_k\pmod{11} \\
				&\equiv d_0-d_2+\cdots\pm d_k\pmod{11}
			\end{align*} where the sign of $d_k$ depends on if $k$ is even or odd.

			If $11|n,$ then $n\in[0]_{11}$ and since $n\equiv (d_0-d_1+\cdots\pm d_k)\pmod{11}$ it follows that $(d_0-d_1+\cdots\pm d_k)\in[0]_{11}$ as well, so $11|(d_0-d_1+\cdots\pm d_k),$ as desired. The converse follows similarly.

		\end{proof}

	\item[28.] Find $x\in\ZZ$ such that $x\equiv8\pmod{10}, x\equiv3\pmod{9}, x\equiv2\pmod{7}.$
		\begin{soln}
			By the Euclidean Algorithm, we have 
			\begin{align*}
				9 &= 1(7) + 2 \\
				7 &= 3(2) + 1 \\
			\end{align*} so we may write 
			\begin{align*}
				1 &= 7 - 3(2) \\
				&= 7 - 3(9 - 7) \\
				&= 4\cdot7 - 3\cdot 9.
			\end{align*} Then \[x\equiv 3(4\cdot 7) - 2(3\cdot 9)\pmod{7\cdot9}\equiv30\pmod{63}\] is a class of solutions to the two equivalences $x\equiv3\pmod{9}$ and $x\equiv2\pmod{7}$ since 7 and 9 are relatively prime. 

			Next, using the Euclidean Algorithm between 63 and 10, we have 
			\begin{align*}
				63 &= 6(10) + 3 \\
				10 &= 3(3) + 1
			\end{align*} so we may write 
			\begin{align*}
				1 &= 10 - 3(3) \\
				&= 10 - 3(63 - 6(10)) \\
				&= 19\cdot10-3\cdot63.
			\end{align*} Then 
			\begin{align*}
				x &\equiv 30(19\cdot10)-8(3\cdot63)\pmod{10\cdot63} \\
				&\equiv 4188 \pmod {630} \\
				&\equiv \boxed{408\pmod{630}}
			\end{align*} is a class of solutions to all three equivalences simultaneously. 
		\end{soln}

	\item[31.] Show that the following conditions on an integer $n\ge2$ are equivalent.

		(1) If $\bar{a}\in\ZZ_n,$ then either $\bar{a}$ is invertible or $\bar{a}^k=\bar{0}$ for some $k\ge 1.$

		(2) $n$ is a power of a prime.
		\begin{proof}
			We need to prove both $(1)\implies(2)$ and $(2)\implies(1).$ 

			Assume (2). If $n$ is a power of a prime, write $n=p^i$ for $i>1.$ If $\bar{a}$ and $n$ are not relatively prime, then it must be that $\bar{a}$ is of the form $bp^j$ for some $b\in\ZZ$ and $j>1.$ Then $\bar{a}^i=(bp^j)^i=b^i(p^i)^j=b^i\cdot \bar{0}^j = \bar{0}.$ On the other hand, if $\gcd(\bar{a}, n)=1,$ by Theorem 5, $\bar{a}$ has an inverse in $\ZZ_n,$ as desired.

			Assume (1). Then for all $\bar{a}\in\ZZ_n,$ either $\bar{a}$ is invertible or $\bar{a}^k=\bar{0}$ for some $k\ge1.$ Let $n$ be of the form $bp^i$ where $p$ is some prime that divides $n,$ and $b$ is a nonzero integer that is not divisible by $p.$  Consider $\bar{a}=b.$ Assume that $\gcd(b, bp^i)=b\neq 1.$ Then $b$ does not have an inverse in $\ZZ_n.$ However, since $\gcd(b, p)=1,$ it follows that $\gcd(b^k, p^i)=1$ as well, so $bp^i$ will never divide $b^k$ for any $k.$ Thus $b^k\neq \bar{0},$ which contradicts our assumption of (1). Thus, $\gcd(b, bp^i)=b=1,$ thus $n=bp^i=p^i$ is a power of a prime, as desired.
			
		\end{proof}

\end{itemize}

\end{document}

\documentclass{article}
\usepackage[sexy, hdr, fancy]{evan}
\setlength{\droptitle}{-4em}

\lhead{Homework 4}
\rhead{Advanced Algebra I}
\lfoot{}
\cfoot{\thepage}

\begin{document}
\title{Homework 4}
\maketitle
\thispagestyle{fancy}

\section*{Section 2.2: Groups}
\begin{itemize}
	\item[13.] If $G$ is any group, define $\alpha:G\to G$ by $\alpha(g)=g^{-1}.$ Show that $\alpha$ is injective and surjective.
		\begin{proof}
			To show $\alpha$ is injective, consider $g_1$ and $g_2$ such that $\alpha(g_1)=\alpha(g_2).$ Then $g_1^{-1}=g_2^{-1},$ and left multiplying by $g_2g_1,$ we have 
			\begin{align*}
				g_2g_1g_1^{-1} &= g_2g_1g_2^{-1} \\
				g_2 &= g_2g_1g_2^{-1} \\
				g_2 g_2 &= g_2g_1g_2^{-1}g_2 \\
				g_2 g_2 &= g_2 g_1 
			\end{align*} and by the cancellation law, we have $g_2=g_1,$ so $\alpha$ is injective, as desired.

			To show $\alpha$ is surjective, we must show that for all $g\in G,$ there exists a $g_0\in G$ such that $\alpha(g_0)=g.$ Since $gg^{-1}=1$ it follows that $(g^{-1})^{-1}=g,$ so then $g_0=g^{0-1}$ will satisfy this, and since $G$ is a group, every element has an inverse, so $\alpha$ is surjective, as desired.
			
		\end{proof}
		
\end{itemize}

\section*{Section 2.3: Subgroups}
\begin{itemize}
	\item[2.] If $H$ is a subset of a group $G,$ show that $H$ is a subgroup if and only if $H$ is nonempty and $ab^{-1}\in H$ whenever $a\in H$ and $b\in H.$

	\item[5.] 
		\begin{enumerate}[(a)]
			\item If $G$ is an abelian group, show that $H=\left\{ a\in G | a^2=1 \right\}$ is a subgroup of $G.$

			\item Give an example where $H$ is not a subgroup.
				
		\end{enumerate}

	\item[8.] If $X$ is a nonempty subset of a group $G,$ let $\left< X\right>$ be the set of all products of powers of elements of $X;$ more formally \[\left< X\right>=\left\{ x_1^{k_1}x_2^{k_2}\cdots x_m^{k_m}\, |\, m\ge 1, x_i\in X \right\}\]
		\begin{enumerate}[(a)]
			\item Show that $\left< X\right>$ is a subgroup of $G$ that contains $X.$

			\item Show that $\left< X\right>\subseteq H$ for every subgroup $H$ such that $X\subseteq H.$ Thus, $\left< X\right>$ is the \textit{smallest} subgroup of $G$ that contains $X,$ and is called the \textbf{subgroup generated} by $X.$
				
		\end{enumerate}

	\item[13.] 
		\begin{enumerate}[(a)]
			\item If $G$ is a group, show that $\left\{ (g, g) | g\in G \right\}$ is a subgroup of $G\times G.$

			\item Determine the groups $G$ such that $\left\{ (g, g^{-1})|g\in G \right\}$ is a subgroup of $G\times G.$
				
		\end{enumerate}

	\item[22.] Find $Z[GL_2(\RR)].$
		
\end{itemize}

\section*{Section 2.4: Cyclic Groups and the Order of an Element}
\begin{itemize}
	\item[6.] If $G$ is a group and $g\in G,$ show that $\left< g\right>=\left< g^{-1}\right>.$

	\item[7.] Let $o(g)=20$ in a group $G.$ Compute
		\begin{enumerate}[(a)]
			\item $o(g^2)$ 
				\begin{answer*}
					Since $o(g)=20,$ that means $g^{20}=1.$ Then $(g^2)^{10}=g^{20}=1,$ so $o(g^2)=\boxed{10.}$
				\end{answer*}

			\item $o(g^8)$
				\begin{answer*}
					Since $o(g)=20,$ that means $g^{20}=1.$ Then $(g^8)^5=g^{40}=(g^{20})^2=1,$ so $o(g^8)=\boxed{5.}.$
				\end{answer*}

			\item $o(g^5)$
				\begin{answer*}
					Since $o(g)=20,$ that means $g^{20}=1.$ Then $(g^5)^4=g^{20}=1,$ so $o(g^5)=\boxed{4.}$
				\end{answer*}

			\item $o(g^3)$
				\begin{answer*}
					Since $o(g)=20,$ that means $g^{20}=1.$ Then $(g^3)^{20}=(g^{20})^3=1,$ so $o(g^3)=\boxed{20.}$
				\end{answer*}
				
		\end{enumerate}

	\item[10.] 
		\begin{enumerate}[(a)]
			\item If $gh=hg$ in a group and $o(g)$ and $o(h)$ are finite, show that $o(gh)$ is finite.

			\item Show that (a) fails if $gh\neq hg$ by considering $\begin{bmatrix}
					0 & -1 \\
					1 & 0
				\end{bmatrix}$ and $\begin{bmatrix}
					0 & 1 \\
					-1 & -1
				\end{bmatrix}.$
				
		\end{enumerate}

	\item[18.] If $G=\left< g\right>$ and $H=\left< h\right>,$ show that $G\times H=\left< (g, 1), (1, h)\right>$
		
\end{itemize}

\end{document}

\documentclass{article}
\usepackage[sexy, hdr, fancy]{evan}
\setlength{\droptitle}{-4em}

\lhead{Homework 4}
\rhead{Advanced Algebra I}
\lfoot{}
\cfoot{\thepage}

\begin{document}
\title{Homework 4}
\maketitle
\thispagestyle{fancy}

\section*{Section 2.2: Groups}
\begin{itemize}
	\item[13.] If $G$ is any group, define $\alpha:G\to G$ by $\alpha(g)=g^{-1}.$ Show that $\alpha$ is injective and surjective.
		\begin{proof}
			To show $\alpha$ is injective, consider $g_1$ and $g_2$ such that $\alpha(g_1)=\alpha(g_2).$ Then $g_1^{-1}=g_2^{-1},$ and left multiplying by $g_2g_1,$ we have 
			\begin{align*}
				g_2g_1g_1^{-1} &= g_2g_1g_2^{-1} \\
				g_2 &= g_2g_1g_2^{-1} \\
				g_2 g_2 &= g_2g_1g_2^{-1}g_2 \\
				g_2 g_2 &= g_2 g_1 
			\end{align*} and by the cancellation law, we have $g_2=g_1,$ so $\alpha$ is injective, as desired.

			To show $\alpha$ is surjective, we must show that for all $g\in G,$ there exists a $g_0\in G$ such that $\alpha(g_0)=g.$ Since $gg^{-1}=1$ it follows that $(g^{-1})^{-1}=g,$ so then $g_0=g^{0-1}$ will satisfy this, and since $G$ is a group, every element has an inverse, so $\alpha$ is surjective, as desired.
			
		\end{proof}
		
\end{itemize}

\section*{Section 2.3: Subgroups}
\begin{itemize}
	\item[2.] If $H$ is a subset of a group $G,$ show that $H$ is a subgroup if and only if $H$ is nonempty and $ab^{-1}\in H$ whenever $a\in H$ and $b\in H.$
		\begin{proof}
			If $H$ is a subgroup, then $H$ must contain at least $1\in G,$ so $H$ is nonempty. Then if $a, b\in H,$ we have $b\inv\in H$ so $ab\inv\in H$ since $H$ is a subgroup, as desired.

			For the other direction, if $H$ is nonempty, then suppose it contains at least 1 element. If it has 1 element, say $a,$ then $aa\inv=1\in H$ which is the trivial subgroup. On the other hand, if $H$ contains more than 1 element, for $a, b\in H,$ we have $ab\inv\in H.$ Since we know $1\in H$ it follows that if $b\in H$ then $1\cdot b\inv=b\inv\in H$ is the inverse of $a$ which is also contained in $H.$ Thus since $a, b\inv\in H,$ we have $a(b\inv)\inv=ab\in H.$ Thus $H$ is a subgroup, as desired.
			
		\end{proof}

		\newpage
	\item[5.] 
		\begin{enumerate}[(a)]
			\item If $G$ is an abelian group, show that $H=\left\{ a\in G | a^2=1 \right\}$ is a subgroup of $G.$
				\begin{proof}
					Clearly $1\cdot1=1$ so $1\in H.$ Then consider $a, b\in H$ so that $a^2=b^2=1.$ Then $aabb=1$ and since $G$ is abelian, we have $(ab)(ab)=1,$ so $(ab)^2$ so $ab\in H$ as well. Finally, if $a\in H$ then $a^2=1$ so $a=a\inv,$ thus $(a\inv)^2=1$ so $a\inv\in H.$ Thus $H$ is a subgroup as desired.

				\end{proof}

			\item Give an example where $H$ is not a subgroup.
				
				\begin{soln}
					When $G$ is not abelian, then $H$ is not necessarily a subgroup. For example, consider the group $S_3=\left\{ \varepsilon, \sigma, \sigma^2, \tau, \tau\sigma, \tau\sigma^2 \right\}$ where $H$ is then $\left\{ \varepsilon, \tau, \tau\sigma, \tau\sigma^2 \right\}.$ However, we have $\tau(\tau\sigma)=\sigma\not\in H$ so $H$ is not a subgroup.

				\end{soln}
		\end{enumerate}

	\item[8.] If $X$ is a nonempty subset of a group $G,$ let $\left< X\right>$ be the set of all products of powers of elements of $X;$ more formally \[\left< X\right>=\left\{ x_1^{k_1}x_2^{k_2}\cdots x_m^{k_m}\, |\, m\ge 1, x_i\in X \right\}\]
		\begin{enumerate}[(a)]
			\item Show that $\left< X\right>$ is a subgroup of $G$ that contains $X.$
				\begin{proof}
					We have $1\in\left< X\right>$ if we take all the $k_i=0.$ Next, if two elements 
					\begin{align*}
						a &= x_1^{k_1}x_2^{k_2}\cdots x_m^{k_m} \\
						b &= x_1^{i_1}x_2^{i_2}\cdots x_m^{i_m}
					\end{align*} are in $\left< X\right>,$ then we have 
					\begin{align*}x_1^{i_1}x_2^{i_2}\cdots x_m^{i_m}
						ab &= \left(x_1^{k_1}x_2^{k_2}\cdots x_m^{k_m}\right)\left(  x_1^{i_1}x_2^{i_2}\cdots x_m^{i_m}\right) \\
						&= x_1^{k_1+i_1}x_2^{k_2+i_2}\cdots x_m^{k_m+i_m} \\
						&= x_1^{n_1}x_2^{n_2}\cdots x_m^{n_m}
					\end{align*} which is in $\left< X\right>$ as well. Then if $a$ is as above, its inverse is given by \[a\inv = x_1^{-k_1}x_2^{-k_2}\cdots x_m^{-k_m}\] which in particular is in $\left< X\right>$ as well. Thus $\left< X\right>$ is a subgroup of $G,$ and it contains $X$ because for each $x_i$ in $X,$ we can represent $x_i$ with $k_i=1$ and all other $k_j=0.$
					
				\end{proof}

			\item Show that $\left< X\right>\subseteq H$ for every subgroup $H$ such that $X\subseteq H.$ Thus, $\left< X\right>$ is the \textit{smallest} subgroup of $G$ that contains $X,$ and is called the \textbf{subgroup generated} by $X.$
				\begin{proof}
					Let $X=\left\{ x_1, x_2, \cdots, x_m \right\}\subset H.$ Then since each of $x_i\in H,$ it must be that $x_i^2\in H$ and by induction $x_i^{k_i}\in H$ for any $k_i.$ Thus since each of $x_1^{k_1}, x_2^{k_2}, \cdots, x_m^{k_m}\in H,$ it must be that their product $x_1^{k_1}x_2^{k_2}\cdots x_m^{k_m}\in H$ as well for all $k_i.$ This is exactly $\left< X\right>,$ so it follows that $\left< X\right>\subseteq H,$ with equality when $X=H.$
					
				\end{proof}
				
		\end{enumerate}
	\item[13.] 
		\begin{enumerate}[(a)]
			\item If $G$ is a group, show that $H=\left\{ (g, g) | g\in G \right\}$ is a subgroup of $G\times G.$
				\begin{proof}
					Since $1\in G$ is the identity, we have $g\cdot 1=g,$ so $(g, f)\cdot(1, 1)=(g, f)$ for all $(g, f)\in G\times G$ thus $(1, 1)$ is the identity in $G\times G$ as well and is in $H.$

					Since $G$ is a group, if $g, h\in G,$ we have $g+h\in G.$ Next, if $(g, g), (h, h)\in H,$ then $(g, g)\cdot(h, h)=(g+h, g+h)\in H$ as well. 

					Finally, since $G$ is a group, if $g\in G,$ its inverse $g\inv\in G$ as well. Thus if $(g, g)\in H,$ its inverse is given by $(g\inv, g\inv)\in H,$ so $H$ is a subgroup, as desired.
					
				\end{proof}

			\item Determine the groups $G$ such that $H=\left\{ (g, g^{-1})|g\in G \right\}$ is a subgroup of $G\times G.$
				\begin{soln}
					If $(g, g\inv), (f, f\inv)\in H,$ then the binary operation on them \[(g, g\inv)\cdot(f, f\inv)=(gf, g\inv f\inv)=(gf, (fg)\inv)\] must also be in $H$ in order for $H$ to be a subgroup. Therefore, we must have $gf=fg,$ so $H$ is a subgroup if and only if \boxed{G\text{ is abelian}.}
					
				\end{soln}
				
		\end{enumerate}

	\item[22.] Find $Z[GL_2(\RR)].$
		\begin{soln}
			Let $A=\begin{bmatrix}
				a & b \\ c & d
			\end{bmatrix}\in GL_2(\RR)$ be an arbitrary matrix, and let $Z=\begin{bmatrix}
				h & i \\ j & k
			\end{bmatrix}\in GL_2(\RR)$ be a center. Then we have
			\begin{align*}
				AZ &= \begin{bmatrix}
					a & b \\ c & d
				\end{bmatrix} \begin{bmatrix}
					h & i \\ j & k
				\end{bmatrix} = \begin{bmatrix}
					ah+bj & ai+bk \\ ch+dj & ci+dk
				\end{bmatrix} \\
				ZA &= \begin{bmatrix}
					h & i \\ j & k
				\end{bmatrix}\begin{bmatrix}
					a & b \\ c & d
				\end{bmatrix} = \begin{bmatrix}
					ah+ic & bh+id \\ aj+ck & bj+dk
				\end{bmatrix}
			\end{align*}

			If $AZ=ZA,$ then we must have $ah+bj=ah+ic\implies bj=ci$ for all $b, c\in\RR.$ Since we have no control over what $b$ and $c$ are, it must be that $i=j=0.$ Then we must have $ai+bk=bk=bh+id=bh$ and $ch+dj=ch=aj+ck=ck.$ Thus, we $h=k,$ and the group of centers is given by the general form
			\[ Z\left( GL_2(\RR) \right)=\boxed{ \left\{ \begin{bmatrix}
				h & 0 \\ 0 & h
		\end{bmatrix}\Bigg\vert\, h\in\RR\right\}}\]
		\end{soln}
		
\end{itemize}

\section*{Section 2.4: Cyclic Groups and the Order of an Element}
\begin{itemize}
	\item[6.] If $G$ is a group and $g\in G,$ show that $\left< g\right>=\left< g^{-1}\right>.$
		\begin{proof}
			For some $f\in\left< g\right>,$ we have $f=g^k$ for some $k\in\ZZ.$ Then $f=(g^{-k})\inv=(g\inv)^{-k}\in\left< g\inv\right>$ so it follows that $g\in \left< g\inv\right>,$ thus $\left< g\right>\subset\left< g\inv\right>.$

			Similarly, if $h\in\left< g\inv\right>,$ then $h=(g\inv)^n$ for some $n\in\ZZ.$ Then $h=g^{-n}$ so $h\in \left< g\right>,$ thus $\left< g\inv\right>\subset\left< g\right>,$ so in fact $\left< g\right>=\left< g\inv\right>,$ as desired.

		\end{proof}

		\newpage
	\item[7.] Let $o(g)=20$ in a group $G.$ Compute
		\begin{enumerate}[(a)]
			\item $o(g^2)$ 
				\begin{answer*}
					Since $o(g)=20,$ that means $g^{20}=1.$ Then $(g^2)^{10}=g^{20}=1,$ so $o(g^2)=\boxed{10.}$
				\end{answer*}

			\item $o(g^8)$
				\begin{answer*}
					Since $o(g)=20,$ that means $g^{20}=1.$ Then $(g^8)^5=g^{40}=(g^{20})^2=1,$ so $o(g^8)=\boxed{5.}.$
				\end{answer*}

			\item $o(g^5)$
				\begin{answer*}
					Since $o(g)=20,$ that means $g^{20}=1.$ Then $(g^5)^4=g^{20}=1,$ so $o(g^5)=\boxed{4.}$
				\end{answer*}

			\item $o(g^3)$
				\begin{answer*}
					Since $o(g)=20,$ that means $g^{20}=1.$ Then $(g^3)^{20}=(g^{20})^3=1,$ so $o(g^3)=\boxed{20.}$
				\end{answer*}
				
		\end{enumerate}

	\item[10.] 
		\begin{enumerate}[(a)]
			\item If $gh=hg$ in a group and $o(g)$ and $o(h)$ are finite, show that $o(gh)$ is finite.
				\begin{proof}
					Let $o(g)=n$ and $o(h)=m,$ so that $g^n=h^m=1.$ Then $(g^n)^m=1=(h^m)^n$ so the product $g^{mn}h^{mn}=1.$ Since $gh=hg,$ we have $(gh)^{mn}=1,$ thus $o(gh)$ must divide $mn.$ Since $mn$ is finite, it follows that $o(gh)$ must be finite too.
					
				\end{proof}

			\item Show that (a) fails if $gh\neq hg$ by considering $g=\begin{bmatrix}
					0 & -1 \\
					1 & 0
				\end{bmatrix}$ and $h=\begin{bmatrix}
					0 & 1 \\
					-1 & -1
				\end{bmatrix}.$
				\begin{proof}
					We have $g^4=I$ so $o(g)=4,$ and $h^3=I$ so $o(h)=3$ so $o(g)$ and $o(h)$ are both finite. Now, \[ gh= \begin{bmatrix}
							0 & -1 \\ 1 & 0
						\end{bmatrix}\begin{bmatrix}
							0 & 1 \\ -1 & -1
						\end{bmatrix}=\begin{bmatrix}
							1 & 1 \\ 0 & 1
					\end{bmatrix}\] and we claim that \[(gh)^n=\begin{bmatrix}
							1 & n \\
							0 & 1
					\end{bmatrix}\] for all $n\ge 1,$ which we show by induction. The base case $n=1$ is already given. Next, suppose this holds for arbitrary $k.$ Then \[(gh)^k (gh)=\begin{bmatrix}
							1 & k \\ 0 & 1
						\end{bmatrix}\begin{bmatrix}
							1 & 1 \\ 0 & 1
						\end{bmatrix} = \begin{bmatrix}
							1 & k+1 \\ 0 & 1
					\end{bmatrix}=(gh)^{k+1} \] 
					so the claim is proved. In particular, $n\neq 0$ for any $n\ge 1,$ so $(gh)^n\neq I$ for any $n,$ thus $o(gh)=\infty.$ 

				\end{proof}
				
		\end{enumerate}

	\item[18.] If $G=\left< g\right>$ and $H=\left< h\right>,$ show that $G\times H=\left< (g, 1), (1, h)\right>$
		\begin{proof}
			Consider some $(a, b)\in G\times H,$ so that $a\in\left< g\right>$ and $b\in\left< h\right>.$ That means $a=g^i$ and $b=h^j$ where $i, j\in \ZZ.$ Then the element $(g^i, h^j)$ can be represented as the product $(g^i, 1)\cdot(1, h^j)= \left[ (g, 1) \right]^i\left[ (1, h) \right]^j$ so \[(a, b)=(g^i, h^j)\in \left< (g, 1), (1, h)\right>\] and thus $G\times H\subset \left< (g, 1), (1, h)\right>.$

			Next, for some element in $\left< (g, 1), (1, h)\right>,$ we can express it in the form $(g, 1)^n\cdot (1, h)^m$ for some $n, m\in \ZZ.$ This is exactly \[(g, 1)^n\cdot(1, h)^m=(g^n, 1)\cdot(1, h^m)=(g^n, h^m)\] which is an element of the Cartesian product $\left< g\right>\times \left< h\right>.$ Thus $\left< (g, 1), (1, h)\right>\subset G\times H,$ so in fact $G\times H=\left< (g, 1), (1, h)\right>,$ as desired.
			
		\end{proof}
		
\end{itemize}

\end{document}

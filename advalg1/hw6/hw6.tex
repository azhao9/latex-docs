\documentclass{article}
\usepackage[sexy, hdr, fancy]{evan}
\setlength{\droptitle}{-4em}

\lhead{Homework 6}
\rhead{Advanced Algebra I}
\lfoot{}
\cfoot{\thepage}

\begin{document}
\title{Homework 6}
\maketitle
\thispagestyle{fancy}

\section*{Section 2.6: Cosets and Lagrange's Theorem}

\begin{itemize}
	\item[4.] If $K\subseteq H\subseteq G$ are finite groups, show that $\abs{G:K}=\abs{G:H}\cdot\abs{H:K}.$
		\begin{proof}
			For finite groups, we have $\abs{G:K}=\abs{G}/\abs{K}$ and similarly for the other two, so we have \[\frac{\abs{G}}{\abs{K}}=\frac{\abs{G}}{\abs{H}}\cdot\frac{\abs{H}}{\abs{K}}\] as desired.
			
		\end{proof}

	\item[15.] If $H$ and $K$ are subgroups of a group and $\abs{H}$ is prime, show that either $H\subseteq K$ or $H\cap K=\left\{ 1 \right\}.$

	\item[27.] Is $D_5\times C_3\cong D_3\times C_5?$ Prove your answer.
		
\end{itemize}

\section*{Section 2.8: Normal Subgroups}

\begin{itemize}
	\item[4.] If $D_4=\left\{ 1, a, a^2, a^3, b, ba, ba^2, ba^3 \right\},K=\left\{ 1, b \right\}$ and $H=\left\{ 1, a^2, b, ba^2\right\}$ show that $K\unlhd H\unlhd D_4,$ but $K\not\unlhd D_4.$
		\begin{proof}
			Since $\abs{H:K}=2,$ by section 2.8 theorem 4, $K$ is normal in $H.$ Similarly, $\abs{D_4:H}=2,$ so $H$ is normal in $D_4.$ However, we have $aK=\left\{ a, ab \right\}\neq \left\{ a, ba \right\}=Ka$ since $ab\neq ba.$
			
		\end{proof}

	\item[11.] Let $p$ and $q$ be distinct primes. If $G$ is a group of order $pq$ that has a unique subgroup of order $p$ and a unique subgroup of order $q,$ show that $G$ is cyclic.

	\item[16.] Show that $\Inn G\unlhd \Aut G$ for any group $G.$

	\item[25.] If $X$ is a nonempty subset of a group $G,$ define the \textbf{normalizer} $N(X)$ of $X$ by \[N(X)=\Set{a\in G}{aXa\inv=X}.\]
		\begin{enumerate}[(a)]
			\item Show that $N(X)$ is a subgroup of $G.$
				\begin{proof}
					Clearly $1_GX1_G\inv=X,$ so $1_G\in N(X).$ Then if $a, b\in N(X),$ we have 
					\begin{align*}
						aXa\inv &= X \\
						bXb\inv &= X \\
						\implies a(bXb\inv)a\inv &= X \\
						\implies (ab)X(ba)\inv &= X
					\end{align*} so $ab\in N(X).$ Then if $a\in N(X),$ we have 
					\begin{align*}
						aXa\inv &= X \\
						aX &= Xa \\
						X &= a\inv Xa
					\end{align*} so $a\inv\in N(X)$ as well. Thus, $N(X)$ is a subgroup of $G,$ as desired.
					
				\end{proof}

			\item If $H$ is a subgroup of $G,$ show that $H\unlhd N(H).$

			\item If $H$ is a subgroup of $G,$ show that $N(H)$ is the largest subgroup of $G$ in which $H$ is normal. That is, if $H\unlhd K,$ and $K$ if a subgroup of $G,$ then $K\subseteq N\left( H \right).$
				
		\end{enumerate}
		
\end{itemize}

\section*{Section 2.10: The Isomorphism Theorem}

\begin{itemize}
	\item[7.] If $\alpha:G\to G_1$ is a group homomorphism and both $\alpha(G)$ and $\ker \alpha$ are finitely generated, show that $G$ is finitely generated.

	\item[9.] If $K=\left\{ \varepsilon, (12)(34), (13)(24), (14)(23) \right\},$ is there a group homomorphism $\alpha:S_4\to A_4$ with $\ker \alpha=K?$

	\item[21.] Show that $\CC^*/\CC^0\cong \RR^+$ where $\CC^0=\Set{z}{\abs{z}=1}$ is the circle group.
		\begin{proof}
			Define the homomorphism $\varphi:\CC^*\to\RR^+$ where $\varphi(z)=\abs{z}.$ This is indeed a homomorphism because $\varphi(z_1z_2)=\abs{z_1z_2}=\abs{z_1}\abs{z_2}=\varphi(z_1)\varphi(z_2).$ 
			
			Then the kernel of $\varphi$ is the set $\Set{z}{\varphi(z)=1}$ which is exactly $\CC^0.$ Finally, $\varphi``(\CC^*)=\RR^+$ since invertible elements in $\CC$ are all except 0, whose magnitudes are all positive. 

			Thus, by the first Isomorphism Theorem, since $\CC^0$ is the kernel of a homomorphism, $\CC^*/\CC^0\cong \RR^+,$ as desired.
			
		\end{proof}
		
\end{itemize}

\end{document}

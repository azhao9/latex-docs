\documentclass{article}
\usepackage[sexy, hdr, fancy]{evan}
\setlength{\droptitle}{-4em}

\lhead{Homework 1}
\rhead{Advanced Algebra I}
\lfoot{}
\cfoot{\thepage}

\begin{document}
\title{Homework 1}
\maketitle
\thispagestyle{fancy}

\section*{Section 0.1: Proofs}
\begin{itemize}
	\item[2.] (a) Prove by cases or provide a counterexample: If $n$ is any integer, then $n^2=4k+1$ for some integer $k.$
		\begin{proof}
			This is a false statement. Let $n=2m$ for $m\in\ZZ.$ Then $n^2=4m^2\neq 4k+1$ for any $k\in\ZZ.$

		\end{proof}

	\item[3.]  (c) Prove by contradiction and prove the converse or provide a counterexample: If $a$ and $b$ are positive numbers and $a\le b,$ then $\sqrt{a}\le \sqrt{b}.$
		\begin{proof}
			Assume the opposite, that $\sqrt{a}>\sqrt{b}.$ Since square roots are non-negative, we may square both sides to get $a>b,$ but we know $a\le b,$ which is a contradiction. Thus $\sqrt{a}\le\sqrt{b},$ as desired. 

			For the converse, we aim to show that if $\sqrt{a}\le\sqrt{b},$ then $a\le b.$ Since square roots are non-negative, square both sides to get $a\le b,$ as desired.

		\end{proof}

	\item[6.] If $p$ is a statement, let $\sim p$ denote the statement ``not $p,$'' called the negation of $p.$ Thus, $\sim p$ is true when $p$ is false, and false when $p$ is true. Show that if $\sim q\implies\sim p,$ then $p\implies q.$
		\begin{proof}
			For two statements $a$ and $b$ we have $a\implies b$ is always a true statement except in the case where $a$ is true and $b$ is false. If $\sim q\implies\sim p$ is true then there are three distinct possibilities:
			\begin{enumerate}
				\ii $\sim q$ is true and $\sim p$ is true. 
				\ii $\sim q$ is false and $\sim p$ is true. 
				\ii $\sim q$ is false and $\sim p$ is false.
			\end{enumerate} We tackle these by cases. 

			Case 1: If $\sim q$ and $\sim p$ are both true, then $q$ and $p$ are both false, so $p\implies q$ is a true statement.

			Case 2: If $\sim q$ is false and $\sim p$ is true, then $q$ is true and $p$ is false, so $p\implies q$ is a true statement.

			Case 3: If $\sim q$ and $\sim p$ are both false, then $q$ and $p$ are both true, so $p\implies q$ is a true statement.

			Thus $\sim q\implies \sim p$ implies that $p\implies q,$ as desired.

		\end{proof}
\end{itemize}

\section*{Section 0.3: Mappings}
\begin{itemize}
	\item[5.]  (a) For $A\xrightarrow{\alpha}A,$ show that $\alpha^2=\alpha$ if and only if $\alpha(x)=x$ for all $x\in\alpha(A).$
		\begin{proof}
			If $\alpha(x)=x$ for all $x\in \alpha(A),$ then $(\alpha\circ\alpha)(x)=x=\alpha(x),$ so $\alpha^2=\alpha,$ as desired.

			For the other direction, if $\alpha^2=\alpha,$ then $(\alpha\circ\alpha)(x)=\alpha(\alpha(x))=\alpha(x).$ Let $y=\alpha(x)\in\alpha(A),$ thus $\alpha(y)=y$ for all $y\in\alpha(A),$ as desired. 

		\end{proof}

		(b) If $A\xrightarrow{\alpha}A$ satisfies $\alpha^2=\alpha,$ show that $\alpha$ is surjective if and only if $\alpha$ is injective. Describe $\alpha$ in this case.
			\begin{proof}
				If $\alpha$ is surjective, then for all $a'\in A,$ there exists $a\in A$ such that $\alpha(a)=a'.$ We know that $\alpha(\alpha(a))=\alpha(a)$ for all $a\in A,$ so we can rewrite this as $\alpha(a')=a'$ for all $a'\in A.$ Then if we have $\alpha(a')=\alpha(a^*),$ that means $a'=a^*,$ so $\alpha$ is injective, as desired.

			If $\alpha$ is injective, then whenever $\alpha(a')=\alpha(a^*),$ it must be true that $a'=a^*.$ Assume that $\alpha``(A)$ the image of $A$ under $\alpha$ is not all of $A.$ Since not all elements are mapped to, there must exist some element $a\in A$ that is the image of more than a single element in $A.$ This is a contradiction since we know $\alpha$ is injective, that distinct elements of $A$ map to distinct elements of $A.$ Thus $\alpha``(A)$ is in fact all of $A.$ We know that $\alpha(\alpha(a))=\alpha(a)$ for all $a\in A.$ Let $y=\alpha(a)\in \alpha``(A),$ then we $\alpha(y)=y.$ Since $\alpha``(A)=A,$ it follows that $\alpha$ is surjective since for any $y\in A,$ we know $\alpha(y)=y,$ as desired.

			In this case, we have $\alpha(a)=a$ for all $a\in A.$
			
			\end{proof}

			(c) Let $A\xrightarrow{\beta}B\xrightarrow{\gamma}A$ satisfy $\gamma\beta=1_A.$ If $\alpha=\beta\gamma,$ show that $\alpha^2=\alpha.$
			\begin{proof}
				We have
				\begin{align*}
					\alpha^2 &= (\beta\gamma)(\beta\gamma) = \beta(\gamma\beta)\gamma \\
					&= \beta\circ1_A\circ\gamma=\beta\gamma=\alpha,
				\end{align*} as desired.
			\end{proof}

	\item[8.] Let $A\xrightarrow{\alpha}B\xrightarrow{\beta}A$ satisfy $\beta\alpha=1_A.$ If either $\alpha$ is surjective or $\beta$ is injective, show that each of them is invertible and that each of them is the inverse of the other.
			\begin{proof}
				We wish to show that $\alpha\beta=1_B,$ which combined with the fact that $\beta\alpha=1_A$ will show that $\alpha$ and $\beta$ are inverses of each other. 

				If $\alpha$ is surjective, then for any $b\in B$ there exists $a\in A$ such that $\alpha(a)=b.$ Since $\beta(\alpha(a))=a,$ that means $\alpha(\beta(\alpha(a)))=\alpha(a), \forall a\in A$ since $\alpha$ is well-defined. Then we let $\alpha(a)=b^*\in B,$ so that $\alpha(\beta(b^*))=b^*.$ Because $\alpha$ is surjective, any value of $b^*$ must have at least a single $a\in A$ such that $\alpha(a)=b^*,$ thus $\alpha(\beta(b^*))=b^*$ holds for all $b^*\in B,$ so $\alpha\beta=1_B,$ as desired.

				We know that $\beta(\alpha(a))=a$ for all $a\in A,$ which means that the image $\beta``(B)$ is actually all of $A,$ so then $\beta$ is surjective as well, and therefore bijective if $\beta$ is also injective. It is a theorem that a bijective mapping always has an inverse, so denote $\beta^{-1}$ to be the inverse of $\beta.$ Then we have $\beta^{-1}(\beta(\alpha(a)))=(\beta^{-1}\circ\beta)(\alpha(a))=\alpha(a)=\beta^{-1}(a)$ for all $a\in A,$ thus $a=\beta^{-1},$ so $a\beta=1_B,$ as desired. 
											
			\end{proof}
\end{itemize}

\section*{Section 0.4: Equivalences}
\begin{itemize}
	\item[1.] In each case, decide whether the relation $\equiv$ is an equivalence on $A.$ If it is, describe the equivalence classes.
		
		(e) $A=\NN; a\equiv b$ means that $b=ka$ for some integer $k.$
			\begin{soln}
				$\equiv$ is not an equivalence because it is not symmetric: \[a\equiv b\implies b=ka\] but then $a=\frac{1}{k}b,$ so $b\not\equiv a,$ since $\frac{1}{k}\notin\NN.$

			\end{soln}

		(h) $A=\RR\times\RR; (x, y) \equiv(x_1, y_1)$ means that $x^2+y^2=x_1^2+y_1^2.$
			\begin{soln}
				$\equiv$ is an equivalence relation because it satisfies:
				\begin{enumerate}
					\ii Reflexivity: $(x, y)\equiv(x, y)$ since $x^2+y^2=x^2+y^2.$

					\ii Symmetry: $(x, y)\equiv(x_1, y_1)$ means $x^2+y^2=x_1^2+y_1^2,$ so then $(x_1, y_1)\equiv(x, y)$ since $x_1^2+y_1^2=x^2+y^2.$

					\ii Transitivity: If $(x, y)\equiv(a, b),$ and $(a, b)\equiv(p, q),$ then $x^2+y^2=a^2+b^2$ and $a^2+b^2=p^2+q^2,$ so then $x^2+y^2=p^2+p^2,$ so $(x, y)\equiv(p, q).$
				\end{enumerate}

				The equivalence classes are concentric circles around the origin.

			\end{soln}

	\item[3.] (d) $A=\RR^+\times\RR^+; (x, y)\equiv(x_1, y_1)$ means that $y/x = y_1/x_1; B=\{x\in\RR\,\vert\, x>0\}.$ Show that $\equiv$ is an equivalence on $A$ and find a (well-defined) bijection $\sigma: A_{\equiv}\to B.$
			\begin{proof}
				$\equiv$ is an equivalence relation because it satisfies:
				\begin{enumerate}
					\ii Reflexivity: $(x, y)\equiv(x, y)$ since $y/x=y/x.$

					\ii Symmetry: $(x, y)\equiv(x_1, y_1)$ means $y/x=y_1/x^1,$ so then $(x_1, y_1)\equiv(x, y)$ since $y_1/x_1=y/x.$

					\ii Transitivity: If $(x, y)\equiv(a, b),$ and $(a, b)\equiv(p, q),$ then $y/x=b/a$ and $b/a=q/p,$ so then $y/x=q/p$ so $(x, y)\equiv(p, q).$
				\end{enumerate}

				The equivalence classes are the lines $y=kx$ for $k\in\RR^+,$ so $\sigma$ can map these lines to the value $k$ of their slope. This is well-defined and bijective because each line only gets mapped to a single value corresponding to its slope, and the slope uniquely determines the line. Specifically, $\sigma([x, y])=y/x$ is the mapping. 

			\end{proof}

	\item[6.] Let $\equiv$ and $\sim$ be two equivalences on the same set $A.$

		(a) If $a\equiv a_1$ implies that $a\sim a_1,$ show that each $\sim$ equivalence class is partitioned by the $\equiv$ equivalence classes it contains.
			\begin{proof}
				Since $\equiv$ equivalence classes already partition $A,$ it suffices to prove that any equivalence class $[a]$ under $\equiv$ cannot be part of more than equivalence class under $\sim.$ Let $[a]_{\equiv}$ represent an equivalence class generated by $a$ under $\equiv,$ and similarly for $[a]_{\sim}.$ 
			
				If $[a]_\equiv$ only has a single element $a,$ then it can only be part of a single equivalence class under $\sim.$ Otherwise, consider two elements $x, y\in[a]_\equiv.$ That means $x\equiv a \implies x\sim a,$ and $y\equiv a\implies y\sim a.$ That means $x\in[a]_\sim$ and $y\in[a]_\sim,$ so any two elements in an equivalence class under $\equiv$ are always in the same equivalence class under $\sim.$ Thus any $[a]_\equiv$ is fully contained within $[a]_\sim,$ so $\equiv$ equivalence classes partition the $\sim$ equivalence classes, as desired.

			\end{proof}

		(b) Define $\cong$ on $A$ by writing $a\cong a_1$ if an only if both $a\equiv a_1$ and $a\sim a_1.$ Show that $\cong$ is an equivalence and describe the $\cong$ equivalence classes in terms of the $\equiv$ and $\sim$ equivalence classes.
			\begin{proof}
				We show that $\cong$ is an equivalence relation because it satisfies:
				\begin{enumerate}
					\ii Reflexivity: $a\cong a$ because $a\equiv a$ and $a\sim a$ since $\equiv$ and $\sim$ are equivalence relations and therefore reflexive.

				\ii Symmetry: If $a\cong b,$ that means $a\equiv b$ and $a\sim b,$ so then since $\equiv$ and $\sim$ are equivalence relations and therefore symmetric, we have $b\equiv a$ and $b\sim a,$ thus $b\cong a.$

				\ii Transitivity: If $a\cong b$ and $b\cong c,$ that means $a\equiv b, a\sim b,$ and $b\equiv c, b\sim c,$ so since $\equiv$ and $\sim$ are equivalence relations and therefore transitive, we have $a\equiv c, a\sim b,$ thus $a\cong c.$
				\end{enumerate}

				A $\cong$ equivalence class is the set of elements such that all elements in the set are equivalent under both $\equiv$ and $\sim.$

			\end{proof}

	\item[7.] In each case, determine whether $\alpha:\QQ^+\to\QQ$ is well defined.

		(a) $\alpha\left( \frac{n}{m} \right) = n$
			\begin{soln}
				This is \boxed{\text{not well defined}} since $\alpha(1/2)=1$ and $\alpha(2/4)=2,$ but $1/2=2/4.$

			\end{soln}

		(b) $\alpha\left( \frac{n}{m} \right)=\frac{n-m}{n+m}$
			\begin{soln}
				This is \boxed{\text{well defined}.} We wish to show that if $n/m=a/b$ then $\alpha\left(\frac{n}{m}\right)=\frac{n-m}{n+m}=\frac{a-b}{a+b}=\alpha\left( \frac{a}{b} \right).$ Cross multiplying, this is equivalent to 
				\begin{align*}
					(n-m)(a+b) &= (n+m)(a-b) \\
					an-am+bn-bm &= an+am-bn-bm \\
					2bn &= 2am \\
					\frac{n}{m} &= \frac{a}{b},
				\end{align*} which is true, as desired.

			\end{soln}

		(c) $\alpha\left( \frac{n}{m} \right)=m+n$
			\begin{soln}
				This is \boxed{\text{not well defined}} since $\alpha(1/2)=1+2=3$ but $\alpha(2/4)=2+4=6.$

			\end{soln}

		(d) $\alpha\left( \frac{n}{m} \right)=\frac{5m+7n}{3n+m}$
			\begin{soln}
				This is \boxed{\text{well defined}.} Similarly to part (b), we wish to show that if $n/m=a/b,$ then $\alpha\left( \frac{n}{m} \right) = \frac{5m+7n}{3n+m} = \frac{5b+7a}{3a+b}=\alpha\left( \frac{a}{b} \right).$ Cross multiplying, this is equivalent to
				\begin{align*}
					(5m+7n)(3a+b) &= (5b+7a)(3n+m) \\
					15am+5bm+21an+7bn &= 15bn+5bm+21an+7am \\
					8am &= 8bn \\
					\frac{a}{b} &= \frac{n}{m}
				\end{align*} which is true, as desired.
			\end{soln}

	\item[9.] For a mapping $\alpha:A\to B,$ let $\equiv$ denote the kernel equivalence of $\alpha$ and let $\varphi:A\to A_{\equiv}$ denote the natural mapping. Define $\sigma:A_{\equiv}\to B$ by $\sigma([a])=\alpha(a)$ for all equivalence classes $[a]$ in $A_\equiv.$
		
		(a) Show that $\sigma$ is well defined and injective, surjective if $\alpha$ is surjective.
		\begin{proof}
			$\sigma$ is well defined if and only if $\sigma([a])=\sigma([b])$ whenever $[a]=[b];$ this just means an element in the pre-image only gets mapped to a single unique element in the image. 

			If $[a]=[b],$ then for any $x\in[a]$ we have $x\in[b]$ as well, so $x\equiv a$ and $x\equiv b,$ which means $a\equiv b.$ Thus, $\alpha(a)=\alpha(b)$ since $\equiv$ is the kernel equivalence. Therefore $\sigma([a])=\alpha(a)=\alpha(b)=\sigma([b]),$ so $\sigma$ is well defined, as desired.

			$\sigma$ is injective if and only if $[a]=[b]$ whenever $\sigma([a])=\sigma([b]).$ If $\sigma([a])=\sigma([b]),$ that means $\sigma([a])=\alpha(a)=\alpha(b)=\sigma([b]).$ Since $\equiv$ is the kernel equivalence, that means $a\equiv b.$ Given some $x\in[a]\implies x\equiv a,$ that means $x\equiv b,$ so $x\in[b],$ thus $[a]\subset[b]$ and similarly $[b]\subset[a],$ so in fact $[a]=[b],$ thus $\sigma$ is injective, as desired.

			$\sigma$ is surjective if and only if for all $b\in B,$ there exists an equivalence class $[a]\in A_\equiv$ such that $\sigma([a])=b.$ We have $\sigma([a])=\alpha(a)$ for all $[a]\in A_\equiv.$ If $\alpha$ is surjective, then for all $b\in B,$ there exists an $a\in A$ such that $\alpha(a)=b.$ Since $\varphi$ is surjective, for any $[a]$ there exists $a\in A$ such that $\varphi(a)=[a],$ so it follows that for any $b,$ there exists an $a$ and therefore an $[a]$ such that $\sigma([a])=b,$ thus $\sigma$ is surjective, as desired. 

		\end{proof}

		(b) Show that $\alpha=\sigma\varphi,$ so that $\alpha$ is the composite of a surjective mapping followed by an injective mapping.
			\begin{proof}
				We wish to show that $\alpha(a)=\sigma(\varphi(a))$ for all $a\in A.$ We have $\varphi(a)=[a]$ for all $a\in A$ by the natural mapping, then $\sigma([a])=\alpha(a)$ so $\sigma(\varphi(a))=\alpha(a),$ thus $\sigma\varphi=\alpha,$ as desired.
				
			\end{proof}

		(c) If $\alpha(A)$ is a finite set, show that the set $A_{\equiv}$ of equivalence classes is also finite and that $\vert A_{\equiv}\vert=\vert\alpha(A)\vert.$
			\begin{proof}
				Suppose the image $\alpha``(A)$ has $k\in\NN$ elements $b_1, b_2, \cdots, b_k$ that are all distinct. Assume there are infinitely many equivalence classes, $[a_i].$ Consider the equivalence class $[a_1].$ Then for any $a_1\in[a_1],$ without loss of generality (WLOG) assume that $\alpha(a_1)=b_1,$ where all elements in the equivalence class map to the same image by the definition of the equivalence class. Then without loss of generality assume that $\alpha(a_i)=b_i$ for $a_i\in[a_i]$, with $i=1, 2, \cdots, k.$ All of these equivalence classes are disjoint, and actually equivalence classes since $b_i$ are all distinct.
				
				Next consider $a_{k+1}\in[a_{k+1}].$ We know that $\alpha(a_{k+1})$ must map to one of $b_1, b_2, \cdots, b_k,$ WLOG let $\alpha(a_{k+1})=b_1.$ But then $a_{k+1}\in[a_1]$ from earlier, but equivalence classes are disjoint, so this is a contradiction. By induction, any $[a_j]$ where $j>k$ cannot exist disjoint from all $[a_i], i=1, 2, \cdots, k,$ so $A_{\equiv}$ is finite, as desired. 
				
				In fact, there are exactly as many equivalence classes as elements of $\alpha``(A)$ by the construction above, so $\vert A_\equiv\vert=\vert\alpha``(A)\vert,$ as desired.
			
			\end{proof}

		(d) In each case, find $\vert A_{\equiv}\vert$ for the given mapping $\alpha.$

		\begin{enumerate}[(i)]
				\ii $A=U\times U$ with $U=\{1, 2, 3, 4, 6, 12\}, \quad \alpha:A\to\QQ$ defined by $\alpha(n, m)=n/m.$
					\begin{soln}
						The set $A_{\equiv}$ is the set of distinct rational numbers we may form with any two elements in $U.$ These are \[\left\{1, 2, 3, 4, 6, 12, \frac{1}{2}, \frac{1}{3}, \frac{1}{4}, \frac{1}{6}, \frac{1}{12}, \frac{3}{2}, \frac{4}{3}, \frac{2}{3}, \frac{3}{4}\right\}\] so $\vert A_{\equiv}\vert=\boxed{15.}.$

					\end{soln}

				\ii $A=\{n\in\ZZ\, \vert\, 1\le n\le99\}, \quad\alpha:A\to\NN$ defined by $\alpha(n) =$ the sum of the digits of $n.$
					\begin{soln}
						The equivalence classes are the distinct values of $\alpha(n).$ The minimum value of $\alpha(n)$ is 1, achieved at $n=1$ or $n=10,$ and the maximum is 18, achieved when $n=99.$ Since every value in between can be achieved (we won't show this explicitly), $\vert A_{\equiv}\vert=\boxed{18.}$

					\end{soln}
		\end{enumerate}

	\item[10.] Let $A=\{\alpha\, \big\vert \, \alpha:P\to Q\text{ is a mapping}\}.$ Given $p\in P,$ define $\equiv$ on $A$ by $\alpha\equiv\beta$ if $\alpha(p)=\beta(p).$

		(a) Show that $\equiv$ is an equivalence on $A.$
			\begin{proof}
				$\equiv$ is an equivalence relation because it satisfies:
				\begin{enumerate}
					\ii Reflexivity: $\alpha\equiv\alpha$ because $\alpha(p)=\alpha(p).$
					\ii Symmetry: If $\alpha\equiv\beta,$ then $\alpha(p)=\beta(p)$ so $\beta(p)=\alpha(p)$ and thus $\beta\equiv\alpha.$
					\ii Transitivity: If $\alpha\equiv\beta$ and $\beta\equiv\gamma,$ then $\alpha(p)=\beta(p)$ and $\beta(p)=\gamma(p),$ so then $\alpha(p)=\gamma(p)$ and thus $\alpha\equiv\gamma.$
				\end{enumerate}
			\end{proof}

		(b) Find a mapping $\lambda:A\to Q$ such that $\equiv$ is the kernel equivalence of $\lambda.$
			\begin{soln}
				Fix some $p^*\in P.$ then let $\lambda(\alpha)=\alpha(p^*)\in Q$ for all $\alpha\in A.$ If $\alpha_1\equiv\alpha_2,$ then $\alpha_1(p)=\alpha_2(p)$ for any $p\in P,$ so $\lambda(\alpha_1)=\alpha_1(p^*)=\alpha_2(p^*)=\lambda(\alpha_2).$ Thus, $\equiv$ is the kernel equivalence of $\lambda$ as desired.

			\end{soln}

		(c) If $\vert Q\vert=n,$ how many equivalence classes does $\equiv$ have?
			\begin{soln}
				There are \boxed{\text{at most }n} equivalence classes, one for each distinct element of Q. If $\alpha$ is not surjective, then there are fewer than $n.$

			\end{soln}
\end{itemize}

\end{document}

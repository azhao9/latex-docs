\documentclass{article}
\usepackage[sexy, hdr, fancy]{evan}

\setlength{\droptitle}{-4em}

\lhead{\leftmark}
\rhead{Advanced Algebra I}
\cfoot{\thepage}

\begin{document}
\title{Advanced Algebra I Lecture Notes}
\maketitle
	This is AS.110.401 Advanced Algebra I, taught by Caterina Consani.

\thispagestyle{fancy}

\tableofcontents

\newpage

\section{26 September, 2016}
\subsection*{Subgroups}

Goal: Find subsets of a given group which inherit the same law as the group itself.

\begin{example*}
	Some examples of a chain of subsets.
	\begin{itemize}
		\ii $\{\pm 1\} \subset \{\pm 1, \pm i\} \subset \CC^0 = \{z\in\CC^* \big\vert\, |z|=1\} \subset \CC^*$

		\ii $\ZZ\subset\QQ\subset\RR\subset(\CC, +)$

		\ii $A_n\subset(S_n, \circ)$
	\end{itemize}
\end{example*}

\begin{definition*}
	$H\subset G$ where $G$ is a group, is a subgroup of $G$ if $H$ is a group with respect to the same operation for $G.$
\end{definition*}

\begin{theorem*}[Subgroup Test]
	A subset $H\subset G$ is a subgroup if:
	\begin{enumerate}
		\ii $1_G\in H$ 

		\ii $h, h'\in H\implies hh'\in H$

		\ii $h\in H\implies h^{-1}\in H$ (where $h\in G$ is the inverse of $h$)
	\end{enumerate}
\end{theorem*}

\begin{proof}
	\begin{enumerate}
		\ii Let $e\in H$ be the identify of $H.$ Then $e^2=e=e\cdot1_G,$ then by the cancellation law in $G$ it follows that $e=1_G.$

		\ii This follows from the fact that $H$ is a subgroup, so it is closed under the operation.

		\ii Let $h\in H$ and $h'\in H$ be its inverse. If $h^{-1}\in G$ is the inverse of $h$ in $G,$ then $hh'=1=hh^{-1}\implies h'=h^{-1}$ by the cancellation law in $G.$
	\end{enumerate}
\end{proof}

\begin{example*}
	$n\ZZ\subset \ZZ$ is a subgroup for fixed $n.$
\end{example*}

\begin{theorem*}[Finite group test]
	If $|H|<\infty,$ $H\neq\{\},$ and $H\subset G$ group. Then $H$ is a subgroup if and only if $H$ is closed.
\end{theorem*}

\begin{proof}
	$H$ is closed and finite. Let $h\in H$ such that $h^n=h^{m+n}$ for some $m, n\ge 1.$ Then by the cancellation in $G,$ this means $1_G=h^m,$ so $1_G\in H.$ Then since $1_G=h^{m-1}h,$ so $h^{-1}=h^{m-1}\in H.$
\end{proof}

\begin{example*}
	Consider $G=\ZZ_3,$ study its subgroups. Trivially the sets $H_1=\{\bar{0}\}$ and $H_2=\ZZ_3$ are subgroups. Does there exist a subgroup of $\ZZ_3$ with cardinality 2?

	Consider the possible sets $H_3.$ If $H_3=\{\bar{0}, \bar{1}\}$ then $H_3$ is not closed wince $\bar{1}+\bar{1}=\bar{2}\not\in H_3.$ Similarly for other combinations.
\end{example*}

An element $g$ is a \vocab{generator} if all elements of $G$ can be generated by $g.$ 

\begin{example*}
	$\ZZ_3=\{n\cdot\bar{1}|\, n\in\ZZ\}$ so $\bar{1}$ is a generator for $\ZZ_3.$
\end{example*}

\begin{example*}[Groups of order 4]
	Let $|G|=4.$ From the Cayley table, we know that \[G=\begin{cases}
			\ZZ_4 \\ K_4
	\end{cases} \] where $K_4$ is the Klein group $\ZZ_2\times\ZZ_2.$

	In the first case, $G=\{1, a, a^2, a^3\}$ such that $a^4=1.$ A trivial subgroup is $H_1=\{1\}.$ If $|H|=2,$ then $H_2=\{1, x\}$ where $x^2\in H_2,$ thus it must be that $H_2=\{1, a^2\}.$ We can't have $x=a$ or $x=a^3$ because $a^2$ would not be in $H_2,$ and because $(a^3)^2=a^6=a^2$ would not be in $H_2.$ Similarly we can't have $|H_3|=3,$ so the only subgroups are $\{1\}, \{1, a^2\}, G.$

	In the second case, $G=\{1, a, b, c\}$ such that $a^2=b^2=c^2=1.$ Then $H_2=\{1, a\}$ and similarly $\{1, b\}$ and $\{1, c\}$ are all subgroups. There are no subgroups of size 3 because elements such as $ab$ and $bc$ would not be contained, so they would not be closed. Thus, the subgroups are $\left\{ 1 \right\}, \left\{ 1, a \right\}, \left\{ 1, b \right\}, \left\{ 1, c \right\}, G.$
\end{example*}

\begin{example*}
	If $|G|=6,$ then either $G$ is commutative or $G=S_3=\left\{ \varepsilon, \sigma, \sigma^2, \tau, \tau\sigma, \tau\sigma^2 \right\}$ with the identities $\sigma^3=\tau^2=\varepsilon$ and $\sigma\tau\sigma=\tau.$ Then the subgroups are 
	\begin{align*}
		H_1 &= \left\{ \varepsilon \right\} \\
		H_2 &= \left\{ \varepsilon, \tau \right\} \\
		H_3 &= \left\{ \varepsilon, \tau\sigma \right\} \\
		H_4 &= \left\{ \varepsilon, \tau\sigma^2 \right\} \\
		H_5 &= \left\{ \varepsilon, \sigma, \sigma^2 \right\} \\
		H_6 &= S_3
	\end{align*}

	On the other hand, if $G$ is commutative, then $G=\ZZ_6.$ Then the subgroups are
	\begin{align*}
		H_1 &= \left\{ 1 \right\} \\
		H_2 &= \left\{ 0, 3 \right\} \\
		H_3 &= \left\{ 0, 2, 4 \right\} \\
		H_4 &= \ZZ_6
	\end{align*}
	so these clearly have different group structures.
\end{example*}

\begin{remark*}
	The cardinality of the subgroups divide the cardinality of the whole group. Coincidence?
\end{remark*}

\begin{remark*}
	We stated without proof that $\ZZ_6$ is the only commutative group of order 6 (up to isomorphism).
\end{remark*}

\begin{definition*}
	The subgroup of \vocab{centers} of $G$ is defined as \[Z(G) := \left\{ z\in G |\, gz=zg, \, \forall g\in G \right\}\]
\end{definition*}

If $G$ is commutative, then $Z(G)\equiv G.$ On the other hand if $G$ is not commutative, then $Z(G)$ can be very small.

\begin{example*}
	Proof in book. $Z(S_n)=\left\{ \varepsilon \right\}$
\end{example*}

\begin{example*}
	$|G|=8$ where $G$ is non-commutative. An example is the group of \vocab{quaternions} which is defined as \[G=Q:=\left\{ \pm 1, \pm i, \pm j, \pm k\right\}\] such that 
	\begin{align*}
		i^2=j^2=k^2 &= -1 \\
		ij = -ji &= k \\
		jk = -kj &= i \\
		ki = -ik &= j \\
	\end{align*} $Q$ can be realized as a subgroup of $\text{GL}_n(\CC).$ Then the subgroup of centers is $Z(Q)=\left\{ \pm 1 \right\},$ so $Q$ is very non-commutative.
\end{example*}

\begin{theorem*}
	If $G$ is a group and $H, K\subset H$ are subgroups, then
	\begin{enumerate}
		\ii $H\cap K:=\left\{ g\in G|\, g\in H, g\in K \right\}$ is a subgroup of $G$

		\ii $\forall g\in G,$ the set $gHg^{-1}:=\left\{ ghg^{-1}|\, h\in H \right\}$ is a subgroup of $G.$ The group $gHg^{-1}$ is called a \vocab{conjugate} of $H$ in $G.$
	\end{enumerate}
\end{theorem*}

\begin{fact*}
	$|H|=|gHg^{-1}|$ if $H$ is finite.
\end{fact*}

\begin{fact*}
	If $G$ is commutative, then $gHg^{-1}\equiv H.$
\end{fact*}

\end{document}

\documentclass{article}
\usepackage[sexy, hdr, fancy, diagrams]{evan}

\setlength{\droptitle}{-4em}

\lhead{\leftmark}
\rhead{Advanced Algebra I}
\cfoot{\thepage}

\begin{document}
\title{Advanced Algebra I Lecture Notes}
\maketitle
	This is AS.110.401 Advanced Algebra I, taught by Caterina Consani.

\thispagestyle{fancy}

\tableofcontents

\newpage

\section{26 September, 2016}
\subsection{Subgroups}

Goal: Find subsets of a given group which inherit the same law as the group itself.

\begin{example*}
	Some examples of a chain of subsets.
	\begin{itemize}
		\ii $\{\pm 1\} \subset \{\pm 1, \pm i\} \subset \CC^0 = \{z\in\CC^* \big\vert\, |z|=1\} \subset \CC^*$

		\ii $\ZZ\subset\QQ\subset\RR\subset(\CC, +)$

		\ii $A_n\subset(S_n, \circ)$
	\end{itemize}
\end{example*}

\begin{definition*}
	$H\subset G$ where $G$ is a group, is a subgroup of $G$ if $H$ is a group with respect to the same operation for $G.$
\end{definition*}

\begin{theorem*}[Subgroup Test]
	A subset $H\subset G$ is a subgroup if:
	\begin{enumerate}
		\ii $1_G\in H$ 

		\ii $h, h'\in H\implies hh'\in H$

		\ii $h\in H\implies h^{-1}\in H$ (where $h\in G$ is the inverse of $h$)
	\end{enumerate}
\end{theorem*}

\begin{proof}
	\begin{enumerate}
		\ii Let $e\in H$ be the identify of $H.$ Then $e^2=e=e\cdot1_G,$ then by the cancellation law in $G$ it follows that $e=1_G.$

		\ii This follows from the fact that $H$ is a subgroup, so it is closed under the operation.

		\ii Let $h\in H$ and $h'\in H$ be its inverse. If $h^{-1}\in G$ is the inverse of $h$ in $G,$ then $hh'=1=hh^{-1}\implies h'=h^{-1}$ by the cancellation law in $G.$
	\end{enumerate}
\end{proof}

\begin{example*}
	$n\ZZ\subset \ZZ$ is a subgroup for fixed $n.$
\end{example*}

\begin{theorem*}[Finite group test]
	If $|H|<\infty,$ $H\neq\{\},$ and $H\subset G$ group. Then $H$ is a subgroup if and only if $H$ is closed.
\end{theorem*}

\begin{proof}
	$H$ is closed and finite. Let $h\in H$ such that $h^n=h^{m+n}$ for some $m, n\ge 1.$ Then by the cancellation in $G,$ this means $1_G=h^m,$ so $1_G\in H.$ Then since $1_G=h^{m-1}h,$ so $h^{-1}=h^{m-1}\in H.$
\end{proof}

\begin{example*}
	Consider $G=\ZZ_3,$ study its subgroups. Trivially the sets $H_1=\{\bar{0}\}$ and $H_2=\ZZ_3$ are subgroups. Does there exist a subgroup of $\ZZ_3$ with cardinality 2?

	Consider the possible sets $H_3.$ If $H_3=\{\bar{0}, \bar{1}\}$ then $H_3$ is not closed wince $\bar{1}+\bar{1}=\bar{2}\not\in H_3.$ Similarly for other combinations.
\end{example*}

An element $g$ is a \vocab{generator} if all elements of $G$ can be generated by $g.$ 

\begin{example*}
	$\ZZ_3=\{n\cdot\bar{1}|\, n\in\ZZ\}$ so $\bar{1}$ is a generator for $\ZZ_3.$
\end{example*}

\begin{example*}[Groups of order 4]
	Let $|G|=4.$ From the Cayley table, we know that \[G=\begin{cases}
			\ZZ_4 \\ K_4
	\end{cases} \] where $K_4$ is the Klein group $\ZZ_2\times\ZZ_2.$

	In the first case, $G=\{1, a, a^2, a^3\}$ such that $a^4=1.$ A trivial subgroup is $H_1=\{1\}.$ If $|H|=2,$ then $H_2=\{1, x\}$ where $x^2\in H_2,$ thus it must be that $H_2=\{1, a^2\}.$ We can't have $x=a$ or $x=a^3$ because $a^2$ would not be in $H_2,$ and because $(a^3)^2=a^6=a^2$ would not be in $H_2.$ Similarly we can't have $|H_3|=3,$ so the only subgroups are $\{1\}, \{1, a^2\}, G.$

	In the second case, $G=\{1, a, b, c\}$ such that $a^2=b^2=c^2=1.$ Then $H_2=\{1, a\}$ and similarly $\{1, b\}$ and $\{1, c\}$ are all subgroups. There are no subgroups of size 3 because elements such as $ab$ and $bc$ would not be contained, so they would not be closed. Thus, the subgroups are $\left\{ 1 \right\}, \left\{ 1, a \right\}, \left\{ 1, b \right\}, \left\{ 1, c \right\}, G.$
\end{example*}

\begin{example*}
	If $|G|=6,$ then either $G$ is commutative or $G=S_3=\left\{ \varepsilon, \sigma, \sigma^2, \tau, \tau\sigma, \tau\sigma^2 \right\}$ with the identities $\sigma^3=\tau^2=\varepsilon$ and $\sigma\tau\sigma=\tau.$ Then the subgroups are 
	\begin{align*}
		H_1 &= \left\{ \varepsilon \right\} \\
		H_2 &= \left\{ \varepsilon, \tau \right\} \\
		H_3 &= \left\{ \varepsilon, \tau\sigma \right\} \\
		H_4 &= \left\{ \varepsilon, \tau\sigma^2 \right\} \\
		H_5 &= \left\{ \varepsilon, \sigma, \sigma^2 \right\} \\
		H_6 &= S_3
	\end{align*}

	On the other hand, if $G$ is commutative, then $G=\ZZ_6.$ Then the subgroups are
	\begin{align*}
		H_1 &= \left\{ 1 \right\} \\
		H_2 &= \left\{ 0, 3 \right\} \\
		H_3 &= \left\{ 0, 2, 4 \right\} \\
		H_4 &= \ZZ_6
	\end{align*}
	so these clearly have different group structures.
\end{example*}

\begin{remark*}
	The cardinality of the subgroups divide the cardinality of the whole group. Coincidence?
\end{remark*}

\begin{remark*}
	We stated without proof that $\ZZ_6$ is the only commutative group of order 6 (up to isomorphism).
\end{remark*}

\begin{definition*}
	The subgroup of \vocab{centers} of $G$ is defined as \[Z(G) := \left\{ z\in G |\, gz=zg, \, \forall g\in G \right\}\]
\end{definition*}

If $G$ is commutative, then $Z(G)\equiv G.$ On the other hand if $G$ is not commutative, then $Z(G)$ can be very small.

\begin{example*}
	Proof in book. $Z(S_n)=\left\{ \varepsilon \right\}$
\end{example*}

\begin{example*}
	$|G|=8$ where $G$ is non-commutative. An example is the group of \vocab{quaternions} which is defined as \[G=Q:=\left\{ \pm 1, \pm i, \pm j, \pm k\right\}\] such that 
	\begin{align*}
		i^2=j^2=k^2 &= -1 \\
		ij = -ji &= k \\
		jk = -kj &= i \\
		ki = -ik &= j \\
	\end{align*} $Q$ can be realized as a subgroup of $\text{GL}_n(\CC).$ Then the subgroup of centers is $Z(Q)=\left\{ \pm 1 \right\},$ so $Q$ is very non-commutative.
\end{example*}

\begin{theorem*}
	If $G$ is a group and $H, K\subset H$ are subgroups, then
	\begin{enumerate}
		\ii $H\cap K:=\left\{ g\in G|\, g\in H, g\in K \right\}$ is a subgroup of $G$

		\ii $\forall g\in G,$ the set $gHg^{-1}:=\left\{ ghg^{-1}|\, h\in H \right\}$ is a subgroup of $G.$ The group $gHg^{-1}$ is called a \vocab{conjugate} of $H$ in $G.$
	\end{enumerate}
\end{theorem*}

\begin{fact*}
	$|H|=|gHg^{-1}|$ if $H$ is finite.
\end{fact*}

\begin{fact*}
	If $G$ is commutative, then $gHg^{-1}\equiv H.$
\end{fact*}


\newpage

\section{5 October, 2016}
\subsection{Cosets}

Suppose $G$ is a group and $H\subset G$ is a subgroup of $G.$ Then for some $a\in G,$ the set \[Ha:=\left\{ ha|h\in H \right\}\] is the \vocab{right coset} generated by $a.$ Similarly, the set \[aH := \left\{ ah|h\in H \right\}\] is the \vocab{left coset} generated by $a.$

\begin{example*}
	Let $4\ZZ\subset\ZZ,$ then since $\ZZ$ is commutative, the left and right cosets are the same, and they are \[1+4\ZZ, \quad 2+4\ZZ, \quad 3+4\ZZ, \quad 4\ZZ.\]
\end{example*}

\begin{remark}
	If $G$ is abelian, then $aH=Ha$ for all $a\in G.$
\end{remark}

\begin{theorem*}
	Let $G$ be a group, $H\subset G$ is a subgroup, and let $a, b\in G.$

	\begin{enumerate}[(1)]
		\ii $H=H\cdot 1=1\cdot H$

		\ii $aH=H\iff a\in H$

		\ii $Ha=Hb\iff ab^{-1}\in H$

		\ii $a\in Hb\implies Ha=Hb$

		\ii $Ha=Hb$ or $Ha\cap Hb=\left\{  \right\}$

		\ii $G=\displaystyle\coprod_{a\in G} Ha$ is a partition of $G.$
	\end{enumerate}
\end{theorem*}

\begin{remark}
	(5) and (6) are very similar to properties of equivalence classes on a set.
\end{remark}

\begin{proof}
	Proof of (3): $Ha=Hb\implies a\in Ha=Hb$ so $a\in Hb$ so $a$ can be written as $a=hb$ for some $h\in H,$ so right multiplying by $b^{-1},$ we have $ab^{-1}=h\in H$ as desired.

	For the reverse direction if $ab^{-1}\in H,$ then $ha=h(ab^{-1})b\in Hb$ since $h, ab^{-1}\in H\implies h(ab^{-1})\in H$ thus since $ha\in Ha$ belongs to $Hb,$ we have $Ha\subset Hb.$ Then $b^{-1}a=(ab^{-1})^{-1}\in H$ so $hb=h(ba^{-1})a\in Ha$ so $Hb\subset Ha$ so in fact $Ha=Hb.$

	(3) $\implies$ (2) by choosing $b=1.$

	Proof of (5): If $Ha\cap Hb\neq \left\{  \right\},$ then there must be an element in common. Let $x\in Ha\cap Hb$ be this element, then $x\in Ha$ so $Hx=Ha$ and $Hx=Hb$ so $Ha=Hb.$
\end{proof}

\begin{example}
	Let $G=\left< a\right>$ be a group, where $o(a)=4$ so $G=\left\{ 1, a, a^2, a^3 \right\}.$ Find the cosets of $H=\left< a^2\right>=\left\{ 1, a^2 \right\}.$ 

	Since $a^2\in H,$ then $Ha^2=H.$ Next, $Ha=\left\{ a, a^3 \right\}$ and notice $Ha\cap H=\left\{  \right\}.$ Then $G$ is partitioned by $H\cdot 1$ and $Ha$ and $|H|=|Ha|.$
\end{example}

\begin{lemma}
	Let $G$ be a group and $H\subset G$ be a subgroup.

	\begin{enumerate}[(1)]
		\ii $|H|=|Ha|=|aH|$ for all $a\in G.$

		\ii $\varphi:\left\{ Ha|a\in G \right\} \to \left\{ bH|b\in G \right\}$ where $\varphi(Ha)=a^{-1}H$ is a bijection of sets.
	\end{enumerate}
\end{lemma}

\begin{proof}
	For (1): $|H|=|Ha|$ since $h\mapsto ha$ defines a bijection. This map is injective since if $h_1a=ha$ then $h_1=h$ by the cancellation law, and surjective because for all $ha\in Ha$ we can recover the $h\in H$ such that $h\mapsto ha.$ Thus $|H|=|Ha|.$
\end{proof}

\begin{remark}
	The sets of right and left cosets for the same $H\subset G$ have the same cardinality.
\end{remark}

\begin{definition}
	$|G:H|$ is the \vocab{index} of $H$ in $G$ and is defined as the cardinality of the set $\left\{ Ha|a\in G \right\}.$
\end{definition}

\begin{remark}
	This applies even for infinite sets, for example $|\ZZ:4\ZZ|=4.$
\end{remark}

\begin{theorem}[Langrange's Theorem]
	Let $G$ be a group and $H\subset G$ is a subgroup, and $|G|$ is finite.

	\begin{enumerate}[(1)]
		\ii $|H| \, \big\vert\, |G|$

		\ii $|G:H|=\displaystyle\frac{|G|}{|H|}$
	\end{enumerate}
\end{theorem}

\begin{proof}
	For (1): Let $k=|G:H|$ where the set of right cosets is $\left\{ Ha_1, Ha_2, \cdots, Ha_k \right\}$ where $a_i\in G.$ Then \[G=Ha_1\sqcup Ha_2\sqcup\cdots\sqcup Ha_k\] and since these are disjoint sets, the cardinality of $G$ is the sum of the cardinalities of the cosets, which are all $|H|$ since $|H|=|Ha|$, so $|G|=k|H|.$
\end{proof}

\begin{corollary}
	If $|G|$ is finite, and $g\in G,$ then $o(g)=\left\lvert \left< g\right> \right\rvert\big \vert\, |G|.$
\end{corollary}

\begin{remark}
	The converse of Lagrange's Theorem fails in general. 
\end{remark}

\begin{example}
	Take $A_4\subset S_4$ the subgroup of even permutations. Then $|A_4|=|S_4|/2=12,$ but there does not exist a subgroup $H\subset A_4$ such that $|H|=6.$

	Indeed $K_4=\left\{ \varepsilon, (12)(34), (13)(24), (14)(23) \right\}$ is a subgroup of $A_4$ and let $H=\left< \sigma\right>=\left< (123)\right>$ be the group generated by the 3-cycles. Then $|H|=3, |K_4|=4$ and the Cartesian product \[H\times K_4=\left\{ hk|h\in H, k\in K_4 \right\}\] is a group of order 12. However, note that $\forall \gamma\in K_4$ and $\forall \sigma'\in H,$ the order \[o(\gamma\sigma')=\begin{cases}
			1 \\ 2 \\ 3
	\end{cases}\] so there is no element $g\in A_4$ such that $o(g)=6$ so there does not exist a subgroup $G\subset A_4$ such that $|G|=6.$
\end{example} 

\subsection{Normal Subgroups}
\begin{definition}
	Given a group $G,$ a subgroup $H\subset G$ is said to be \vocab{normal} if $aH=Ha, \forall a\in G.$
\end{definition}

\begin{example}[non-example]
	$G=S_3$ and take $H=\left\{ \varepsilon, \tau \right\}$ where $\tau^2=\varepsilon$ and let $\sigma\in S_3$ such that $\sigma^3=\varepsilon.$ Then $\sigma H\neq H\sigma.$
\end{example}




\newpage

\section{12 October, 2016}
\subsection{Normal Subgroups}

\begin{definition}
	If $G$ is a group, and $H\subset G$ is a subgroup, $H$ is a \vocab{normal subgroup} if $Hg=gH$ for all $g\in G.$ That is, the left and right cosets coincide. This is $H\unlhd G.$
\end{definition}

Then we can define the \vocab{quotient groups} as $G/H:=\left\{ gH\, |\, g\in G \right\}.$ We may also define the product of cosets as $gH\cdot g_1H:=(gg_1)H.$ 

We can write a chain of normal subgroups \[\left\{ 0 \right\}\subset H_1\unlhd H_2\unlhd \cdots \unlhd H_n\unlhd G\] where all $H_i$ are normal in $G.$ We can write \[G=\bigoplus_{i=1}^n H_i/H_{i-1}\] which is a direct sum.

\begin{definition}
	When $G$ has no non-trivial normal subgroup then $G$ is called \vocab{simple.} 
\end{definition}

\begin{example*}
	$A_n\subset S_n$ for all $n\ge 5$.
\end{example*}

\begin{remark*}
	$H=\left< (123)\right>\unlhd A_4$ is normal.
\end{remark*}

\begin{example*}
	Let $G=\left< a, b\, |\, o(a)=4, o(b)=2, aba=b\right>.$ Then explicitly \[G=\left\{ 1, a, a^2, a^3, b, ba, ba^2, ba^3 \right\}.\] Let $K=\left< a\right>,$ so $|K|=4,$ and $|G:K|=2,$ hence $K$ is normal in $G.$ Then $G/K=\left\{ K, Kb \right\}.$ 
	
	Next, let $H=\left< b\right>,$ so $|G:H|=4.$ Let $S=\left\{ H, Ha, Ha^2, Ha^3 \right\}$ with $Ha=\left\{ a, ba \right\}\neq aH=\left\{ a, ab \right\}$ since $ab\neq ba.$ Thus $H$ is not normal in $G$ and $S$ is not actually a group.
\end{example*}

\begin{theorem*}
	Let $K\unlhd Z(G)\unlhd G$ st $G/K$ is cyclic. Then $G$ is abelian.
\end{theorem*}

\begin{proof}
	We have $G/K=\left< Kg\right>$ for some $g\in G.$ Let $a, b\in G,$ we need to show $ab=ba.$ Consider $Ka, Kb$ in the quotient $G/K.$ Then $Ka=(Kg)^m=Kg^m$ and $Kb=Kg^n$ for some $m, n\in\ZZ.$ Since these are cosets that are either disjoint or equal, we have $b\in Kg^n\implies b=k_1g^n$ and $a=kg^m$ for some $k, k_1\in K.$ Now \[ab=(kg^m)(k_1g^n)=(kk_1)g^{m+n}=(k_1g^n)(kg^m)=ba\] since as cosets, $KaKb=Kab=abK=KbKa.$ 
\end{proof}

\subsection{The Isomorphism Theorem}

Let $\varphi:G\to H$ where $\varphi$ is a group homomorphism. Then $\ker(\varphi):=\left\{ g\in G\, |\, \varphi(g)=1_H \right\}\unlhd G$ is a normal subgroup. Clearly $\ker(\varphi)$ is a subgroup. Now we want to see if $\ker(\varphi)$ is normal, that is if \[g\ker(\varphi)g\inv=\ker(\varphi)\] for all $g\in G.$ 

Let $h\in\ker(\varphi),$ so $\varphi(h)=1.$ Then $\varphi(ghg\inv)=\varphi(g)\varphi(h)\varphi(g\inv)=\varphi(g)\varphi(g\inv)=1$ for all $h\in\ker(\varphi)$ so in fact $g\ker(\varphi)g\inv=\ker(\varphi).$

For the converse, given $K\unlhd G$ a normal subgroup, where $\varphi:G\to G/K,$ we define $\ker(\varphi)=K$ by construction of $G/K.$

\textbf{Conclusion:} There is a 1-1 correspondence between normal subgroups of $G$ and group homomorphisms $\varphi:G\to H.$

\begin{example*}
	Let $\varphi:G\to H$ be a group homomorphism. Then \[H\supset \varphi``(G)=\left\{ \varphi(g)\, |\, g\in G \right\}\] is a subgroup, but in general not normal
\end{example*}

\begin{example*}
	Take $H\subset G$ be not normal, and consider the natural embedding $\varphi=i:H\to G,$ then $\varphi``(G)=i``(G)=H,$ which we said was not normal in $G.$
\end{example*}

\begin{remark*}
	Sometimes one shows that $H\subset G$ is normal by showing that $H=\ker(\varphi)$ for a suitable $\varphi.$
\end{remark*}

\begin{theorem}
	Let $\varphi:G\to H$ be a group homomorphism. Then the following are equivalent:
	\begin{itemize}
		\ii $\varphi$ is injective
		\ii $\ker(\varphi)=\left\{ 1_G \right\}$
	\end{itemize}
\end{theorem}

\begin{proof}
	Let $g\in\ker(\varphi),$ so $\varphi(g)=1=\varphi(1_G).$ Then since $\varphi$ is injective, we must have $g=1_G.$

	Next, assume $\ker(\varphi)=1_G.$ Then let $a, b\in G$ such that $\varphi(a)=\varphi(b)\iff \varphi(ab\inv)=1_H.$ Thus $ab\inv=1_G,$ so $a=b,$ thus $\varphi$ is injective.
	
\end{proof}

\begin{remark*}
	The statement for monoids would fail since the proof assumes the existence of inverses.
\end{remark*}

\begin{example*}
	Let $\varphi:M\to\left\{ 0, 1 \right\}$ be a monoid homomorphism, where $1+1=1,$ and $0+1=1+0=1.$ Define $\varphi(m)=1$ for all $m\neq 1_M$ and $\varphi(1_M)=0.$ Then $\ker(\varphi)=\left\{ 1_M \right\}$ but $\varphi$ is clearly not injective.
	
\end{example*}

\begin{theorem}[Isomorphism Theorem]
	Let $\varphi:G\to H$ be a group homomorphism, and $K:=\ker(\varphi)\unlhd G.$ Then \[\varphi``(G)\cong G/K\]

	\begin{diagram}
		G & \rTo^{\varphi} & H \\
		\dTo & & \uTo^{i} \\
		G/K & \rTo^{\bar{\varphi}} & \varphi``(G)
	\end{diagram}
\end{theorem}

\end{document}

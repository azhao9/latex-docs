\documentclass{article}
\usepackage[sexy, hdr, fancy]{evan}
\setlength{\droptitle}{-4em}

\lhead{Homework 5}
\rhead{Advanced Algebra I}
\lfoot{}
\cfoot{\thepage}

\begin{document}
\title{Homework 5}
\maketitle
\thispagestyle{fancy}

\section*{Section 2.4: Cyclic Groups and the Order of an Element}
\begin{itemize}
	\item[4.] In each case determine whether $G$ is cyclic.
		\begin{enumerate}[(a)]
			\item $G=\ZZ_7^*$
				\begin{soln}
					Here, $G=\left\{ 1, 2, 3, 4, 5, 6 \right\},$ where these are understood to be the equivalence classes, and the operation is multiplication. Then we have
					\begin{align*}
						1 &\equiv 1 \\
						2 &\equiv 3^2 \\
						3 &\equiv 3^1 \\
						4 &\equiv 3^4 \\
						5 &\equiv 3^5 \\
						6 &\equiv 3^3
					\end{align*} so $G=\left< 3\right>,$ and \boxed{G\text{ is cyclic.}}
					
				\end{soln}

			\item $G=\ZZ_{12}^*$
				\begin{soln}
					Here, $G=\left\{ 1, 5, 7, 11 \right\},$ where these are understood to be equivalence classes, so the order of $G$ is 4. However, $\left< 5\right>=\left\{ 1, 5 \right\}$ and $\left< 7\right>=\left\{ 1, 7 \right\},$ and these subgroups both have order 2, so \boxed{G\text{ is not cyclic.}}
					
				\end{soln}

			\item $G=\ZZ_{16}^*$
				\begin{soln}
					Here, $G=\left\{ 1, 3, 5, 7, 9, 11, 13, 15 \right\}$ so the order of $G$ is 8. Now, we have
					\begin{align*}
						\left< 3\right> &= \left\{ 1, 3, 9, 11 \right\} \\
						\left< 5\right> &= \left\{ 1, 5, 9, 13 \right\}
					\end{align*} so $G$ has two distinct subgroups of order 4, so \boxed{G\text{ is not cyclic.}}
					
				\end{soln}

			\item $G=\ZZ_{11}^*$
				\begin{soln}
					Here, $G=\left\{ 1, 2, 3, 4, 5, 6, 7, 8, 9, 10 \right\},$ and we have
					\begin{align*}
						1 &\equiv 1 \\
						2 &\equiv 2^1 \\
						3 &\equiv 2^8 \\
						4 &\equiv 2^2 \\
						5 &\equiv 2^4 \\
						6 &\equiv 2^9 \\
						7 &\equiv 2^7 \\
						8 &\equiv 2^3 \\
						9 &\equiv 2^6 \\
						10 &\equiv 2^5
					\end{align*} so $G=\left< 2\right>$ so \boxed{G\text{ is cyclic.}}

				\end{soln}
				
		\end{enumerate}

	\item[20.] \begin{enumerate}[(a)]
			\item Find three elements of $C_6\times C_{15}$ of maximum order.
				\begin{soln}
					Let $C_6=\left\{ 1, g, \cdots, g^5 \right\}$ and $C_{15}=\left\{ 1, f, \cdots, f^{14} \right\}.$ Then the element of max order in $C_6$ is $g^5,$ where $o(g^5)=6$ since 5 and 6 are relatively prime. Similarly, the elements of max order in $C_{15}$ are the elements $f^k$ where $k$ is relatively prime to 15, which are $k=7, 11, 13.$ Then the elements in $C_6\times C_{15}$ of max order are \[\boxed{(g^5, f^7), \quad (g^5, f^{11}), \quad (g^5, f^{13})}\] which all have order $\lcm(6, 15)=30.$
					
				\end{soln}

			\item Find one element of maximum order in $C_m\times C_n.$
				\begin{soln}
					If $C_m=\left< g\right>$ and $C_n=\left< f\right>$ then we are guaranteed $o(g^{m-1})=m$ and $o(f^{n-1})=n$ since $m-1$ and $n-1$ are relatively prime to $m$ and $n,$ respectively. Thus, the element $\boxed{(g^{m-1}, f^{n-1})}$ is of maximum order $\lcm(m, n).$
					
				\end{soln}
				
		\end{enumerate}

	\item[28.] Let $H$ be a subgroup of a group $G$ and let $a\in G, o(a)=n.$ If $m$ is the smallest positive integer such that $a^m\in H,$ show that $m|n.$
		
\end{itemize}

\section*{Section 2.5: Homomorphisms and Isomorphisms}
\begin{itemize}
	\item[3.] If $G$ is any group, define $\alpha:G\to G$ by $\alpha(g)=g\inv.$ Show that $G$ is abelian if and only if $\alpha$ is a homomorphism.
		\begin{proof}
			If $G$ is abelian, then $gf=fg$ for any $f, g\in G.$ Then $\alpha(gf)=\alpha(fg)=(fg)\inv=g\inv f\inv = \alpha(g) \alpha(f)$ so $\alpha(gf)=\alpha(g)\alpha(f),$ so $\alpha$ is a homomorphism, as desired.

			If $\alpha$ is a homomorphism, then $\alpha(fg)=\alpha(f)\alpha(g)$ for all $f, g\in G.$ Then $(fg)\inv=f\inv g\inv = (gf)\inv$ so in fact $fg=gf$ since inverses are unique, and $G$ is abelian, as desired.
			
		\end{proof}

	\item[6.] Show that there are exactly two homomorphisms $\alpha:C_6\to C_4.$

	\item[13.] Show that $G=\left\{ \begin{bmatrix}
			1 & 0 \\ 0 & 1
		\end{bmatrix}, \begin{bmatrix}
			-1 & 0 \\ 0 & -1
		\end{bmatrix}, \begin{bmatrix}
			0 & -1 \\ 1 & 0 
		\end{bmatrix}, \begin{bmatrix}
			0 & 1 \\ -1 & 0
		\end{bmatrix}\right\}$ is a subgroup of $GL_2(\ZZ)$ isomorphic to $\left\{ 1, -1, i, -i \right\}.$

	\item[25.] Are the additive groups $\ZZ$ and $\QQ$ isomorphic? 

	\item[33.] If $Z(G)=\left\{ 1 \right\},$ show that $G\cong\text{inn}G.$

\end{itemize}

\section*{Section 2.6: Cosets and Lagrange's Theorem}
\begin{itemize}
	\item[1.] In each case find the right and left cosets in $G$ of the subgroups $H$ and $K$ of $G.$ 
		\begin{itemize}
			\item[(e)] $G=D_4=\left\{ 1, a, a^2, a^3, b, ba, ba^2, ba^3 \right\}, o(a)=4, o(b)=2,$ and $aba=b;$ $H=\left< a^2\right>, K=\left< b\right>.$

			\item[(f)] $G=$ any group; $H$ is any subgroup of index 2.
				
		\end{itemize}

	\item[17.] Let $|G|=p^2,$ where $p$ is a prime. Show that every proper subgroup of $G$ is cyclic.
		
\end{itemize}

\end{document}

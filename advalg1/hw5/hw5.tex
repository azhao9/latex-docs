\documentclass{article}
\usepackage[sexy, hdr, fancy]{evan}
\setlength{\droptitle}{-4em}

\lhead{Homework 5}
\rhead{Advanced Algebra I}
\lfoot{}
\cfoot{\thepage}

\begin{document}
\title{Homework 5}
\maketitle
\thispagestyle{fancy}

\section*{Section 2.4: Cyclic Groups and the Order of an Element}
\begin{itemize}
	\item[4.] In each case determine whether $G$ is cyclic.
		\begin{enumerate}[(a)]
			\item $G=\ZZ_7^*$
				\begin{soln}
					Here, $G=\left\{ 1, 2, 3, 4, 5, 6 \right\},$ where these are understood to be the equivalence classes, and the operation is multiplication. Then we have
					\begin{align*}
						1 &\equiv 1 \\
						2 &\equiv 3^2 \\
						3 &\equiv 3^1 \\
						4 &\equiv 3^4 \\
						5 &\equiv 3^5 \\
						6 &\equiv 3^3
					\end{align*} so $G=\left< 3\right>,$ and \boxed{G\text{ is cyclic.}}
					
				\end{soln}

			\item $G=\ZZ_{12}^*$
				\begin{soln}
					Here, $G=\left\{ 1, 5, 7, 11 \right\},$ where these are understood to be equivalence classes, so the order of $G$ is 4. However, $\left< 5\right>=\left\{ 1, 5 \right\}$ and $\left< 7\right>=\left\{ 1, 7 \right\},$ and these subgroups both have order 2, so \boxed{G\text{ is not cyclic.}}
					
				\end{soln}

			\item $G=\ZZ_{16}^*$
				\begin{soln}
					Here, $G=\left\{ 1, 3, 5, 7, 9, 11, 13, 15 \right\}$ so the order of $G$ is 8. Now, we have
					\begin{align*}
						\left< 3\right> &= \left\{ 1, 3, 9, 11 \right\} \\
						\left< 5\right> &= \left\{ 1, 5, 9, 13 \right\}
					\end{align*} so $G$ has two distinct subgroups of order 4, so \boxed{G\text{ is not cyclic.}}
					
				\end{soln}

				\newpage
			\item $G=\ZZ_{11}^*$
				\begin{soln}
					Here, $G=\left\{ 1, 2, 3, 4, 5, 6, 7, 8, 9, 10 \right\},$ and we have
					\begin{align*}
						1 &\equiv 1 \\
						2 &\equiv 2^1 \\
						3 &\equiv 2^8 \\
						4 &\equiv 2^2 \\
						5 &\equiv 2^4 \\
						6 &\equiv 2^9 \\
						7 &\equiv 2^7 \\
						8 &\equiv 2^3 \\
						9 &\equiv 2^6 \\
						10 &\equiv 2^5
					\end{align*} so $G=\left< 2\right>$ so \boxed{G\text{ is cyclic.}}

				\end{soln}
				
		\end{enumerate}

	\item[20.] \begin{enumerate}[(a)]
			\item Find three elements of $C_6\times C_{15}$ of maximum order.
				\begin{soln}
					Let $C_6=\left\{ 1, g, \cdots, g^5 \right\}$ and $C_{15}=\left\{ 1, f, \cdots, f^{14} \right\}.$ Then the element of max order in $C_6$ is $g^5,$ where $o(g^5)=6$ since 5 and 6 are relatively prime. Similarly, the elements of max order in $C_{15}$ are the elements $f^k$ where $k$ is relatively prime to 15, which are $k=7, 11, 13.$ Then the elements in $C_6\times C_{15}$ of max order are \[\boxed{(g^5, f^7), \quad (g^5, f^{11}), \quad (g^5, f^{13})}\] which all have order $\lcm(6, 15)=30.$
					
				\end{soln}

			\item Find one element of maximum order in $C_m\times C_n.$
				\begin{soln}
					If $C_m=\left< g\right>$ and $C_n=\left< f\right>$ then we are guaranteed $o(g^{m-1})=m$ and $o(f^{n-1})=n$ since $m-1$ and $n-1$ are relatively prime to $m$ and $n,$ respectively. Thus, the element $\boxed{(g^{m-1}, f^{n-1})}$ is of maximum order $\lcm(m, n).$
					
				\end{soln}
				
		\end{enumerate}

	\item[28.] Let $H$ be a subgroup of a group $G$ and let $a\in G, o(a)=n.$ If $m$ is the smallest positive integer such that $a^m\in H,$ show that $m|n.$
		\begin{proof}
			We know that $0\le m\le n-1,$ so suppose $n=qm+r$ where $q, r\in\ZZ$ and $0\le r\le m-1.$ Then \[a^n=a^{qm+r}=a^r(a^m)^q=1_G.\] Since $a^m\in H,$ by induction $(a^m)^q\in H$ for any $q\ge 1,$ and the inverse $((a^m)^q)\inv\in H$ as well since $H$ is a subgroup. However, from above, we know this inverse is exactly $a^r,$ which must exist in $H.$ However, since by assumption, $m$ is the smallest positive integer such that $a^m$ is in $H,$ so it must be the case that $r=0,$ so $a^r=1_G.$ Thus, $n=qm,$ so $m|n,$ as desired.
			
		\end{proof}
		
\end{itemize}

\section*{Section 2.5: Homomorphisms and Isomorphisms}
\begin{itemize}
	\item[3.] If $G$ is any group, define $\alpha:G\to G$ by $\alpha(g)=g\inv.$ Show that $G$ is abelian if and only if $\alpha$ is a homomorphism.
		\begin{proof}
			If $G$ is abelian, then $gf=fg$ for any $f, g\in G.$ Then $\alpha(gf)=\alpha(fg)=(fg)\inv=g\inv f\inv = \alpha(g) \alpha(f)$ so $\alpha(gf)=\alpha(g)\alpha(f),$ so $\alpha$ is a homomorphism, as desired.

			If $\alpha$ is a homomorphism, then $\alpha(fg)=\alpha(f)\alpha(g)$ for all $f, g\in G.$ Then $(fg)\inv=f\inv g\inv = (gf)\inv$ so in fact $fg=gf$ since inverses are unique, and $G$ is abelian, as desired.
			
		\end{proof}

	\item[6.] Show that there are exactly two homomorphisms $\alpha:C_6\to C_4.$
		\begin{proof}
			Let $C_6=\left\{ 1, g, \cdots, g^5 \right\}$ and $C_4=\left\{ 1, f, f^2, f^3 \right\}.$ Then since $C_6=\left< g\right>,$ a homomorphism is uniquely determined by where $g$ is mapped to. Clearly $\alpha_1(g)=1$ works, then $\alpha_1$ maps all elements of $C_6$ to 1, which is a homomorphism. 

			Then, the mapping $\alpha_2(g)=f^2$ also determines a homomorphism. Then $\alpha_2(1)=\alpha_2(g^2)=\alpha_2(g^4)=1$ and $\alpha_2(g)=\alpha_2(g^3)=\alpha_2(g^5)=f^2.$ It's easy to check that $\alpha_2(ab)=\alpha_2(a)\alpha_2(b)$ for any $a, b\in C_6.$ 

			Consider the mapping $\alpha_3(g)=f.$ Then $\alpha_3(g^2)=f^2$ and $\alpha_3(g^3)=f^3$ and $\alpha_3(g^4)=1.$ This is a problem because $\alpha_3(g^2 g^4)=\alpha_3(g^6)=\alpha_3(1)=1,$ but $\alpha_3(g^2)\alpha(g^4)=f^2\cdot 1=f^2,$ so $\alpha_3$ is not a homomorphism.

			Similarly, the mapping $\alpha_4(g)=f^3$ encounters the same problems. Then $\alpha(g^5)=f^3,$ so $\alpha_4(gg^5)=\alpha_4(1)=1$ but $\alpha_4(g)\alpha_4(g^5)=f^3\cdot f^3=f^2,$ so $\alpha_4$ is not a homomorphism.

			Thus, there are only two homomorphisms from $C_6$ to $C_4,$ as desired.
			
		\end{proof}

	\item[13.] Show that $G=\left\{ \begin{bmatrix}
			1 & 0 \\ 0 & 1
		\end{bmatrix}, \begin{bmatrix}
			-1 & 0 \\ 0 & -1
		\end{bmatrix}, \begin{bmatrix}
			0 & -1 \\ 1 & 0 
		\end{bmatrix}, \begin{bmatrix}
			0 & 1 \\ -1 & 0
		\end{bmatrix}\right\}$ is a subgroup of $GL_2(\ZZ)$ isomorphic to $\left\{ 1, -1, i, -i \right\}.$
		\begin{proof}
			We first show that $G$ is a subgroup. Here, $1_G=\begin{bmatrix}
				1 & 0 \\ 0 & 1
			\end{bmatrix}$ is the identity in $GL_2(\ZZ).$ Then we have
			\begin{align*}
				\begin{bmatrix}
					-1 & 0 \\ 0 & -1
				\end{bmatrix} \begin{bmatrix}
					0 & -1 \\ 1 & 0 
				\end{bmatrix} &= \begin{bmatrix}
					0 & 1 \\ -1 & 0
				\end{bmatrix}\in G \\
				\begin{bmatrix}
					-1 & 0 \\ 0 & -1
				\end{bmatrix}\begin{bmatrix}
					0 & 1 \\ -1 & 0
				\end{bmatrix} &= \begin{bmatrix}
					0 & -1 \\ 1 & 0
				\end{bmatrix}\in G \\
				\begin{bmatrix}
					0 & -1 \\ 1 & 0
				\end{bmatrix}\begin{bmatrix}
					0 & 1 \\ -1 & 0
				\end{bmatrix} &= \begin{bmatrix}
					1 & 0 \\ 0 & 1
				\end{bmatrix}\in G
			\end{align*} 

			We can also find each element's inverse. From above, we have $\begin{bmatrix}
				0 & -1 \\ 1 & 0
			\end{bmatrix}$ and $\begin{bmatrix}
				0 & 1 \\ -1 & 0
			\end{bmatrix}$ are inverses of each other, and $\begin{bmatrix}
				-1 & 0 \\ 0 & -1
			\end{bmatrix}\inv=\begin{bmatrix}
				-1 & 0 \\ 0 & -1
			\end{bmatrix}.$ Thus, $G$ is indeed a subgroup.

			If we define the mapping $\alpha:G\to \left\{ 1, -1, i, -i \right\}$ such that
			\begin{align*}
				\alpha\left( \begin{bmatrix}
					1 & 0 \\ 0 & 1
				\end{bmatrix}\right)=1 & &\alpha\left( \begin{bmatrix}
					-1 & 0 \\ 0 & -1
				\end{bmatrix}\right)=-1 \\
				\alpha\left( \begin{bmatrix}
					0 & -1 \\ 1 & 0
				\end{bmatrix}\right)= i & & \alpha\left( \begin{bmatrix}
					0 & 1 \\ -1 & 0
				\end{bmatrix}\right)= -i
			\end{align*}
			it's clear that $\alpha$ is surjective and injective, and it satisfies the properties of a group homomorphism (easy to check). The inverse is just the same mapping in the other direction, so $G$ is isomorphic to $\left\{ 1, -1, i, -i \right\}$ as desired.
			
		\end{proof}

	\item[25.] Are the additive groups $\ZZ$ and $\QQ$ isomorphic? 
		\begin{proof}
			Suppose there exists an isomorphism $\varphi:\QQ\to\ZZ.$ Then since $\varphi$ is surjective, there exists a $q\in\QQ$ such that $\varphi(q)=1.$ Now consider $\varphi(q/2)+\varphi(q/2).$ Since $\varphi$ is a group homomorphism, this must equal $\varphi(q/2+q/2)=\varphi(q)=1.$ However, the equation $2\varphi(q/2)=1$ has no solution since $\varphi(q/2)\in\ZZ,$ Thus, there is no isomorphism from $\QQ$ to $\ZZ,$ so the two groups are not isomorphic.
			
		\end{proof}

	\item[33.] If $Z(G)=\left\{ 1 \right\},$ show that $G\cong\text{inn}G.$
		\begin{proof}
			Define the mapping $\varphi:G\to\text{inn}G$ such that $a\mapsto \sigma_a$ for all $a\in G$ where $\sigma_a(g)=aga\inv$ for all $g\in G.$ Clearly, $\varphi$ is surjective by the way we've defined it. To show $\varphi$ is injective, suppose $\varphi(a)=\varphi(b),$ so that $\sigma_a=\sigma_b.$ That means \[\sigma_a(g)=aga\inv=bgb\inv=\sigma_b(g)\] for all $g\in G.$ Manipulating, we have 
			\begin{align*}
				aga\inv &= bgb\inv \\
				ag &= bgb\inv a\\
				(b\inv a)g &= g(b\inv a) \\
			\end{align*} and since $Z(G)=\left\{ 1 \right\},$ it must be that $b\inv a=1,$ so $b=a,$ thus $\varphi$ is injective.

			Now, we must show $\varphi$ is a homomorphism. We have \[\varphi(a)\varphi(b)=\sigma_a\sigma_b\] and we must show this is equal to \[\varphi(ab)=\sigma_{ab}.\] That is, we must show \[\sigma_a(\sigma_b(g))=\sigma_{ab}(g)\] for all $g\in G.$ This is
			\begin{align*}
				\sigma_a(\sigma_b(g))&=\sigma_a(bgb\inv) \\
				&= abgb\inv a\inv \\
				&= (ab)g(ab)\inv \\
				&= \sigma_{ab}(g)
			\end{align*} so $\sigma_a\sigma_b=\sigma_{ab}$ and $\varphi$ is a homomorphism.

			Combining these, we have $\varphi$ is an isomorphism, so $G\cong \text{inn}G,$ as desired.

		\end{proof}

\end{itemize}

\section*{Section 2.6: Cosets and Lagrange's Theorem}
\begin{itemize}
	\item[1.] In each case find the right and left cosets in $G$ of the subgroups $H$ and $K$ of $G.$ 
		\begin{itemize}
			\item[(e)] $G=D_4=\left\{ 1, a, a^2, a^3, b, ba, ba^2, ba^3 \right\}, o(a)=4, o(b)=2,$ and $aba=b;$ $H=\left< a^2\right>, K=\left< b\right>.$
				\begin{soln}
					We have $H=\left\{ 1, a^2 \right\},$ so the left cosets of $H$ in $G$ are given by
					\begin{align*}
						1H &= \left\{ 1, a^2 \right\} \\
						aH &= \left\{ a, a^3 \right\} \\
						bH &= \left\{ b, ba^2 \right\} \\
						(ba)H &= \left\{ ba, ba^3 \right\}
					\end{align*} and the right cosets of $H$ in $G$ are 
					\begin{align*}
						H1 &= \left\{ 1, a^2 \right\} \\
						Ha &= \left\{ a, a^3 \right\} \\
						Hb &= \left\{ b, a^2b \right\} = \left\{ b, ba^2 \right\} \\
						H(ba) &= \left\{ ba, a^2ba \right\} = \left\{ ba, ba^3 \right\}
					\end{align*}

					We have $K=\left\{ 1, b \right\}$ so the left cosets of $K$ in $G$ are given by
					\begin{align*}
						1K &= \left\{ 1, b \right\} \\
						aK &= \left\{ a, ab \right\} = \left\{ a, ba^3 \right\} \\
						(a^2)K &= \left\{ a^2, a^2b \right\} = \left\{ a^2, ba^2 \right\} \\
						(a^3)K &= \left\{ a^3, a^3b \right\} = \left\{ a^3, ba \right\}
					\end{align*} and the right cosets of $K$ in $G$ are 
					\begin{align*}
						K1 &= \left\{ 1, b \right\} \\
						Ka &= \left\{ a, ba \right\} \\
						K(a^2) &= \left\{ a^2, ba^2 \right\} \\
						K(a^3) &= \left\{ a^3, ba^3 \right\}
					\end{align*}
				\end{soln}

			\item[(f)] $G=$ any group; $H$ is any subgroup of index 2.
				\begin{soln}
					Since the index of $H$ in $G$ is 2, there are two distinct cosets. Clearly $H1=1H=H$ is a coset, and since cosets are disjoint and partition $G,$ the other coset is exactly $G\setminus H.$ In this case, the left and right cosets are the same.
					
				\end{soln}
				
		\end{itemize}

	\item[17.] Let $|G|=p^2,$ where $p$ is a prime. Show that every proper subgroup of $G$ is cyclic.
		\begin{proof}
			By Lagrange's Theorem, if $H$ is a subgroup, then $|H|$ divides $|G|,$ so $|H|=1, p, p^2.$ The proper subgroups are where $|H|=p.$ By Corollary 3, $H$ is necessarily cyclic, as desired.
			
		\end{proof}
		
\end{itemize}

\end{document}

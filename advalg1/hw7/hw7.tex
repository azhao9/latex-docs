\documentclass{article}
\usepackage[sexy, hdr, fancy]{evan}
\setlength{\droptitle}{-4em}

\lhead{Homework 7}
\rhead{Advanced Algebra I}
\lfoot{}
\cfoot{\thepage}

\begin{document}
\title{Homework 7}
\maketitle
\thispagestyle{fancy}

\section*{Section 2.10: The Isomorphism Theorem}

\begin{itemize}
	\item[22.] Show that $\RR^*/\left\{ 1, -1 \right\}\cong \RR^+.$
		\begin{soln}
			Define the mapping $\varphi:\RR^*\to \RR^+$ given by $\varphi(x)=x^2$ for $x\in \RR^*.$ This is indeed a homomorphism:
			\begin{align*}
				\varphi(xy) &= (xy)^2 = x^2 y^2 = \varphi(x)\varphi(y)
			\end{align*} and the kernel is the set $\left\{ 1, -1 \right\}$ since $\varphi(1)=\varphi(-1)=1.$ Here, the image of $\RR^*$ under $\varphi$ is exactly $\RR^+,$ since the square of non-zero elements of $\RR$ are positive. Thus, by the Isomorphism theorem, \[\varphi``(\RR^*)=\RR^+\cong \RR^*/\ker \varphi = \RR^*/\left\{ 1, -1 \right\}\] as desired.
			
		\end{soln}

	\item[29.] Let $G=\left\{ \begin{bmatrix}
		1 & a & b \\ 0 & 1 & c \\ 0 & 0 & 1
	\end{bmatrix}\Bigg\vert a, b, c\in \RR\right\}.$ 
	\begin{enumerate}[(a)]
		\item Show that $G$ is a subgroup of $M_3(\RR)^*$ and that $Z(G)\cong \RR.$
			\begin{proof}
				Clearly $\begin{bmatrix}
					1 & 0 & 0 \\ 0 & 1 & 0 \\ 0 & 0 & 1
				\end{bmatrix}\in G$ which is the identity in $\GL_3(\RR).$ Then let
				\begin{align*}
					A &= \begin{bmatrix}
						1 & a & b \\
						0 & 1 & c \\
						0 & 0 & 1
					\end{bmatrix} \\
					M &= \begin{bmatrix}
						1 & m & n \\
						0 & 1 & p \\
						0 & 0 & 1
					\end{bmatrix}
				\end{align*} be in $G,$ so their product \[AM = \begin{bmatrix}
						1 & a & b \\
						0 & 1 & c \\
						0 & 0 & 1
					\end{bmatrix} \begin{bmatrix}
						1 & m & n \\
						0 & 1 & p \\
						0 & 0 & 1
				\end{bmatrix} = \begin{bmatrix}
				1 & m+a & n+ap+b \\
				0 & 1 & p+c \\
				0 & 0 & 1
				\end{bmatrix}\] is also in $G.$ Finally, the inverse of $A$ is given by
				\[A\inv = \begin{bmatrix}
				1 & -a & ac-b \\
				0 & 1 & -c \\
				0 & 0 & 1
				\end{bmatrix}\] which is also in $G.$ Thus, $G$ is a subgroup of $\GL_3(\RR_),$ as desired.

				Let $M\in Z(G).$ Then we have 
				\begin{align*}
					AM &= \begin{bmatrix}
						1 & m+a & n+ap+b \\
						0 & 1 & p+c \\
						0 & 0 & 1 
					\end{bmatrix} \\
					MA &= \begin{bmatrix}
						1 & a+m & b+mc+n \\
						0 & 1 & c+p \\
						0 & 0 & 1
					\end{bmatrix}
				\end{align*} so since $M\in Z(G),$ we must have $AM=MA,$ which is equivalent to having $n+ap+b=b+mc+n$ or $ap=mc.$ Since $a$ and $c$ can be anything, it must be the case that $m=p=0.$ Thus, the general form of $M\in Z(G)$ is \[M=\begin{bmatrix}
						1 & 0 & n \\ 0 & 1 & 0 \\ 0 & 0 & 1
					\end{bmatrix}, \quad n\in \RR
			\] and we can construct a mapping $\varphi:Z(G)\to\RR$ where \[\varphi\left( \begin{bmatrix}
					1 & 0 & n \\ 0 & 1 & 0 \\ 0 & 0 & 1
			\end{bmatrix}\right) = n\] which is obviously bijective. It is also a homomorphism because 
			\begin{align*}
				\varphi\left( \begin{bmatrix}
					1 & 0 & n \\ 0 & 1 & 0 \\ 0 & 0 & 1
				\end{bmatrix}\begin{bmatrix}
					1 & 0 & m \\ 0 & 0 & 1 \\ 0 & 0 & 1
				\end{bmatrix}\right) &= \varphi\left( \begin{bmatrix}
					1 & 0 & m+n \\ 0 & 1 & 0 \\ 0 & 0 & 1
				\end{bmatrix}\right) = m+n \\
				\varphi\left( \begin{bmatrix}
					1 & 0 & n \\ 0 & 1 & 0 \\ 0 & 0 & 1
				\end{bmatrix}\right)+\varphi\left( \begin{bmatrix}
					1 & 0 & m \\ 0 & 1 & 0 \\ 0 & 0 & 1
				\end{bmatrix}\right) &= m+n
			\end{align*}
			Thus $Z(G)\cong \RR,$ as desired.

		\end{proof}
		\item Show that $G/Z(G)\cong \RR\times\RR.$
			\begin{proof}
				Construct a mapping $\varphi:G\to \RR\times \RR$ where \[\varphi\left( \begin{bmatrix}
						1 & a & b \\ 0 & 1 & c \\ 0 & 0 & 1
				\end{bmatrix}\right) = (a, c)\] If we take $A$ and $M$ as above, then \[AM=\begin{bmatrix}
						1 & m+a & n+ap+b \\ 0 & 1 & p+c \\ 0 & 0 & 1
				\end{bmatrix}\] so $\varphi(AM)=(m+a, p+c)$ while \[\varphi(A)+\varphi(M)=(a, c) + (m, p) = (a+m, c+p)\] so $\varphi$ is a homomorphism where $\varphi``(G)=\RR\times\RR$ since $a, c\in \RR.$ Here, $\ker \varphi$ is exactly the set of matrices where $b\in \RR$ and $(a, c)=(0, 0),$ so $a=c=0,$ which is exactly $Z(G).$ Thus, by the Isomorphism theorem, \[\varphi``(G) = \RR\times \RR \cong G/\ker \varphi = G/Z(G)\] as desired.
				
			\end{proof}
			
	\end{enumerate}
		
\end{itemize} 
\section*{Section 8.2: Cauchy's Theorem}

\begin{itemize}
	\item[7.] If $H$ and $K$ are conjugate subgroups in $G,$ show that $N(H)$ and $N(K)$ are conjugate.
		\begin{proof}
			Let $H=g_0 K g_0\inv$ for some $g_0\in G.$ Then we have
			\begin{align*}
				N(H) &= \Set{g\in G}{gHg\inv=H} \\
				&= \Set{g\in G}{g(g_0 Kg_0\inv)g\inv = g_0 Kg_0\inv} \\
				&= \Set{g\in G}{(g_0\inv gg_0) K (g_0\inv g\inv g_0) = K} \\
				&= \Set{g\in G}{(g_0\inv gg_0) K (g_0\inv gg_0)\inv = K}
			\end{align*} so the set given by $g_0\inv N(H) g_0$ is exactly
			\begin{align*}
				g_0\inv N(H) g_0 &= \Set{g_0\inv gg_0 \in G}{(g_0\inv gg_0) K(g_0\inv gg_0)=K} = \Set{g_1\in G}{g_1 K g_1\inv = K} = N(K)
			\end{align*} so $N(H)=g_0 N(K) g_0\inv,$ thus $N(H)$ and $N(K)$ are conjugate, as desired.
			
		\end{proof}

	\item[14.] Let $D_3=\left\{ 1, a, a^2, b, ba, ba^2 \right\}$ where $o(a)=3, o(b)=2, aba=b.$ If $H=\left\{ 1, b \right\},$ show that $N(H)=H.$
		\begin{proof}
			Since $N(H)$ is a subgroup of $D_3$ so its order must divide 6. Since $H$ is a subgroup of $D_3,$ it is also a subgroup of $N(H),$ so $|H|=2\mid |N(H)|.$ Thus, $|N(H)|$ is even and divides 6, so $|N(H)|=2$ or 6. $N(H)$ can't possibly be all of $D_3$ since $ab\neq ba,$ so we must have $|N(H)|=2,$ so in fact $N(H)=H$ since $N(H)$ contains $H.$
			
		\end{proof}

	\item[23.] Let $G^{\omega}$ be the group of sequences $[g_i)=(g_0, g_1, \cdots)$ from a group $G$ with component-wise multiplication $[g_i)\cdot [h_i)=[g_ih_i).$ Show that if $G\neq \left\{ 1 \right\}$ is a finite $p$-group, then $G^{\omega}$ is an infinite $p$-group.
			\begin{proof}
				Since $G$ is a finite $p$-group, suppose $\abs{G}=p^n$ for some $n\ge 1.$ Thus, $g^{p^n}=1$ for any $g\in G.$ Thus, for any sequence $[g_i)=(g_0, g_1, \cdots),$ we have
					\begin{align*}
						[g_i)^{p^n} &= (g_0^{p^n}, g_1^{p^n}, \cdots) \\
							&=  (1, 1, \cdots)
						\end{align*} thus $o\left([g_i)\right)$ must divide $p^n$ for any $[g_i)\in G^{\omega},$ so $G^{\omega}$ is a $p$-group. Clearly $G^{\omega}$ is not a finite group, in fact it is not even countable.
							
			\end{proof}

	\item[26.] Let $G$ be a non-abelian group of order $p^3$ where $p$ is a prime. Show that
		\begin{enumerate}[(a)]
			\item $Z(G)=G'$ and this is the unique normal subgroup of $G$ of order $p.$
				\begin{proof}
					Since $Z(G)$ is a subgroup of $G,$ we must have $\abs{Z(G)}\mid p^3.$ By Theorem 6, since $G$ is a finite $p$-group, its center is not $\left\{ 1 \right\}.$ Thus, we must have one of the following: \[\abs{Z(G)} = p, p^2, p^3\] 
					
					We can't have $\abs{Z(G)}=p^3$ because then $Z(G)=G$ but we assumed $G$ was non-abelian. If $\abs{Z(G)}=p^2,$ then $G/Z(G)$ is a well-defined group since $Z(G)\unlhd G.$ However, this means the $G/Z(G)$ has order $p^3/p^2=p,$ so $G/Z(G)$ must be an abelian cyclic group. This implies $G$ was abelian to begin with, which is a contradiction. 

					Thus, we must have $\abs{Z(G)}=p.$ Since $Z(G)$ is normal in $G,$ it is the only subgroup of order $p,$ as desired.
					
				\end{proof}

			\item $G$ has exactly $p^2+p-1$ distinct conjugacy classes.
				\begin{proof}
					By the class equation, we have \[\abs{G} = \abs{Z(G)}+\sum_{}^{}\abs{G:N(a_i)}\] where $a_i$ are the representatives of the conjugacy classes. We must have $\abs{G:N(a_i)}$ divides the order of $G,$ so $\abs{G:N(a_i)}=p, p^2, p^3$ since we assume $a_i$ is not in the center. We can't have it be $p^3,$ since the center is non-empty. If $\abs{G:N(a_i)}=p^2$ for some $a_i,$ then \[\frac{\abs{G}}{\abs{N(a_i)}} = \frac{p^3}{\abs{N(a_i)}} = p^2 \implies \abs{N(a_i)}=p\] Thus, since $Z(G)$ is the unique subgroup of order $p,$ we must have $N(a_i)=Z(G).$ Thus, $a_i\in Z(G),$ but we assumed otherwise, contradiction.

					Thus, we must have $\abs{G:N(a_i)}=p$ for all $a_i$ representatives not in the center. By the class equation, we have \[p^3=p+\sum_{}^{}p\implies p^3-p=p(p^2-1)=\sum_{}^{}p\] so there are $p^2-1$ distinct conjugacy classes that are not in the center, and $\abs{Z(G)}=p,$ so there are $p$ singleton conjugacy classes. Thus, the total number of conjugacy classes is $p^2+p-1,$ as desired.
					
				\end{proof}
				
		\end{enumerate}
		
\end{itemize}

\section*{Section 8.3: Group Actions}

\begin{itemize}
	\item[3.] If $p$ and $q$ are primes, show that no group of order $pq$ is simple.
		\begin{proof}
			By Cauchy's theorem, $p$ divides $pq$ so there exists an element of order $p.$ Suppose $o(g)=p$ for some $g\in G.$ Then $\abs{\left< g\right>}=p,$ so $\abs{G:\left< g\right>}=q.$ WLOG $q\le p,$ so by Corollary 1 of Theorem 1, $\left< g\right>\unlhd G,$ so $G$ is not simple.
			
		\end{proof}

	\item[13.] Let $G=\left( \RR, + \right)$ and define $a\cdot z=e^{ia}z$ for all $z\in\CC$ and $a\in G.$ Show that $\CC$ is a $G$-set, describe the action geometrically, and find all orbits and stabilizers.
		\begin{proof}
			We have $0\in \RR$ is the identity, so $0\cdot z = e^{i\cdot 0} z = z.$ If $a, b\in\RR,$ we have
			\begin{align*}
				a\cdot(b\cdot z) &= a\cdot (e^{ib}z) = e^{ia}(z e^{ib}) = e^{i(a+b)}z \\
				(ab)\cdot z &= (a+b)\cdot z = e^{i(a+b)}z
			\end{align*}
			Thus, $G$ acts on $\CC$ so $\CC$ is a $G$-set, as desired. 

			This action is equivalent to a rotation by an angle $a\in \RR$ where $a$ is in radians. The orbits are the concentric circles about the origin with $r\ge 0,$ each circle being an orbit. The stabilizers are of the form $2k\pi\in \RR$ where $k\in\ZZ,$ since a rotation by a multiple of $2\pi$ does not change the position.
			
		\end{proof}

		\newpage
	\item[21.] If $H$ is a subgroup of $G,$ find a $G$-set $X$ and an element $x\in X$ such that $H=S(x).$
		\begin{soln}
			Let $X=G,$ and define the group action as follows for all $x\in G:$
			\[g\cdot x = \begin{cases}
					x \quad\text{if } g\in H \\
					1 \quad\text{if } g\not\in H
			\end{cases}\]
			We may check that this is indeed a group action. We have $1\cdot x = x$ since $1\in H$ because $H$ is a subgroup. Next, consider two elements $g, h\in G.$ There are 4 cases:
			\begin{align*}
				\tag{1} g, h &\in H\implies gh\in H \\
				\implies g\cdot (h\cdot x) &= g\cdot x = x = (gh)\cdot x \\
				\tag{2} g&\in H, h\not\in H\implies gh\not\in H \\
				\implies g\cdot (h\cdot x) &= g\cdot 1 = 1 = (gh)\cdot x \\
				\tag{3} g&\not\in H, h\in H\implies gh\not\in H \\
				\implies g\cdot (h\cdot x) &= g\cdot x = 1 = (gh)\cdot x \\
				\tag{4} g, h &\not\in H\implies gh\not\in H \\
				\implies g\cdot (h\cdot x) &= g\cdot 1 = 1 = (gh)\cdot x
			\end{align*} thus, $X=G$ is indeed a $G$-set under this action. Then, for any $h\in G$ such that $h\neq 1,$ the stabilizer $S(h)$ all elements $g\in G$ such that $g\cdot h = h.$ The $g$ that fit this are the $g\in H,$ so $H=S(h),$ as desired.
			
		\end{soln}

	\item[23.] Let $X$ be a $G$-set and let $x$ and $y$ denote elements of $X.$
		\begin{enumerate}[(a)]
			\item Show that $S(x)$ is a subgroup of $G.$
				\begin{proof}
					Clearly $1_g\cdot x = x$ because $X$ is a $G$-set. Next, if $g, h\in S(x),$ we have 
					\begin{align*}
						g\cdot x &= x = h\cdot x\\
						\implies g\cdot(h\cdot x) &= x \\
						\implies (gh)\cdot x &= x
					\end{align*} so $gh\in S(x)$ as well. Finally, if $g\in S(x),$ we have
					\begin{align*}
						g\cdot x &= x \\
						\implies g\inv(g\cdot x) &= g\inv\cdot x \\
						\implies x &= g\inv\cdot x
					\end{align*} so $g\inv\in S(x)$ as well. Thus, $S(x)$ is a subgroup of $G$, as desired
					
				\end{proof}

			\item If $x\in X$ and $b\in G,$ show that $S(b\cdot x)=bS(x)b\inv.$
				\begin{proof}
					We have 
					\begin{align*}
						S(b\cdot x) &= \Set{g\in G}{g\cdot (b\cdot x) = b\cdot x} \\
						&= \Set{g\in G}{(gb)\cdot x = b\cdot x} \\
						&= \Set{g\in G}{(b\inv gb)\cdot x = x} \\
						b\inv S(b\cdot x)b &= \Set{b\inv gb\in G}{(b\inv gb)\cdot x = x} \\
						&= \Set{h\in G}{h\cdot x = x} \\
						&= S(x)
					\end{align*} so $S(b\cdot x) = bS(x)b\inv$ as desired.
					
				\end{proof}

			\item If $S(x)$ and $S(y)$ are conjugate subgroups, show that $\abs{G\cdot x}=\abs{G\cdot y}.$
				\begin{proof}
					Let $S(x)=aS(y)a\inv$ for some $a\in G.$ Then define the mapping \[\varphi:G\cdot x\to G\cdot y\] by $\varphi(g\cdot x) = (ag)\cdot x.$ This is well defined: if $g_1\cdot x = g_2\cdot x,$ then $g_2 g_1\inv\in S(x),$ so the conjugate \[a(g_2 g_1\inv)a\inv = (ag_2)(ag_1)\inv \in S(y)\] so \[(ag_2)\cdot x = (ag_1)\cdot x\] This also means that $\varphi$ is 1-1 because we can simply recover the $g_1$ and $g_2.$ Clearly $\varphi$ is surjective because for any $(ag)\cdot x\in G\cdot y,$ we can recover the $g\cdot x\in G\cdot x.$ Thus, $\varphi$ is a bijective map, so the two groups have the same cardinality.
					
				\end{proof}
				
		\end{enumerate}
		
\end{itemize}

\end{document}

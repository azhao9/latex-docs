\documentclass{article}
\usepackage[sexy, hdr, fancy]{evan}
\setlength{\droptitle}{-4em}

\lhead{Homework 7}
\rhead{Advanced Algebra I}
\lfoot{}
\cfoot{\thepage}

\begin{document}
\title{Homework 7}
\maketitle
\thispagestyle{fancy}

\section*{Section 2.10: The Isomorphism Theorem}

\begin{itemize}
	\item[22.] Show that $\RR^*/\left\{ 1, -1 \right\}\cong \RR^+.$

	\item[29.] Let $G=\left\{ \begin{bmatrix}
		1 & a & b \\ 0 & 1 & c \\ 0 & 0 & 1
	\end{bmatrix}\Bigg\vert a, b, c\in \RR\right\}.$ 
	\begin{enumerate}[(a)]
		\item Show that $G$ is a subgroup of $M_3(\RR)^*$ and that $Z(G)\cong \RR.$

		\item Show that $G/Z(G)\cong \RR\times\RR.$
			
	\end{enumerate}
		
\end{itemize}

\section*{Section 8.2: Cauchy's Theorem}

\begin{itemize}
	\item[7.] If $H$ and $K$ are conjugate subgroups in $G,$ show that $N(H)$ and $N(K)$ are conjugate.

	\item[14.] Let $D_3=\left\{ 1, a, a^2, b, ba, ba^2 \right\}$ where $o(a)=3, o(b)=2, aba=b.$ If $H=\left\{ 1, b \right\},$ show that $N(H)=H.$

	\item[23.] Let $G^{\omega}$ be the group of sequences $[g_i)=(g_0, g_1, \cdots)$ from a group $G$ with component-wise multiplication $[g_i)\cdot [h_i)=[g_ih_i).$ Show that if $G\neq \left\{ 1 \right\}$ is a finite $p$-group, then $G^{\omega}$ is an infinite $p$-group.

	\item[26.] Let $G$ be a non-abelian group of order $p^3$ where $p$ is a prime. Show that
		\begin{enumerate}[(a)]
			\item $Z(G)=G'$ and this is the unique normal subgroup of $G$ of order $p.$

			\item $G$ has exactly $p^2+p-1$ distinct conjugacy classes.
				
		\end{enumerate}
		
\end{itemize}

\section*{Section 8.3: Group Actions}

\begin{itemize}
	\item[3.] If $p$ and $q$ are primes, show that no group of order $pq$ is simple.

	\item[13.] Let $G=\left( \RR, + \right)$ and define $a\cdot z=e^{ia}z$ for all $z\in\CC$ and $a\in G.$ Show that $\CC$ is a $G$-set, describe the action geometrically, and find all orbits and stabilizers.

	\item[21.] If $H$ is a subgroup of $G,$ find a $G$-set $X$ and an element $x\in X$ such that $H=S(x).$

	\item[23.] Let $X$ be a $G$-set and let $x$ and $y$ denote elements of $X.$
		\begin{enumerate}[(a)]
			\item Show that $S(X)$ is a subgroup of $G.$

			\item If $x\in X$ and $b\in G,$ show that $S(b\cdot x)=bS(x)b\inv.$

			\item If $S(x)$ and $S(y)$ are conjugate subgroups, show that $\abs{G\cdot x}=\abs{G\cdot y}.$
				
		\end{enumerate}
		
\end{itemize}

\end{document}

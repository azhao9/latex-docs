\documentclass{article}
\usepackage[sexy, hdr, fancy]{evan}
\setlength{\droptitle}{-4em}

\lhead{Homework 9}
\rhead{Advanced Algebra I}
\lfoot{}
\cfoot{\thepage}

\begin{document}
\title{Homework 9}
\maketitle
\thispagestyle{fancy}

\section*{Section 3.1: Examples and Basic Properties}

\begin{itemize}
	\item[1.] In each case explain why $R$ is not a ring.
		\begin{enumerate}[(a)]
			\item $R=\left\{ 0, 1, 2, 3, \cdots \right\},$ operations of $\ZZ.$
				\begin{answer*}
					$R$ does not contain the additive inverses.
				\end{answer*}

			\item $R=2\ZZ.$
				\begin{answer*}
					$R$ does not contain a multiplicative identity.
				\end{answer*}

			\item $R=$ the set of all mappings $f:\RR\to\RR;$ addition is point-wise but using composition as the multiplication.
				\begin{answer*}
					If $f, g, h\in R,$ the condition $f(g+h)=fg+fh$ does not hold. For example, if $f(x)=\sqrt{x}$ and $g(x)=h(x)=x,$ we have $f(g+h)=\sqrt{2x}$ but $fg+gh=2\sqrt{x},$ and the two are not equal.
				\end{answer*}
				
		\end{enumerate}

	\item[3.] (c) Show that $S=\left\{ \begin{bmatrix}
		a & 0 & b \\ 0 & c & d \\ 0 & 0 & a
	\end{bmatrix}\Bigg\vert a, b, c, d\in \RR\right\}$ is a subring of $R=M_3(\RR).$
		\begin{proof}
			We have \[0_R=\begin{bmatrix}
					0 & 0 & 0 \\ 0 & 0 & 0 \\ 0 & 0 & 0
			\end{bmatrix}\in S\] and \[1_R=\begin{bmatrix}
					1 & 0 & 0 \\ 0 & 1 & 0 \\ 0 & 0 & 1
			\end{bmatrix}\in S\] Now, let \[S=\begin{bmatrix}
					a & 0 & b \\ 0 & c & d \\ 0 & 0 & a
				\end{bmatrix}, \quad T =\begin{bmatrix}
					w & 0 & x \\ 0 & y & z \\ 0 & 0 & w
			\end{bmatrix}\] so then
			\begin{align*}
				S-T &= \begin{bmatrix}
					a-w & 0 & b-x \\ 
					0 & c-y & d-z \\
					0 & 0 & a-w
				\end{bmatrix}\in S \\
				ST &= \begin{bmatrix}
					aw & 0 & ax+bw \\
					0 & cy & cz+dw \\
					0 & 0 & aw
				\end{bmatrix}\in S
			\end{align*} 
			Thus, $S$ is a subring of $R,$ as desired.
			
		\end{proof}
		
\end{itemize}

\section*{Section 3.2: Integral Domains and Fields}

\begin{itemize}
	\item[1.] Find all the roots of $x^2+3x-4$ in
		\begin{enumerate}[(a)]
			\item $\ZZ$
				\begin{soln}
					This quadratic factors as \[x^2+3x-4=(x+4)(x-1)\] so the roots are $1, -4\in \ZZ.$
				\end{soln}

			\item $\ZZ_6$
				\begin{soln}
					Similarly to part (a), this quadratic factors as \[x^2+3x-4=(x+4)(x-1)\] Thus $x+4=\bar{0}$ so $x=\bar{2}$ and $x-1=0$ so $x=\bar{1}$ are solutions. 
				\end{soln}

			\item $\ZZ_4$
				\begin{soln}
					Similarly to part (a), this quadratic factors as \[x^2+3x-4=(x+4)(x-1)\] Thus $x+4=\bar{0}$ so $x=\bar{0}$ and $x-1=0$ so $x=\bar{1}$ are solutions. 
				\end{soln}
				
		\end{enumerate}

	\item[5.] Show that $M_n(R)$ is never a domain if $n\ge 2.$
		\begin{proof}
			Consider the element $A=\begin{bmatrix}
				0 & \cdots & r \\
				\vdots & \ddots & \vdots \\
				0 & \cdots & 0
			\end{bmatrix}\in M_n(R)$ where $0\neq r\in R.$ Then $A^2$ is a matrix of all 0's, but $A$ itself is not a matrix of all 0's. Thus, $M_n(R)$ is never a domain if $n\ge 2.$
		\end{proof}

	\item[10.] If $F=\left\{ 0, 1, a, b \right\}$ is a field, fill in the addition and multiplication tables for $F.$
		\begin{soln}
			Since $F$ is a field, it must contain the multiplicative inverses of $a$ and $b.$ Thus, $a\inv=b$ and vice versa. Similarly, it must contain the additive inverses of $a$ and $b,$ which are again each other. 
		\end{soln}<++>
		
\end{itemize}

\section*{Section 3.3: Ideals and Factor Rings}

\begin{itemize}
	\item[1.] (a) Decide whether $\ZZ$ is an ideal of $\CC.$ Support your answer.
		\begin{soln}
			$\ZZ$ is not an ideal of $\CC.$ Consider $z=1+i\in\CC.$ Then if $a=2\in \ZZ$ we have $az=2+i\not\in\ZZ.$
		\end{soln}

	\item[4.] (a) If $m$ is an integer, show that $mR=\Set{mr}{r\in R}$ and $A_m=\Set{r\in R}{mr=0}$ are ideals of $R.$

	\item[6.] If $A$ is an ideal of $R,$ show that $M_2(A)$ is an ideal of $M_2(R).$
		
\end{itemize}

\section*{Section 3.4: Homomorphisms}

\begin{itemize}
	\item[1.] In each case determine whether the map $\theta$ is a ring homomorphism. Support your answer.
		\begin{enumerate}[(a)]
			\item $\theta:\ZZ_3\to\ZZ_{12},$ where $\theta(r)=4r.$

			\item $\theta:\ZZ_4\to\ZZ_{12},$ where $\theta(r)=3r.$

			\item $\theta:R\times R\to R,$ where $\theta(r, s)=r+s.$

			\item $\theta:R\times R\to R,$ where $\theta(r, s)=rs.$

			\item $\theta:F(\RR, \RR)\to \RR,$ where $\theta(f)=f(1).$
				
		\end{enumerate}

	\item[15.] If $\sigma:R\to S$ is a ring isomorphism, show that the same is true of the inverse map $\sigma\inv:S\to R.$
		
\end{itemize}

\end{document}

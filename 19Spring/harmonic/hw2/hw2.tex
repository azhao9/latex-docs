\documentclass{article}
\usepackage[sexy, hdr, fancy]{evan}
\setlength{\droptitle}{-4em}

\lhead{Homework 2}
\rhead{Harmonic Analysis}
\lfoot{}
\cfoot{\thepage}

\begin{document}
\title{Homework 2}
\maketitle
\thispagestyle{fancy}

\begin{enumerate}
  \item Consider the $2\pi$-period odd function defined on $[0, \pi]$ by $f(\theta)=\theta(\pi-\theta).$ NOTE: Discussed with Mauro, I am using the $\pi$-periodic even function defined on just $[0, \pi].$ The results will not be identical but they should be okay.

    \begin{enumerate}[(a)]
	\ii Draw the graph of $f$
	\begin{center}
	  \begin{asy}
	    settings.outformat="pdf";
	    unitsize(2cm);
	    import graph;
	    real f(real x) {
	      return x*(2-x);
	    }

	    real g(real x) {
	      return -x*(2+x);
	    }

	    real k(real x) {
	      return -(x+2)*(x+4);
	    }

	    real j(real x) {
	      return -(x-2)*(x-4);
	    }

	    path p1 = graph(f, 0, 2);
	    path p2 = graph(g, 0, -2);
	    path p3 = graph(j, 2, 3);
	    path p4 = graph(k, -2, -3);
	    draw(p1);
	    draw(p2);
	    draw(p3);
	    draw(p4);

	    draw( (-3, 0) -- (3, 0));
	    draw( (0, 1.5) -- (0, -0.5));
	    label("$\pi$", (2, 0), S);
	    label("$-\pi$", (-2, 0), S);
	    label("0", origin, SE);
	    
	  \end{asy}

	\end{center}	

	\ii Compute the Fourier coefficients of $f,$ and show that
	\begin{align*}
	  f(\theta)=\frac{8}{\pi} \sum_{k\text{ odd }\ge1}^{} \frac{\sin k\theta}{k^3}
	\end{align*}

	Is this function continuous on the circle? Continuously differentiable on the circle? In $L^2$ of the circle?
	\begin{soln}
	  We have
	  \begin{align*}
	    \hat f(n) &= \frac{1}{\pi} \int_0^\pi \theta(\pi-\theta) e^{-2in\theta}\, d\theta = \int_0^\pi \theta e^{-2in\theta}\, d\theta - \frac{1}{\pi}\int_0^\pi \theta^2 e^{-2in\theta}\, d\theta
	  \end{align*}

	  Integrating by parts, we have
	  \begin{align*}
	    \int_0^\pi \theta e^{-2in\theta}\, d\theta &= -\frac{\theta}{2in} e^{-2in\theta}\bigg\vert_0^\pi - \int_0^\pi -\frac{1}{2in} e^{-2in\theta}\, d\theta \\
	    &= -\frac{\pi}{2in} + \frac{1}{4n^2}e^{-2in\theta}\bigg\vert_0^\pi = -\frac{\pi}{2in} \\
	    \int_0^\pi \theta^2 e^{-2in\theta}\, d\theta &= -\frac{\theta^2}{2in}e^{-2in\theta}\bigg\vert_0^\pi - \int_0^\pi -\frac{1}{2in}e^{-2in\theta}\cdot 2\theta\, d\theta \\
	    &= -\frac{\pi^2}{2in} + \frac{1}{in}\int_0^\pi \theta e^{-2in\theta}\, d\theta = -\frac{\pi^2}{2in} + \frac{1}{in} \cdot \left( -\frac{\pi}{2in} \right) = -\frac{\pi^2}{2in} + \frac{\pi}{2n^2} \\
	    \implies \hat f(n) &= -\frac{\pi}{2in} - \frac{1}{\pi}\left( -\frac{\pi^2}{2in} + \frac{\pi}{2n^2} \right) = -\frac{1}{2n^2}
	  \end{align*}

	  Given a fixed $n\neq0,$ we have
	  \begin{align*}
	    \hat g(n, \theta) &= \hat f(n)e^{2in\theta} - \hat f(-n) e^{-2in\theta} = -\frac{1}{2n^2} \left( \cos 2n\theta + i\sin 2n\theta \right) + \frac{1}{2(-n)^2}\left( \cos(-2n\theta) + i\sin(-2n\theta) \right) \\
	    &= -\frac{1}{n^2}\cos2n\theta
	  \end{align*}
	  due to the even and odd properties of $\cos$ and $\sin,$ respectively. We also have
	  \begin{align*}
	    \hat f(0) &= \frac{1}{\pi}\int_0^\pi \theta(\pi-\theta)\, d\theta = \frac{1}{\pi}\left( \frac{\theta^2\pi}{2} - \frac{\theta^3}{3} \right)\bigg\vert_0^\pi = \frac{1}{\pi} \left( \frac{\pi^3}{2} - \frac{\pi^3}{3} \right) = \frac{\pi^2}{6}
	  \end{align*}

	  and thus the Fourier series for $f$ is given by
	  \begin{align*}
	    \sum_{n=-\infty}^{\infty} \hat f(n)e^{2in\theta} = \frac{\pi^2}{6} + \sum_{k=1}^{\infty} \hat g(k, \theta) = \frac{\pi^2}{6} - \sum_{k=1}^{\infty} \frac{1}{n^2}\cos 2n\theta
	  \end{align*}

	  Now, since $f$ is continuous, we have
	  \begin{align*}
	    \sum_{k=-\infty}^{\infty} \abs{\hat f(k)} = 2\sum_{k=1}^{\infty} \abs{\frac{1}{n^2}} + \abs{\hat f(0)} = 2\cdot\frac{\pi^2}{6}+\frac{\pi^2}{6} < \infty
	  \end{align*}
	  and thus the Fourier series converges uniformly to $f.$ 

	  It is not continuously differentiable, it has a cusp at $k\pi$ for $k\in\ZZ.$ It is also in $L^2$ (trivial to show, just integrating a quartic polynomial).
	\end{soln}

    \end{enumerate}

  \item Prove that if $f$ is an even $2\pi$-period function, i.e. $f(\theta)=f(-\theta)$ for all $\theta\in[-\pi, \pi],$ then the Fourier series can be written as a cosine series. 
    \begin{proof}
      We have
      \begin{align*}
	\hat f(n) &= \frac{1}{2\pi}\int_{-\pi}^\pi f(x) e^{-inx}\, dx \\
	\hat f(-n) &= \frac{1}{2\pi}\int_{-\pi}^\pi f(y) e^{iny}\, dy
      \end{align*}
      We can see that these two integrals are equal with the change of variable $y=-x, dy=-dx,$ since
      \begin{align*}
	\hat f(-n) &= \frac{1}{2\pi} \int_{\pi}^{-\pi} -f(-x) e^{-inx}\, dx = \frac{1}{2\pi} \int_{-\pi}^\pi f(-x) e^{-inx}\, dx = \hat f(n)
      \end{align*}
      due to the even-ness of $f.$ Thus, for any $n\neq 0,$ we have
      \begin{align*}
	\hat f(n)e^{in\theta} + \hat f(-n) e^{-in\theta} &= \hat f(n) \left[ \left( \cos n\theta + i\sin n\theta \right) + \left( \cos (-n\theta) + i\sin (-n\theta) \right) \right] \\
	&= \hat f(n) \left[ \left( \cos n\theta + i\sin n\theta \right) + \left( \cos n\theta - i\sin n\theta \right) \right] \\
	&= 2\hat f(n) \cos n\theta
      \end{align*}

      Thus, in the infinite sum expansion of the Fourier series for $f,$ we see that it will only contain cosine terms, as desired. 
    \end{proof}

\end{enumerate}

\end{document}

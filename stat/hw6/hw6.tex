\documentclass{article}
\usepackage[sexy, hdr, fancy]{evan}
\usepackage{bm}
\setlength{\droptitle}{-4em}

\lhead{Homework 6}
\rhead{Introduction to Statistics}
\lfoot{}
\cfoot{\thepage}

\newcommand{\var}{\mathrm{Var}}
\newcommand{\cov}{\mathrm{Cov}}

\begin{document}
\title{Homework 6}
\maketitle
\thispagestyle{fancy}

\begin{enumerate}
	\item Prove the converse of the factorization theorem, namely prove that if $T$ is a sufficient statistic, then the joint density can be factored as \[f(x_1, \cdots, x_n\mid\theta)=g(T, \theta)h(x_1, \cdots, x_n)\] Also show that if $T$ is sufficient for $\theta,$ then the MLE must be a function of $T.$
		\begin{proof}
			If $T$ is a sufficient statistic, then by definition the distribution of $X_1, \cdots, X_n$ given $T=t$ does not depend on $\theta$ and only depends on the data $X_1,\cdots X_n.$ Thus, let \[f(X_1, \cdots, X_n\mid T) = h(X_1, \cdots, X_n)\] be a function of just the data. Next, let \[g(T, \theta) = g(T(X_1, \cdots, X_n), \theta) = f(T(X_1, \cdots, X_n)\mid \theta)=f(T\mid \theta)\] be the density of $T.$ Thus, we have
			\begin{align*}
				g(T, \theta) h(X_1, \cdots, X_n) &= f(X_1, \cdots, X_n\mid T) f(T\mid \theta) \\
				&= f(X_1, \cdots, X_n\mid \theta)
			\end{align*} as desired.

		\end{proof}

	\item Let $\hat{\theta}$ be an estimator for a parameter $\theta,$ and suppose that $\var(\hat{\theta})<\infty.$ Let $T$ be a sufficient statistic for $\theta.$ Consider the random variable \[Y=E[\hat{\theta}\mid T]\] Prove that \[E\left[ \left( Y-\theta \right)^2 \right]\le E\left[ \left( \hat{\theta}-\theta \right)^2 \right]\] Explain why this suggests that a sufficient statistic can be particularly useful in parametric estimation.
		\begin{proof}
			We have \[E[Y]=E\left[ E[\hat{\theta}\mid T] \right] = E[\hat{\theta}]\] so
			\begin{align*}
				E\left[ (Y-\theta)^2 \right] &= E[Y^2]-2\theta E[Y] + \theta^2 \\
				&= E[Y^2] - 2\theta E[\hat{\theta}]+\theta^2 \\
				E\left[ \left( \hat{\theta}-\theta \right)^2 \right] &= E[\hat{\theta}^2] - 2\theta E[\hat{\theta}] + \theta^2
			\end{align*}
			so we must show that $E[Y^2] \le E[\hat{\theta}^2].$

			We have \[ E[\hat{\theta}^2]-E[\theta]^2 = \var(\hat{\theta})=\var(E[\hat{\theta}\mid T])+E[\var(\hat{\theta}\mid T)]\] Thus, we have
			\begin{align*}
				E[\hat{\theta}^2] &= E[\hat{\theta}]^2+\var(Y)+E[\var(\hat{\theta}\mid T)] \\
				&= E[\hat{\theta}]^2+E[Y^2]-E[Y]^2+E[\var(\hat{\theta}\mid T)] \\
				&= E[Y^2] + E[\var(\hat{\theta}\mid T)]
			\end{align*} since $E[\hat{\theta}] = E[Y].$ Since a variance is always non-negative, it follows that $E[\var(\hat{\theta}\mid T)]\ge 0,$ so \[E[\hat{\theta}^2]\ge E[Y^2]\] as desired.

			We have $\hat{\theta}=\hat{\theta}(X_1, \cdots, X_n)$ is a function of the data. Thus, the density of $\hat{\theta}$ is the same as the joint density $f(X_1, \cdots, X_n).$ Since $T$ is sufficient for $\theta,$ it follows that $f(X_1, \cdots, X_n\mid T)$ is a function of purely the data, and therefore, the density of $(\hat{\theta}\mid T)$ is a function of purely the data. Thus, $E[\hat{\theta}\mid T]$ does not depend on $\theta,$ so it can be said to be an estimator for $\theta.$ Our result shows that no matter what $\hat{\theta}$ is, the MSE of $E[\hat{\theta}\mid T]$ is at most the MSE of $\hat{\theta},$ so it is a better estimator. 
			
		\end{proof}

	\item Complete all the details of the example we discussed in lecture. Let $X_1, \cdots, X_n$ be iid data from a normal distribution with unknown mean $\theta$ and known variance $\sigma^2.$ Suppose that $\theta$ is assumed to be random, with prior distribution also normal; assume that the mean and variance of the prior distribution of $\theta_0$ and $\sigma_{pr}^2,$ where both $\theta_0$ and $\sigma_{pr}^2$ are known.

		\begin{enumerate}[(a)]
			\item Compute the posterior distribution \[f_{\theta\mid\bm{X}} (\theta\mid x_1, \cdots, x_n)\] where $\bm{X}=(X_1, \cdots, X_n),$ and specify all the parameters of this distribution.
				\begin{soln}
					The posterior distribution is given by
					\begin{align*}
						f_{\theta|X}(\theta\mid X_1, \cdots, X_n) &= \frac{f_{X|\theta}(X_1, \cdots, X_n\mid \theta)f(\theta)}{\displaystyle\int f_{X|\theta}(X_1, \cdots, X_n\mid \theta)f(\theta)\, d\theta} \\
						&= \frac{\displaystyle\prod_{i=1}^{n} \frac{1}{\sqrt{2\pi}\sigma}\exp\left( -\frac{(X_i-\theta)^2}{2\sigma^2} \right) \frac{1}{\sqrt{2\pi}\sigma_{pr}} \exp\left( -\frac{(\theta-\theta_0)^2}{2\sigma_{pr}^2} \right)}{\displaystyle\int \displaystyle\prod_{i=1}^{n} \frac{1}{\sqrt{2\pi}\sigma}\exp\left( -\frac{(X_i-\theta)^2}{2\sigma^2} \right) \frac{1}{\sqrt{2\pi}\sigma_{pr}} \exp\left( -\frac{(\theta-\theta_0)^2}{2\sigma_{pr}^2} \right)\, d\theta} \\
						&= \frac{\displaystyle\exp\left( -\frac{1}{2\sigma^2}\sum_{i=1}^{n}(X_i-\theta)^2 - \frac{1}{2\sigma_{pr}^2}(\theta-\theta_0)^2 \right)}{\displaystyle\int \exp\left( -\frac{1}{2\sigma^2}\sum_{i=1}^{n}(X_i-\theta)^2 - \frac{1}{2\sigma_{pr}^2}(\theta-\theta_0)^2 \right)\, d\theta}
					\end{align*}
					The numerator can be expanded to
					\begin{align*}
						\exp\left( -\frac{1}{2\sigma^2}\left[ \sum_{i=1}^{n} X_i^2 - 2\theta\sum_{i=1}^{n}X_i + n\theta^2 \right] - \frac{1}{2\sigma_{pr}^2}\left[ \theta^2-2\theta_0\theta+\theta_0^2] \right] \right)
					\end{align*} and similarly for the integrand in the denominator, and in this case, we may cancel anything not involving $\theta$ since the integral treats those as constants. We may write $\sum_{}^{}X_i=n\bar{X}$ so the numerator (and integrand) is 
					\[ \exp\left( \frac{2\theta n\bar{X}-n\theta^2}{2\sigma^2}+\frac{2\theta_0\theta-\theta^2}{2\sigma_{pr}^2} \right)\] 
					From here, we wish to complete the square within the exponent with respect to $\theta:$
					\begin{align*}
						\frac{2\theta n\bar{X}-n\theta^2}{2\sigma^2}+\frac{2\theta_0 \theta-\theta^2}{2\sigma_{pr}^2} &= \frac{2\theta n\bar{X}\sigma_{pr}^2-n\theta^2\sigma_{pr}^2+2\theta_0\theta\sigma^2-\sigma^2\theta^2}{2\sigma^2\sigma_{pr}^2} \\
						&= -\frac{1}{2\sigma^2\sigma_{pr}^2}\left[ \theta^2\left( n\sigma_{pr}^2+\sigma^2 \right) - \theta\left( 2n\bar{X}\sigma_{pr}^2+2\theta_0\sigma^2 \right) \right] \\
						&= -\frac{n\sigma_{pr}^2+\sigma^2}{2\sigma^2\sigma_{pr}^2}\left[ \theta^2 - 2\left( \frac{n\bar{X}\sigma_{pr}^2+\theta_0\sigma^2}{n\sigma_{pr}^2+\sigma^2}\right)\theta \right] \\
						&= -\frac{n\sigma_{pr}^2+\sigma^2}{2\sigma^2\sigma_{pr}^2}\left[\left( \theta-\frac{n\bar{X}\sigma_{pr}^2+\theta_0\sigma^2}{n\sigma_{pr}^2+\sigma^2} \right)^2 - \left( \frac{n\bar{X}\sigma_{pr}^2 + \theta_0\sigma^2}{n\sigma_{pr}^2+\sigma^2} \right)^2\right]
					\end{align*}
					Note that the final term does not include any $\theta,$ so since this same expression is in the denominator, it will also cancel. Finally, the numerator simplifies to
					\begin{align*}
						\exp\left( -\frac{n\sigma_{pr}^2+\sigma^2}{2\sigma^2\sigma_{pr}^2}\left( \theta-\frac{n\bar{X}\sigma_{pr}^2 + \theta_0\sigma^2}{n\sigma_{pr}^2+\sigma^2} \right)^2 \right) &= \exp\left( -\frac{\left( \theta-\frac{n\bar{X}\sigma_{pr}^2+\theta_0\sigma^2}{n\sigma_{pr}^2+\sigma^2} \right)^2}{2\cdot \frac{\sigma^2\sigma_{pr}^2}{n\sigma_{pr}^2+\sigma^2}} \right)
					\end{align*}
					If we take 
					\begin{align*}
						\theta_{post} &= \frac{n\bar{X}\sigma_{pr}^2+\theta_0\sigma^2}{n\sigma_{pr}^2+\sigma^2} \\
						\sigma_{post}^2 &= \frac{\sigma^2\sigma_{pr}^2}{n\sigma_{pr}^2+\sigma^2}
					\end{align*} then this expression is just missing the factor of \[\frac{1}{\sqrt{2\pi}\sigma_{post}}\] in front of the exponential. Then if we do this in the denominator, the integral evaluates to 1 since it is a normal density, and finally the posterior distribution of $\theta$ is given by a normal distribution with the above parameters.

				\end{soln}

			\item For what value of $\theta$ is this posterior density maximized? Given this, what would you choose as an estimate for $\theta?$
				\begin{soln}
					The value of $\theta$ that maximizes this posterior density is clearly the posterior mean, thus \[\theta=\frac{n\bar{X}\sigma_{pr}^2+\theta_0\sigma^2}{n\sigma_{pr}^2+\sigma^2}\] is the desired estimate (note the dependence on the data in $\bar{X}$)
					
				\end{soln}

			\item How do the prior variance $\sigma_{pr}^2$ and the posterior variance compare? Which one is larger? Does this make sense? Why?
				\begin{soln}
					The posterior variance is given by \[\sigma_{post}^2 = \frac{\sigma^2\sigma_{pr}^2}{n\sigma_{pr}^2+\sigma^2}\] which less than $\sigma_{pr}^2.$ We can confirm this by clearing the denominators. It makes sense that the posterior variance is smaller because the data should have given us a better idea of what the actual value is.
					
				\end{soln}

			\item How does the estimator you obtained in part b compare to the MLE?
				\begin{soln}
					The MLE for the mean of a normal distribution is simply the sample mean. For large $n,$ the sample mean should approach its true value of $\theta_0,$ so these estimators are asymptotically the same.
					
				\end{soln}
				
		\end{enumerate}

	\item Suppose we are in the Bayesian framework and we wish to estimate a parameter $\theta$ with prior distribution $f$ from some family of distributions $G.$ If, conditional on the value of the parameter, the data have some distribution $H$ and the posterior distribution is again in the family $G,$ we say that $G$ and $H$ are conjugate.

		\begin{enumerate}[(a)]
			\item Show that if $X_i$ are iid Bernoulli $(p)$ and $p$ has a Beta-distributed prior, so that \[f_p(x)=\frac{\Gamma(\alpha+\beta)}{\Gamma(\alpha)\Gamma(\beta)}x^{\alpha-1}(1-x)^{\beta-1}\] where, as usual, \[\Gamma(u)=\int_0^\infty t^{u-1}e^{-t}\, dt\] then the Bernoulli and Beta families are conjugate.
				\begin{proof}
					The distribution of $X_i$ given $p$ is a Bernoulli $p$ distribution.

					We have the posterior distribution of $p$ is 
					\begin{align*}
						f_{P|X}(p\mid x) &= \frac{f_{X|P}(x_1, \cdots, x_n\mid p)f(p)}{\int f_{X|P}(x_1, \cdots, x_n\mid p) f(p)\, dp} \\
						&= \frac{\displaystyle \prod_{i=1}^{n} p^{x_i}(1-p)^{1-x_i}\frac{\Gamma(\alpha+\beta)}{\Gamma(\alpha)\Gamma(\beta)}p^{\alpha-1}(1-p)^{\beta-1}}{\displaystyle \int\prod_{i=1}^{n} p^{x_i}(1-p)^{1-x_i}\frac{\Gamma(\alpha+\beta)}{\Gamma(\alpha)\Gamma(\beta)}p^{\alpha-1}(1-p)^{\beta-1}} \\
						&= \frac{p^{\sum_{}^{}x_i} (1-p)^{n-\sum_{}^{}x_i} p^{\alpha-1}(1-p)^{\beta-1}}{\displaystyle \int p^{\sum_{}^{}x_i} (1-p)^{n-\sum_{}^{}x_i} p^{\alpha-1}(1-p)^{\beta-1}} \\ 
						&= \frac{p^{\alpha+\sum_{}^{}x_i - 1} (1-p)^{\beta+n-\sum_{}^{}x_i-1}}{\int p^{\alpha+\sum_{}^{}x_i-1}(1-p)^{\beta+n-\sum_{}^{}x_i-1}\, dp} \\
						&= \frac{\frac{\Gamma\left( \alpha+\sum_{}^{}x_i \right)\Gamma\left( \beta+n-\sum_{}^{}x_i \right)}{\Gamma\left( \alpha+\beta+n \right)}p^{\alpha+\sum_{}^{}x_i - 1} (1-p)^{\beta+n-\sum_{}^{}x_i-1}}{\displaystyle\int \frac{\Gamma\left( \alpha+\sum_{}^{}x_i \right)\Gamma\left( \beta+n-\sum_{}^{}x_i \right)}{\Gamma(\alpha+\beta+n)} p^{\alpha+\sum_{}^{}x_i - 1} (1-p)^{\beta+n-\sum_{}^{}x_i-1}\, dp}
					\end{align*}
					The integrand in the denominator evaluates to 1 since it is the density of the Beta with parameters $\alpha+\sum_{}^{}x_i$ and $\beta+n-\sum_{}^{}x_i.$ Thus, the posterior distribution of $p$ is this same Beta distribution. 

					Thus, the Bernoulli and Beta families are conjugate, as desired.
					
				\end{proof}

			\item What if the $X_i$ are binomial with parameters $n, p$ where $n$ is known and $p$ has, again, a Beta distribution? Are the binomial and Beta families conjugate?
				\begin{soln}
					The distribution of $X_i$ given $p$ is a Binomial $n, p$ distribution.

					We have the posterior distribution of $p$ is
					\begin{align*}
						f_{P|X}(p\mid x) &= \frac{f_{X|P}(x_1, \cdots, x_n\mid p) f(p)}{\int f_{X|P} (x_1, \cdots, x_n\mid p)f(p)\, dp} \\
						&= \frac{\displaystyle\prod_{i=1}^{m}\binom{n}{x_i} p^{x_i}(1-p)^{n-x_i} \frac{\Gamma(\alpha+\beta)}{\Gamma(\alpha)\Gamma(\beta)}p^{\alpha-1}(1-p)^{\beta-1}}{\displaystyle\int\prod_{i=1}^{m}\binom{n}{x_i} p^{x_i}(1-p)^{n-x_i} \frac{\Gamma(\alpha+\beta)}{\Gamma(\alpha)\Gamma(\beta)}p^{\alpha-1}(1-p)^{\beta-1}\, dp } \\
						&= \frac{p^{\sum_{}^{}x_i} (1-p)^{n^2-\sum_{}^{}x_i} p^{\alpha-1}(1-p)^{\beta-1}}{\displaystyle\int p^{\sum_{}^{}x_i} (1-p)^{n^2-\sum_{}^{}x_i} p^{\alpha-1}(1-p)^{\beta-1}\, dp} \\
						&= \frac{p^{\alpha+\sum_{}^{}x_i-1} (1-p)^{\beta+n^2-\sum_{}^{}x_i-1}}{\displaystyle\int p^{\alpha+\sum_{}^{}x_i-1} (1-p)^{\beta+n^2-\sum_{}^{}x_i-1} \, dp}
					\end{align*}
					Note the similarity to the form in the previous problem, so we may conclude this has a Beta distribution with parameters $\alpha+\sum_{}^{}x_i$ and $\beta+n^2-\sum_{}^{}x_i.$ 

					Thus, the binomial and Beta families are conjugate.
					
				\end{soln}

				\newpage
			\item Show that if $X_i$ are iid exponential with parameter $\lambda,$ and $\lambda$ has a Gamma-distributed prior, then the posterior also has a Gamma distribution. What is a reasonable estimate for $\lambda$ in this Bayesian setting? How doe sit compare to the MLE for the exponential?
				\begin{proof}
					Suppose the distribution of $\lambda$ is given by \[f(\lambda)=\frac{\beta^{\alpha}}{\Gamma(\alpha)}\lambda^{\alpha-1} e^{-\lambda\beta}\] The posterior distribution of $\lambda$ is given by
					\begin{align*}
						f_{L|X}(\lambda\mid x) &= \frac{f_{X|L}(x_1, \cdots, x_n\mid \lambda) f(\lambda)}{\int f_{X|L}(x_1, \cdots, x_n\mid\lambda) f(\lambda)\, d\lambda} \\
						&= \frac{\displaystyle \prod_{i=1}^{n} \lambda e^{-\lambda x_i} \frac{\beta^{\alpha}}{\Gamma(\alpha)}\lambda^{\alpha-1}e^{-\lambda\beta}}{\displaystyle\int \prod_{i=1}^{n} \lambda e^{-\lambda x_i} \frac{\beta^{\alpha}}{\Gamma(\alpha)}\lambda^{\alpha-1}e^{-\lambda\beta}\, d\lambda} \\
						&= \frac{e^{-\lambda\sum_{}^{}x_i} \lambda^{\alpha-1} e^{-\lambda\beta}}{\displaystyle \int e^{-\lambda\sum_{}^{}x_i} \lambda^{\alpha-1} e^{-\lambda\beta}\, d\lambda} \\
						&= \frac{e^{-\lambda\left( \beta+\sum_{}^{}x_i \right)}\lambda^{\alpha-1} }{\displaystyle\int e^{-\lambda\left( \beta+\sum_{}^{}x_i \right)}\lambda^{\alpha-1}\, d\lambda} \\
						&= \frac{\frac{\left( \beta+\sum_{}^{}x_i \right)^\alpha}{\Gamma(\alpha)}e^{-\lambda\left( \beta+\sum_{}^{}x_i \right)} \lambda^{\alpha-1}}{\displaystyle\int \frac{\left( \beta+\sum_{}^{}x_i \right)^\alpha}{\Gamma(\alpha)}e^{-\lambda\left( \beta+\sum_{}^{}x_i \right)} \lambda^{\alpha-1}\, d\lambda}
					\end{align*} 
					Note that the denominator is the Gamma distribution with parameters $\alpha$ and $\beta+\sum_{}^{}x_i,$ so it evaluates to 1. Thus, the posterior distribution of $\lambda$ is this same Gamma distribution (which is specified in the numerator). 

					A reasonable estimate for $\lambda$ is the mean of this distribution, which is given by \[\hat{\lambda}=\frac{\alpha}{\beta+\sum_{}^{}x_i}\] The MLE for the exponential is given by \[\hat{\lambda} = \frac{n}{\sum_{}^{}x_i}\] so these are plausibly similar.
					
				\end{proof}
				
		\end{enumerate}

		\newpage
	\item Suppose we observe an iid sample $X_1, \cdots, X_n$ from the distribution that is uniform in the interval $[-\theta, \theta]$ for some unknown $\theta>0.$

		\begin{enumerate}[(a)]
			\item Find the MLE for $\theta.$
				\begin{soln}
					Since these are uniform variables, we must have \[\theta\ge \max_{1\le i\le n} \left\lvert X_i \right\rvert\] otherwise there would be an impossible data element. The likelihood function is given by 
					\begin{align*}
						f(X_1, \cdots, X_n\mid \theta) &= \prod_{i=1}^{n} \frac{1}{2\theta} = \frac{1}{2^n \theta^n}
					\end{align*} which is a decreasing function in $\theta,$ so the MLE is in fact given by \[\hat{\theta}=\max_{1\le i\le n} \left\lvert X_i \right\rvert\] 
				\end{soln}

			\item Show that the pair $T=\max\left\{ X_1, \cdots, X_n \right\}$ and $S=\min\left\{ X_1, \cdots, X_n \right\}$ are sufficient for $\theta.$
				\begin{proof}
					The distribution of $X_i$ given $T$ and $S$ is simply a uniform distribution from $S$ to $T.$ We know they were initially drawn from a uniform distribution, but we don't know anything about its endpoints, so if we are given $S$ and $T$ as the endpoints, the distribution is uniform $[S, T],$ which in particular does not depend on $\theta.$ Thus, $T$ and $S$ are sufficient for $\theta.$
					
				\end{proof}
				
		\end{enumerate}

	\item Suppose $(U, V)$ is a uniformly distributed point in the unit circle $\Set{(x, y)}{x^2+y^2\le 1}$ in the plane. 

		\begin{enumerate}[(a)]
			\item Determine the marginal PDFs of $U$ and $V$ and expectations $E[U]$ and $E[V].$ Also determine the covariance $\cov(U, V)$ and decide if $U, V$ are independent.
				\begin{soln}
					The area of the unit circle is $\pi,$ so the joint density is given by \[f_{U, V}(u, v) = \frac{1}{\pi}\] The marginal density of $u$ is given by \[f_U(u)=\int f_{U, V}(u, v)\, dv = \int_{-\sqrt{1-u^2}}^{\sqrt{1-u^2}} \frac{1}{\pi}\, dv = \frac{2\sqrt{1-u^2}}{\pi}\] Similarly, the marginal density of $v$ is given by \[f_V(v)=\frac{2\sqrt{1-v^2}}{\pi}.\] It's easy to see that these densities are symmetric about the origin, so $E[U]=E[V]=0.$ The covariance is given by 
					\begin{align*}
						\cov(U, V) &= E[UV]-E[U]E[V] = E[UV] \\
						&= \int \int uv\cdot f_{U, V}(u, v)\, dv\, du \\
						&= \frac{1}{\pi}\int_{-1}^1 \int_{-\sqrt{1-u^2}}^{\sqrt{1-u^2}}uv\, dv\, du \\
						&= 0
					\end{align*} but the product of the marginal densities is \[f_U(u)f_V(v)=\frac{2\sqrt{1-u^2}}{\pi}\cdot\frac{2\sqrt{1-v^2}}{\pi}=\frac{4(1-u^2)(1-v^2)}{\pi^2}\neq f_{U, V}(u, v)\] so $U$ and $V$ are not independent.
					
				\end{soln}

			\item Let $W=U^2+V^2.$ Compute the density $f_W(w)$ for $W.$
				\begin{soln}
					Consider the probability $F_W(w)=P(W\le w)=P(U^2+V^2\le w).$ This is a circle of radius $w$ centered at the origin, but $U^2+V^2$ can be anywhere in the unit circle, so this probability is given by \[P(W\le w)=\frac{w^2\pi}{\pi}=w^2\] so the density is given by 
					\begin{align*}
						f_W(w)=\frac{d}{dw}F_W(w) = \frac{d}{dw}\left[ w^2 \right]=2w, \quad0\le 1\le w
					\end{align*}
				\end{soln}

			\item Let $R=\theta U,$ and $T=\theta V,$ where $\theta>0$ is some non-random parameter. Compute the joint distribution of $(R, T).$
				\begin{soln}
					We have \[f_{R, T}(r, t) = f_{U, V}(u, v) \left\lvert \frac{d(u, v)}{d(r, t)} \right\rvert\] where $U=R/\theta$ and $V=T/\theta,$ so the joint density of $R, T$ is given by \[f_{R, T}(r, t) = \frac{1}{\pi}\left\lvert \begin{bmatrix}
							1/\theta & 0 \\ 0 & 1/\theta
					\end{bmatrix}\right\rvert = \frac{1}{\theta^2\pi}\]
				\end{soln}
				
		\end{enumerate}

	\item Suppose we observe independent pairs $(X_i, Y_i)$ where each $(X_i, Y_i)$ has a uniform distribution in the circle of unknown radius $\theta$ and centered at $(0, 0)$ in the plane.

		\begin{enumerate}[(a)]
			\item Show that $(X_i/\theta, Y_i/\theta)$ has a uniform distribution in the unit circle, and find the PDF of $X_i^2+Y_i^2.$
				\begin{proof}
					The joint density of $X_i, Y_i$ is given by \[f_{X_i, Y_i}(x_i, y_i)=\frac{1}{\theta^2\pi}\] so letting $X_i=\theta A, Y_i=\theta B,$ we have the joint density of $A, B$ is \[f_{A, B}(a, b)=f_{X_i, Y_i}(x_i, y_i)\left\lvert \frac{d(x_i, y_i)}{d(a, b)} \right\rvert = \frac{1}{\theta^2\pi} \left\lvert \begin{bmatrix}
						\theta & 0 \\ 0 & \theta
				\end{bmatrix}\right\rvert = \frac{1}{\pi}\] which is exactly the joint density of a uniform distribution on the unit circle, as desired.

				Let $W=X_i^2+Y_i^2.$ Then the CDF of $W$ is given by \[F_W(w)=P(W\le w)=P(X_i^2+Y_i^2\le w)\] which is a circle of radius $w$ centered on the origin, and since $X_i, Y_i$ is uniformly distributed on a circle of radius $\theta,$ this probability is \[F_W(w)=\frac{w^2\pi}{\theta^2\pi}=\frac{w^2}{\theta^2}.\] Thus, the density of $W$ is given by \[f_W(w\mid\theta)=\frac{d}{dw}F_W(w)=\frac{d}{dw}\left[ \frac{w^2}{\theta^2} \right]=\frac{2w}{\theta^2}, \quad 0\le w\le \theta\]
				
				\end{proof}

			\item Show that $(X_1^2+Y_1^2, \cdots, X_n^2+Y_n^2)$ is a sufficient statistic for $\theta.$
				\begin{proof}
					Let $W_i=X_i^2+Y_i^2.$ Then the likelihood function is given by \[f(W_1, \cdots, W_n\mid \theta) = \prod_{i=1}^{n}f(W_i\mid\theta)=\prod_{i=1}^{n}\frac{2W_i}{\theta^2} = \frac{2^n}{\theta^{2n}}\prod_{i=1}^{n}W_i\] Since \[T\left[ (X_1, Y_1), \cdots, (X_n, Y_n) \right] = (X_1^2+Y_1^2, \cdots, X_n^2+Y_n^2)=(W_1, \cdots, W_n)\] is a function of the data, we may write $h\left[ (X_1, Y_1), \cdots, (X_n, Y_n) \right] = 1$ and \[g(T, \theta) = g[(W_1, \cdots, W_n), \theta)]= \frac{2^n}{\theta^{2n}}\prod_{i=1}^{n} W_i\] so $T$ is a sufficient statistic by the Factorization theorem.
					
				\end{proof}

			\item Find the MLE and determine its density function and its bias. Are the regularity assumptions were require on the MLE satisfied here?
				\begin{soln}
					As above, the joint density \[f\left[ (X_1, Y_1), \cdots, (X_n, Y_n)\mid \theta \right] = \prod_{i=1}^{n}f\left[ (X_i, Y_i)\mid \theta \right]=\prod_{i=1}^{n}\frac{1}{\theta^2\pi} = \frac{1}{\theta^{2n}\pi^{n}}\] Since $X_i^2+Y_i^2\le \theta^2,$ the MLE $\hat{\theta}$ is \[\hat{\theta}=\max_{1\le i\le n} \sqrt{X_i^2+Y_i^2}\] Consider the CDF of $\hat{\theta}$ 
					\begin{align*}
						F(t)&=P(\hat{\theta}\le t) = P\left( \max_{1\le i\le n} \sqrt{X_i^2+Y_i^2} \le t\right) \\
						&= P\left( \sqrt{X_1^2+Y_1^2},\cdots, \displaystyle \sqrt{X_n^2+Y_n^2}\le t \right) \\
						&= \prod_{i=1}^{n}P\left(\sqrt{X_i^2+Y_i^2}\le t\right) \\
						&= \prod_{i=1}^{n} P(W_i\le t^2) = \prod_{i=1}^{n} \frac{t^2}{\theta^2} = \frac{t^{2n}}{\theta^{2n}} 
					\end{align*} and the density of $\hat{\theta}$ is the derivative of this wrt to $t:$
					\[f_{\hat{\theta}}(t) = \frac{\partial}{\partial t}\left[ \frac{t^{2n}}{\theta^{2n}} \right]=\frac{2n t^{2n-1}}{\theta^{2n}}\] 

					Then $E[\hat{\theta}]$ is given by
					\begin{align*}
						E[ \hat{\theta}] &= \int_0^\theta t\frac{2nt^{2n-1}}{\theta^{2n}}\, dt = \int_0^\theta \frac{2nt^{2n}}{\theta^{2n}}\, dt \\
						&= \frac{2nt^{2n+1}}{\theta^{2n}(2n+1)}\bigg\vert_0^\theta =\frac{2n\theta}{(2n+1)} 
					\end{align*} so the bias of $\hat{\theta}$ is \[E[\hat{\theta}]-\theta=\frac{2n\theta}{2n+1}-\theta = -\frac{\theta}{2n+1}.\]

					The support of the distribution of $(X_i, Y_i)$ is \[\Set{(x_i, y_i)}{f[(x_i, y_i)\mid\theta]>0}=\Set{(x_i, y_i)}{1/\theta^2\pi>0}\] which is the entire domain, and doesn't depend on $\theta,$ so MLE satisfies the regularity conditions.
					
				\end{soln}

			\item Compute the variance of the MLE and simplify it so that it is clear how this variance decays with the sample size $n.$
				\begin{soln}
					The variance of the MLE is given by 
					\begin{align*}
						\var(\hat{\theta}) &= E[\hat{\theta}^2]-(E[\hat{\theta}])^2 
					\end{align*} where 
					\begin{align*}
						E[\hat{\theta}^2] &= \int_0^\theta t^2 \frac{2nt^{2n-1}}{\theta^{2n}}\, dt = \int_0^\theta \frac{2nt^{2n+1}}{\theta^{2n}}\, dt \\
						&= \frac{2nt^{2n+2}}{\theta^{2n}(2n+2)}\bigg\vert_0^\theta = \frac{n\theta^2}{n+1}
					\end{align*} so the variance is 
					\begin{align*}
					\var(\hat{\theta}) &= \frac{n\theta^2}{n+1}-\left( \frac{2n\theta}{2n+1} \right)^2 \\ 
					&= \theta^2\left( \frac{n}{n+1}-\frac{4n^2}{(2n+1)^2} \right) = \frac{n\theta^2}{(n+1)(2n+1)^2}
					\end{align*} Clearly, this diminishes very quickly as $n$ increases.
					
				\end{soln}

			\item Find the MSE of the MLE. As $n\to\infty,$ which term contributes more to the MSE, the squared bias or the variance?
				\begin{soln}
					The MSE is given by 
					\begin{align*}
						E[(\hat{\theta}-\theta)^2] &= \var(\hat{\theta}) + \left( E[\hat{\theta}-\theta] \right)^2 \\
						&= \frac{n\theta^2}{(n+1)(2n+1)^2} + \left( -\frac{\theta}{2n+1} \right)^2 \\
						&\tag{1}=\frac{\theta^2}{(2n+1)^2}\left( \frac{n}{n+1}+1 \right) \\ 
						&= \frac{\theta^2}{(n+1)(2n+1)}
					\end{align*}
					In (1), since \[\frac{n}{n+1}\to1\] as $n\to\infty,$ the squared bias and the variance contribute equally to the MSE.
					
				\end{soln}

			\item Find a method of moments estimator for $\theta$ based on the $X_i$ and call this $\hat{\theta}_X.$
				\begin{soln}
					The marginal density of $X_i$ is given by \[f_X(x)=\frac{2\sqrt{\theta^2-x^2}}{\theta^2\pi}\] which is symmetric about the origin, so $\mu_1=E[X_i]=0.$ Then
					\begin{align*}
						\mu_2 &= E[X_i^2] = \int_{-\theta}^\theta x^2\cdot\frac{2\sqrt{\theta^2-x^2}}{\theta^2\pi}\, dx = \frac{\theta^2}{4}
					\end{align*} according to Wolfram, so the method of moments estimate is \[\hat{\theta}_x=2\sqrt{\hat{\mu}_2}.\] 
				\end{soln}

			\item Compare the performance of the MLE and the method of moments estimator as follows: In R, do the following 10000 times. Sample the uniform distribution in the unit circle using a sample of size 10, and compute the three estimators (MLE, MoM $X_i,$ MoM $Y_i$). Compute estimates of the bias, the variance, and the MSE of each. Estimate the correlation coefficient between $\hat{\theta}_x$ and $\hat{\theta}_y.$ Assuming your estimate in the previous parts are correct, how much should we impove the variance of one of $\hat{\theta}_x$ or $\hat{\theta}_y$ by averaging them?

			\item Show that for the method of moments estimator and the MLE, is it the case that the distribution of $\hat{\theta}/\theta$ does not depend on $\theta.$ Explain why this means we can write \[MSE_\theta(\hat{\theta})=\theta^2\left( MSE_{\theta=1}(\hat{\theta}) \right)\] From this, explain why it suffices that we compare the two estimators when $\theta=1.$
				\begin{proof}
					The MLE was \[\hat{\theta}=\max_{1\le i\le n}\sqrt{X_i^2+Y_i^2}\] so consider the probability \[P(\hat{\theta}/\theta \le t) = P(\hat{\theta}\le t\theta) = \frac{(t\theta)^{2n}}{\theta^{2n}} = t^{2n}\] from the density derived in part (c), so the density of $\hat{\theta}/\theta$ does not depend on $\theta.$ 
				\end{proof}
				
		\end{enumerate}
		
\end{enumerate}

\section*{Chapter 9: Testing Hypotheses and Assessing Goodness of Fit}

\begin{itemize}
	\item[2.] Which of the following hypotheses are simple, and which are composite?

		\begin{enumerate}[a.]
			\item $X$ follows a uniform distribution on $[0, 1].$
				\begin{answer*}
					This is simple, since it specifies the entire distribution of $X.$
				\end{answer*}

			\item A die is unbiased.
				\begin{answer*}
					This is simple, since it specifies the distribution of the roll (each has probability $1/6$).
				\end{answer*}

			\item $X$ follows a normal distribution with mean 0 and variance $\sigma^2>10.$
				\begin{answer*}
					This is composite, since the variance is not specified entirely.
				\end{answer*}

			\item $X$ follows a normal distribution with mean $\mu=0.$
				\begin{answer*}
					This is composite, because the variance is not specified at all.
				\end{answer*}
				
		\end{enumerate}

	\item[5.] True or false, and state why:

		\begin{enumerate}[a.]
			\item The significance level of a statistical test is equal to the probability that the null hypothesis is true.
				\begin{answer*}
					This is \boxed{\text{false.}} The significance level is the probability of a type I error, not the probability the null hypothesis is true.
				\end{answer*}

			\item If the significance level of a test is decreased, the power would be expected to increase.
				\begin{answer*}
					This is \boxed{\text{false.}} Decreasing the significance level increases the chance of a type II error, which decreases the power.
				\end{answer*}

			\item If a test is rejected at the significance level $\alpha,$ the probability that the null hypothesis is true equals $\alpha.$
				\begin{answer*}
					This is \boxed{\text{false.}} We know nothing about the probability that the null hypothesis is true, only that the test statistic falls into the rejection region.
				\end{answer*}

			\item The probability that the null hypothesis is falsely rejected is equal to the power of the test.
				\begin{answer*}
					This is \boxed{\text{false.}} Falsely rejecting the null hypothesis means rejecting the null when it is actually true. This is a type I error and is the significance level, not the power.
				\end{answer*}

			\item A type I error occurs when the test statistic falls in the rejection region of the test.
				\begin{answer*}
					This is \boxed{\text{false.}} When the statistic falls in the rejection region, we reject the null hypothesis. This says nothing about whether the null was actually true to begin with or not.
				\end{answer*}

			\item A type II error is more serious than a type I error.
				\begin{answer*}
					This can't be answered definitively, but it is not in general true.
				\end{answer*}

			\item The power of a test is determined by the null distribution of the test statistic.
				\begin{answer*}
					This is \boxed{\text{false.}} The power of a test is determined by the alternative distribution.
				\end{answer*}

			\item The likelihood ratio is a random variable.
				\begin{answer*}
					This is \boxed{\text{true.}} It is a function of the data, which are random variables.
				\end{answer*}
				
		\end{enumerate}

	\item[4.] Let $X$ have one of the following distributions:
		\begin{center}
			\begin{tabular}{ccc}
				$X$ & $H_0$ & $H_A$ \\
				\hline
				$x_1$ & 0.2 & 0.1 \\
				$x_2$ & 0.3 & 0.4 \\
				$x_3$ & 0.3 & 0.1 \\
				$x_4$ & 0.2 & 0.4
			\end{tabular}
		\end{center}

		\begin{enumerate}[a.]
			\item Compare the likelihood ratio, $\Lambda,$ for each possible value $X$ and order the $x_i$ according to $\Lambda.$
				\begin{soln}
					We have 
					\begin{align*}
						\Lambda_1 &= \frac{P(X=x_1\mid H_0)}{P(X=x_1\mid H_A)} = \frac{0.2}{0.1} = 2 \\
						\Lambda_2 &= \frac{P(X=x_2\mid H_0)}{P(X=x_2\mid H_A)}=\frac{0.3}{0.4}=\frac{3}{4} \\
						\Lambda_3 &= \frac{P(X=x_3\mid H_0)}{P(X=x_3\mid H_A)} = \frac{0.3}{0.1} = 3 \\
						\Lambda_4 &= \frac{P(X=x_4\mid H_0)}{P(X=x_4\mid H_A)} = \frac{0.2}{0.4}=\frac{1}{2}
					\end{align*}
					So the ordering from least to greatest is \[\Lambda_4<\Lambda_2<\Lambda_1<\Lambda_3\]
				\end{soln}

			\item What is the likelihood ratio test of $H_0$ versus $H_A$ at the level $\alpha=0.2?$ What is the test at the level $\alpha=0.5?$
				\begin{soln}
					If $H_0$ is true, then the PMF of $\Lambda$ is given by 
					\begin{center}
						\begin{tabular}{c||c|c|c|c}
							$\Lambda$ & 0.5 & 0.75 & 2 & 3 \\
							\hline
							$p(\lambda)$ & 0.2 & 0.3 & 0.2 & 0.3
						\end{tabular}
					\end{center}
					Thus, at the level $\alpha=0.2,$ we have \[P(\Lambda\le c \mid H_0) = 0.2\] so $c\in [0.5, 0.75)$ defines the upper bound of the rejection region. We would reject if we picked $x_4.$
						
					At the level $\alpha=0.5,$ we have \[P(\Lambda\le c\mid H_0) = 0.5\] so $c\in [0.75, 2).$ We would reject if we picked $x_2$ or $x_4.$

				\end{soln}

			\item If the prior probabilities are $P(H_0)=P(H_A),$ which outcomes favor $H_0?$
				\begin{soln}
					If $P(H_0)=P(H_A),$ then the likelihood ratio is given by \[\frac{P(X\mid H_0)}{P(X\mid H_A)} = \frac{P(H_A)}{P(H_0)}\cdot \frac{P(H_0\mid X)}{P(H_A\mid X)}=\frac{P(H_0\mid X)}{P(H_A\mid X)}\]
					The outcomes that favor $H_0$ are the outcomes where this ratio is greater than 1. These occur exactly when the probability of $x_i$ under $H_0$ is greater than the probability under $H_a,$ which are when $X=x_1$ and $X=x_3.$
					
				\end{soln}

			\item What prior probabilities correspond to the decision rules with $\alpha=0.2$ and $\alpha=0.5?$
				
		\end{enumerate}

		\newpage
	\item[7.] Let $X_1, \cdots, X_n$ be a sample from a Poisson distribution. Find the likelihood ratio for testing $H_0:\lambda=\lambda_0$ versus $H_a:\lambda=\lambda_1,$ where $\lambda_1>\lambda_0.$ Use the fact that the sum of independent Poisson random variables follows a Poisson distribution to explain how to determine a rejection region for a test at level $\alpha.$
		\begin{soln}
			Let $S_n=\displaystyle \sum_{i=1}^{n} X_i.$ The likelihood ratio is given by
			\begin{align*}
				\Lambda &= \frac{P(X_1, \cdots, X_n\mid \lambda=\lambda_0)}{P(X_1, \cdots, X_n\mid \lambda=\lambda_1} \\
				&= \frac{\displaystyle \prod_{i=1}^{n} \frac{\lambda_0^{X_i} e^{-\lambda_0}}{X_i!}}{\displaystyle \prod_{i=1}^{n} \frac{\lambda_1^{X_i} e^{-\lambda_1}}{X_i!}} \\
				&= e^{-n(\lambda_0-\lambda_1)} \left( \frac{\lambda_0}{\lambda_1} \right)^{S_n}
			\end{align*}

			To determine a rejection region, we consider the probability \[P(\Lambda\le c\mid H_0)=\alpha\] which is
			\begin{align*}
				P(\Lambda\le c\mid H_0) &= P\left( e^{-n(\lambda_0-\lambda_1)} \left( \frac{\lambda_0}{\lambda_1} \right)^{S_n} \le c\bigg\vert \lambda=\lambda_0 \right) \\
				&= P\left( \left( \frac{\lambda_0}{\lambda_1} \right)^{S_n}\le ce^{n(\lambda_0-\lambda_1)} \right) \\
				&= P\left( S_n\log \left( \frac{\lambda_0}{\lambda_1}\right)\le n(\lambda_0-\lambda_1) \log c  \right) \\
				&= P\left( S_n\ge n(\lambda_0-\lambda_1)\frac{\log c}{\log \left( \frac{\lambda_0}{\lambda_1} \right)} \right) \\
				&= P\left( S_n\le n(\lambda_1-\lambda_0) \frac{\log c}{\log \lambda_0-\log \lambda_1} \right)	
			\end{align*} Since $S_n$ is the sum of Poisson random variables, its density is given by \[f(s) = \frac{(n\lambda_0)^s e^{-n\lambda_0}}{s!}\] if we assume that $\lambda=\lambda_0.$ We must have $c<1$ otherwise the RHS will be negative, and the probability is 0. Suppose $M$ is the largest integer less than or equal to the RHS, so the probability is 
			\begin{align*}
				P\left( S_n\le n(\lambda_1-\lambda_0) \frac{\log c}{\log \lambda_0-\log \lambda_1} \right) &= \sum_{s=0}^{M} \frac{(n\lambda_0)^s e^{-n\lambda}}{s!}=\alpha
			\end{align*} so we may solve explicitly for $c$ in terms of $\alpha$ since $M$ is a function of $c.$ 

		\end{soln}

		\newpage
	\item[9.] Let $X_1,\cdots, X_{25}$ be a sample from a normal distribution having a variance of 100. Find the rejection region for a test at level $\alpha=0.10$ of $H_0:\mu=0$ versus $H_A: \mu=1.5.$ What is the power of the test? Repeat for $\alpha=0.01.$
		\begin{soln}
			The variance is 100, so the density is given by \[f(x) = \frac{1}{10\sqrt{2\pi}}\exp\left( -\frac{(x-\mu)^2}{200} \right)\] Thus, the likelihood ratio is given by
			\begin{align*}
				\Lambda &= \frac{P(X_1, \cdots, X_{25}\mid H_0)}{P(X_1, \cdots, X_{25}\mid H_A)} \\
				&= \frac{\displaystyle \prod_{i=1}^{25}\frac{1}{10\sqrt{2\pi}}\exp\left( -\frac{X_i^2}{200} \right)}{\displaystyle \prod_{i=1}^{25} \frac{1}{10\sqrt{2\pi}}\exp\left( -\frac{(X_i-1.5)^2}{200} \right)} \\
				&= \exp\left( -\frac{1}{200}\left[ \sum_{i=1}^{25}X_i^2 - \sum_{i=1}^{25}(X_i-1.5)^2 \right] \right) \\
				&= \exp\left( -\frac{1}{200}\left[ 3\sum_{i=1}^{25}X_i - 25(1.5)^2 \right] \right) \\
				&= \exp\left( -\frac{1}{200}\left[ 3(25\bar{X})-25(1.5)^2 \right] \right) \\
				&= \exp\left( \frac{2.25-3\bar{X}}{8} \right)
			\end{align*}

			At the level $\alpha=0.10,$ we have the rejection region given by
			\begin{align*}
				P(\Lambda\le c\mid H_0) &= P\left( \exp\left( \frac{2.25-3\bar{X}}{8} \right)\le c \right) \\
				&= P\left( \frac{2.25-3\bar{X}}{8} \le \log c\right) \\
				&= P\left( \bar{X} \ge \frac{2.25-8\log c}{3} \right)= 0.10
			\end{align*}

			If we assume the null hypothesis to be true, that is $\mu=0$, then the distribution of $\bar{X}$ is \[N\left( 0, \frac{10^2}{25} \right) = N(0, 2^2).\] Thus, the probability is
			\begin{align*}
				P\left(\bar{X}\ge \frac{2.25-8\log c}{3}\right) &= P\left( \frac{\bar{X}}{2}\ge \frac{2.25-8\log c}{6} \right) = 1-P\left( \frac{\bar{X}}{2}< \frac{2.25-8\log c}{6} \right) \\
				&= 1-\Phi\left( \frac{2.25-8\log c}{6} \right) = 0.10 
			\end{align*} so at this point we may solve for $c:$
			\begin{align*}
				\Phi\left( \frac{2.25-8\log c}{6} \right) &= 0.90 \\
				\implies \frac{2.25-8\log c}{6} &= 1.282 \\
				\implies c &= 0.506
			\end{align*}

			Thus, we reject $H_0$ if $\Lambda\in (0, 0.506].$ Let $\beta$ be the probability of a type II error, that is, \[\beta=P(\Lambda>0.506\mid H_A)\] If the alternative is true, then $\mu=1.5,$ and the distribution of $\bar{X}$ is $N(1.5, 2^2).$ Thus the probability is given by
			\begin{align*}
				P(\Lambda>0.506\mid H_A) &= P\left( \exp\left( \frac{2.25-3\bar{X}}{8}\right)>0.506 \right) \\
				&= P\left( \frac{2.25-3\bar{X}}{8}>\log 0.506 \right) \\
				&= P\left(\bar{X}<2.567\right) \\
				&= P\left(\frac{\bar{X}-1.5}{2} < \frac{2.567-1.5}{2}\right) \\
				&= \Phi(0.533) = 0.71
			\end{align*} so the power is $1-\beta=1-0.71=0.29.$
		

			At the level $\alpha=0.01,$ basically everything is the same, except the distribution of $\bar{X}$ is \[N\left( 0, \frac{10^2}{100} \right) = N(0, 1)\] so
			\begin{align*}
				P\left( \bar{X}\ge \frac{2.25-8\log c}{3} \right) &= 1-\Phi\left( \frac{2.25-8\log c}{3} \right) = 0.01 \\
				\Phi\left( \frac{2.25-8\log c}{3} \right) &= 0.99 \\
				\implies \frac{2.25-8\log c}{3} &= 2.327 \\
				\implies c &= 0.554
			\end{align*}
			Thus, we reject $H_0$ if $\Lambda\in (0, 0.554].$ If the alternative is true, then the distribution of $\bar{X}$ is $N(1.5, 1),$ so the probability of a type II error is
			\begin{align*}
				P(\Lambda>0.554\mid H_A) &= P\left( \exp\left( \frac{2.25-3\bar{X}}{8}>0.554 \right) \right) \\
				&= P\left( \frac{2.25-3\bar{X}}{8}>\log 0.554 \right) \\
				&= P(\bar{X}<2.325) \\
				&= P(\bar{X}-1.5 < 2.325-1.5) \\
				&= \Phi(0.825) = 0.795
			\end{align*} so the power is $1-\beta=1-0.795=0.205.$
			
		\end{soln}
		
\end{itemize}

\end{document}

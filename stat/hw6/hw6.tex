\documentclass{article}
\usepackage[sexy, hdr, fancy]{evan}
\usepackage{bm}
\setlength{\droptitle}{-4em}

\lhead{Homework 6}
\rhead{Introduction to Statistics}
\lfoot{}
\cfoot{\thepage}

\newcommand{\var}{\mathrm{Var}}
\newcommand{\cov}{\mathrm{Cov}}

\begin{document}
\title{Homework 6}
\maketitle
\thispagestyle{fancy}

\begin{enumerate}
	\item Prove the converse of the factorization theorem, namely prove that if $T$ is a sufficient statistic, then the joint density can be factored as \[f(x_1, \cdots, x_n\mid\theta)=g(T, \theta)h(x_1, \cdots, x_n)\] Also show that if $T$ is sufficient for $\theta,$ then the MLE must be a function of $T.$

	\item Let $\hat{\theta}$ be an estimator for a parameter $\theta,$ and suppose that $\var(\hat{\theta})<\infty.$ Let $T$ be a sufficient statistic for $\theta.$ Consider the random variable \[Y=E[\hat{\theta}\mid T]\] Prove that \[E\left[ \left( Y-\theta \right)^2 \right]\le E\left[ \left( \hat{\theta}-\theta \right)^2 \right]\]
		\begin{proof}
			We have \[E[Y]=E\left[ E[\hat{\theta}\mid T] \right] = E[\hat{\theta}]\] so
			\begin{align*}
				E\left[ (Y-\theta)^2 \right] &= E[Y^2]-2\theta E[Y] + \theta^2 \\
				&= E[Y^2] - 2\theta E[\hat{\theta}]+\theta^2
			\end{align*}<++>
		\end{proof}<++>

	\item Complete all the details of the example we discussed in lecture. Let $X_1, \cdots, X_n$ be iid data from a normal distribution with unknown mean and known variance $\sigma^2.$ Suppose that $\theta$ is assumed to be random, with prior distribution also normal; assume that the mean and variance of the prior distribution of $\theta_0$ and $\sigma_{pr}^2,$ where both $\sigma_0$ and $\sigma_{pr}^2$ are known.

		\begin{enumerate}[(a)]
			\item Compute the posterior distribution \[f_{\theta\mid\bm{X}} (\theta\mid x_1, \cdots, x_n)\] where $\bm{X}=(X_1, \cdots, X_n),$ and specify all the parameters of this distribution.

			\item For what value of $\theta$ is this posterior density maximized? Given this, what would you choose as an estimate for $\theta?$

			\item How do the prior variance $\sigma_{pr}^2$ and the posterior variance compare? Which one is larger? Does this make sense? Why?

			\item How does the estimator you obtained in part b compare to the MLE?
				
		\end{enumerate}

	\item Suppose we are in the Bayesian framework and we wish to estimate a parameter $\theta$ with prior distribution $f$ from some family of distributions $G.$ If, conditional on the value of the parameter, the data have some distribution $H$ and the posterior distribution is again in the family $G,$ we say that $G$ and $H$ are conjugate.

		\begin{enumerate}[(a)]
			\item Show that if $X_i$ are iid Bernoulli $(p)$ and $p$ has a Beta-distributed prior, so that \[f_p(x)=\frac{\Gamma(\alpha+\beta)}{\Gamma(\alpha+\beta)\Gamma(\beta)}x^{\alpha-1}(1-x)^{\beta-1}\] where, as usual, \[\Gamma(u)=\int_0^\infty t^{u-1}e^{-t}\, dt\] then the Bernoulli and Beta families are conjugate.

			\item What if the $X_i$ are binomial with parameters $n, p$ where $n$ is known and $p$ has, again, a Beta distribution? Are the binomial and Beta families conjugate?

			\item Show that if $X_i$ are iid exponential with parameter $\lambda,$ and $\lambda$ has a Gamma-distributed prior, then the posterior also has a Gamma distribution. What is a reasonable estimate for $\lambda$ in this Bayesian setting? How doe sit compare to the MLE for the exponential?
				
		\end{enumerate}

	\item Suppose we observe an iid sample $X_1, \cdots, X_n$ from the distribution that is uniform in the interval $[-\theta, \theta]$ for some unknown $\theta>0.$

		\begin{enumerate}[(a)]
			\item Find the MLE for $\theta.$

			\item Show that the pair $T=\max\left\{ X_1, \cdots, X_n \right\}$ and $S=\min\left\{ X_1, \cdots, X_n \right\}$ are sufficient for $\theta.$
				
		\end{enumerate}

	\item Suppose $(U, V)$ is a uniformly distributed point in the unit circle $\Set{(x, y)}{x^2+y^2\le 1}$ in the plane. 

		\begin{enumerate}[(a)]
			\item Determine the marginal PDFs of $U$ and $V$ and expectations $E[U]$ and $E[V].$ Also determine the covariance $\cov(U, V)$ and decide if $U, V$ are independent.
				\begin{soln}
					The area of the unit circle is $\pi,$ so the joint density is given by \[f_{U, V}(u, v) = \frac{1}{\pi}\] The marginal density of $u$ is given by \[f_U(u)=\int f_{U, V}(u, v)\, dv = \int_{-\sqrt{1-u^2}}^{\sqrt{1-u^2}} \frac{1}{\pi}\, dv = \frac{2\sqrt{1-u^2}}{\pi}\] Similarly, the marginal density of $v$ is given by \[f_V(v)=\frac{2\sqrt{1-v^2}}{\pi}.\] It's easy to see that these densities are symmetric about the origin, so $E[U]=E[V]=0.$ The covariance is given by 
					\begin{align*}
						\cov(U, V) &= E[UV]-E[U]E[V] = E[UV] \\
						&= \int \int uv\cdot f_{U, V}(u, v)\, dv\, du \\
						&= \frac{1}{\pi}\int_{-1}^1 \int_{-\sqrt{1-u^2}}^{\sqrt{1-u^2}}uv\, dv\, du \\
						&= 0
					\end{align*} but the product of the marginal densities is \[f_U(u)f_V(v)=\frac{2\sqrt{1-u^2}}{\pi}\cdot\frac{2\sqrt{1-v^2}}{\pi}=\frac{4(1-u^2)(1-v^2)}{\pi^2}\neq f_{U, V}(u, v)\] so $U$ and $V$ are not independent.
					
				\end{soln}

			\item Let $W=U^2+V^2.$ Compute the density $f_W(w)$ for $W.$
				\begin{soln}
					Consider the probability $F_W(w)=P(W\le w)=P(U^2+V^2\le w).$ This is a circle of radius $w$ centered at the origin, but $U^2+V^2$ can be anywhere in the unit circle, so this probability is given by \[P(W\le w)=\frac{w^2\pi}{\pi}=w^2\] so the density is given by 
					\begin{align*}
						f_W(w)=\frac{d}{dw}F_W(w) = \frac{d}{dw}\left[ w^2 \right]=2w, \quad0\le 1\le w
					\end{align*}
				\end{soln}

			\item Let $R=\theta U,$ and $T=\theta V,$ where $\theta>0$ is some non-random parameter. Compute the joint distribution of $(R, T).$
				\begin{soln}
					We have \[f_{R, T}(r, t) = f_{U, V}(u, v) \left\lvert \frac{d(u, v)}{d(r, t)} \right\rvert\] where $U=R/\theta$ and $V=T/\theta,$ so the joint density of $R, T$ is given by \[f_{R, T}(r, t) = \frac{1}{\pi}\left\lvert \begin{bmatrix}
							1/\theta & 0 \\ 0 & 1/\theta
					\end{bmatrix}\right\rvert = \frac{1}{\theta^2\pi}\]
				\end{soln}
				
		\end{enumerate}

	\item Suppose we observe independent pairs $(X_i, Y_i)$ where each $(X_i, Y_i)$ has a uniform distribution in the circle of unknown radius $\theta$ and centered at $(0, 0)$ in the plane.

		\begin{enumerate}[(a)]
			\item Show that $(X_i/\theta, Y_i/\theta)$ has a uniform distribution in the unit circle, and find the PDF of $X_i^2+Y_i^2.$
				\begin{proof}
					The joint density of $X_i, Y_i$ is given by \[f_{X_i, Y_i}(x_i, y_i)=\frac{1}{\theta^2\pi}\] so letting $X_i=\theta A, Y_i=\theta B,$ we have the joint density of $A, B$ is \[f_{A, B}(a, b)=f_{X_i, Y_i}(x_i, y_i)\left\lvert \frac{d(x_i, y_i)}{d(a, b)} \right\rvert = \frac{1}{\theta^2\pi} \left\lvert \begin{bmatrix}
						\theta & 0 \\ 0 & \theta
				\end{bmatrix}\right\rvert = \frac{1}{\pi}\] which is exactly the joint density of a uniform distribution on the unit circle, as desired.

				Let $W=X_i^2+Y_i^2.$ Then the CDF of $W$ is given by \[F_W(w)=P(W\le w)=P(X_i^2+Y_i^2\le w)\] which is a circle of radius $w$ centered on the origin, and since $X_i, Y_i$ is uniformly distributed on a circle of radius $\theta,$ this probability is \[F_W(w)=\frac{w^2\pi}{\theta^2\pi}=\frac{w^2}{\theta^2}.\] Thus, the density of $W$ is given by \[f_W(w\mid\theta)=\frac{d}{dw}F_W(w)=\frac{d}{dw}\left[ \frac{w^2}{\theta^2} \right]=\frac{2w}{\theta^2}, \quad 0\le w\le \theta\]
				
				\end{proof}

			\item Show that $(X_1^2+Y_1^2, \cdots, X_n^2+Y_n^2)$ is a sufficient statistic for $\theta.$
				\begin{proof}
					Let $W_i=X_i^2+Y_i^2.$ Then the likelihood function is given by \[f(W_1, \cdots, W_n\mid \theta) = \prod_{i=1}^{n}f(W_i\mid\theta)=\prod_{i=1}^{n}\frac{2w_i}{\theta^2} = \frac{2^n}{\theta^{2n}}\prod_{i=1}^{n}w_i\]
				\end{proof}<++>

			\item Find the MLE and determine its density function and its bias. Are the regularity assumptions were require on the MLE satisfied here?
				\begin{soln}
					As above, the joint density \[f\left[ (X_1, Y_1), \cdots, (X_n, Y_n)\mid \theta \right] = \prod_{i=1}^{n}f\left[ (X_i, Y_i)\mid \theta \right]=\prod_{i=1}^{n}\frac{1}{\theta^2\pi} = \frac{1}{\theta^{2n}\pi^{n}}\] Since $X_i^2+Y_i^2\le \theta^2,$ the MLE $\hat{\theta}$ is \[\hat{\theta}=\max_{1\le i\le n} \sqrt{X_i^2+Y_i^2}\] Consider the CDF of $\hat{\theta}$ 
					\begin{align*}
						F(t)&=P(\hat{\theta}\le t) = P\left( \max_{1\le i\le n} \sqrt{X_i^2+Y_i^2} \le t\right) \\
						&= P\left( \sqrt{X_1^2+Y_1^2},\cdots, \displaystyle \sqrt{X_n^2+Y_n^2}\le t \right) \\
						&= \prod_{i=1}^{n}P\left(\sqrt{X_i^2+Y_i^2}\le t\right) \\
						&= \prod_{i=1}^{n} P(W_i\le t^2) = \prod_{i=1}^{n} \frac{t^2}{\theta^2} = \frac{t^{2n}}{\theta^{2n}} 
					\end{align*} and the density of $\hat{\theta}$ is the derivative of this wrt to $t:$
					\[f_{\hat{\theta}}(t) = \frac{\partial}{\partial t}\left[ \frac{t^{2n}}{\theta^{2n}} \right]=\frac{2n t^{2n-1}}{\theta^{2n}}\] 

					Then $E[\hat{\theta}]$ is given by
					\begin{align*}
						E[ \hat{\theta}] &= \int_0^\theta t\frac{2nt^{2n-1}}{\theta^{2n}}\, dt = \int_0^\theta \frac{2nt^{2n}}{\theta^{2n}}\, dt \\
						&= \frac{2nt^{2n+1}}{\theta^{2n}(2n+1)}\bigg\vert_0^\theta =\frac{2n\theta}{(2n+1)} 
					\end{align*} so the bias of $\hat{\theta}$ is \[E[\hat{\theta}]-\theta=\frac{2n\theta}{2n+1}-\theta = -\frac{\theta}{2n+1}.\]

					The support of the distribution of $(X_i, Y_i)$ is \[\Set{(x_i, y_i)}{f[(x_i, y_i)\mid\theta]>0}=\Set{(x_i, y_i)}{1/\theta^2\pi>0}\] which is the entire domain, and doesn't depend on $\theta.$ Thus the MLE satisfies the regularity conditions.
					
				\end{soln}

			\item Compute the variance of the MLE and simplify it so that it is clear how this variance decays with the sample size $n.$
				\begin{soln}
					The variance of the MLE is given by 
					\begin{align*}
						\var(\hat{\theta}) &= E[\hat{\theta}^2]-(E[\hat{\theta}])^2 
					\end{align*} where 
					\begin{align*}
						E[\hat{\theta}^2] &= \int_0^\theta t^2 \frac{2nt^{2n-1}}{\theta^{2n}}\, dt = \int_0^\theta \frac{2nt^{2n+1}}{\theta^{2n}}\, dt \\
						&= \frac{2nt^{2n+2}}{\theta^{2n}(2n+2)}\bigg\vert_0^\theta = \frac{n\theta^2}{n+1}
					\end{align*} so the variance is 
					\begin{align*}
					\var(\hat{\theta}) &= \frac{n\theta^2}{n+1}-\left( \frac{2n\theta}{2n+1} \right)^2 \\ 
					&= \theta^2\left( \frac{n}{n+1}-\frac{4n^2}{(2n+1)^2} \right) = \frac{n\theta^2}{(n+1)(2n+1)^2}
					\end{align*} Clearly, this diminishes very quickly as $n$ increases.
					
				\end{soln}

			\item Find the MSE of the MLE. As $n\to\infty,$ which term contributes more to the MSE, the squared bias or the variance?
				\begin{soln}
					The MSE is given by 
					\begin{align*}
						E[(\hat{\theta}-\theta)^2] &= \var(\hat{\theta}) + \left( E[\hat{\theta}-\theta] \right)^2 \\
						&= \frac{n\theta^2}{(n+1)(2n+1)^2} + \left( -\frac{\theta}{2n+1} \right)^2 \\
						&\tag{1}=\frac{\theta^2}{(2n+1)^2}\left( \frac{n}{n+1}+1 \right) \\ 
						&= \frac{\theta^2}{(n+1)(2n+1)}
					\end{align*}
					Since \[\frac{n}{n+1}\to1\] as $n\to\infty,$ the squared bias and the variance contribute equally to the MSE.
					
				\end{soln}

			\item Find a method of moments estimator for $\theta$ based on the $X_i$ and call this $\hat{\theta}_X.$
				\begin{soln}
					The marginal density of $X_i$ is given by \[f_X(x)=\frac{2\sqrt{\theta^2-x^2}}{\theta^2\pi}\] which is symmetric about the origin, so $\mu_1=E[X_i]=0.$ Then
					\begin{align*}
						\mu_2 &= E[X_i^2] = \int_{-\theta}^\theta x^2\cdot\frac{2\sqrt{\theta^2-x^2}}{\theta^2\pi}\, dx = \frac{\theta^2}{4}
					\end{align*} according to Wolfram, so the method of moments estimate is \[\hat{\theta}_x=2\sqrt{\hat{\mu}_2}.\] 
				\end{soln}

			\item Compare the performance of the MLE and the method of moments estimator as follows: In R, do the following 10000 times. Sample the uniform distribution in the unit circle using a sample of size 10, and compute the three estimators (MLE, MoM $X_i,$ MoM $Y_i$). Compute estimates of the bias, the variance, and the MSE of each. Estimate the correlation coefficient between $\hat{\theta}_x$ and $\hat{\theta}_y.$ Assuming your estimate in the previous parts are correct, how much should we impove the variance of one of $\hat{\theta}_x$ or $\hat{\theta}_y$ by averaging them?

			\item Show that for the method of moments estimator and the MLE, is it the case that the distribution of $\hat{\theta}/\theta$ does not depend on $\theta.$ Explain why this means we can write \[MSE_\theta(\hat{\theta})=\theta^2\left( MSE_{\theta=1}(\hat{\theta}) \right)\] From this, explain why it suffices that we compare the two estimators when $\theta=1.$
				
		\end{enumerate}
		
\end{enumerate}

\section*{Chapter 9: Testing Hypotheses and Assessing Goodness of Fit}

\begin{itemize}
	\item[2.] Which of the following hypotheses are simple, and which are composite?

		\begin{enumerate}[a.]
			\item $X$ follows a uniform distribution on $[0, 1].$

			\item A die is unbiased.

			\item $X$ follows a normal distribution with mean 0 and variance $\sigma^2>10.$

			\item $X$ follows a normal distribution with mean $\mu=0.$
				
		\end{enumerate}

	\item[5.] True or false, and state why:

		\begin{enumerate}[a.]
			\item The significance level of a statistical test is equal to the probability that the null hypothesis is true.

			\item If the significance level of a test is decreased, the power would be expected to increase.

			\item If a test is rejected at the significance level $\alpha,$ the probability that the null hypothesis is true equals $\alpha.$

			\item The probability that the null hypothesis is falsely rejected is equal to the power of the test.

			\item A type I error occurs when the test statistic falls in the rejection region of the test.

			\item A type II error is more serious than a type I error.

			\item The power of a test is determined by the null distribution of the test statistic.

			\item The likelihood ratio is a random variable.
				
		\end{enumerate}

	\item[4.] Let $X$ have one of the following distributions:
		\begin{center}
			\begin{tabular}{ccc}
				$X$ & $H_0$ & $H_A$ \\
				\hline
				$x_1$ & 0.2 & 0.1 \\
				$x_2$ & 0.3 & 0.4 \\
				$x_3$ & 0.3 & 0.1 \\
				$x_4$ & 0.2 & 0.4
			\end{tabular}
		\end{center}

		\begin{enumerate}[a.]
			\item Compare the likelihood ratio, $\Lambda,$ for each possible value $X$ and order the $x_i$ according to $\Lambda.$

			\item What is the likelihood ratio test of $H_0$ versus $H_A$ are the level $\alpha=0.2?$ What is the test at the level $\alpha=0.5?$

			\item If the prior probabilities are $P(H_0)=P(H_A),$ which outcomes favor $H_0?$

			\item What prior probabilities correspond to the decision rules with $\alpha=0.2$ and $\alpha=0.5?$
				
		\end{enumerate}

	\item[7.] Let $X_1, \cdots, X_n$ be a sample from a Poisson distribution. Find the likelihood ratio for testing $H_0:\lambda=\lambda_0$ versus $H_a:\lambda=\lambda_1,$ where $\lambda_1>\lambda_0.$ Use the fact that the sum of independent Poisson random variables follows a Poisson distribution to explain how to determine a rejection region for a test at level $\alpha.$

	\item[9.] Let $X_1,\cdots, X_{25}$ be a sample from a normal distribution having a variance of 100. Find the rejection region for a test at level $\alpha=0.10$ of $H_0:\mu=0$ versus $H_A: \mu=1.5.$ What is the power of the test? Repeat for $\alpha=0.01.$
		
\end{itemize}

\end{document}

\documentclass{article}
\usepackage[sexy, hdr, fancy]{evan}
\setlength{\droptitle}{-4em}

\lhead{Homework 2}
\rhead{Introduction to Stochastic Processes}
\lfoot{}
\cfoot{\thepage}

\newcommand{\var}{\mathrm{Var}}
\newcommand{\cov}{\mathrm{Cov}}

\begin{document}
\title{Homework 1}
\maketitle
\thispagestyle{fancy}

\begin{enumerate}[(1)]
	\item In a simple symmetric random walk, let $T$ denote the time of the first return to the origin. Use the tail probability representation of the expectation to show that $E[T]=+\infty.$

	\item Let $X$ denote a random variable which has the arc sine distribution.

		\begin{enumerate}[(a)]
			\item Calculate $P\left[ \frac{1}{4}<X<\frac{3}{4} \right].$ 

			\item Calculate $E[X].$

			\item Calculate $\var(X).$
				
		\end{enumerate}

	\item Consider a simple symmetric random walk of length 12. Let $L_{12}$ denote the amount of time that the random walk is positive.

		\begin{enumerate}[(a)]
			\item Use the formula given in class to calculate the values of the frequency function of $L_{12}$ to three decimal places.

			\item To see how good the asymptotic approximation is, find the difference
				\[\abs{P\left[ \frac{1}{4}<\frac{L_{12}}{12}<\frac{3}{4} \right]-P\left[ \frac{1}{4}<X<\frac{3}{4} \right]}\]
				where the latter value was calculated in problem 2a.

		\end{enumerate}

	\item Find the conditional probability that a simple symmetric random walk of length $2n$ is always positive, given that it ends at 0.

		\begin{enumerate}[(a)]
			\item Write an expression in terms of $S_1, S_2, S_3, \cdots, S_{2n}$ for the desired conditional probability, as a ratio of two unconditional probabilities, using the definition of conditional probability.

			\item Write an exact formula for the denominator of the fraction in (a).

			\item To derive an expression for the numerator consider the (relative) complementary event that the random walk goes below the $x$-axis at some time but ends at 0.

			\item Calculate an expression for the probability that a simple symmetric random walk of length $2n$ ends at height -2.

			\item Use parts (b), (c), and (d) to calculate the desired numerator.

			\item Calculate the answer to the original question.
				
		\end{enumerate}
		
\end{enumerate}

\end{document}

\documentclass{article}
\usepackage[sexy, hdr, fancy]{evan}
\setlength{\droptitle}{-4em}

\lhead{Homework 7}
\rhead{Introduction to Stochastic Processes}
\lfoot{}
\cfoot{\thepage}

\newcommand{\var}{\mathrm{Var}}
\newcommand{\cov}{\mathrm{Cov}}

\begin{document}
\title{Homework 7}
\maketitle
\thispagestyle{fancy}

\section*{Chapter 6: Continuous-Time Markov Chains}

\begin{itemize}
	\item[2.] Suppose that a one-celled organism can be in one of two states - either $A$ or $B.$ An individual in state $A$ will change to state $B$ at an exponential rate $\alpha;$ an individual in state $B$ divides into two new individuals of type $A$ at an exponential rate $\beta.$ Define an appropriate continuous-tine Markov chain for a population of such organisms and determine the appropriate parameters for this model.

	\item[5.] There are $N$ individuals in a population, some of whom have a certain infection that spreads as follows. Contacts between two members of this population occur in accordance with a Poisson process having rate $\lambda.$ When a contact occurs, it is equally likely to involve any of the $\binom{N}{2}$ pairs of individuals in the population. If a contact involves an infected and non-infected individual, then with probability  $p$ the non-infected individual becomes infected. Once infected, an individual remains infected throughout. Let $X(t)$ denote the number of infected members of the population at time $t.$
		\begin{enumerate}[(a)]
			\item Is $\left\{ X(t), t\ge0 \right\}$ a continuous-time Markov chain?

			\item Specify its type.

			\item Starting with a single infected individual, what is the expected time until all members are infected?
				
		\end{enumerate}

	\item[6.] Consider a birth and death process with birth rates $\lambda_i=(i+1)\lambda, i\ge0,$ and death rates $\mu_i=i\mu, i\ge0.$
		\begin{enumerate}[(a)]
			\item Determine the expected time to go from state 0 to state 4.

			\item Determine the expected time to go from state 2 to state 5.

			\item Determine the variances in parts (a) and (b).
				
		\end{enumerate}

	\item[9.] The birth and death process with parameters $\lambda_n=0$ and $\mu_n=\mu, n<0$ is called a pure death process. Find $P_{ij}(t).$
		
\end{itemize}

\end{document}

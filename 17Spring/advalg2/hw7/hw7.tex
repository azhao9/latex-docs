\documentclass{article}
\usepackage[sexy, hdr, fancy]{evan}
\setlength{\droptitle}{-4em}

\lhead{Homework 7}
\rhead{Advanced Algebra II}
\lfoot{}
\cfoot{\thepage}

\DeclareMathOperator{\cha}{char}
\DeclareMathOperator{\aut}{aut}

\begin{document}
\title{Homework 7}
\maketitle
\thispagestyle{fancy}

\begin{enumerate}
	\item Let $R$ be a ring, and let $\sigma$ be an automorphism of $R.$ Show that $\Set{a\in R}{\sigma(a)=a}$ is a subring of $R,$ and a subfield if $R$ is a field.
		\begin{proof}
			Call the subset $S.$ Any automorphism must fix 1, so $1\in S.$ Now if $a, b\in S,$ we have
			\begin{align*}
				\sigma(a+b) &= \sigma(a)+\sigma(b) = a+b \\
				\sigma(ab) &= \sigma(a)\sigma(b)=ab
			\end{align*}
			so $a+b, ab\in S,$ so $S$ is indeed a subring. Now, if $R$ is a field, then for all nonzero $a\in R,$
			\[1=\sigma(1)=\sigma\left( a\cdot \frac{1}{a} \right) = \sigma(a) \sigma\left( \frac{1}{a} \right)\]
			Now, if $a\in S,$ then $\sigma(a)=a,$ so
			\[\sigma\left( \frac{1}{a} \right) = \frac{1}{\sigma(a)}=\frac{1}{a}\]
			so $\frac{1}{a}\in S$ as well, and thus $S$ is a field.
		\end{proof}

	\item Let $F$ be a finite field with $p^n$ elements for $p$ a prime. Show that each element $a\in F$ has a $p$th root in $F,$ i.e. there exists $b\in F$ such that $b^p=a.$ Is $b$ unique? By contrast, for $K:=F(x)$ the fraction field of the polynomial ring $F[x],$ show that $x$ has no $p$th root in $K.$
		\begin{proof}
			Since $F=\FF_{p^n}$ is the splitting field of $x^{p^n}-x$ over $\FF_p,$ we have $a^{p^n}=a$ for all $a\in \FF_{p^n}.$ Thus, if $b=a^{p^{n-1}},$ we have
			\[b^p=(a^{p^{n-1}})^p = a^{p^n}=a\]
			so $b$ is a $p$th root of $a.$ If $b^p=c^p=a,$ then $\left( \frac{b}{c} \right)^p=1.$ Since the nonzero elements of $F$ form a cyclic group of order $p^n-1,$ so since $b/c\in F^\times,$ it must be the case that $b/c=1\implies b=c$ so the $p$th root is unique.

			Suppose $x$ had a $p$th root in $K,$ so that for some $f, g\in F[x],$ we have $x=\left( \frac{f}{g} \right)^p.$ Then $g^px=f^p.$ Note that $a^p\neq 0$ for any $0\neq a\in F$ since $F$ is a field and therefore an integral domain. Thus, if $\deg f = m, \deg g=n,$ we have $\deg (g^px)=pn+1=pm=\deg f^p$ which is clearly impossible. Thus, there is no $p$th root of $x,$ as desired.
		\end{proof}
		
\end{enumerate}

\section*{Section 6.4: Finite Fields}

\begin{itemize}
	\item[8.] Find $[\FF_{p^n}:\FF_{p^m}]$ where $m\mid n.$

	\item[18.]
		\begin{enumerate}[(a)]
			\item Show that a monic irreducible polynomial $f\in F[x]$ has no repeated root in any splitting field over $F$ if and only if $f\ncong 0$ in $F[x].$ 

			\item If $\cha F=0,$ show that no irreducible polynomial has a repeated root in any splitting field over $F.$
				
		\end{enumerate}

	\item[19.] If $\cha F=p,$ show that a monic irreducible polynomial $f\in F[x]$ has a repeated root in some splitting field if and only if $f=g(x^p$ for some $g\in F[x].$ (Hint: Ex 18)

	\item[21.] Let $p$ be a prime and write $f=x^p-x-1.$ Show that the splitting field of $f$ over $\FF_p$ is $\FF_p(u),$ where $u$ is any root of $f.$ (Hint: Compute $f(u+a), a\in\FF_p$)
tin
	\item[22.]
		\begin{enumerate}[(a)]
			\item Let $f$ be a monic irreducible polynomial of degree $n$ in $\FF_p[x].$ Show that $f$ divides $x^{p^n}-x$ in $\FF_p[x].$ (Hint: First work over $\FF_p(u), f(u)=0.$ Use the uniqueness in Theorem 4 $\S$ 4.1.)

			\item Show that the degree of each monic irreducible divisor $f$ of $x^{p^n}-x$ is a divisor of $n.$ (Hint: Theorem 5)

			\item Factor $x^8-x$ into irreducibles in $\FF_2[x].$

		\end{enumerate}
		
\end{itemize}

\section*{Section 4.5: Symmetric Polynomials}

\begin{itemize}
	\item[14.] Given $\sigma\in S_n,$ define $\theta_\sigma: R[x_1, \cdots, x_n]\to R[x_1, \cdots, x_n]$ by $\theta_\sigma\left[ f(x_1, \cdots, x_n) \right]=f(x_{\sigma 1}, \cdots, x_{\sigma n}).$
		\begin{enumerate}[(a)]
			\item Show that $\theta_\sigma$ is a ring automorphism of $R[x_1, \cdots, x_n].$
				\begin{proof}
					First we show this is a ring homomorphism. Clearly $\theta_\sigma(1)=1.$ Now, for $f, g\in R[x_1, \cdots, x_n],$
					\begin{align*}
						\theta_\sigma\left[f(x_1, \cdots, x_n)+g(x_1, \cdots, x_n)\right] &= \theta_\sigma\left[ (f+g)(x_1, \cdots, x_n) \right] \\
						&= (f+g)(x_{\sigma1}, \cdots, x_{\sigma n}) \\
						&= f(x_{\sigma1}, \cdots, x_{\sigma n}) + g(x_{\sigma1}, \cdots, x_{\sigma n}) \\
						&= \theta_\sigma(f) + \theta_\sigma(g) \\
						\theta_\sigma\left[f(x_1, \cdots, x_n)\cdot g(x_1, \cdots, x_n)\right] &= \theta_\sigma\left[ (fg)(x_1, \cdots, x_n) \right] \\
						&= (fg)(x_{\sigma1}, \cdots, x_{\sigma n}) \\
						&= f(x_{\sigma1}, \cdots, x_{\sigma n}) \cdot g(x_{\sigma 1}, \cdots, x_{\sigma n}) \\
						&= \theta_\sigma(f)\cdot \theta_\sigma(g)
					\end{align*}

					Now if 
					\[\theta_\sigma(f)=f(x_{\sigma1}, \cdots, x_{\sigma n}) = g(x_{\sigma 1}, \cdots, x_{\sigma n}) = \theta_\sigma(g)\]
					then consider $\sigma\inv$ and its associated $\theta_{\sigma\inv}.$ Then applying $\theta_{\sigma\inv}$ to both of these polynomials, 
					\begin{align*}
						&\theta_{\sigma\inv} \left[ f(x_{\sigma1}, \cdots, x_{\sigma n}) \right] = f(x_{\sigma\inv\sigma1}, \cdots, x_{\sigma\inv\sigma n}) = f(x_1, \cdots, x_n) \\
						=&\theta_{\sigma\inv} \left[ g(x_{\sigma1}, \cdots, x_{\sigma n}) \right] = g(x_{\sigma\inv\sigma 1}, \cdots, x_{\sigma\inv\sigma n}) = g(x_1, \cdots, x_n) 
					\end{align*}
					so $\theta_\sigma$ is injective. Now, for any $f(x_1, \cdots, x_n),$ we have
					\[\theta_\sigma\left[ f(x_{\sigma\inv 1}, \cdots, x_{\sigma\inv n}) \right] = f(x_1, \cdots, x_n)\]
					so $\theta_\sigma$ is surjective. Thus, $\theta_\sigma$ is a bijective ring homomorphism from $R[x_1, \cdots, x_n]$ to itself, so it is a ring automorphism.
				\end{proof}

			\item Show that $\sigma\mapsto \theta_\sigma$ is a group homomorphism $S_n\to \aut R[x_1, \cdots, x_n],$ which is injective.
				\begin{proof}
					Let $\sigma, \tau\in S_n.$ Then consider $\theta_{\sigma\tau}.$ For some $f(x_1, \cdots, x_n)\in R[x_1, \cdots, x_n],$ we have
					\begin{align*}
						\theta_{\sigma\tau}\left[ f(x_1, \cdots, x_n) \right] &= f(x_{\sigma\tau 1}, \cdots, x_{\sigma\tau n}) \\
						&= \theta_\sigma\left[ f(x_{\tau1}, \cdots, x_{\tau n}) \right] \\
						&= \theta_\sigma(\theta_\tau\left[ f(x_1, \cdots, x_n) \right]) \\
						&= (\theta_\sigma \circ \theta_\tau)\left[ f(x_1, \cdots, x_n) \right]
					\end{align*}
					so $(\sigma\tau)\mapsto \theta_{\sigma\tau} = \theta_\sigma\circ\theta_\tau$ and this is indeed a group homomorphism. Now consider the kernel of this homomorphism. The identity in $\aut R[x_1, \cdots, x_n]$ is the identity map, which is
					\[\theta_\varepsilon\left[ f(x_1, \cdots, x_n) \right] = f(x_1, \cdots, x_n)\]
					So the kernel only contains the identity permutation, $\varepsilon.$ Thus, we have $S_n/\left\{ \varepsilon \right\} \cong S_n$ which is isomorphic to the image of this homomorphism, so it is indeed injective.
				\end{proof}

			\item If $G\subseteq \aut R[x_1, \cdots, x_n]$ is a subgroup, show that $S_G=\Set{f}{\theta(f)=f, \forall \theta\in G}$ is a subring of $R[x_1, \cdots, x_n].$
				\begin{proof}
					Clearly $1\in S_G$ since all automorphisms must fix 1. Now if $f, g\in S_G,$ then.
					\begin{align*}
						\theta(f+g) &= \theta(f) + \theta(g) = f + g \\
						\theta(f\cdot g) &= \theta(f)\cdot\theta(g) = f\cdot g
					\end{align*}
					so $f+g, fg\in S_G,$ and thus $S_G$ is indeed a subring of $R[x_1, \cdots, x_n].$
				\end{proof}
				
		\end{enumerate}
		
\end{itemize}

\end{document}

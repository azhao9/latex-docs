\documentclass{article}
\usepackage[sexy, hdr, fancy]{evan}
\setlength{\droptitle}{-4em}

\lhead{Homework 9}
\rhead{Advanced Algebra II}
\lfoot{}
\cfoot{\thepage}

\newcommand{\PP}{\mathbb P}
\DeclareMathOperator{\cha}{char}

\begin{document}
\title{Homework 9}
\maketitle
\thispagestyle{fancy}

\begin{itemize}
	\item[1.] Let $F$ be a field, and define projective $n$-space $\PP^n(F)$ to be the set of 1-dimensional $F$-subspaces in $F^{n+1}.$ Give a group $G$ and a $G$-set $X$ such that the set of orbits for the action is in natural bijection with $\PP^n(F).$ When $F$ is a finite field with $q$ elements, deduce from this that
		\[\# \PP^n(F)=\frac{q^{n+1}-1}{q-1}\] 
		\begin{soln}
			Consider the group $G=F^\times$ and the set $X=F^{n+1}\setminus 0.$ Then the orbits are exactly the 1-dimensional $F$ subspaces of $F^{n+1}.$ If $\# F=q,$ then by the orbit decomposition theorem, we have
			\begin{align*}
				\# X &= \# X_f + \sum_{i=1}^{n} \# (G\cdot x_i)
			\end{align*}
			Here, $X_f$ is empty because nothing in $F^{n+1}\setminus0$ is fixed by every element in $F,$ and $\# X=q^{n+1}-1.$ Then $\#(G\cdot x_i)=q-1$ because if $a, b\in G$ and $x_i=(g_1, \cdots, g_{n+1}),$ then
			\begin{align*}
				a\cdot(g_1, \cdots, g_{n+1}) = (ag_1, \cdots, ag_{n+1}) &= (bg_1, \cdots, bg_{n+1}) = b\cdot(g_1, \cdots, g_{n+1}) \\
				\iff ag_i &= bg_i, \forall i \\
				\iff a &= b
			\end{align*}
			Thus, we have
			\begin{align*}
				q^{n+1}-1 &= 0 + \sum_{i=1}^{n} (q-1) = n(q-1) \\
				\implies n &= \frac{q^{n+1}-1}{q-1}
			\end{align*}
			where $n$ is the number of orbits, which is equal to $\#\PP^n(F),$ as desired.
		\end{soln}
		
\end{itemize}

\section*{Section 10.1: Galois Groups and Separability}

\begin{itemize}
	\item[2.] Prove: If $E\supseteq F$ are fields, $G=\Aut_F(E), u\in E,$ and $\sigma\in G,$ then
		\begin{enumerate}[(1)]
			\item $\sigma[f(u)] = f[\sigma(u)]$ for all $f\in F[x].$
				\begin{proof}
					Let $f=a_0+a_1x_1+\cdots+a_n x^n$ with $a_0, \cdots, a_n\in F.$ Then since $\sigma\in \Aut_F(E),$ it must fix $F,$ so $\sigma(a_i)=a_i$ for all $i.$ Then 
					\begin{align*}
						\sigma[f(u)] &= \sigma(a_0+a_1u + \cdots +a_n u^n) = \sigma(a_0) + \sigma(a_1u) + \cdots + \sigma(a_n u^n) \\
						&= \sigma(a_0) + \sigma(a_1)\sigma(u) + \cdots + \sigma(a_n)\sigma(u)^n \\
						&= a_0 + a_1 \sigma(u) + \cdots + a_n \sigma(u)^n \\
						&= f[\sigma(u)]
					\end{align*}
				\end{proof}
				
			\item In particular, if $u$ is a root of $f,$ then $\sigma(u)$ is also a root of $f.$
				\begin{proof}
					If $u$ is a root of $f,$ then $f(u)=0,$ so
					\[f[\sigma(u)] = \sigma[f(u)] = \sigma(0) = 0\]
					so $\sigma(u)$ is also a root of $f.$
				\end{proof}

			\item If $u$ is algebraic over $F,$ and $\sigma, \tau\in \Aut_F\left( F(u) \right),$ then $\sigma=\tau$ if and only if $\sigma(u)=\tau(u).$
				\begin{proof}
					$(\implies):$ This is trivial. If two maps are the same, then they send $u$ to the same thing.

					$(\impliedby):$ Since $F(u)=\Set{f(u)}{f\in F[x]},$ we have
					\begin{align*}
						\sigma``(F(u)) &= \Set{\sigma[f(u)]}{f\in F[x]} = \Set{f[\sigma(u)]}{f\in F[x]} \\
						&= \Set{f[\tau(u)]}{f\in F[x]} = \Set{\tau[f(u)]}{f\in F[x]} \\
						&= \tau``(F(u))
					\end{align*}
					so $\sigma=\tau.$
				\end{proof}
				
		\end{enumerate}

	\item[13.] If $E=\QQ(\sqrt[4]{2}, i),$ show that $\Aut_{\QQ}(E)\cong D_4.$
		\begin{proof}
			Let $u=\sqrt[4]{2}.$ Then the minimal polynomials of $u$ and $i$ are $x^4-2$ and $x^2+1,$ respectively, with roots $\left\{ u, -u, iu, -iu \right\}$ and $\left\{ i, -i \right\},$ respectively. Then any $\sigma\in \Aut_\QQ(E)$ must have $\sigma(u)\in \left\{ u, -u, iu, -iu \right\}$ and $\sigma(i) \in \left\{ i, -i \right\}.$ So we may find $\sigma, \tau\in \Aut_\QQ(E)$ such that $\sigma(u)=iu, \sigma(i)=i$ and $\tau(u)=u, \tau(i)=-i.$ Then $o(\sigma) = 4$ and $o(\tau) = 2,$ and
			\begin{align*}
				\sigma\tau\sigma(u) &= \sigma\tau(iu) = \sigma\tau(i)\sigma\tau(u) \\
				&= \sigma(-i)\sigma(u) = -\sigma(i)\sigma(u) = (-i)(iu) = u \\
				\tau(u) &= u
			\end{align*}
			so $\left< \sigma, \tau\right>=\Aut_\QQ(E)\cong D_4,$ as desired.
		\end{proof}

	\item[20.] Let $F=K(t)$ denote the field of rational forms over a field $K$ in an indeterminate $t.$ Show that $x^2-t$ is irreducible over $F$ but is not separable if $\cha K=2.$
		\begin{proof}
			Suppose $x^2-t=(x-a)(x-b)$ for $a, b\in K(t).$ Then comparing coefficients, we have
			\begin{align*}
				a+b &= 0 \\
				ab &= -t \\
				\implies a^2=t
			\end{align*}
			Now, if $a=p/q$ for $p, q\in K[t],$ then $t=a^2=p^2/q^2\implies tq^2 = p^2.$ However, $\deg p^2$ is even and $\deg tq^2$ is odd, so this is impossible. Thus $x^2-t$ is irreducible. If $\cha K=2,$ then $(x^2-t)' = 2x \equiv 0,$ so $x^2-t$ would not be separable.
		\end{proof}

	\item[22.] 
		\begin{enumerate}[(a)]
			\item Show that the following are equivalent for a polynomial $f\in F[x;].$ 
				\begin{enumerate}[(1)]
					\ii $f$ has no repeated root in any extension field of $f.$
					\ii $f$ has no repeated root in some splitting field over $F.$
					\ii $f$ and $f'$ are relatively prime in $F[x].$
				\end{enumerate}
				\begin{proof}
					$(1\implies 2):$ This is trivial, since splitting fields are extension fields.

					$(2\implies 3):$ Suppose $f$ splits in a splitting field $E.$ If $f$ and $f'$ were not relatively prime in $F[x],$ then there exists some $d\in F[x]$ such that $d\mid f$ and $d\mid f'$ where $\deg d\ge 1.$ Since $d\mid f,$ it must also split in $E,$ so suppose $d$ has a root $u\in E.$ Then $(x-u)\mid d$ so $(x-u)\mid f$ and $(x-u)\mid f',$ so it must be the case that $(x-u)^2\mid f,$ and thus $f$ has a repeated root. This is a contradiction, so $f$ and $f'$ are relatively prime.

					$(3\implies 1):$ If $f$ has a repeated root $u$ in some extension field $E$ of $f.$ Then $(x-u)^2\mid f\iff (x-u)\mid f, f'.$ If $f$ and $f'$ are relatively prime, then $1=fg + f'h$ for some $g, h\in F[x].$ Since $E$ is an extension field of $F,$ this equation also holds in $E.$ Now, we have $1=f(u)g(u) + f'(u)h(u) = 0,$ a contradiction, so $f$ has no repeated roots in any extension field.
				\end{proof}

			\item If $f$ is as in (a), show that $f$ is separable, but not conversely.
				\begin{proof}
					If $f$ was not separable, then one of its irreducible factors is not separable, say $p\in F[x].$ If $f=pg$ for $g\in F[x],$ then $f'=pg' + p'g,$ and since $p$ is not separable, $p'=0,$ so $f'=pg'.$ Then $\gcd(f, f')=p,$ so $f$ and $f'$ are not relatively prime, which contradicts (3). Thus, $f$ is separable. 
					
					However, consider $f=(x-1)^2.$ Then $f$ is separable because its irreducible factors are both $(x-1),$ which are both separable. However, $f$ has a repeated root, contradicting (1).
				\end{proof}
				
		\end{enumerate}

	\item[25.] If $E\supseteq F$ and $f\in F[x]$ is separable over $F,$ show that $f$ is separable over $E.$
		\begin{proof}
			Suppose $f=pg$ for some irreducible $p\in E[x].$ Since $f$ is separable over $F,$ all of its irreducible factors must be separable. Suppose $f=q_1\cdots q_r$ for irreducible and separable $q_i\in F[x].$ Then since $E$ is an extension field of $F,$ this factorization holds in $E$ as well. Then $f=pg=q_1\cdots q_r$ in $E[x],$ and since $p$ is irreducible in $E[x]$ is it prime, so we must have $p\mid q_i$ for some $i.$ If $p$ was not separable, then it would have a repeated root in $E,$ but then $q_i$ would also have a repeated root in $E,$ which is an extension field of $F,$ which would mean $q_i$ is not separable. This is a contradiction, so $p$ is separable in $E,$ so $f$ is separable in $E,$ as desired.
		\end{proof}


	\item[26.] If $E\supseteq K\supseteq F$ and $E\supseteq F$ is a separable extension, show that both $E\supseteq K$ and $K\supseteq F$ are separable extensions.
		\begin{proof}
			Since $E\supseteq F$ is separable, every $u\in E$ has a separable minimal polynomial over $F.$ Since $K\supseteq E,$ it follows that every $u\in K$ also has a separable minimal polynomial over $F,$ so $K\supseteq F$ is a separable extension. 
			
			For $u\in E,$ let the minimal polynomial of $u$ over $F$ be $f,$ and the minimal polynomial over $K$ be $k.$ Then it follows that $k\mid f$ in $E[x],$ and since $f$ is separable, $k$ must also be separable, and thus $E\supseteq K$ is a separable extension.
		\end{proof}

	\item[27.] Let $F$ have characteristic $p.$ If $f=x^p-a$ where $a\in F,$ show that $f$ is irreducible or a power of a linear polynomial. (Hint: Lemma 5 and Theorem 4)
		\begin{proof}
			Let $f$ have a root $u$ in some extension field $E.$ Then $f(u)=u^p-a=0\implies u^p=a,$ so we have $f=x^p-u^p=(x-u)^p$ since $\cha F=p.$ If $f$ is not irreducible, then this is its factorization in $F[x],$ so then $f$ is a power of a linear polynomial.
			
			If $f$ is not a power of a linear polynomial, then it must be that $u\notin F,$ so $F(u)$ is a splitting field of $f$ over $F.$ Suppose $f$ has a nontrivial irreducible factor $g\in F[x].$ Then $g=(x-u)^q$ for some $1< q<p,$ since $u\notin F.$ Then since $g$ has a repeated root $u,$ we must have $g'\equiv0$ by Lemma 5, so $g$ is not separable, and thus $g=h(x^p)$ by Theorem 4, for some $h\in F[x].$ Since every irreducible factor of $f$ takes this form, we have
			\begin{align*}
				f = h_1(x^p) \cdots h_r (x^p)= x^p-a
			\end{align*}
			Thus we must have $h_i(x^p) = x^p-a$ for some $i$ and the rest are 1, so $f$ is irreducible. 
		\end{proof}
		
\end{itemize}

\end{document}p

\documentclass{article}
\usepackage[sexy, hdr, fancy]{evan}
\setlength{\droptitle}{-4em}

\lhead{Homework 8}
\rhead{Discrete Math (Section 05)}
\lfoot{}
\cfoot{\thepage}

\newcommand{\var}{\mathrm{Var}}
\newcommand{\cov}{\mathrm{Cov}}

\begin{document}
\title{Homework 8}
\maketitle
\thispagestyle{fancy}

\begin{itemize}
	\item[1.] Let $p, x, n\in \ZZ$ with $p$ prime and $n>0.$
		\begin{enumerate}[(a)]
			\item Prove that if $p\mid x^n$ then $p\mid x.$
				\begin{proof}
					By the FTA, $x$ must factor uniquely into a product of primes. Suppose
					\begin{align*}
						x &= p_1^{e_1} \cdots p_k^{e_k}
					\end{align*}
					for $p_1, \cdots p_k$ distinct primes, and $1\le e_1, \cdots, e_k\in\ZZ.$ Then we have
					\begin{align*}
						x^n &= \left(p_1^{e_1}\cdots p_k^{e_k}\right)^n = p_1^{ne_1}\cdots p_k^{ne_k}
					\end{align*}
					If $p\mid x^n,$ then since $p$ is a prime, it must be one of the $p_i$'s in the factorization. But then since this $p_i$ appears in the factorization of $x,$ it follows that $p\mid x.$
				\end{proof}

			\item Show that the statement need not be true if $p$ is not prime.
				\begin{proof}
					If $p=4, x=6, n=2,$ then $4\mid 6^2=36$ but $4\nmid 6.$
				\end{proof}
				
		\end{enumerate}

	\item[2.] Use the Fundamental Theorem of Arithmetic (FTA) to show that $\log_{21} 143$ is irrational.
		\begin{proof}
			Suppose $\log_{21}143$ was rational. Then $\log_{21}143=p/q$ for some $p, q\in\ZZ, q\neq 0,$ and
			\begin{align*}
				q\log_{21}143 &= p \\
				\implies 21^{q\log_{21}143} &= 21^p \\
				\implies 143\cdot 21^q &= 21^p \\
				\implies 11\cdot 13\cdot 3^q\cdot 7^q &= 3^p \cdot 7^p
			\end{align*}
			By the FTA, since 11 and 13 are primes and appear on the LHS, they must appear in the factorization on the RHS, but since they don't, this is a contradiction. Thus $\log_{21}143$ is irrational.
		\end{proof}

	\item[3.] Let $a, b, c, n\in\ZZ$ with $n>1.$
		\begin{enumerate}[(a)]
			\item Prove that if $\gcd(c, n)=1$ and $ac\equiv bc\pmod n,$ then $a\equiv b\pmod n.$
				\begin{proof}
					Since $\gcd(c, n)=1,$ the inverse of $c$ exists in modulo $n.$ Thus, we have
					\begin{align*}
						ac &\equiv bc \pmod n \\
						\implies ac(c\inv) &\equiv bc(c\inv) \pmod n \\
						\implies a &\equiv b \pmod n
					\end{align*}
				\end{proof}

			\item Show that the statement in part (a) need not be true if $c$ and $n$ are not relatively prime.
				\begin{proof}
					If $c=6$ and $n=4,$ then $1\cdot 6 \equiv 3\cdot 6\pmod 4$ but $1\not\equiv 3\pmod 4.$
				\end{proof}
				
		\end{enumerate}

		\newpage
	\item[4.] Let $a, b, c, d, n\in \ZZ$ with $n>1.$ Prove that if $a\equiv b\pmod n$ and $c\equiv d\pmod n$ then
		\begin{enumerate}[(a)]
			\item $a+c\equiv b+d\pmod n$
				\begin{proof}
					If $a\equiv b\pmod n$ then $n\mid (a-b)$ and similarly $n\mid (c-d).$ Thus, $n\mid \left[ (a-b)+(c-d) \right] = (a+c-b-d),$ so $a+c\equiv b+d\pmod n.$
				\end{proof}

			\item $ac\equiv bd\pmod n$
				\begin{proof}
					If $a\equiv b\pmod n$ then $ac\equiv bc\pmod n.$ Similarly, if $c\equiv d\pmod n$ then $bc\equiv bd\pmod n.$ Since $\equiv$ is transitive, it follows that $ac\equiv bd\pmod n.$
				\end{proof}
				
		\end{enumerate}

	\item[5.] In this problem we will use direct proof to prove the following statement: Let $a, p\in\ZZ.$ If $p$ is prime and $p\nmid a$ then $a^{p-1}\equiv1\pmod p.$
		\begin{enumerate}[(a)]
			\item We are given that $a, p\in\ZZ, p$ is prime, and $p\nmid a.$ Explain why this mean $a\neq 0.$
				\begin{answer*}
					If $a=0,$ then $p\mid a$ since 0 is divisible by nonzero integer, which is a contradiction.
				\end{answer*}

			\item Let $S=\left\{ a, 2a, 3, \cdots, (p-1)a \right\}.$ Let $x, y\in S$ with $x\neq y.$ Prove $x\not\equiv y\pmod p.$
				\begin{proof}	
					Suppose $x\equiv y\pmod p,$ then $p\mid (x-y).$ Since $x, y\in S,$ we have $x=ma$ and $y=na$ for $1\le m, n\le p-1.$ Thus $x-y=ma-na=a(m-n),$ so $p\mid a(m-n).$ Since $p\nmid a,$ we must have $p\mid (m-n),$ but this is only possible if $m-n=0\implies m=n,$ which contradicts the fact that $x\neq y.$ Thus $x\not\equiv y\pmod p.$
				\end{proof}

			\item Let $T=\ZZ/p\ZZ\setminus\left\{ 0 \right\}$ and let $f:S\to T$ be given by the rule $f(s)=s\mod p.$ Prove $f$ is a bijection.
				\begin{proof}
					To show $f$ is well defined, suppose $ma, na\in S$ and $ma=na.$ Then
					\begin{align*}
						ma &=q_1p + r_1 \\
						na &= q_2p + r_2
					\end{align*}
					but since this representation is unique and $ma=na,$ it follows that $r_1=r_2$ so 
					\begin{align*}
						f(ma) = ma\mod p = na\mod p = f(na)
					\end{align*}

					Now, suppose for $ma, na\in S,$ we have $r=f(ma)=f(na).$ Then we have
					\begin{align*}
						ma &= q_1p + r \\
						na &= q_2p+r \\
						\implies ma-na &= (q_1p+r)-(q_2p+r) = q_1p-q_2p = (q_1-q_2)p
					\end{align*}
					Thus, $p\mid (ma-na)$ so $ma\equiv na\pmod p,$ but from part (b), this is impossible if $ma\neq na,$ so we must have $ma=na,$ and thus $f$ is injective.

					The distinct values in $T$ are $1, \cdots, p-1,$ and since $\abs{S}=p-1=\abs{T}$ and $f$ is injective, it must also be surjective. Thus, $f$ is a bijection.
				\end{proof}

			\item Explain why
				\[\prod_{s\in S}^{}s \equiv \prod_{t\in T}^{} t\pmod p\]
				\begin{answer*}
					Since the elements of $S$ and the elements of $T$ are in bijection under $f,$ for every $s\in S,$ there is a corresponding $t\in T$ such that $f(s)=s\mod p=t$ Thus, taking the product of all of these pairs, we obtain the required relation.
				\end{answer*}

				\newpage
			\item Explain why $p$ and $(p-1)!$ are relatively prime.
				\begin{answer*}
					Since $p$ is a prime, it is not divisible by anything less than it other than 1. Thus, none of $2, \cdots, p-1$ share any common prime factors with $p$ since otherwise $p$ would have a factor less than $p,$ which is impossible since $p$ is a prime. Thus, the product $2\cdots (p-1)=(p-1)!$ doesn't share any common prime factors with $p,$ so they are relatively prime.
				\end{answer*}

			\item Based on your work in parts (d) and (e), conclude that $a^{p-1}\equiv 1\pmod p.$
				\begin{soln}
					From (d), we have
					\begin{align*}
						\prod_{s\in S}^{}s &= \prod_{i=1}^{p-1} ia = a\cdot 2a\cdots (p-1)a = (p-1)! \cdot a^{p-1} \\
						\prod_{t\in T}^{}t \pmod p &= \prod_{t=1}^{p-1} t \pmod p \equiv (p-1)!\pmod p \\
						\implies (p-1)! \cdot a^{p-1} &\equiv (p-1)!\pmod p
					\end{align*}
					Since $\gcd\left[ p, (p-1)! \right]=1,$ the inverse of $(p-1)!$ exists in modulo $p,$ so multiplying by that inverse on both sides, we get $a^{p-1}\equiv 1\pmod p,$ as desired.
				\end{soln}
				
		\end{enumerate}

	\item[6.] In a transposition cipher we use permutations to help encode text. First we select a positive integer $m.$ Let $M=\Set{x\in\NN}{1\le x\le m}.$ Next, we create a permutation $f:M\to M.$ we then take the text message and split its letters into blocks of size $m.$ We encode the block $b_1b_2\cdots b_m$ as $c_1c_2\cdots c_m$ where $c_i=b_{f(i)}.$ 
		\begin{enumerate}[(a)]
			\item Suppose $m=4$ and $f(1)=3, f(2)=1, f(3)=4, f(4)=2.$ Use the transposition cipher to encode PIRATE ATTACK.
				\begin{soln}
					The blocks of size $m=4$ are PIRA, TEAT, TACK. Thus, we have
					\begin{align*}
						f:\begin{cases}
							1 &\mapsto 3 \\
							2 &\mapsto 1 \\
							3 &\mapsto 4 \\
							4 &\mapsto 2
						\end{cases} \implies \begin{cases}
							PIRA &\mapsto IAPR \\
							TEAT &\mapsto ETTA \\
							TACK &\mapsto AKTC
						\end{cases}
					\end{align*}
					so the encoded message is IAPRETTAAKTC.
				\end{soln}

			\item We decrypt an encoded transposition cipher message by using $f\inv.$ For the function provided in part (a), what is $f\inv?$
				\begin{soln}
					$f\inv \circ f =\id$ so we must have
					\begin{align*}
						f\inv:\begin{cases}
							1&\mapsto 2 \\
							2 &\mapsto 4 \\
							3 &\mapsto 1 \\
							4 &\mapsto 3
						\end{cases}
					\end{align*}
				\end{soln}

			\item Using the decryption function you obtained in part (b), decode SWUETRAEOEHS.
				\begin{soln}
					The blocks of size $m=4$ are SWUE, TRAE, OEHS. Thus, we have
					\begin{align*}
						f\inv:\begin{cases}
							1&\mapsto 2 \\
							2 &\mapsto 4 \\
							3 &\mapsto 1 \\
							4 &\mapsto 3
						\end{cases} \implies \begin{cases}
							SWUE &\mapsto USEW \\
							TRAE &\mapsto ATER \\
							OEHS &\mapsto HOSE
						\end{cases}
					\end{align*}
					so the decoded message is USE WATER HOSE.
				\end{soln}
				
		\end{enumerate}

	\item[7.] Suppose $n=713=23\times 31.$
		\begin{enumerate}[(a)]
			\item Let $e=43.$ Bob's encryption function is $E(M)=M^e\mod n.$ what is his decryption function?
				\begin{soln}
					This is RSA encryption. We have $\varphi(713) = (23-1)(31-1) = 660.$ Now, suppose the decryption function is $D(N)=N^d\mod n.$ Then we must have $de\equiv 1\implies d\equiv e\inv\pmod {660}.$ Thus, we must find $e\inv$ in modulo 660, which exists because $\gcd(43, 660) = 1.$ We have
					\begin{align*}
						660 &= 15\cdot 43 + 15 \\
						43 &= 2\cdot 15 + 13 \\
						15 &= 1\cdot 13 + 2 \\
						13 &= 6\cdot 2 + 1 \\
						\implies 1 &= 13 - 6\cdot 2 \\
						&= 13 - 6\cdot(15-1\cdot 13) = 7\cdot 13 - 6\cdot 15 \\
						&= 7\cdot(43 - 2\cdot 15) - 6\cdot15 = 7\cdot 43 - 20\cdot 15 \\
						&= 7\cdot 43 - 20\cdot(660-15\cdot 43) = 307\cdot 43 - 20\cdot 660
					\end{align*}
					Thus, $43\inv = 307,$ so $d= 307,$ and thus $D(N)=N^{307}\mod{713}$ is the decryption function.
				\end{soln}

			\item Encrypt the word $I$ using Bob's encryption function.
				\begin{soln}
					$I$ is equivalent to 9, so we have $E(9) = 9^{43}\mod 713.$ We have
					\begin{align*}
						9^2 &= 9\cdot 9 = 81 \mod 731 \\
						9^4 &= 9^2\cdot 9^2 = 81\cdot 81 = 144\mod 713 \\
						9^8 &= 9^4\cdot 9^4 \equiv 144\otimes 144 = 59\mod 713 \\
						9^{16} &= 9^8\cdot 9^8 \equiv 59\otimes 59 = 629\mod 713 \\
						9^{32} &= 9^{16}\cdot 9^{16}\equiv 629\otimes 629 = 639 \mod 713 \\
						9^{11} &= 9\cdot 9^2\cdot 9^8 \equiv 9\otimes 81\otimes 59 = 231\mod 731 \\
						9^{43} &= 9^{32}\cdot 9^{11} \equiv 639\cdot 231=18\mod 713
					\end{align*}
					Thus, $E(9) = 18\to R.$
				\end{soln}

			\item Let $d=43.$ Sue's decryption function is $D(N)=N^d\mod n.$ What is Sue's encryption function?
				\begin{soln}
					If the encryption function is $E(M)=M^e\mod 713,$ then we have $D(E(M)) = M^{ed}\mod 713 = M.$ If $d=43,$ then by switching the roles of $d$ and $e$ in part (a), we have $e=307,$ so the encryption function is $E(M)=M^{307}\mod 713.$
				\end{soln}
				
		\end{enumerate}
		
\end{itemize}

\end{document}

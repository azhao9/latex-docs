\documentclass{article}
\usepackage[sexy, hdr, fancy]{evan}
\setlength{\droptitle}{-4em}

\lhead{Homework 8}
\rhead{Discrete Math (Section 05)}
\lfoot{}
\cfoot{\thepage}

\newcommand{\var}{\mathrm{Var}}
\newcommand{\cov}{\mathrm{Cov}}

\begin{document}
\title{Homework 8}
\maketitle
\thispagestyle{fancy}

\begin{itemize}
	\item[1.] Let $p, x, n\in \ZZ$ with $p$ prime and $n>0.$
		\begin{enumerate}[(a)]
			\item Prove that if $p\mid x^n$ then $p\mid x.$

			\item Show that the statement need not be true if $p$ is not prime.
				
		\end{enumerate}

	\item[2.] Use the Fundamental Theorem of Arithmetic (FTA) to show that $\log_{21} 143$ is irrational.

	\item[3.] Let $a, b, c, n\in\ZZ$ with $n>1.$
		\begin{enumerate}[(a)]
			\item Prove that if $\gcd(c, n)=1$ and $ac\equiv bc\pmod n,$ then $a\equiv b\pmod n.$

			\item Show that the statement in part (a) need not be true if $c$ and $n$ are not relatively prime.
				
		\end{enumerate}

	\item[4.] Let $a, b, c, d, n\in \ZZ$ with $n>1.$ Prove that if $a\equiv b\pmod n$ and $c\equiv d\pmod n$ then
		\begin{enumerate}[(a)]
			\item $a+c\equiv b+d\pmod n$

			\item $ac\equiv bd\pmod n$
				
		\end{enumerate}

	\item[5.] In this problem we will use direct proof to prove the following statement: Let $a, p\in\ZZ.$ If $p$ is prime and $p\nmid a$ then $a^{p-1}\equiv1\pmod p.$
		\begin{enumerate}[(a)]
			\item We are given that $a, p\in\ZZ, p$ is prime, and $p\nmid a.$ Explain why this mean $a\neq 0.$

			\item Let $S=\left\{ a, 2a, 3, \cdots, (p-1)a \right\}.$ Let $x, y\in S$ with $x\neq y.$ Prove $x\not\equiv y\pmod p.$

			\item Let $T=\ZZ/p\ZZ\setminus\left\{ 0 \right\}$ and let $f:S\to T$ be given by the rule $f(s)=s\mod p.$ Prove $f$ is a bijection.

			\item Explain why
				\[\prod_{s\in S}^{}s \equiv \prod_{t\in T}^{} t\pmod p\]

			\item Explain why $p$ and $(p-1)!$ are relatively prime.

			\item Based on your work in parts (d) and (e), conclude that $a^{p-1}\equiv 1\pmod p.$
				
		\end{enumerate}

	\item[6.] In a transposition cipher we use permutations to help encode text. First we select a positive integer $m.$ Let $M=\Set{x\in\NN}{1\le x\le m}.$ Next, we create a permutation $f:M\to M.$ we then take the text message and split its letters into blocks of size $m.$ We encode the block $b_1b_2\cdots b_m$ as $c_1c_2\cdots c_m$ where $c_i=b_{f(i)}.$ 
		\begin{enumerate}[(a)]
			\item Suppose $m=4$ and $f(1)=3, f(2)=1, f(3)=4, f(4)=2.$ Use the transposition cipher to encode PIRATE ATTACK.

			\item We decrypt an encoded transposition cipher message by using $f\inf.$ For the function provided in part (a), what is $f\inv?$

			\item Using the decryption function you obtained in part (b), decode SWUETRAEOEHS.
				
		\end{enumerate}

	\item Suppose $n=713=23\times 31.$
		\begin{enumerate}[(a)]
			\item Let $e=43.$ Bob's encryption function is $E(M)=M^e\mod n.$ what is his decryption function?

			\item Encrypt the word $I$ using Bob's encryption function.

			\item Let $d=43.$ Sue's decryption function is $D(N)=N^d\mod n.$ What is Sue's encryption function?
				
		\end{enumerate}
		
\end{itemize}

\end{document}

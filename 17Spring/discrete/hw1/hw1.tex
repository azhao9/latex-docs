\documentclass{article}
\usepackage[sexy, hdr, fancy]{evan}
\setlength{\droptitle}{-4em}

\lhead{Homework 1}
\rhead{Discrete Math}
\lfoot{}
\cfoot{\thepage}

\newcommand{\var}{\mathrm{Var}}
\newcommand{\cov}{\mathrm{Cov}}

\begin{document}
\title{Homework 1}
\maketitle
\thispagestyle{fancy}

\begin{enumerate}
	\item Provide a definition for each of the following terms.

		\begin{enumerate}[(a)]
			\item convex set
				\begin{definition*}
					A convex set $A$ is a subset of $\RR^n$ such that for any two points $p_1, p_2\in A,$ the line joining the two points is also contained in $A.$
				\end{definition*}

			\item convex function
				\begin{definition*}
					A mapping $f:\RR\to\RR$ is a convex function if for any two values $a, b\in\RR,$ the line joining the points $(a, f(a))$ and $(b, f(b))$ lies on or above the graph of $f.$
				\end{definition*}

			\item linear combination
				\begin{definition*}
					Given $a_1, \cdots, a_n\in\RR,$ a linear combination of the $a_i$ is given by
					\[a_1c_1+a_2c_2+\cdots+a_nc_n\]
					where $c_i\in\RR, \forall i.$
				\end{definition*}

			\item convex combination
				\begin{definition*}
					Given $a_1, \cdots, a_n\in \RR,$ a convex combination of the $a_i$ is a linear combination
					\[a_1\lambda_1 + a_2\lambda_2+\cdots+a_n\lambda_n\]
					where $\lambda_i\in\left[ 0, 1 \right], \forall i$ such that $\displaystyle\sum_{i=1}^{n}\lambda_i=1.$
				\end{definition*}

		\end{enumerate}

	\item Based on your definitions in the previous problem,

		\begin{enumerate}[(a)]
			\item which of the following six sets are convex and which at not? Justify.
				\begin{soln}
					Set 1 is not convex, consider a segment joining the top-most point and the top-left corner; this segment is not contained in the set.

					Set 2 is not convex, consider a segment joining the top-most point and the left-most point; this segment is not contained in the set.

					Set 3 is convex. For any two points in the set, the line joining them will always be in the set. 

					Set 4 is not convex, consider a point above the hole and a point beneath the hole within the set; the segment joining these two points is not contained in the set.

					Set 5 is convex. For any two points on the line, the segment between them will still be a subset of the line.

					Set 6 is convex. For any two points in the set, the line joining them will always be in the set.

				\end{soln}

				\newpage
			\item which of the following functions are convex and which are not? Justify.
				\begin{soln}
					Function 1 is convex. If the two points are on different branches, the segment joining them will lie above the function. If they are on the same branch, the segment joining them will be a subset of the graph.

					Function 2 is not convex, consider a segment joining the left and right endpoints; this segment intersects the graph so it does not lie on or above it.

					Function 3 is not convex, consider a segment joining the left and right endpoints; this segment lies entirely below the graph.
				\end{soln}
				
		\end{enumerate}

	\item How would you define the

		\begin{enumerate}[(a)]
			\item binary representation for a natural number $n?$
				\begin{definition*}
					For $n\in\NN,$ the binary representation of $n$ is 
					\[n=d_k 2^k + d_{k-1}2^{k-1}+\cdots+d_1(2)+d_0\]
					where $d_i\in\left\{ 0, 1 \right\}, 0\le i\le k$ and $d_k\neq 0.$
				\end{definition*}

			\item ternary representation for a natural number $n?$
				\begin{definition*}
					For $n\in\NN,$ the ternary representation of $n$ is 
					\[n=d_k 3^k + d_{k-1}3^{k-1}+\cdots+d_1(3)+d_0\]
					where $d_i\in\left\{ 0, 1, 2 \right\}, 0\le i\le k$ and $d_k\neq 0.$
				\end{definition*}

			\item decimal representation for a rational number $q$ such that $0<q<1?$
				\begin{definition*}
					For $q\in\QQ$ where $0<q<1,$ the decimal representation of $q$ is
					\[\sum_{i=1}^{\infty}d_i 10^{-i}\]
					where $d_i\in\left\{ 0, 1, \cdots, 9 \right\}.$ Note that this is an infinite sum because some rational numbers do not terminate. For the ones that do, $d_k=0$ for $k>N$ sufficiently large. 
				\end{definition*}
				
		\end{enumerate}

	\item Let $w$ be a positive integer, Then the sum of any $w$ consecutive integers is always divisible by $w.$

		\begin{enumerate}[(a)]
			\item How would you define two consecutive integers?
				\begin{answer*}
					Two integers $a$ and $b$ are consecutive if $\abs{a-b}=1.$
				\end{answer*}

			\item Prove that the sum of any two consecutive integers is always odd.
				\begin{proof}
					Let $a, b$ be consecutive integers. WLOG, $a<b,$ so $b-a=1\implies b=a+1.$ Then 
					\[a+b=a+(a+1)=2a+1\equiv1\pmod 2\]
					so the sum of any two consecutive integers is odd, as desired.
				\end{proof}
				
			\item Prove that the sum of any three consecutive integers is always divisible by 3.
				\begin{proof}
					Let $a, b, c$ be consecutive integers. WLOG, $a<b<c,$ so $b=a+1$ and $c=b+1=a+2.$ Then
					\[a+b+c=a+(a+1)+(a+2)=3a+3\equiv0\pmod 3\]
					so the sum of any three consecutive integers is divisible by 3, as desired.
				\end{proof}

				\newpage
			\item Prove that the sum of any five consecutive integers is always divisible by 5.
				\begin{proof}
					Let $a, b, c, d, e$ be consecutive integers. WLOG, $a<b<c<d<e,$ so similarly to parts (b) and (c), the sum is given by
					\[a+b+c+d+e=a+(a+1)+(a+2)+(a+3)+(a+4)=5a+10\equiv0\pmod 5\]
					so the sum of any five consecutive integers is divisible by 5, as desired.
				\end{proof}

			\item Disprove the statement given at the start of this exercise.
				\begin{soln}
					If $w=2,$ the statement is false. The sum of two consecutive integers is odd, which is never divisible by $w$ which is 2.
				\end{soln}

			\item Conjecture: for what values of $w$ do you think the statement is true?
				\begin{answer*}
					The statement is true for odd values of $w.$
				\end{answer*}

		\end{enumerate}

	\item Prove for all $x\in\RR$ and for all $m\in\ZZ,$ 
		\[\left\lfloor x+m \right\rfloor=\left\lfloor x \right\rfloor+m\]
		\begin{proof}
			If $x$ is an integer, then
			\[\left\lfloor x+m \right\rfloor=x+m=\left\lfloor x \right\rfloor+m\]
			Otherwise, $n<x<n+1$ for some integer $n,$ and $n+m<x+m<n+m+1.$ Then
			\[\left\lfloor x+m \right\rfloor=n+m=\left\lfloor x \right\rfloor+m\]
			as desired.
		\end{proof}

	\item Let $x\in\RR.$ Prove that if $\left\lfloor x \right\rfloor=\left\lceil x \right\rceil$ then $x$ is an integer.
		\begin{proof}
			We prove the contrapositive. If $x$ is not an integer, it is between two integers $m<x<n+1.$ Then
			\[\left\lfloor x \right\rfloor=m\neq m+1=\left\lceil x \right\rceil\]
			Thus, it must be that $x$ is an integer, as desired.
		\end{proof}

	\item Prove/Disprove: For all primes $p, 2p+1$ is also prime.
		\begin{proof}
			This is false. A counterexample is $p=7,$ which is prime, but $2\cdot7+1=15$ is not prime.
		\end{proof}

	\item Prove using direct proof by cases: Every integer that is a perfect cube is either a multiple of 9, or 1 more than a multiple of 9, or 1 less than a multiple of 9.
		\begin{proof}
			Any integer can be represented as one of $3k-1, 3k,$ or $3k+1$ for $k\in\ZZ.$ It follows that any perfect cube can be represented as one of $(3k-1)^3, (3k)^3,$ or $(3k+1)^3.$ 

			Case 1:
			\[(3k-1)^3=27k^3-27k^2+9k-1\equiv-1\pmod 9\]

			Case 2:
			\[(3k)^3=27k^3\equiv0\pmod 9\]

			Case 3:
			\[(3k+1)^3=27k^3+27k^2+9k+1\equiv1\pmod 9\]

			Thus, the statement is proven.
		\end{proof}
		
\end{enumerate}

\end{document}

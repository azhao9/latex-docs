\documentclass{article}
\usepackage[sexy, hdr, fancy]{evan}
\setlength{\droptitle}{-4em}

\lhead{Homework 6}
\rhead{Discrete Math (Section 05)}
\lfoot{}
\cfoot{\thepage}

\newcommand{\var}{\mathrm{Var}}
\newcommand{\cov}{\mathrm{Cov}}
\DeclareMathOperator{\ima}{im}
\DeclareMathOperator{\dom}{dom}

\begin{document}
\title{Homework 1}
\maketitle
\thispagestyle{fancy}

\begin{itemize}
	\item[22.12] Prove: For every positive integer $n,$ the Tower of Hanoi puzzle with $n$ disks can be solved in $2^n-1$ moves.
		\begin{proof}
			$n=1:$ If there is only 1 disk, it can be moved to the correct position in $1=2^1-1$ moves, so the base case is satisfied. Suppose a puzzle with $k$ disks is solved in $2^k-1$ moves. Then for a puzzle with $k+1$ discs, we can equivalently move the first $k$ disks to the middle position in $2^k-1$ moves. Then we move the bottom disk to the correct position, and then move the $k$ middle discs to the correct position in $2^k-1$ moves. The total number of moves is $(2^k-1)+1+(2^k-1)=2^{k+1}-1,$ so the formula holds for $k+1,$ and the statement is proved by induction.
		\end{proof}

	\item[22.16]
		\begin{itemize}
			\item[(e)] Let $e_0=1, e_1=4,$ and for $n>1,$ let $e_n=4(e_{n-1}-e_{n-2}).$ What are the first five terms of the sequence $e_0, e_1, e_2, \cdots?$ Prove $e_n=(n+1)2^n.$
				\begin{proof}
					we have
					\begin{align*}
						e_0 &= 1 \\
						e_2 &= 4 \\
						e_2 &= 4(4-1)=12 \\
						e_3 &= 4(12-4)=32 \\
						e_4 &= 4(32-12)=80
					\end{align*}
					Now proceed by strong induction. For $n=0,$ we have $e_0=1=(0+1)2^0,$ so the base case is satisfied. Then suppose that the formula holds for all of 0 to $k.$ That means $e_k=(k+1)2^k$ and $e_{k-1}=k2^{k-1}.$ Then
					\begin{align*}
						e_{k+1} &= 4(e_k-e_{k-1}) = 4\left[ (k+1)2^k-k2^{k-1} \right] \\
						&= 4k2^k+4\cdot 2^k - 4k2^{k-1} = k2^{k+2}+2^{k+2}-k2^{k+1} \\
						&= 2^{k+1}(2k+2-k)=\left[ (k+1)+1 \right]2^{k+1}
					\end{align*}
					so the formula holds for $k+1$ and the statement is proved by strong induction.
				\end{proof}
				
		\end{itemize}

	\item[3.] Let $n$ be a positive integer. Use induction to prove that
		\[\sum_{j=1}^{n} j^4=\frac{6n^5+15n^4+10n^3-n}{30}\]
		\begin{proof}
			$n=1:$ The base case is satisfied because
			\[1^4=1=\frac{6+15+10-1}{30}\]
			Now suppose the formula holds for arbitrary $k.$ Then we have
			\begin{align*}
				\sum_{j=1}^{k+1}j^4 &= \sum_{j=1}^{k} j^4+(k+1)^4 = \frac{6k^5+15k^4+10k^3-k}{30} + (k+1)^4 \\
				&= \frac{(6k^5+15k^4+10k^3-k) + 30(k^4+4k^3+6k^2+4k+1)}{30} \\
				&= \frac{6k^5 + 45k^4+130k^3+180k^2+119k+30}{30} \\
				&= \frac{6(k+1)^5+15(k+1)^4+10(k+1)^3-(k+1)}{30}
			\end{align*}
			so the formula holds for $k+1$ and the statement is proved by induction.
		\end{proof}

	\item[4.] Consider the following nonlinear recurrence relation defined for $n\in\NN:$
		\[a_0=1, \quad a_n=na_0+(n-1)a_1+(n-2)a_2+\cdots+2a_{n-2}+1a_{n-1}\]
		\begin{enumerate}[(a)]
			\item Calculate $a_1, a_2, a_3, a_4$
				\begin{soln}
					\begin{align*}
						a_1 &= 1a_0 = 1 \\
						a_2 &= 2a_0+1a_1=3 \\
						a_3 &= 3a_0+2a_1+1a_2 = 8 \\
						a_4 &= 4a_0+3a_1+2a_2+1a_3 = 21
					\end{align*}
				\end{soln}

			\item Use induction to prove for all positive integers $n$ that
				\[a_n=\frac{1}{\sqrt{5}}\left[ \left( \frac{3+\sqrt{5}}{2} \right)^n-\left( \frac{3-\sqrt{5}}{2} \right)^n \right]\]
				\begin{proof}
					$n=1:$ The base case is satisfied because
					\begin{align*}
						1 = a_1 &= \frac{1}{\sqrt{5}}\left[ \left( \frac{3+\sqrt{5}}{2} \right)^1-\left( \frac{3-\sqrt{5}}{2} \right)^1 \right] = \frac{1}{\sqrt{5}}\cdot \frac{2\sqrt{5}}{2}
					\end{align*}
					Now suppose the formula holds for arbitrary $k.$ Note that
					\begin{align*}
						a_k &= ka_0+(k-1)a_1+(k-2)a_2+\cdots+2a_{k-2}+1a_{k-1} \\
						a_{k+1} &= (k+1)a_0+ka_1+(k-1)a_2+\cdots+3a_{k-2}+2a_{k-1}+1a_k \\
						\implies a_{k+1}-a_k &= a_0+a_1+a_2+\cdots+a_{k-2}+a_{k-1}+a_k
					\end{align*}
					The RHS is given by
					\begin{align*}
						\sum_{i=0}^{k} \frac{1}{\sqrt{5}}\left[ \left( \frac{3+\sqrt{5}}{2} \right)^i - \left( \frac{3-\sqrt{5}}{2} \right)^i \right] &= \frac{1}{\sqrt{5}}\left[\sum_{i=0}^{k} \left( \frac{3+\sqrt{5}}{2} \right)^i - \sum_{i=0}^{k} \left( \frac{3-\sqrt{5}}{2} \right)^i\right]
					\end{align*}
					These are the sums of two geometric series, and the closed form is
					\begin{align*}
						&\frac{1}{\sqrt{5}}\left[ \frac{\left( \frac{3+\sqrt{5}}{2} \right)^{k+1}-1}{\frac{3+\sqrt{5}}{2} - 1} - \frac{\left( \frac{3-\sqrt{5}}{2} \right)^{k+1}-1}{\frac{3-\sqrt{5}}{2} - 1} \right] = \frac{1}{\sqrt{5}}\left[ \frac{\left( \frac{3+\sqrt{5}}{2} \right)^{k+1}-1}{\frac{1+\sqrt{5}}{2}} - \frac{\left( \frac{3-\sqrt{5}}{2} \right)^{k+1}-1}{\frac{1-\sqrt{5}}{2}} \right] \\
						&\quad = \frac{1}{\sqrt{5}\left( \frac{1+\sqrt{5}}{2} \right)\left( \frac{1-\sqrt{5}}{2} \right)}\left( \left( \frac{1-\sqrt{5}}{2} \right)\left[ \left( \frac{3+\sqrt{5}}{2} \right)^{k+1}-1 \right] - \left( \frac{1+\sqrt{5}}{2}\right) \left[ \left( \frac{3-\sqrt{5}}{2} \right)^{k+1}-1 \right] \right) \\
						&\quad = -\frac{1}{\sqrt{5}}\left[ \left( \frac{1-\sqrt{5}}{2} \right) \left( \frac{3+\sqrt{5}}{2} \right)^{k+1}- \left( \frac{1+\sqrt{5}}{2} \right)\left( \frac{3-\sqrt{5}}{2} \right)^{k+1} + \sqrt{5} \right] \\
						&\quad = \frac{1}{\sqrt{5}}\left[ \left( \frac{1+\sqrt{5}}{2} \right) \left( \frac{3-\sqrt{5}}{2} \right)^{k+1}- \left( \frac{1-\sqrt{5}}{2} \right)\left( \frac{3+\sqrt{5}}{2} \right)^{k+1} - \sqrt{5}\right]
					\end{align*}
					Then $a_{k+1}$ is obtained by adding $a_k$ to the result above, which is
					\begin{align*}
						&\frac{1}{\sqrt{5}}\left[ \left( \frac{1+\sqrt{5}}{2} \right) \left( \frac{3-\sqrt{5}}{2} \right)^{k+1}- \left( \frac{1-\sqrt{5}}{2} \right)\left( \frac{3+\sqrt{5}}{2} \right)^{k+1} - \sqrt{5}\right] + \frac{1}{\sqrt{5}} \left[ \left( \frac{3+\sqrt{5}}{2}  \right)^k - \left( \frac{3-\sqrt{5}}{2} \right)^k\right] \\
						&\quad=\frac{1}{\sqrt{5}}\left[ \left( \frac{1+\sqrt{5}}{2}\cdot \frac{3-\sqrt{5}}{2} - 1 \right)\left( \frac{3-\sqrt{5}}{2} \right)^k - \left( \frac{1-\sqrt{5}}{2}\cdot \frac{3+\sqrt{5}}{2} -1 \right)\left( \frac{3+\sqrt{5}}{2} \right)^k - \sqrt{5}\right] \\
						&\quad= \frac{1}{\sqrt{5}}\left[ \left( \frac{\sqrt{5}-5}{2} \right)\left( \frac{3-\sqrt{5}}{2} \right)^k - \left( \frac{-\sqrt{5}-5}{2} \right)\left( \frac{3+\sqrt{5}}{2} \right)^k - \sqrt{5} \right] \\
						&= \frac{1}{\sqrt{5}}\left[ \left( \frac{5+\sqrt{5}}{2} \right)\left( \frac{3+\sqrt{5}}{2} \right)^k - \left( \frac{5-\sqrt{5}}{2} \right)\left( \frac{3-\sqrt{5}}{2} \right)^k - \sqrt{5} \right] \\
						&= \frac{1}{\sqrt{5}}\left[ \left( 1+\frac{3+\sqrt{5}}{2} \right)\left( \frac{3+\sqrt{5}}{2} \right)^k - \left( 1+\frac{3-\sqrt{5}}{2} \right)\left( \frac{3-\sqrt{5}}{2} \right)^k - \sqrt{5} \right] \\
						&= \frac{1}{\sqrt{5}}\left[ \left( \frac{3+\sqrt{5}}{2} \right)^{k+1} - \left( \frac{3-\sqrt{5}}{2} \right)^{k+1} + \frac{3+\sqrt{5}}{2} - \frac{3-\sqrt{5}}{2} - \sqrt{5} \right] \\
						&= \frac{1}{\sqrt{5}}\left[ \left( \frac{3+\sqrt{5}}{2} \right)^{k+1} - \left( \frac{3-\sqrt{5}}{2} \right)^{k+1} \right]
					\end{align*}
					Thus, the formula holds for $k+1,$ so the statement is proved by induction.
				\end{proof}
				
		\end{enumerate}

	\item[24.1] For each of the following relations, please answer these questions:
		\begin{enumerate}[(1)]
			\item Is it a function? If not, explain why and stop.
			\item What are its domain and image?
			\item Is the function one-to-one? If not, explain why and stop.
			\item What is its inverse function?
		\end{enumerate}
		\begin{enumerate}[(a)]
			\item $\left\{ (1, 2), (3, 4) \right\}$
				\begin{answer*}
					This is a function. Its domain is $\left\{ 1, 3 \right\}$ and its range is $\left\{ 2, 4 \right\}.$ The function is one-to-one. The inverse function is $\left\{ (2, 1), (4, 3) \right\}.$
				\end{answer*}

			\item $\Set{(x, y)}{x, y\in\ZZ, y=2x}$
				\begin{answer*}
					This is a function. Its domain is $\ZZ$ and its image is $2\ZZ.$ The function is one-to-one. The inverse function is $\Set{(x, y)}{x, y\in\ZZ, x=2y}.$
				\end{answer*}

			\item $\Set{(x, y)}{x, y\in \ZZ, x+y=0}$
				\begin{answer*}
					This is a function. Its domain is $\ZZ$ and its image is $\ZZ.$ The function is one-to-one. The inverse function is $\Set{(x, y)}{x, y\in \ZZ, x+y=0}.$
				\end{answer*}

			\item $\Set{(x, y)}{x, y\in\ZZ, xy=0}$
				\begin{answer*}
					This is not a function. When $x=0,$ then $(0, y)$ satisfies the relation for all $y\in\ZZ,$ so $x$ is mapped to more than a single value.
				\end{answer*}

			\item $\Set{(x, y)}{x, y\in\ZZ, y=x^2}$
				\begin{answer*}
					This is a function. Its domain is $\ZZ$ and its image is $\ZZ_{\ge0}.$ The function is not one-to-one because $(2, 4)$ and $(-2, 4)$ are both in the relation, but $2\neq -2.$
				\end{answer*}

			\item $\varnothing$
				\begin{answer*}
					This is a function. Its domain is $\varnothing$ and its image is $\varnothing.$ The function is one-to-one. The inverse function is $\varnothing.$
				\end{answer*}

			\item $\Set{(x, y)}{x, y\in\QQ, x^2+y^2=1}$
				\begin{answer*}
					This is not a function. The pairs $(0.6, 0.8)$ and $(0.6, -0.8)$ are both in the relation, so $0.6$ is mapped to more than a single value.
				\end{answer*}

			\item $\Set{(x, y)}{x, y\in\ZZ, x\mid y}$
				\begin{answer*}
					This is not a function. The pairs $(1, 2)$ and $(1, 3)$ are both in the relation, so 1 is mapped to more than a single value.
				\end{answer*}

			\item $\Set{(x, y)}{x, y\in\NN, x\mid y, y\mid x}$
				\begin{answer*}
					This is a function since the condition is equivalent to $x=y.$ The domain is $\NN$ and the image is $\NN.$ The function is one-to-one. The inverse function is $\Set{(x, y)}{x, y\in\NN, x=y}.$
				\end{answer*}

			\item $\Set{(x, y)}{x, y\in\NN, \binom{x}{y}=1}$
				\begin{answer*}
					This is not a function. Since $\binom{2}{0}=\binom{2}{2}=1,$ the pairs $(2, 0)$ and $(2, 2)$ are in the relation, so 2 is mapped to more than a single value.
				\end{answer*}

		\end{enumerate}

	\item[24.23]
		\begin{enumerate}[(a)]
			\item Let $f:\ZZ\to\ZZ$ by $f(x)=\abs{x}.$ If $X=\left\{ -1, 0, 1, 2 \right\},$ find $f(X).$
				\begin{answer*}
					We have $f(X)=\left\{ f(-1), f(0), f(1), f(2) \right\} = \left\{ 0, 1, 2 \right\}.$
				\end{answer*}

			\item Let $f:\RR\to\RR$ by $f(x)=\sin x.$ If $X=[0, \pi],$ find $f(X).$
				\begin{answer*}
					The $\sin$ function takes on values from 0 to 1 inclusive over $[0, \pi],$ so $f(X)=[0, 1].$
				\end{answer*}

			\item Let $f:\RR\to \RR$ by $f(x)=2^x.$ If $X=[-1, 1],$ find $f(X).$
				\begin{answer*}
					Since $2^x$ is an increasing function, its minimum value over $X$ is $2^{-1}=1/2$ and its maximum value is $2^1=2,$ so $f(X)=\left[ \frac{1}{2}, 2 \right].$
				\end{answer*}

			\item Let $f:\ZZ\to\ZZ$ by $f(x)=3x-1.$ What is $f(\left\{ 1 \right\})?$ Is it the same as $f(1)?$
				\begin{answer*}
					We have $f(\left\{ 1 \right\}) = \left\{ f(1) \right\} = \left\{ 2 \right\}.$ It is not the same as $f(1)=2$ because the former is a set, while the latter is a number.
				\end{answer*}

			\item Let $f:A\to B$ be a function. What is $f(A)?$
				\begin{answer*}
					Here, $f(A)$ is the image of $f$ as a function.
				\end{answer*}
				
		\end{enumerate}

	\item[24.24]
		\begin{enumerate}[(a)]
			\item Let $f:\ZZ\to\ZZ$ by $f(x)=\abs{x}.$ If $Y=\left\{ 1, 2, 3 \right\}$ find $f\inv(Y).$
				\begin{answer*}
					Under absolute value, both 1 and $-1$ are mapped to 1, and similarly for 2 and 3. So $f\inv(Y)=\left\{ -3, -2, -1, 1, 2, 3 \right\}.$
				\end{answer*}

			\item Let $f:\RR\to\RR$ by $f(x)=x^2.$ If $Y=[1, 2],$ find $f\inv(Y).$
				\begin{answer*}
					Under $f,$ the interval $[1, \sqrt{2}]$ maps to $Y$ since $f(1)=1$ and $f(\sqrt{2})=2$ and $f$ is increasing over $[1, \sqrt{2}].$ The interval $[-\sqrt{2}, -1]$ also maps to $Y$ since $f(-\sqrt{2})=2$ and $f(-1)=1$ and $f$ is decreasing over $[-\sqrt{2}, -1].$ Thus $f\inv(Y)=[-\sqrt{2}, -1]\cup [1, \sqrt{2}].$
				\end{answer*}

			\item Let $f:\RR\to\RR$ by $f(x)=1/(1+x^2).$ Find $f\inv\left( \left\{ \frac{1}{2} \right\} \right).$
				\begin{answer*}
					We have
					\[\frac{1}{2} = \frac{1}{1+1^2} = \frac{1}{1+(-1)^2}\]
					so $f\inv\left( \left\{ \frac{1}{2} \right\} \right) = \left\{ -1, 1 \right\}.$
				\end{answer*}

			\item Let $f:\RR\to\RR$ by $f(x)=1/(1+x^2).$ Find $f\inv\left( \left\{ -\frac{1}{2} \right\} \right).$
				\begin{answer*}
					Since $f$ is strictly positive over $\RR,$ there are no values of $x$ such that $f(x)=-1/2,$ so $f\inv\left( \left\{ -\frac{1}{2} \right\} \right)=\varnothing.$
				\end{answer*}
				
		\end{enumerate}

	\item[26.1] For each pair of functions $f$ and $g$ please do the following:
		\begin{itemize}
			\ii Determine which of $g\circ f$ and $f\circ g$ is defined.
			\ii If one or both are defined, find the resulting function(s).
			\ii If both are defined, determine whether $g\circ f=f\circ g.$
		\end{itemize}
		\begin{itemize}
			\item[(e)] $f=\left\{ (1, 2), (2,  3), (3, 4), (4, 5), (5, 1) \right\}$ and $g=\left\{ (1, 1), (2, 1), (3, 4), (4, 4) \right\}$
				\begin{soln}
					$g\circ f$ is not defined because
					\[\ima f = \left\{ 1, 2, 3, 4, 5 \right\}\nsubseteq\left\{ 1, 2, 3, 4 \right\} = \dom g\]
					On the other hand, $f\circ g$ is defined because $\ima g\subseteq \dom f.$ Then we have
					\[f\circ g=\left\{ (1, 2), (2, 2), (3, 5), (4, 5) \right\}\]
				\end{soln}

			\item[(g)] $f(x)=x+3$ and $g(x)=x-7$ (both for all $x\in\ZZ$)
				\begin{soln}
					We have $\ima f=\ima g=\dom f=\dom g=\ZZ$ so both $g\circ f$ and $f\circ g$ are defined. Then
					\begin{align*}
						(g\circ f)(x) &= g(x+3) = x-4 \\
						(f\circ g)(x) &= f(x-7) = x-4
					\end{align*}
					and thus $g\circ f = f\circ g.$
				\end{soln}

			\item[(i)] $f(x)=\frac{1}{x}$ for $x\in\QQ$ except $x=0$ and $g(x)=x+1$ for all $x\in\QQ$
				\begin{soln}
					$g\circ f$ is defined because
					\[\ima f = \QQ\setminus\left\{ 0 \right\} \subseteq \QQ = \dom g\]
					However, $f\circ g$ is not defined because
					\[\ima g = \QQ \nsubseteq \QQ\setminus\left\{ 0 \right\}\]
					Then we have
					\[(g\circ f)(x) = g\left( \frac{1}{x} \right) = \frac{1}{x} + 1\]
				\end{soln}
				
		\end{itemize}
		
\end{itemize}

\end{document}

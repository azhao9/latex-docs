\documentclass{article} 
\usepackage[sexy, hdr, fancy]{evan}
\usepackage{tikz}
\usetikzlibrary{arrows}
\setlength{\droptitle}{-4em}

\lhead{Homework 10}
\rhead{Discrete Math (Section 05)}
\lfoot{}
\cfoot{\thepage}

\newcommand{\var}{\mathrm{Var}}
\newcommand{\cov}{\mathrm{Cov}}

\begin{document}
\title{Homework 10}
\maketitle
\thispagestyle{fancy}

\begin{itemize}
	\item[50.7] Let $T$ be a tree with at least two vertices and let $v\in V(T).$ If $T-v$ is a tree, then $v$ is a leaf.
		\begin{proof}
			Suppose $d(v)\neq 1.$ Then it is at least 2, and since $T$ is a tree, every edge from $v$ is a cut edge. Suppose we have a path $u\to v\to w.$ Then since $T-v$ is a tree, it must be connected, but this is impossible since $(u, v)$ and $(v, w)$ are cut edges. Thus, $d(v)=1$ so $v$ is a leaf.
		\end{proof}

	\item[50.11] 
		\begin{enumerate}[(a)]
			\item First prove, using strong induction and the fact that every edge of a tree is a cut edge, that a tree with $n$ vertices has exactly $n-1$ edges.
				\begin{proof}
					We proceed by strong induction. In the case $n=1,$ there is obviously $1-1=0$ edges. Suppose a tree with $n$ vertices has exactly $n-1$ edges for $n=1, \cdots, k.$ Let $T$ be a tree with $n=k+1$ vertices, and let $v$ be a leaf of $T.$ Let $T'=T-v$ so $T'$ is a tree on $k$ vertices, with $k-1$ edges. Since $v$ was a leaf, it had degree 1, so $T$ must have had $(k-1)+1=(k+1)-1$ edges, so the claim is proved by induction.
				\end{proof}

			\item Use (a) to prove that the average degree of a vertex in a tree is less than 2.
				\begin{proof}
					Suppose a tree has $n$ vertices, and thus $n-1$ edges. Then 
					\begin{align*}
						\sum_{v\in V}^{}d(v) &= 2(\# E) = 2(n-1) \\
						\implies \frac{1}{n} \sum_{v\in V}^{} d(v) &= 2\cdot \frac{n-1}{n} < 2
					\end{align*}
					where the LHS is the average degree of a vertex.
				\end{proof}

			\item Use (b) to prove that every tree (with at least two vertices) has a leaf.
				\begin{proof}
					If every vertex had degree at least 2, then the average degree of a vertex would be at least 2. However, from (b), we know that the average degree is strictly less than 2, which is a contradiction. Thus, there must exist a vertex of degree 1, which is a leaf.
				\end{proof}

		\end{enumerate}

	\item[50.16] Let $G$ be a graph. A cycle of $G$ that contains all the vertices in $G$ is called a Hamiltonian cycle.
		\begin{enumerate}[(a)]
			\item Show that if $n\ge 5,$ then $\overline{C_n}$ has a Hamiltonian cycle.
				\begin{proof}
					Label the vertices in order as $v_1, \cdots, v_n.$ If $n$ is odd, then there is an obvious Hamiltonian cycle by incrementing vertex index by 2 each time modulo $n.$ Every vertex is exhausted because $\gcd(n, 2)=1.$

					If $n$ is even, then consider the cycle on $n-1$ vertices, as well as the Hamiltonian cycle of the complement obtained as above. Adjoin the vertex $v_n$ such that it lies between $v_1$ and $v_2$ in the cycle. Then replace the edge in the Hamiltonian cycle between $v_3$ and $v_5$ with the path $v_3\to v_n\to v_5.$ Then this is a Hamiltonian cycle, and we are on the complement of $C_n$ now. Thus, $\overline{C_n}$ has a Hamiltonian cycle for all $n\ge 5.$
				\end{proof}

				\newpage
			\item Prove that the graph in the figure does not have a Hamiltonian cycle.
				\begin{center}
					\begin{tikzpicture}[shorten >=1pt, auto, node distance=1cm, thick, main node/.style={circle, draw, font=\sffamily\Large\bfseries}]
						\node[main node] (1) {};
						\node[main node] (2) [right of=1] {};
						\node[main node] (3) [right of=2]{};
						\node[main node] (4) [right of=3]{};
						\node[main node] (5) [right of=4]{};
						\node[main node] (6) [below of=1]{};
						\node[main node] (7) [right of=6]{};
						\node[main node] (8) [right of=7]{};
						\node[main node] (9) [right of=8]{};
						\node[main node] (10) [right of=9] {};
						\node[main node] (11) [below of=6]{};
						\node[main node] (12) [right of=11] {};
						\node[main node] (13) [right of=12]{};
						\node[main node] (14) [right of=13]{};
						\node[main node] (15) [right of=14]{};
						\node[main node] (16) [below of=11]{};
						\node[main node] (17) [right of=16]{};
						\node[main node] (18) [right of=17]{};
						\node[main node] (19) [right of=18]{};
						\node[main node] (20) [right of=19]{};
						\node[main node] (21) [below of=16]{};
						\node[main node] (22) [right of=21]{};
						\node[main node] (23) [right of=22]{};
						\node[main node] (24) [right of=23]{};
						\node[main node] (25) [right of=24]{};

						\path[every node/.style={font=\sffamily\small}]
						(7) edge (2) edge (6) edge (8) edge (12)
						(9) edge (8) edge (10) edge (4) edge (14)
						(17) edge (12) edge (16) edge (18) edge (22)
						(19) edge (14) edge (18) edge (20) edge (24)
						(13) edge (8) edge (12) edge (14) edge (18)
						(3) edge (2) edge (4) edge (8)
						(11) edge (6) edge (12) edge (16)
						(15) edge (10) edge (14) edge (20)
						(23) edge (18) edge (22) edge (24)
						(1) edge (2) edge (6)
						(5) edge (4) edge (10)
						(21) edge (16) edge (22)
						(25) edge (20) edge (24);
					\end{tikzpicture}
				\end{center}
				\begin{proof}
					If there did exist a Hamiltonian cycle, then it would consist of 25 vertices and 25 edges. Since the graph itself has 28 edges, we would need to delete 3 edges to create a cycle, that is, every vertex would have degree 2. However, this is clearly impossible, so there is no Hamiltonian cycle.
				\end{proof}

		\end{enumerate}

	\item[50.18] Consider the following algorithm. 
		\begin{itemize}
				\ii Input: A connected graph $G.$
				\ii Output: A spanning tree of $G.$
		\end{itemize}
		\begin{enumerate}
				\ii Let $T$ be a copy of $G.$
				\ii Let $e_1, e_2, \cdots, e_m$ be the edges of $G.$
				\ii For $k=1, 2, \cdots, m,$ do: If edge $e_k$ is not a cut edge of $T,$ then delete $e_k$ from $T.$
				\ii Output $T.$
		\end{enumerate}
		Prove that this algorithm is correct. 
		\begin{proof}
			Since $G$ is connected, if we are deleting edges that are not cut edges, the result at the end of the algorithm is a connected graph. Then since we never deleted any vertices, the result is spanning of $G.$ At termination, all edges that are not cut edges are deleted, so everything remaining is a cut edge, and thus the result is a tree. Thus, $T$ is a spanning tree of $G,$ as desired.
		\end{proof}

	\item[52.4] Let $a, b$ be integers with $a, b\ge 3.$ The torus graph $T_{a, b}$ has vertex set $V=\Set{(x, y)}{0\le x<a, 0\le y<b}.$ Every vertex $(x, y)$ in $T_{a, b}$ has exactly four neighbors: $(x+1, y), (x-1, y), (x, y+1),$ and $(x, y-1)$ where arithmetic in the first position is modulo $a$ and arithmetic in the second position is modulo $b.$ Determine $\chi(T_{a, b}).$
		\begin{soln}
			Suppose $a$ represents the number of rows, and $b$ represents the number of columns in the graph, if we were to enumerate the vertices that way.

			If $a$ and $b$ are both even, then $\chi(T_{a, b})=2$ since there are no odd-length cycles. If $a$ is even and $b$ is odd, then consider the following construction. Begin with a coloring by alternating two colors.
			\begin{center}
				\begin{tikzpicture}[shorten >=1pt, auto, node distance=1cm, thick, main node/.style={circle, draw, font=\sffamily\Large\bfseries}]
					\node[main node] (1) {1};
					\node[main node] (2) [right of=1] {2};
					\node[main node] (3) [right of=2] {1};
					\node[main node] (4) [right of=3] {2};
					\node[main node] (5) [right of=4] {1};
					\node[main node] (6) [below of=1] {2};
					\node[main node] (7) [right of=6] {1};
					\node[main node] (8) [right of=7] {2};
					\node[main node] (9) [right of=8] {1};
					\node[main node] (10) [right of=9] {2};
					\node[main node] (11) [below of=6] {1};
					\node[main node] (12) [right of=11] {2};
					\node[main node] (13) [right of=12] {1};
					\node[main node] (14) [right of=13] {2};
					\node[main node] (15) [right of=14] {1};
					\node[main node] (16) [below of=11] {2};
					\node[main node] (17) [right of=16] {1};
					\node[main node] (18) [right of=17] {2};
					\node[main node] (19) [right of=18] {1};
					\node[main node] (20) [right of=19] {2};
				\end{tikzpicture}
			\end{center}
			Then since there are adjacencies between the first and last columns, we must correct this coloring by introducing a new color in the first and last columns that eliminates conflicts.
			\begin{center}
				\begin{tikzpicture}[shorten >=1pt, auto, node distance=1cm, thick, main node/.style={circle, draw, font=\sffamily\Large\bfseries}]
					\node[main node] (1) {1};
					\node[main node] (2) [right of=1] {2};
					\node[main node] (3) [right of=2] {1};
					\node[main node] (4) [right of=3] {2};
					\node[main node] (5) [right of=4] {3};
					\node[main node] (6) [below of=1] {3};
					\node[main node] (7) [right of=6] {1};
					\node[main node] (8) [right of=7] {2};
					\node[main node] (9) [right of=8] {1};
					\node[main node] (10) [right of=9] {2};
					\node[main node] (11) [below of=6] {1};
					\node[main node] (12) [right of=11] {2};
					\node[main node] (13) [right of=12] {1};
					\node[main node] (14) [right of=13] {2};
					\node[main node] (15) [right of=14] {3};
					\node[main node] (16) [below of=11] {3};
					\node[main node] (17) [right of=16] {1};
					\node[main node] (18) [right of=17] {2};
					\node[main node] (19) [right of=18] {1};
					\node[main node] (20) [right of=19] {2};
				\end{tikzpicture}
			\end{center}
			This alternating and replacing construction can be extended to any values $a$ and $b$ with $a$ even and $b$ odd, and similarly for when $a$ is odd and $b$ is even, so $\chi(T_{a, b})=3$ in these cases. 

			If both $a$ and $b$ are odd, then $\chi(T_{a, b})=3$ as well, by a similar construction.
		\end{soln}

	\item[52.8] Let $G$ be a graph with $n$ vertices. Prove that $\chi(G)\ge \omega(G)$ and $\chi(G)\ge n/\alpha(G).$
		\begin{proof}
			Since $\omega(G)$ is the size of the largest clique, these vertices are all adjacent to each other. Thus, to color this clique and by extension the entire graph, we need at least $\omega(G)$ colors.

			Suppose we needed fewer than $n/\alpha(G)$ colors to color $G.$ Then, if we consider all vertices of the same color as a set, one of these sets must have more than $\alpha(G)$ vertices, since
			\begin{align*}
				\frac{n}{\frac{n}{\alpha(G)}-\varepsilon} > \alpha(G)
			\end{align*}
			is the average number of vertices in a color set, which is greater than $\alpha(G).$ However, since all of the vertices in this set are the same color, none can be adjacent, and thus it is an independent set. However, $\alpha(G)$ was the size of the largest independent set in $G,$ which is a contradiction. Thus, we need at least $n/\alpha(G)$ colors.
		\end{proof}

	\item[52.13] Let $n$ be a positive integer. The $n$-cube is a graph, denoted $Q_n,$ whose vertices are the $2^n$ possible length-$n$ lists of 0s and 1s. Two vertices of $Q_n$ are adjacent if their lists differ in exactly one position. 
		\begin{enumerate}[(a)]
			\item Show that $Q_2 $ is a four-cycle.
				\begin{proof}
					$Q_2$ is shown below, and is clearly a four-cycle.
					\begin{center}
						\begin{tikzpicture}[shorten >=1pt, auto, node distance=3cm, thick, main node/.style={circle, draw, font=\sffamily\Large\bfseries}]
							\node[main node] (1) {00};
							\node[main node] (2) [right of=1] {01};
							\node[main node] (3) [below of=1] {10};
							\node[main node] (4) [right of=3] {11};

							\path
							(1) edge (2) edge (3)
							(2) edge (4)
							(3) edge (4);
						\end{tikzpicture}
					\end{center}
				\end{proof}

				\newpage
			\item Draw a picture of $Q_3$ and explain why this graph is called a cube.
				\begin{soln}
					$Q_3$ is shown below. It is called a cube because it resembles a cube.
					\begin{center}
						\begin{tikzpicture}[shorten >=1pt, auto, node distance=4cm, thick, main node/.style={circle, draw, font=\sffamily\Large\bfseries}]
							\node[main node] (1) {000};
							\node[main node] (2) [right of=1] {001};
							\node[main node] (3) [above right of=1] {010};
							\node[main node] (4) [below of=1] {100};
							\node[main node] (5) [right of=3] {011};
							\node[main node] (6) [below of=2] {101};
							\node[main node] (7) [below of=3] {110};
							\node[main node] (8) [below of=5] {111};

							\path
							(1) edge (2) edge (3) edge (4)
							(8) edge (5) edge (6) edge (7)
							(3) edge (5) edge (7)
							(6) edge (4) edge (2)
							(4) edge (7)
							(2) edge (5);
						\end{tikzpicture}
					\end{center}
				\end{soln}

			\item How many edges does $Q_n$ have?
				\begin{soln}
					Every vertex has $n$ bits, so it is connected to $n$ other vertices, since there are $n$ positions for a single bit change. Since there are $2^n$ vertices, the total number of edges is
					\begin{align*}
						\frac{1}{2} \sum_{v\in V}^{}d(v) = \frac{1}{2} \left( 2^n\cdot n \right) = n2^{n-1}
					\end{align*}
				\end{soln}

			\item Prove that $Q_n$ is bipartite.
				\begin{proof}
					Suppose there existed an odd-length cycle $v_1\to \cdots \to v_n\to v_1.$ Then along each adjacency in this cycle, there is a single bit change, so there is an odd number of bit changes. However, this is impossible because the number of bit changes must be even in order to return to $v_1.$ This is a contradiction, so there are no odd-length cycles, and thus $Q_n$ is bipartite.
				\end{proof}

			\item Prove that $K_{2, 3}$ is not a subgraph of $Q_n$ for any $n.$
				\begin{proof}
					Suppose $K_{2, 3}$ is a subgraph, and let $v_1, v_2$ and $u_1, u_2, u_3$ be the two sets of vertices in $K_{2, 3}.$ Since $u_1, u_2, u_3$ differ from $v_1$ in exactly one place each and they are distinct, $n\ge 3,$ and since we can ignore the bits that are not changed, WLOG $n=3.$ Then encode $v_1=a_1a_2a_3,$ and $u_1=b_1a_2a_3, u_2=a_1b_2a_3, u_3=a_1a_2b_3,$ where $a_i$'s represent the original values, and $b_i$'s represent a change in that position from the original $v_1$ (no matter what $v_1$ was originally). Then since $v_2$ differs from $b_1a_2a_3$ and $a_1b_2a_3$ in exactly one position each, it must be exactly $b_1b_2a_3$ since the only other option is $a_1a_2a_3$ but $v_2\neq v_1.$ However, then $v_2$ is not adjacent to $u_3$ since they differ in more than one position, and thus $K_{2, 3}$ is not a subgraph.
				\end{proof}

		\end{enumerate}

		\newpage
	\item[52.18] A proper $k$-edge coloring of a graph $G$ is a function $f:E(G)\to \left\{ 1, 2, \cdots, k \right\}$ with the property that if $e$ and $e'$ are distinct edges that have a common end point, then $f(e)\neq f(e').$ The edge chromatic number of $G,$ denoted $\chi'(G),$ is the least $k$ such that $G$ has a proper $k$-edge coloring.
		\begin{enumerate}[(a)]
			\item Show that the edge chromatic number of the graph in the figure is 4.
				\begin{center}
					\begin{tikzpicture}[shorten >=1pt, auto, node distance=1cm, thick, main node/.style={circle, draw, font=\sffamily\Large\bfseries}]
						\node[main node] (1) {};
						\node[main node] (2) [right of=1] {};
						\node[main node] (3) [right of=2]{};
						\node[main node] (4) [right of=3]{};
						\node[main node] (5) [right of=4]{};
						\node[main node] (6) [below of=1]{};
						\node[main node] (7) [right of=6]{};
						\node[main node] (8) [right of=7]{};
						\node[main node] (9) [right of=8]{};
						\node[main node] (10) [right of=9] {};
						\node[main node] (11) [below of=6]{};
						\node[main node] (12) [right of=11] {};
						\node[main node] (13) [right of=12]{};
						\node[main node] (14) [right of=13]{};
						\node[main node] (15) [right of=14]{};
						\node[main node] (16) [below of=11]{};
						\node[main node] (17) [right of=16]{};
						\node[main node] (18) [right of=17]{};
						\node[main node] (19) [right of=18]{};
						\node[main node] (20) [right of=19]{};
						\node[main node] (21) [below of=16]{};
						\node[main node] (22) [right of=21]{};
						\node[main node] (23) [right of=22]{};
						\node[main node] (24) [right of=23]{};
						\node[main node] (25) [right of=24]{};

						\path[every node/.style={font=\sffamily\small}]
						(7) edge (2) edge (6) edge (8) edge (12)
						(9) edge (8) edge (10) edge (4) edge (14)
						(17) edge (12) edge (16) edge (18) edge (22)
						(19) edge (14) edge (18) edge (20) edge (24)
						(13) edge (8) edge (12) edge (14) edge (18)
						(3) edge (2) edge (4) edge (8)
						(11) edge (6) edge (12) edge (16)
						(15) edge (10) edge (14) edge (20)
						(23) edge (18) edge (22) edge (24)
						(1) edge (2) edge (6)
						(5) edge (4) edge (10)
						(21) edge (16) edge (22)
						(25) edge (20) edge (24);
					\end{tikzpicture}
				\end{center}
				\begin{proof}
					The maximum degree of any vertex is 4. If we required fewer than 4 colors, then some edges coming out of a degree 4 vertex would have the same color, which is invalid. A 4-edge coloring is shown:
					\begin{center}
						\begin{tikzpicture}[shorten >=1pt, auto, node distance=1cm, thick, main node/.style={circle, draw, font=\sffamily\Large\bfseries}]
							\node[main node] (1) {};
							\node[main node] (2) [right of=1] {};
							\node[main node] (3) [right of=2]{};
							\node[main node] (4) [right of=3]{};
							\node[main node] (5) [right of=4]{};
							\node[main node] (6) [below of=1]{};
							\node[main node] (7) [right of=6]{};
							\node[main node] (8) [right of=7]{};
							\node[main node] (9) [right of=8]{};
							\node[main node] (10) [right of=9] {};
							\node[main node] (11) [below of=6]{};
							\node[main node] (12) [right of=11] {};
							\node[main node] (13) [right of=12]{};
							\node[main node] (14) [right of=13]{};
							\node[main node] (15) [right of=14]{};
							\node[main node] (16) [below of=11]{};
							\node[main node] (17) [right of=16]{};
							\node[main node] (18) [right of=17]{};
							\node[main node] (19) [right of=18]{};
							\node[main node] (20) [right of=19]{};
							\node[main node] (21) [below of=16]{};
							\node[main node] (22) [right of=21]{};
							\node[main node] (23) [right of=22]{};
							\node[main node] (24) [right of=23]{};
							\node[main node] (25) [right of=24]{};

							\path[every node/.style={font=\sffamily\small}]
							(7) edge node {1} (2) edge node [above] {2} (6) edge node {3} (8) edge node [left] {4} (12)
							(9) edge node [above] {1} (8) edge node {2} (10) edge node {3} (4) edge node [left] {4} (14)
							(17) edge node {1} (12) edge node [above] {2} (16) edge node {3} (18) edge node [left] {4} (22)
							(19) edge node {1} (14) edge node [above] {2} (18) edge node {3} (20) edge node [left] {4} (24)
							(13) edge node {1} (8) edge node [above] {2} (12) edge node {3}(14) edge node [left] {4} (18)
							(3) edge node [above] {3} (2) edge node {1} (4) edge node [left] {4}(8)
							(11) edge node {4} (6) edge node {3} (12) edge node [left] {1} (16)
							(15) edge node {4} (10) edge node [above] {2} (14) edge node [left] {1} (20)
							(23) edge node {1} (18) edge node [above] {3} (22) edge node {2} (24)
							(1) edge node {2} (2) edge node [left] {1} (6)
							(5) edge node [above] {2} (4) edge node [left] {3} (10)
							(21) edge node {4} (16) edge node {2} (22)
							(25) edge node {4} (20) edge node [above] {3} (24);
						\end{tikzpicture}
					\end{center}

				\end{proof}

			\item Prove that if $T$ is a tree, then $\chi'(T)=\Delta(T).$
				\begin{proof}
					Clearly we can't have fewer that $\Delta(T)$ colors, otherwise the vertex with degree $\Delta(T)$ would have two edges of the same color coming from it. We proceed by induction on the number of vertices $n.$ If $n=1,$ then $\chi'(T)=0=\Delta(T).$ 

					Suppose a tree $G$ with $k$ vertices has $\chi'(G)=\Delta(G).$ Then consider $T$ a tree with $k+1$ vertices, and let $v$ be a leaf of $T$ with parent $u.$ Now let $T'=T-v,$ so $T'$ is a tree with $k$ vertices. Thus, $T'$ can be colored with $\Delta(T')$ colors. Now, if $d_{T'}(u)=\Delta(T'),$ then $d_T(u)=1+\Delta(T') = \Delta(T),$ and we can take the coloring of $T'$ and simply color the edge $(u, v)$ with a new color to obtain a coloring of $T,$ which uses $\Delta(T)$ colors. 
					
					If $d_{T'}(u)< \Delta(T'),$ then $\Delta(T')=\Delta(T),$ and to color $T,$ we take the coloring of $T'$ and color the edge $(u, v)$ with one of the existing colors, since we won't violate the property on $u.$ In both cases, we use $\Delta(T)$ colors, so the claim is proven by induction.
				\end{proof}

			\item Give an example of a graph $G$ for which $\chi'(G)>\Delta(G).$
				\begin{soln}
					Consider the graph $K_3.$ Every vertex has degree 2, but $\chi'(K_3)=3.$
				\end{soln}

		\end{enumerate}

\end{itemize}

\end{document}

\documentclass{article}
\usepackage[sexy, hdr, fancy]{evan}
\usepackage{tikz}
\usetikzlibrary{arrows}
\setlength{\droptitle}{-4em}

\lhead{Homework 9}
\rhead{Discrete Math (Section 05)}
\lfoot{}
\cfoot{\thepage}

\newcommand{\var}{\mathrm{Var}}
\newcommand{\cov}{\mathrm{Cov}}

\begin{document}
\title{Homework 9}
\maketitle
\thispagestyle{fancy}

\begin{itemize}
	\item[1.] Create an intersection graph for each collection of sets.
		\begin{enumerate}[(a)]
			\item \begin{align*}
					A_1 &= \Set{x\in\RR}{x<0} \\
					A_2 &= \Set{x\in \RR}{-1<x<0} \\
					A_3 &= \Set{x\in\RR}{0<x<1} \\
					A_4 &= \Set{x\in\RR}{-1<x<1} \\
					A_5 &= \Set{x\in\RR}{x>-1} \\
					A_6 &= \RR
				\end{align*}
				\begin{soln}
					The non-empty intersections are
					\begin{align*}
						A_1\cap A_2 \quad\quad A_1\cap A_4 \quad\quad A_1\cap A_5 \quad\quad A_1\cap A_6 \\
						A_2\cap A_4 \quad\quad A_2\cap A_5 \quad\quad A_2\cap A_6 \\
						A_3\cap A_4 \quad\quad A_3\cap A_5 \quad\quad A_3\cap A_6 \\
						A_4\cap A_5 \quad\quad A_4\cap A_6 \\
						A_5\cap A_6
					\end{align*}
					The intersection graph is given by
					\begin{center}
						\begin{tikzpicture}[shorten >=1pt, auto, node distance=3cm, thick, main node/.style={circle, draw, font=\sffamily\Large\bfseries}]
							\node[main node] (1) {$A_1$};
							\node[main node] (2) [right of=1] {$A_2$};
							\node[main node] (3) [right of=2] {$A_3$};
							\node[main node] (4) [below of=1] {$A_4$};
							\node[main node] (5) [right of=4] {$A_5$};
							\node[main node] (6) [right of=5] {$A_6$};

							\path[every node/.style={font=\sffamily\small}]
							(2) edge (1)
							(4) edge (1)
							edge (2)
							edge (3)
							(5) edge (1)
							edge (2) 
							edge (3)
							edge (4)
							(6) edge (1)
							edge (2)
							edge (3)
							edge [bend left] (4)
							edge (5);
						\end{tikzpicture}
					\end{center}
				\end{soln}

				\newpage
			\item \begin{align*}
					A_1 &= \Set{x\in\ZZ}{-4\le x\le 0} \\
					A_2 &= \Set{x\in\ZZ}{\exists n\in \NN\text{ such that } x=2^n\text{ or } x=-(2)^n} \\
					A_3 &= \Set{x\in\ZZ}{2\mid x} \\
					A_4 &= \Set{x\in\ZZ}{x\text{ is odd}} \\
					A_5 &= \Set{x\in\ZZ}{3\mid x}
				\end{align*}
				\begin{soln}
					We have
					\begin{align*}
						A_1 &= \left\{ -4, -3, -2, -1, 0 \right\} \\
						A_2 &= \left\{ \cdots, -8, -4, -2, -1, 1, 2, 4, 8, \cdots \right\} \\
						A_3 &= \left\{ \cdots, -8, -6, -4, -2, 0, 2, 4, 6, 8, \cdots \right\} \\
						A_4 &= \left\{ \cdots, -7, -5, -3, -1, 1, 3, 5, 7, \cdots \right\} \\
						A_5 &= \left\{ \cdots, -6, -3, 0, 3, 6, \cdots \right\}
					\end{align*}
					and the non-empty intersections are
					\begin{align*}
						A_1\cap A_2 \quad\quad A_1\cap A_3 \quad\quad A_1\cap A_4 \quad\quad A_1\cap A_5 \\
						A_2\cap A_3 \quad\quad A_2\cap A_4 \\
						A_3\cap A_5 \\
						A_4\cap A_5
					\end{align*}
					The intersection graph is given by
					\begin{center}
						\begin{tikzpicture}[shorten >=1pt, auto, node distance=3cm, thick, main node/.style={circle, draw, font=\sffamily\Large\bfseries}]
							\node[main node] (1) {$A_1$};
							\node[main node] (2) [right of=1] {$A_2$};
							\node[main node] (3) [right of=2] {$A_3$};
							\node[main node] (4) [below of=1] {$A_4$};
							\node[main node] (5) [right of=4] {$A_5$};

							\path[every node/.style={font=\sffamily\small}]
							(2) edge (1)
							(3) edge [bend right](1)
							edge (2)
							(4) edge (1)
							edge (2)
							(5) edge (1)
							edge (3)
							edge (4);
						\end{tikzpicture}
					\end{center}
				\end{soln}

		\end{enumerate}
	\item[47.16] Prove that in any graph with two or more vertices, there must be two vertices of the same degree.
		\begin{proof}
			If the graph $G$ has two or more 0-degree vertices, then we are done. Otherwise, consider the graph $G'$ which is obtained by removing all 0-degree vertices from $G.$ If $G'$ has $n$ vertices, then the possible values of degrees of vertices is $\left\{ 1, 2, \cdots, n-1 \right\},$ but since there are $n$ vertices, by the Pigeonhole principle there must be two vertices of the same degree.
		\end{proof}

	\item[47.21] Let $G$ and $H$ be graphs. We say that $G$ is isomorphic to $H$ provided there is a bijection $f:V(G)\to V(H)$ such that for all $a, b\in V(G),$ we have $a\sim b$ in $G$ if and only if $f(a)\sim f(b)$ in $H.$ 
		\begin{enumerate}[(a)]
			\item Prove that isomorphic graphs have the same number of vertices.
				\begin{proof}
					Since $f$ is a bijection between finite sets $V(G)$ and $V(H),$ the cardinalities of the two sets must be equal, so $G$ and $H$ must have the same number of vertices.
				\end{proof}

			\item Prove that if $f:V(G)\to V(H)$ is an isomorphism of graphs $G$ and $H$ and if $v\in V(G),$ then he degree of $v$ in $G$ equals the degree of $f(v)$ in $H.$
				\begin{proof}
					Suppose $d_G(v)=n,$ and the neighbors of $v$ are $\left\{ u_1, \cdots, u_n \right\}\in V(G).$ Then since $u_i\sim v, \forall i$ in $G,$ it follows that $f(u_i)\sim f(v), \forall i$ in $H$ since $f$ is an isomorphism. Furthermore, this condition is bijective, so the neighbors of $f(v)$ are exactly $\left\{ f(u_1), \cdots, f(u_n) \right\}$ so $d_H(f(v))=n.$
				\end{proof}

			\item Prove that isomorphic graphs have the same number of edges.
				\begin{proof}
					If an edge $(a, b)$ is in $E(G),$ then $a\sim b\iff f(a)\sim f(b)$ so the edge $(f(a), f(b))$ is in $E(H),$ and conversely as well. Thus isomorphic graphs have the same number of edges.
				\end{proof}

			\item Give an example of two graphs that have the same number of vertices and the same number of edges but that are not isomorphic.
				\begin{soln}
					Here are two 4-vertex, 2-edge graphs that are not isomorphic.
					\begin{center}
						\begin{tikzpicture}[shorten >=1pt, auto, node distance=3cm, thick, main node/.style={circle, draw, font=\sffamily\Large\bfseries}]
							\node[main node] (1) {1};
							\node[main node] (2) [right of=1] {2};
							\node[main node] (5) [right of=2] {1};
							\node[main node] (6) [right of=5] {2};
							\node[main node] (3) [below of=1] {3};
							\node[main node] (4) [right of=3] {4};
							\node[main node] (7) [right of=4] {3};
							\node[main node] (8) [right of=7] {4};

							\path[every node/.style={font=\sffamily\small}]
							(1) edge (2)
							edge (3)
							(5) edge (7)
							(6) edge (8);
						\end{tikzpicture}
					\end{center}
				\end{soln}

			\item Let $G$ be the graph whose vertex set is $\left\{ 1, 2, 3, 4, 5, 6 \right\}.$ In this graph, there is an edge from $v$ to $w$ if and only if $v-w$ is odd. Let $H$ be the graph in the figure. Find an isomorphism $f:V(G)\to V(H).$
				\begin{center}
					\begin{tikzpicture}[shorten >=1pt, auto, node distance=3cm, thick, main node/.style={circle, draw, font=\sffamily\Large\bfseries}]
						\node[main node] (1) {a};
						\node[main node] (2) [right of=1] {b};
						\node[main node] (3) [right of=2] {c};
						\node[main node] (4) [below of=1] {d};
						\node[main node] (5) [right of=4] {e};
						\node[main node] (6) [right of=5] {f};

						\path[every node/.style={font=\sffamily\small}]
						(1) edge (4)
						edge (5)
						edge (6)
						(2) edge (4)
						edge (5)
						edge (6)
						(3) edge (4)
						edge (5)
						edge (6)
						(4) edge (1)
						edge (2)
						edge (3)
						(5) edge (1)
						edge (2)
						edge (3)
						(6) edge (1)
						edge (2)
						edge (3);
					\end{tikzpicture}
				\end{center}
				\begin{soln}
					$G$ is given by
					\begin{center}
						\begin{tikzpicture}[shorten >=1pt, auto, node distance=3cm, thick, main node/.style={circle, draw, font=\sffamily\Large\bfseries}]
							\node[main node] (1) {1};
							\node[main node] (2) [right of=1] {3};
							\node[main node] (3) [right of=2] {5};
							\node[main node] (4) [below of=1] {2};
							\node[main node] (5) [right of=4] {4};
							\node[main node] (6) [right of=5] {6};

							\path[every node/.style={font=\sffamily\small}]
							(1) edge (4)
							edge (5)
							edge (6)
							(2) edge (4)
							edge (5)
							edge (6)
							(3) edge (4)
							edge (5)
							edge (6)
							(4) edge (1)
							edge (2)
							edge (3)
							(5) edge (1)
							edge (2)
							edge (3)
							(6) edge (1)
							edge (2)
							edge (3);
						\end{tikzpicture}
					\end{center}
					So an isomorphism $f$ is
					\begin{align*}
						f:\begin{cases}
							1 &\mapsto a \\
							2 &\mapsto d \\
							3 &\mapsto b \\
							4 &\mapsto e \\
							5 &\mapsto c \\
							6 &\mapsto f
						\end{cases}
					\end{align*}
				\end{soln}
		\end{enumerate}

	\item[48.11] Recall the definition of graph isomorphism from Exercise 47.21. We call a graph $G$ self-complementary if $G$ is isomorphic to $\overline{G}.$ 
		\begin{enumerate}[(a)]
			\item Show that the graph $G=(\left\{ a, b, c, d \right\}, \left\{ ab, bc, cd \right\})$ is self-complementary.
				\begin{proof}
					We have 
					\begin{align*}
						V(\overline{G})&=\left\{ a, b, c, d \right\} \\
						E(\overline{G}) &= \left\{ ac, ad, bd \right\}
					\end{align*}
					and we have an isomorphism
					\begin{align*}
						f:\begin{cases}
							a &\mapsto c \\
							b &\mapsto a \\
							c &\mapsto d \\
							d &\mapsto b
						\end{cases}
					\end{align*}
					This is an isomorphism because
					\begin{align*}
						a \sim_G b &\iff f(a)=c\sim_H a = f(b) \\
						b\sim_G c &\iff f(b)=a\sim_H d=f(c) \\
						c\sim_G d &\iff f(c)=d\sim_H b=f(d)
					\end{align*}
				\end{proof}

			\item Find a self-complementary graph with five vertices.
				\begin{soln}
					If $G=(\left\{ a, b, c, d, e \right\}, \left\{ ab, bc, cd, de, ae \right\}),$ then $G$ is self-complementary because $\overline{G}=(\left\{ a, b, c, d, e \right\}, \left\{ ac, ad, bd, be, ce \right\})$ and we have an isomorphism (easy to check\ldots)
					\begin{align*}
						f:\begin{cases}
							a &\mapsto a \\
							b &\mapsto c \\
							c &\mapsto e \\
							d &\mapsto b \\
							e &\mapsto d
						\end{cases}
					\end{align*}
				\end{soln}

				\newpage
			\item Prove that if a self-complementary graph has $n$ vertices, then $n\equiv0\pmod 4$ or $n\equiv1\pmod 4.$
				\begin{proof}
					There are a total of $\binom{n}{2}=\frac{n(n-1)}{2}$ possible edges. If a graph $G$ is self-complementary, then its complement $\overline{G}$ must have the same number of edges, and combined these exhaust all $\binom{n}{2}$ edges. Thus $\frac{n(n-1)}{2}$ must be even.

					Suppose $n\equiv 2\pmod 4.$ Then $n=4k+2$ for some $k\in\NN,$ and
					\begin{align*}
						\frac{n(n-1)}{2} &= \frac{(4k+2)(4k+1)}{2} = (2k+1)(4k+1)
					\end{align*}
					which is odd, which can't work. Now if $n\equiv 3\pmod 4,$ then $n=4k+3$ for some $k\in\NN$ and
					\begin{align*}
						\frac{n(n-1)}{2} &= \frac{(4k+3)(4k+2)}{2} = (4k+3)(2k+1)
					\end{align*}
					which is odd, which can't work. Thus, we must have either $n\equiv 0\pmod 4$ or $n\equiv 1\pmod 4,$ and as demonstrated from parts (a) and (b), graphs that satisfy these conditions ($n=4$ and $n=5$) exist.
				\end{proof}

		\end{enumerate}

	\item[49.4] Let $n\ge 2$ be an integer. Form a graph $G_n,$ whose vertices are all the two-element subsets of $\left\{ 1, 2, \cdots, n \right\}.$ In this graph we have an edge between distinct vertices $\left\{ a, b \right\}$ and $\left\{ c, d \right\}$ exactly when $\left\{ a, b \right\}\cap \left\{ c, d \right\}=\varnothing.$
		\begin{enumerate}[(a)]
			\item How many vertices does $G_n$ have?
				\begin{soln}
					The number of vertices is the number of 2-element subsets of $\left\{ 1, 2, \cdots, n \right\},$ or $\binom{n}{2}.$
				\end{soln}

			\item How many edges does $G_n$ have?
				\begin{soln}
					For a vertex $\left\{ a, b \right\},$ the number of disjoint 2-element subsets of $\left\{ 1, 2, \cdots, n \right\}$ is just the number of 2-element subsets of $\left\{ 1, 2, \cdots, n \right\}\smallsetminus \left\{ a, b \right\},$ which is $\binom{n-2}{2}.$ There are $\binom{n}{2}$ total vertices, but this double counts every edge, so the total number of edges is $\frac{1}{2} \binom{n}{2} \binom{n-2}{2}.$
				\end{soln}

			\item For which values of $n\ge 2$ is $G_n$ connected? Prove your answer.
				\begin{soln}
					Assume different variables represent different values. Starting at $\left\{ a, b \right\}$ we can travel to any $\left\{ c, d \right\}$ because there is an edge between them. Suppose we wish to travel to $\left\{ a, c \right\}$ from $\left\{ a, b \right\}.$ There is no edge because these are not disjoint. Thus we must find an intermediate edge disjoint from both of these: $\left\{ d, e \right\},$ and then there is a path $\left\{ a, b \right\}\to \left\{ d, e \right\}\to \left\{ a, c \right\}.$ Thus, there must be distinct values $a, b, c, d, e,$ so $n$ must be at least 5.
				\end{soln}

		\end{enumerate}

	\item[6.] Let $A$ be the set of all simple undirected graphs. Let $R$ be the relation on $A$ defined as follows: $(G, H)\in R$ if there is an isomorphism $f:V(G)\to V(H).$ Prove $R$ is an equivalence relation.
		\begin{proof}
			Reflexive: Clearly $(G, G)\in R$ because there is the identity isomorphism $f=\id_G.$

			Symmetric: If $(G, H)\in R,$ then there exists an isomorphism $f:V(G)\to V(H).$ Then its inverse $f\inv: V(H)\to V(G)$ is also an isomorphism, so $(H, G)\in R.$

			Transitive: If $(G, H)\in R$ and $(H, K)\in R,$ then there exists isomorphisms $f:V(G)\to V(H)$ and $g:V(H)\to V(K).$ Then the composition $g\circ f: V(G)\to V(K)$ is also an isomorphism, so $(G, K)\in R$ as well.
		\end{proof}

	\item[7.] Let $k, n\in\NN$ with $n>0.$ Derive a formula using $n$ and $k$ the number of subgraphs of $K_n$ with exactly $k$ vertices.
		\begin{soln}
			There are $\binom{n}{k}$ ways to choose $k$ vertices to be in the sub-graph. Of all of the possible $\binom{k}{2}$ edges between these vertices (all of which are also in $K_n$), we choose a subset of these to be in the sub-graph, in $2^{\binom{k}{2}}$ ways. Thus the number of subgraphs of $K_n$ with exactly $k$ vertices is $\binom{n}{k} 2^{\binom{k}{2}}.$
		\end{soln}

	\item[8.] 
		\begin{enumerate}[(a)]
			\item Give one advantage and one disadvantage each for the star, ring, and hybrid LAN topologies.
				\begin{answer*}
					Star:
					\begin{itemize}
						\ii Every node is connected to every other node.
						\ii High dependence on central node.
					\end{itemize}
					Ring:
					\begin{itemize}
						\ii Every node is connected to every other node.
						\ii Connections are inefficient because most connections must pass through other nodes.
					\end{itemize}
					Hybrid:
					\begin{itemize}
						\ii Every node is connected to every other node.
						\ii It is a complicated system.
					\end{itemize}
				\end{answer*}

			\item In the basic mesh network, the number of devices, $n,$ is a perfect square, so $n=m^2$ where $m\in\NN.$ The devices are labeled $D(i, j)$ where $0\le i, j\le m-1.$ There is an edge joining $D(i, j)$ to any of the vertices $D(i\pm 1, j)$ and $D(i, j\pm 1)$ that exist. Draw the mesh network for $n=16.$
				\begin{soln}
					Here, $m=4.$ The mesh network is
					\begin{center}
						\begin{tikzpicture}[shorten >=1pt, auto, node distance=3cm, thick, main node/.style={circle, draw, font=\sffamily\Large\bfseries}]
							\node[main node] (1) {$D(0, 0)$};
							\node[main node] (2) [right of=1] {$D(0, 1)$};
							\node[main node] (3) [right of=2] {$D(0, 2)$};
							\node[main node] (4) [right of=3] {$D(0, 3)$};
							\node[main node] (5) [below of=1] {$D(1, 0)$};
							\node[main node] (6) [right of=5] {$D(1, 1)$};
							\node[main node] (7) [right of=6] {$D(1, 2)$};
							\node[main node] (8) [right of=7] {$D(1, 3)$};
							\node[main node] (9) [below of=5] {$D(2, 0)$};
							\node[main node] (10) [right of=9] {$D(2, 1)$};
							\node[main node] (11) [right of=10] {$D(2, 2)$};
							\node[main node] (12) [right of=11] {$D(2, 3)$};
							\node[main node] (13) [below of=9] {$D(3, 0)$};
							\node[main node] (14) [right of=13] {$D(3, 1)$};
							\node[main node] (15) [right of=14] {$D(3, 2)$};
							\node[main node] (16) [right of=15] {$D(3, 3)$};
	
							\path[every node/.style={font=\sffamily\small}]
							(1) edge (2)
							edge (5)
							edge (6)
							(2) edge (3)
							edge (5)
							edge (6) 
							edge (7)
							(3) edge (4)
							edge (6)
							edge (7)
							edge (8)
							(4) edge (7)
							edge (8)
							(5) edge (6)
							edge (9)
							edge (10)
							(6) edge (7)
							edge (9)
							edge (10) 
							edge (11)
							(7) edge (8)
							edge (10)
							edge (11)
							edge (12)
							(8) edge (11)
							edge (12)
							(9) edge (10)
							edge (13)
							edge (14)
							(10) edge (11)
							edge (13)
							edge (14) 
							edge (15)
							(11) edge (12)
							edge (14)
							edge (15)
							edge (16)
							(12) edge (15)
							edge (16)
							(13) edge (14)
							(14) edge (15)
							(15) edge (16);
						\end{tikzpicture}
					\end{center}
				\end{soln}

				\newpage
			\item In a variant of the mesh network, there is an edge joining $D(i, j)$ to any of the vertices $D( (i\pm 1) \mod m, j)$ and $D(i, (j\pm 1)\mod m).$ Draw this variant of the mesh network for $n=16.$
				\begin{soln}
					Here, $m=4.$ This variant mesh network is \begin{center}
						\begin{tikzpicture}[shorten >=1pt, auto, node distance=3cm, thick, main node/.style={circle, draw, font=\sffamily\Large\bfseries}]
							\node[main node] (1) {$D(0, 0)$};
							\node[main node] (2) [right of=1] {$D(0, 1)$};
							\node[main node] (3) [right of=2] {$D(0, 2)$};
							\node[main node] (4) [right of=3] {$D(0, 3)$};
							\node[main node] (5) [below of=1] {$D(1, 0)$};
							\node[main node] (6) [right of=5] {$D(1, 1)$};
							\node[main node] (7) [right of=6] {$D(1, 2)$};
							\node[main node] (8) [right of=7] {$D(1, 3)$};
							\node[main node] (9) [below of=5] {$D(2, 0)$};
							\node[main node] (10) [right of=9] {$D(2, 1)$};
							\node[main node] (11) [right of=10] {$D(2, 2)$};
							\node[main node] (12) [right of=11] {$D(2, 3)$};
							\node[main node] (13) [below of=9] {$D(3, 0)$};
							\node[main node] (14) [right of=13] {$D(3, 1)$};
							\node[main node] (15) [right of=14] {$D(3, 2)$};
							\node[main node] (16) [right of=15] {$D(3, 3)$};
	
							\path[every node/.style={font=\sffamily\small}]
							(1) edge (2)
							edge (5)
							edge (6)
							edge [bend left] (4)
							edge [bend right] (13)
							(2) edge (3)
							edge (5)
							edge (6) 
							edge (7)
							edge [bend left] (14)
							(3) edge (4)
							edge (6)
							edge (7)
							edge (8)
							edge [bend left] (15)
							(4) edge (7)
							edge (8)
							edge [bend left] (16)
							(5) edge (6)
							edge (9)
							edge (10)
							edge [bend right] (8)
							(6) edge (7)
							edge (9)
							edge (10) 
							edge (11)
							(7) edge (8)
							edge (10)
							edge (11)
							edge (12)
							(8) edge (11)
							edge (12)
							(9) edge (10)
							edge (13)
							edge (14)
							edge [bend right] (12)
							(10) edge (11)
							edge (13)
							edge (14) 
							edge (15)
							(11) edge (12)
							edge (14)
							edge (15)
							edge (16)
							(12) edge (15)
							edge (16)
							(13) edge (14)
							edge [bend right] (16)
							(14) edge (15)
							(15) edge (16);
						\end{tikzpicture}
					\end{center}
				\end{soln}

			\item Are there any advantages of a mesh network (either type) over the topologies discussed in part (a)? Explain.
				\begin{answer*}
					In the variant mesh network, the maximum distance between any two nodes is 2 in the case $n=16,$ which is way better than the ring and star topologies for $n=16,$ and also better than the hybrid topology for $n=16,$ which is also worse than the regular mesh network.
				\end{answer*}

		\end{enumerate}

\end{itemize}

\end{document}

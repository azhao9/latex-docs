\documentclass{article}
\usepackage[sexy, hdr, fancy]{evan}

\setlength{\droptitle}{-4em}

\lhead{\leftmark}
\rhead{Introduction to Optimization}
\lfoot{}
\cfoot{\thepage}

\begin{document}
\title{Introduction to Optimization Lecture Notes}
\maketitle
	This is EN.550.361 Intro to Optimization, taught by Donniell Fishkind.
\thispagestyle{fancy}

\tableofcontents

\newpage

\section{September 2, 2016}
\subsection{What is Optimization?}

\begin{definition}
	When presented with choices, optimization is choosing the best, where there is some measure of ``goodness'' and ``badness'' to each choice.
\end{definition}

\subsection{Example: Growing Plants}
\begin{example}
	Up to 6 units of two nutrients can be added to soil. We require the number of units of 2nd nutrient to be at least the log of the number of units of 1st nutrient.

Select \begin{align*}
		x_1 &= \text{\# units of 1st nutrient} \\
		x_2 &= \text{\# units of 2nd nutrient}
	\end{align*} to maximize the expected height of the plant 
	\[ H(x_1, x_2) = 1 + x_1^2(x_2-1)^3e^{-x_1-x_2}. \]
\end{example}

We frame the problem as 
\begin{align*}
	\text{(P) maximize (or minimize) } \quad H(x_1, x_2) &= 1 + x_1^2(x_2-1)^3e^{-x_1-x_2} \\
	\text{s.t. (subject to, such that) } \quad x_1 + x_2 &\le 6 \\
	x_2 &\ge \ln{x_1} \\
	x_1 &\ge 0 \\
	x_2 &\ge 0
\end{align*}

Any pairs $(x_1, x_2)$ that satisfy the constraints are called \vocab{feasible.} We would like to visualize the set of all feasible points, so we turn to graphing on the Euclidean plane. 

\begin{center}
	\begin{asy}
		import graph;
		real f(real x) {
			return log(x);
		}

		real g(real x) {
			return 6 - x;
		}

		path p1 = graph(f, 0.5, 6);
		path p2 = graph(g, -1, 7);
		real [][] i1 = intersections(p1, p2);
		pair i2 = intersectionpoint(p1, p2);

		path xaxis = (-1, 0)--(7, 0);
		path yaxis = (0, -1)--(0, 7);
		Label xlabel = Label('$x_1$', position=EndPoint);
		Label ylabel = Label('$x_2$', position=EndPoint);

		path toFill = (0, 0) -- (1, 0) -- subpath(p1, 1, i1[0][0]) -- (0, 6) -- cycle;
		fill(toFill, mediumgray);

		draw(p1);
		draw(p2);
		draw(xaxis, arrow=Arrow, L=xlabel);
		draw(yaxis, arrow=Arrow, L=ylabel);
		dotfactor = dotfactor * 2;
		dot( (1, 0) );
		dot( (0, 0) );
		dot( i2 );
		dot( (0, 6) );
		dot( (6, 0) );
		dot( (2, 4), L='(2, 4)', NE, red);
	
	\end{asy}
\end{center}

The shaded area above simultaneously satisfies all of the constraints, and is called the \vocab{feasible region}. Any feasible solution must lie within this region or on its boundary. From here, we may imagine the objective function as a surface above the $x_1x_2-$plane, whose height at each point $\vec{x}$ is $H(\vec{x}).$  

\subsection{The General Optimization Problem}

We now define a \vocab{general optimization problem} (the difference between minimization and maximization problems is rather superficial, so we will usually be working with minimization):

\begin{definition}[General Optimization Problem]\label{Definition 1.3}
	Given a feasible region $S\subseteq \RR^n$ and an objective function $f: S\to \RR, $ the problem is stated as 
	\[ \text{(P) min } f(\vec{x}) \text{ s.t. } \vec{x}\in S. \] An \vocab{optimal solution} $\vec{x}^*$ to (P) satisfies the following conditions:
	\begin{enumerate}
			\ii $\vec{x}^* \in S \quad(\vec{x}^*$ is feasible.)
			\ii $f(\vec{x}^*) \le f(y) \quad\forall y\in S \quad (\vec{x}^*$ is minimal).
	\end{enumerate}
\end{definition} 

\begin{remark}[Warning]
	The \textbf{solution} to the problem is the selection $\vec{x}^*$ that minimizes $f,$ not the value of $f(\vec{x}^*)$ itself.
\end{remark}

\begin{example}
	The optimal solution to our plant problem is $\vec{x}^*=\begin{bmatrix}
		2 & 4
	\end{bmatrix},$ thus the maximum possible height of the plant is $H(\vec{x}^*)\approx1.2677.$
\end{example}


\newpage

\section{September 7, 2016}
\subsection{Review}
In optimization, a vector is a \textbf{decision} that summarizes multiple choices. The \textbf{goal} is to find the vector with the lowest objective function value.

\subsection{Topology Review}
\begin{definition}
	For any $\vec{x}\in\RR^n,$ define the \vocab{Euclidean length} or \vocab{Euclidean norm} of $\vec{x}=[x_1, x_2, \cdots, x_n]$ to be $\left\Vert \cdot \right\Vert : \RR^n\to\RR$ that maps $\displaystyle\vec{x}\mapsto \left( \sum_{i=1}^n x_i^2 \right)^{1/n}.$
\end{definition}

\begin{definition}
	For any $\vec{x}, \vec{y}\in\RR^n,$ define the \vocab{Euclidean distance} between $\vec{x}$ and $\vec{y}$ to be $\Vert \vec{x}-\vec{y}\Vert.$
\end{definition}

We make use of these ideas of length and distance to one of the most important constructions in analysis:

\begin{definition}
	For any $\vec{x}\in\RR^n$ and $\varepsilon>0,$ the \vocab{epsilon-neighborhood} of $\vec{x}$ is defined as \[N_{\varepsilon}(\vec{x}) := \left\{\vec{y}\in\RR^n \big\vert \Vert \vec{x}-\vec{y}\Vert < \varepsilon\right\}. \]
\end{definition}

The $\varepsilon-$neighborhood is also sometimes referred to as a ``ball,'' and this term extends to not just 3 dimensions.

\begin{example}[Examples of Neighborhoods]
	Some examples of neighborhoods\ldots
	\begin{itemize}
		\ii $N_3(7)$
		\begin{center}
			\begin{asy}
				import graph;
				path x = (-4, 0) -- (4, 0);

				Label rlabel = Label( '$\mathbb{R}$', position=EndPoint);
				draw(x, arrow=Arrows, L=rlabel, NE);
				Label A = Label('(', position=(-1, 0) );
				Label B = Label('\vert', position=MidPoint );
				Label B = Label(')', position=(1, 0) );
				label(A, x);
				label(B, x);
				label(C, x);
				
			\end{asy}
		\end{center}<++>

		\ii $N_{\sqrt{2}}\begin{bmatrix}
			1 \\ -2
		\end{bmatrix}$

		\ii $N_1\begin{bmatrix}
			1 \\ 1 \\ 1
		\end{bmatrix}$
	\end{itemize}<++>
\end{example}<++>

\begin{definition}
	For any set $S\subseteq \RR^n,$ then $\vec{x}\in S$ is called an \vocab{interior point} of $S$ if there exists a neighborhood of $\vec{x}$ contained in $S.$
\end{definition}

\begin{definition}
	$\vec{x}$ is a \vocab{boundary point} if every neighborhood of $\vec{x}$ contains a point in $S$ and a point not in $S.$
\end{definition}

Given a topological space $X$ with $S\subseteq X,$ all points $x\in X$ fall into one of three categories:
\begin{enumerate}
	\ii interior point of $S$
	\ii interior point of $S^c$ (complement of $S$ in $X$)
	\ii boundary point of $S$
\end{enumerate}

\begin{definition}
	$S$ is \vocab{open} if all $\vec{x}\in S$ are interior points of $S.$
\end{definition}

\begin{definition}
	$S$ is \vocab{closed} if $S$ contains all of its boundary points.
\end{definition}

\begin{fact}
	$S$ is open $\iff S^{c}$ is closed. In fact, this is actually a commonly given definition for closed sets - a set is closed if and only if its complement is open.
\end{fact}

\begin{remark}[Warning]
	A set is only open or closed relative to its parent space. So it doesn't make sense to say a standalone set is ``open'' or ``closed''. 
\end{remark}

\begin{example}
	To give some stranger examples and clarify\ldots
	\begin{enumerate}
		\ii The set (0, 1) is open in $\RR,$ but not open in $\RR^2$ (to see why, draw the segment on the $\RR^2$ plane and try to find a neighborhood of any point that is fully contained within the segment).

		\ii The set [0, 1] is closed in $\RR.$ Its boundary points are 0 and 1, which are both contained in itself.

		\ii The set [0, 1] is open in the set [0, 1]. To understand this, any neighborhood of a point in [0, 1] (even the endpoints) is fully contained in [0, 1], since the neighborhood cannot extend beyond the entire space. 

		\ii Some sets are \textbf{both open and closed.} Such sets are called \vocab{clopen} sets. For example the null set $\{\}$ and $\RR^n$ are both clopen in $\RR^n.$

		\ii Other clopen sets include (0, 1) and [2, 3] in the space $(0, 1)\cup[2, 3].$ 

		\ii On the other hand, some sets are \textbf{neither open nor closed}. For example (0, 1] in $\RR$ is neither open nor closed.
	\end{enumerate}
\end{example}

Great, now we defined all these terms, what are they good for? It turns out that open-ness and closed-ness of sets is a really nice property to have, depending on the context of the problem. For example, 
\begin{itemize}
	\ii Open sets are nice for calculus - you will never ``run out of space'' near a boundary
	\ii Closed sets are nice for optimization - you are \textbf{guaranteed} a solution exists in the set
\end{itemize}

\subsection{Extended Definition}
With the new machinery we've just acquired, we can extend Definition \ref{Definition 1.3} from last time and introduce a few more terms! (as if we haven't had enough yet)

\begin{definition}[Characterizing Solutions]
	Suppose $\vec{x}^*$ is feasible. 
	\begin{enumerate}
			\ii $\vec{x}^*$ is a \vocab{global minimizer} if $\forall y\in S, f(\vec{x}^*)\le f(y).$ 
			\ii $\vec{x}^*$ is a \vocab{strict global minimizer} if $\forall y\in S\sim\vec{x}^*, f(\vec{x}^*)<f(y).$ 
			
			\ii $\vec{x}^*$ is a \vocab{local minimizer} if $\exists \varepsilon>0$ such that $\forall y\in S\cap N_{\varepsilon}(\vec{x}^*), f(\vec{x}^*)\le f(y).$ That is, $f(\vec{x}^*)$ is the smallest possible value on some neighborhood around $\vec{x}^*.$ 

			\ii $\vec{x}^*$ is a \vocab{strict local minimizer} if $\exists \varepsilon>0$ such that $\forall y\in S\cap N_{\varepsilon}(\vec{x}^*)\sim\vec{x}^*, f(\vec{x}^*)< f(y)$
	\end{enumerate}
\end{definition}

\subsection{More Examples}
\begin{example}
	(P) min $\log x,$ s.t. $0<x\le7$
	\begin{center}
		\begin{asy}
			import graph;
			real f(real x) {
				return log(x);
			}

			path p1 = graph(f, 7, 0.3);
			path xaxis = (-1, 0)--(8, 0);
			path yaxis = (0, -1)--(0, 3);
			Label xlabel = Label('$x$', position=EndPoint);
			Label ylabel = Label('$y$', position=EndPoint);
			draw(p1, arrow=Arrow);
			draw(xaxis, arrow=Arrow, L=xlabel);
			draw(yaxis, arrow=Arrow, L=ylabel);
			dot( (7, log(7))  );
		\end{asy}
	\end{center}
	There is \ul{no minimizer}, and the problem is \vocab{unbounded}. 
\end{example}

\begin{example}
	(P) min $\log x,$ s.t. $1<x\le7$
	\begin{center}
		\begin{asy}
			import graph;
			real f(real x) {
				return log(x);
			}

			path p1 = graph(f, 7, 1);
			path xaxis = (-1, 0)--(8, 0);
			path yaxis = (0, -1)--(0, 3);
			path xint = circle( (1, 0), 0.1 );
			Label xlabel = Label('$x$', position=EndPoint);
			Label ylabel = Label('$y$', position=EndPoint);
			draw(p1);
			draw(xaxis, arrow=Arrow, L=xlabel);
			draw(yaxis, arrow=Arrow, L=ylabel);
			draw(xint, 'black');
			dot( (7, log(7))  );
			dot( (1, 0), 'white');
		\end{asy}
	\end{center}
	In this case, there is still \ul{no minimizer}, but at least the problem is bounded.
\end{example}

The issues that plagued our two examples, that caused them to not have minimizers was the fact that their feasible regions are \bf{not closed}. 

\end{document}

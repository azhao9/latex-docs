\documentclass{article}
\usepackage[sexy, fancy]{evan}
\usepackage{lastpage}
\setlength{\droptitle}{-4em}
\setcounter{MaxMatrixCols}{12}

\lhead{Advanced Algebra I}
\rhead{12/9-12/2016}
\chead{2nd Midterm Exam (take home) - Page \thepage\ of \pageref{LastPage}}

\begin{document}
\title{Final Exam}
\maketitle
\thispagestyle{fancy}

\newpage

\begin{itemize}
	\item[1.] (20 points) Let $G$ be a group of order 6.
		\begin{enumerate}[(a)]
			\item (5 points) How many 3-Sylow subgroups are there in G?
				\begin{soln}
					We have $\abs{G}=3^1\cdot 2.$ By Sylow's Third Theorem, we have
					\begin{align*}
						n_3 &\equiv 1\pmod 3 \\
						n_3 &\mid 2
					\end{align*}
					From the second condition, we must have either $n_3=1$ or $n_3=2.$ Only $n_3=1$ satisfies the first condition, so there is exactly \boxed{1} 3-Sylow subgroup in $G.$	
				\end{soln}

			\item (5 points) Show that $G$ contains at least one subgroup of order 2.
				\begin{proof}
					Since $\abs{G}=2^1\cdot 3,$ by Sylow's First Theorem, $G$ contains a subgroup of order $2^1=2,$ as desired.
				\end{proof}

				Assume, for the remaining part of the exercise, that $G$ is not cyclic.

			\item (5 points) Let $H$ be a subgroup of $G$ of order 2. Consider the set $\Omega=\Set{aH}{a\in G}$ of left cosets of $H$ in $G.$ $G$ acts on $\Omega$ as follows:
				\[G\times \Omega\to \Omega, \quad (g, aH)\mapsto gaH, \quad \forall g\in G, \forall a\in G\]
				Determine the cardinality $\abs{\Omega}$ of $\Omega.$
				\begin{soln}
					We have 
					\[\abs{G:H}=\frac{\abs{G}}{\abs{H}}=\frac{6}{2}=3\]
					and $\abs{G:H}$ counts the number of cosets of $H$ in $G,$ which is exactly $\abs{\Omega}.$
				\end{soln}

			\item (5 points) Let
				\[\varphi:G\to S_{\abs{\Omega}}, \quad \varphi(g)(aH) = gaH\]
				be the group homomorphism of $G$ into the group of permutations of $\Omega.$ Determine $\ker(\varphi).$
				\begin{soln}
					Since $\abs{G}=2^1\cdot 3$ and $H$ is a Sylow 2-subgroup, by Sylow's Third Theorem, we have the following:
					\begin{align*}
						n_2 &\equiv 1\pmod 2 \\
						n_2 &\mid 3
					\end{align*}
					From the second equation, there must be either 1 or 3 Sylow 2-subgroups. From part (a), there is exactly 1 Sylow 3-subgroup, suppose it is $K$ and it is normal in $G.$ If $H$ is the only Sylow 2-subgroup, it is also normal in $G,$ and $\abs{G}=\abs{H}\cdot\abs{K},$ so by a theorem, $G\cong HK.$ However, $H\cong C_2$ and $K\cong C_3$ since these subgroups have prime order, so $G\cong C_2\times C_3,$ which is cyclic. This is a contradiction, since we assumed $G$ was not cyclic. Thus, $n_2=3$ and there are three Sylow 2-subgroups.

					If $g\in \ker(\varphi),$ then
					\[\varphi(g)(aH)=aH\implies (ga)H=aH, \quad \forall a\in G\]
					Since these are cosets, we must have
					\[a\inv ga\in H\implies g\in aHa\inv, \quad \forall a\in G\]
					Since there are three unique Sylow 2-subgroups that are all conjugates of $H,$ the only element contained in all conjugates of $H$ is the identity element. Thus, $g=1,$ so $\ker(\varphi)=\left\{ 1 \right\}.$
				\end{soln}
				
		\end{enumerate}

		\newpage

	\item[2.] (20 points) Let
		\[\sigma=\begin{pmatrix}
				1 & 2 & 3 & 4 & 5 & 6 & 7 & 8 & 9 & 10 & 11 & 12 \\
				10 & 9 & 8 & 11 & 7 & 3 & 2 & 6 & 12 & 5 & 4 & 1
		\end{pmatrix}\]
		be a permutation of the set $X_{12}=\left\{ 1, 2, 3, \cdots, 12 \right\}.$ Compute $\sigma^{2000}.$
		\begin{soln}
			We may decompose $\sigma$ into disjoint cycles:
			\[\sigma=(1, 10, 5, 7, 2, 9, 12)(3, 8, 6)(4, 11)\]
			Since disjoint cycles commute with each other, we have
			\[\sigma^{2000} = (1, 10, 5, 7, 2, 9, 12)^{2000}(3, 8, 6)^{2000}(4, 11)^{2000}\]
			The first cycle has 7 elements, so it has order 7. Similarly, the second cycle has order 3, and the third cycle has order 2. Thus, we have
			\begin{align*}
				\sigma^{2000} &= (1, 10, 5, 7, 2, 9, 12)^{2000}(3, 8, 6)^{2000}(4, 11)^{2000} \\
				&= (1, 10, 5, 7, 2, 9, 12)^{7\cdot 285 + 5}(3, 8, 6)^{3\cdot 666 + 2}(4, 11)^{2\cdot 1000} \\
				&= (1, 10, 5, 7, 2, 9, 12)^5 (3, 8, 6)^2
			\end{align*}
			Let $\tau=(1, 10, 5, 7, 2, 9, 12)$ and $\lambda=(3, 8, 6).$ We have
			\[\tau=\begin{cases}
					1\mapsto 10 \\
					10\mapsto 5 \\
					5\mapsto 7 \\
					7\mapsto 2 \\
					2\mapsto 9 \\
					9\mapsto 12 \\
					12\mapsto 1
				\end{cases} \implies \tau^5 = \begin{cases}
					1\mapsto 9 \\
					9\mapsto 7 \\
					7\mapsto 10 \\
					10\mapsto 12 \\
					12\mapsto 2 \\
					2\mapsto 5 \\
					5\mapsto 1
			\end{cases}\]
			and
			\[\lambda=\begin{cases}
					3\mapsto 8 \\
					8\mapsto 6 \\
					6\mapsto 3
				\end{cases}\implies \lambda^2 = \begin{cases}
					3\mapsto 6 \\
					6\mapsto 8 \\
					8\mapsto 3
			\end{cases}\]
			Thus, we conclude that \[\sigma^{2000} = \begin{pmatrix}
					1 & 2 & 3 & 4 & 5 & 6 & 7 & 8 & 9 & 10 & 11 & 12 \\
					9 & 5 & 6 & 4 & 1 & 8 & 10 & 3 & 7 & 12 & 11 & 2
			\end{pmatrix}\]
		\end{soln}

		\newpage

	\item[3.] (20 points) Let $A=C([0, 1], \RR)$ be the ring of continuous (for the Euclidean topology) functions $f:[0, 1]\to\RR$ and let $I\subset A$ be the subset of functions $f\in A$ such that $f(1/2)=0.$
		\begin{enumerate}[(a)]
			\item (5 points) Show that $I$ an ideal of $A.$
				\begin{proof}
					We first show that $I$ is an additive subgroup of $A.$ The additive identity in $A$ is $f_0(x)\equiv0$ which is in $I$ because $f_0(1/2)=0.$ Next, for two functions $f, g\in I,$ we have
					\[(f+g)(1/2) = f(1/2) + g(1/2) = 0\]
					so $f+g\in I.$ Finally, if $h\in I,$ then $h(1/2)=0.$ The additive inverse of $h$ is $-h,$ and 
					\[(-h)(1/2)=-h(1/2)=0\]
					so $-h\in I$ as well. Thus, $I$ is an additive subgroup of $A.$

					Let $f\in A.$ We know that $A$ is a commutative ring, so it suffices to consider a single direction of multiplication. Let $g\in I,$ so for the product $fg,$ we have
					\[(fg)(1/2) = f(1/2)\cdot g(1/2) = f(1/2) \cdot 0 = 0.\]
					Thus, $fg\in I$ as well, so $fI\subset I$ thus $I$ is an ideal, as desired.
				\end{proof}

			\item (5 points) Is $I$ a prime ideal? Prove or disprove it.
				\begin{proof}
					Since $A$ is a commutative ring, $I$ being a prime ideal is equivalent to $A/I$ being an integral domain. Let $f+I, g+I\in A/I$ where $f, g\in A.$ Then the product is
					\[(f+I)(g+I)=fg+I\]
					If this product is equal to 0 coset, then it is equal to $I.$ Thus, $fg\in I,$ so,
					\[(fg)(1/2) = f(1/2)\cdot g(1/2) = 0\]
					Since $f(1/2), g(1/2)\in \RR$ it must be that either $f(1/2)=0$ or $g(1/2)=0.$ Thus, $f\in I$ or $g\in I,$ which means either $f+I=I$ or $g+I=I,$ so $A/I$ is an integral domain. Thus, $I$ is indeed a prime ideal.
				\end{proof}

			\item (10 points) Is $I$ a maximal ideal? Prove or disprove it.
				\begin{proof}
					Clearly $I\neq A$ since not all continuous functions evaluate to 0 at $1/2.$ Let $J$ be an ideal in $A$ such that $I\subsetneq J\subseteq A.$ Then there exists an element $g\in J$ such that $g\notin I\implies g(1/2)\neq 0.$ Then consider some $f\in A,$ which we may write as
					\[f=\left( f-\frac{f(1/2)}{g(1/2)}\cdot g \right) + \frac{f(1/2)}{g(1/2)}\cdot g\]
					Note that
					\begin{align*}
						\left( f-\frac{f(1/2)}{g(1/2)}\cdot g \right)(1/2) = f(1/2) - \frac{f(1/2)}{g(1/2)}\cdot g(1/2) = 0&\implies f-\frac{f(1/2)}{g(1/2)}\cdot g\in I \\
						&\implies f-\frac{f(1/2)}{g(1/2)}\cdot g\in J
					\end{align*}
					since $I$ is a subset of $J,$ and 
					\[\frac{f(1/2)}{g(1/2)}\cdot g\in J\]
					since it is a constant times $g\in J.$ Thus, $f$ is a sum of elements in $J,$ which is an additive subgroup of $A$ and therefore closed under addition, so we conclude that $f\in J$ as well, so $A\subseteq J.$ Combining this with the fact that $J\subseteq A$ we get $A=J.$ Thus, the only ideal of $A$ containing $I$ is $A$ itself, so $I$ is indeed maximal.

				\end{proof}
				
		\end{enumerate}

		\newpage

	\item[4.] (20 points) Consider the polynomial $f(x)=x^2+2x+3$ in $\ZZ_5[x].$
		\begin{enumerate}[(a)]
			\item (5 points) Is $f(x)$ irreducible in $\ZZ_5[x]?$ If yes, prove it, if not determine a proper factorization of $f(x)$ in $\ZZ_5[x].$
				\begin{proof}
					If $f$ is reducible in $\ZZ_5[x],$ then $f$ factors as $(x-a)(x-b).$ However, we have
					\begin{align*}
						f(0) &= 3 \\
						f(1) &= 1+2+3\equiv 1 \\
						f(2) &= 4+4+3\equiv 1 \\
						f(3) &= 9+6+3\equiv3 \\
						f(4) &= 16+8+3\equiv 2
					\end{align*}
					so there does not exist a value $a\in \ZZ_5$ such that $f(a)=0$ since $\ZZ_5$ is an integral domain. Thus, $f$ does not factor as a product of linear terms, so it is irreducible.
				\end{proof}

			\item (10 points) Let $I=(f(x))$ be the principal ideal in $\ZZ_5[x]$ generated by $f(x).$ Consider the factor ring $F=\ZZ_5[x]/I.$

				Prove that the coset $\overline{x}:=x+I$ is invertible in $F$ (i.e. find its multiplicative inverse) and determine the order of $\overline{x}$ in the multiplicative group $F^\times$ of units of $F.$
				\begin{proof}
					The multiplicative identity in $F$ is $1+I,$ since for any coset $f+I,$ we have 
					\[(f+I)(1+I)=f+I\]
					Thus, we must find an element $g+I\in F$ such that
					\[(g+I)(x+I)=gx+I=1+I\]
					which means that $gx-1\in I$ where $g\in \ZZ_5[x].$ Thus, we must have 
					\[gx-1=h(x^2+2x+3)\]
					for some $h\in \ZZ_5[x]$ with degree at most 1. For simplicity, let $h=3,$ so
					\begin{align*}
						gx-1 &= 3(x^2+2x+3) = 3x^2+6x+9 \\
						&\equiv 3x^2+x-1 \\
						\implies gx &= 3x^2+x \\
						\implies g&= 3x+1
					\end{align*}
					Thus, the multiplicative inverse of $\overline x$ is given by $3x+1+I.$

					If $o(\overline x)=n,$ then we have
					\[(x+I)^n=x^n+I=1+I\implies x^n-1\in I\]
					Note that $x^5\equiv x\pmod 5$ by Fermat's Little Theorem, so
					\[x^4=1\implies x^4-1=0\in I\]
					so $o(\overline x)\mid 4,$ so the order is 1, 2, or 4. 

					If $n=1,$ then 
					\[x-1\in I\implies x-1=h(x^2+2x+3)\]
					for some $h\in \ZZ_5[x].$ This is impossible, because $\ZZ_5$ is an integral domain, so the degree of the RHS is greater than 1. Thus, $n\neq 1.$

					If $n=2,$ then
					\[x^2-1\in I\implies x^2-1=h(x^2+2x+3)\]
					Here, we must have $\deg h=0$ and $h$ monic, so $h=1,$ but this does not satisfy the equality. Thus, $n\neq 2.$

					Thus, $n=4$ is the smallest integer that satisfies $x^n-1\in I,$ and we know this is true because $x^4-1\equiv1-1=0$ in $\ZZ_5.$ Thus, the order of $\overline x$ is \boxed{4.}
				\end{proof}

			\item (5 points) Find, if exists, a coset of order 3 in $F^\times.$
				\begin{soln}
					Let $f\in \ZZ_5[x]$ such that $\deg f\le 1.$ Then suppose the coset $f+I$ has order 3, then 
					\[(f+I)^3=f^3+I=1+I\implies f^3-1\in I\]
					We can simplify by assuming $\deg f\le 1,$ so $f=ax+b,$ and 
					\begin{align*}
						(ax+b)^3-1 &= (ax+b-1)\left[ (ax+b)^2+(ax+b)+1 \right] \in I
					\end{align*}
					If this is in $I,$ then $x^2+2x+3$ divides this product, and since $x^2+2x+3$ is irreducible in $\ZZ_5[x],$ it must divide the quadratic part:
					\[q(x^2+2x+3) = (ax+b)^2+(ax+b)+1 = a^2x^2 + (2ab+a)x + (b^2+b+1)\]
					The only possibility is $q=a^2,$ so 
					\[a^2(x^2+2x+3)=a^2+2a^2x+3a^2 = a^2x^2 + (2ab+a)x + (b^2+b+1)\]
					and equating coefficients, we have
					\begin{align*}
						2a^2 &= 2ab+a \\
						3a^2 &= b^2+b+1
					\end{align*}
					Since $\ZZ_5$ is an integral domain, the first equation implies that $2a=2b+1.$ For the second equation, we have the following:
					\[3a^2 = \begin{cases}
							0 \mapsto 0 \\
							1\mapsto 3 \\
							2\mapsto 2 \\
							3\mapsto 2\\
							4\mapsto 3
						\end{cases}, \quad\quad b^2+b+1 =\begin{cases}
							0\mapsto 1 \\
							1\mapsto 3 \\
							2\mapsto 2 \\
							3\mapsto 3 \\
							4\mapsto 1
					\end{cases}\]
					so the possible pairs satisfying the second equation and corresponding equation 1 results are
					\[(a, b) = \begin{cases}
							(1, 1) \implies 2(1)\neq 2(1)+1\\
							(1, 3) \implies 2(1) = 2(3)+1\\
							(2, 2) \implies 2(2) \neq 2(2)+1\\
							(3, 2) \implies 2(3) \neq 2(2)+1 \\
							(4, 1) \implies 2(4) = 2(1)+1 \\
							(4, 3) \implies 2(4) \neq 2(3)+1
					\end{cases}\]
					Thus, $f=x+3$ and $f=4x+1$ both work, so the coset $x+3+I$ has order 3 in $F^\times.$
				\end{soln}
				
		\end{enumerate}

		\newpage

	\item[5.] (20 points) \textbf{Answer this question OR 6}

		A \textit{local ring} $A$ is a commutative, unital ring with a unique maximal ideal. Which of the following rings is local? For each ring, show or provide a counterexample to the statement: "the ring is local."
		\begin{enumerate}[(a)]
			\item (10 points) $A=\ZZ/p^r\ZZ$

			\item (10 points) $A_1=\ZZ_p[x]$ ring of polynomials in $x$ with coefficients in $\ZZ_p=\ZZ/p\ZZ.$
				
		\end{enumerate}

		\newpage

	\item[6.] (20 points) \textbf{Answer this question OR 5.}
		\begin{enumerate}[(a)]
			\item (10 points) Let $p$ be a prime number. Consider the polynomial $f(x)=x^p-px-1.$ Prove or disprove the following statement:
				\[f(x)\text{ is irreducible in }\QQ[x].\]
				\begin{proof}
					If $p=2,$ then $f(x)=x^2-2x-1$ whose roots are $1\pm \sqrt{2}\notin \QQ,$ so $f$ is irreducible when $p=2.$ 

					If $p>2,$ then $\deg f$ is odd. Thus, if $f$ is reducible, then it must contain at least a single linear term, since polynomials in $\ZZ[x]$ factor as a product of linear terms and irreducible quadratic terms. By the Rational Root Theorem, the only possible roots of $f$ are $\pm 1,$ where $f(1)=-p\neq 0$ and $f(-1)=p-2\neq 0.$ Thus, there are no rational roots, so the factorization of $f$ cannot contain a linear term. Thus, $f$ is irreducible.
				\end{proof}

			\item (10 points) Consider the polynomial $g(x)=x^4+5x^2+3x+2.$ Prove or disprove the following statement:
				\[g(x)\text{ is irreducible in }\QQ[x].\]
				\begin{proof}
					By the Rational Root Theorem, the only possible rational roots are $\pm 1, \pm 2.$ We have
					\begin{align*}
						g(1) &= 1+5+3+2\neq 0 \\
						g(2) &= 16+20+6+2\neq 0 \\
						g(-1) &= 1+5-3+2\neq 0 \\
						g(-2) &= 16+20-6+2\neq 0
					\end{align*}
					Thus, $g$ has no rational roots. Suppose $g$ factorizes as the product of two irreducible quadratics. Thus, 
					\[g(x)=x^4+5x^2+3x+2=(x^2+ax+b)(x^2+cx+d)=x^4+(a+c)x^3+(b+d+ac)x^2+(ad+bc)x+bd\]
					Equating coefficients, we have
					\begin{align*}
						a+c &= 0 \\
						b+d+ac &= 5 \\
						ad+bc &= 3 \\
						bd &= 2
					\end{align*}
					From the last condition, we can have either $b=1, d=2$ or $b=-1, d=-2.$ The other possibilities $b=2, d=1$ and $b=-2, d=1$ are symmetric with the former 2.

					If $b=1, d=2,$ from the first equation we also have $c=-a,$ so the system becomes
					\begin{align*}
						1+2-a^2 &= 5 \\
						2a-a &= 3 
					\end{align*}
					From the second equation, we have $a=3,$ but this does not satisfy the first equation. 

					If $b=-1, d=-2,$ the system becomes
					\begin{align*}
						-1-2-a^2 &= 5 \\
						-2a+a &= 3
					\end{align*}
					but there is no solution because the LHS in the first equation is negative.

					Thus, $g$ cannot factorize as a product of two irreducible quadratics, and cannot have any linear factors, so $g$ is irreducible.
				\end{proof}
		\end{enumerate}

\end{itemize}

\end{document}
